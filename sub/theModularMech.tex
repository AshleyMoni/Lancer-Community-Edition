\section{The Modular Mech}
  THE MODULAR MECH  

Mechs in LANCER are powerful machines, but what makes them more powerful, and your  
character more unique than the regular mech pilot, is your ability to source mech weapons, parts,  
and gear from many different manufacturers and combine them together. LANCER has a  
modular mech system, where each mech comes with a list of gear, gated by license, that once  
acquired can be freely swapped between any other mechs a pilot owns.  

Pilots in LANCER are as much an expert at building their machines as piloting them (or else have  
a team supporting them). Currency is not tracked in LANCER, so a pilot’s ability to get better parts  
and build a stronger mech mainly increases with License Level. License level represents wealth,  
resources, connections, influence, and all the powers a pilot has at their disposal to get better and  
more advanced mech parts. A pilot’s mech and their proficiency with mechs are tracked with a  
character’s mech skills and their license level.   

When you level up from levels 1-12, your pilot gains +1 license point to spend on a license from  
any mech manufacturer, from rank I to rank III (with each rank requiring the previous one). Each  
level of a license you unlock grants you access to various systems and weaponry, with rank II  
giving you a FRAME (a base for your mech), and rank I-III giving you more advanced weapons  
and systems. These weapons and systems are interchangeable. You can add as many to your  
mech as you have mounts or system points to do so, from any source among licenses you have  
unlocked. This is the main way you build ‘your’ mech.
 

After picking a frame and gear, you apply bonuses from your mech skills, and any core  
bonuses you have unlocked (every 3 levels) to get a finished mech.
 

All pilots, even level 0 pilots, always have access to the GMS ‘Everest’ FRAME and the general  
GMS gear list. At level 0, this is the only gear available for pilots until they gain license levels and  
can unlock more gear.
 
\subsection{Without Limits}
                                          WITHOUT LIMITS
 

You never need to re-acquire mech gear or FRAMEs that you have access to, as long as you  
have the license for that gear. You are assumed to generally always have access to that gear,  
through influence, patronage, wealth, or rank. In addition, modern 3-d printing technology has  
advanced enough in LANCER that entire mechs can be printed wholesale. Even if you lose your  
mech in combat, it can easily be re-printed during downtime. You can think about your licenses  
as your character’s ‘class’ if this were a more traditional RPG.
 
\subsection{Mech Structure}
                                        MECH STRUCTURE  

                                                                                                             


The basic structure of a mech is 2 arms and 2 legs. You can modify this however you choose,  
within reason (ask your GM). The general look, structure and layout of your mech has no  
bearing on game play.
 

Let’s go through each mech component in detail:
 
\subsection{Frame}
                                                 FRAME  

Your mech’s FRAME is its chassis, armor, mounts, and reinforcement. It determines your mech’s  
appearance and function, from a heavy siege fighter, to an agile flier, to a cloaking technical- 
focused mech. Think about the FRAME as choosing ‘which’ mech you’re going to pilot.
 

In game terms, a FRAME is the base of your mech. FRAMEs becomes available by unlocking  
level II of mech licenses. 
 

Your mech’s FRAME determines its size and armor. Your FRAME also has base statistics,  
system points and mounts, which determine what kinds of weapons and systems you can add  
to your mech. The kinds of weapons and systems you can add depend on the kinds of mounts  
your FRAME has and the free system points it has available.
 

Finally, your FRAME gives you a powerful CORE system ability to use in combat that they can  
typically only activate once per mission. Activating this core system requires Core Power, which  
you start every mission with, but can only use once.
 
\subsection{Size and Armor}
                                        SIZE AND ARMOR  

Let’s go over the base FRAME features. Every FRAME starts with differing number of mounts,  
system points, armor, a different CORE system and can be size 1/2 to the enormous size 3.:
 

Size: How big your mech is, in spaces, on each size. Humans and the smallest mechs are size  
1/2. Typical mechs are size 1. Size does not directly correlate to how big an actor is, but the  
space it controls around it. Most mechs are size 1 (about 10’ by 10’ in default size) while being  
about 15-20’ tall in the fiction.
 
Armor: A mech’s armor reduces all incoming sources of damage by that amount. Armor mostly  
depends on FRAME, and can’t go higher than 4. Damage with the AP tag or burn ignores armor.
 
\subsection{Mounts and Weapons}
                                   MOUNTS AND WEAPONS  
Your mech can add weapons as long as it has space for them on a mount. Weapons come in  
several sizes, and each mount will only take a certain number and size of weapons.
 

All mech weapons in LANCER have size, type, and damage.  

                                                                                                           


•  Pilot (size): A pilot scale weapon. Pilot weapons are small enough that they cannot Critical Hit.
 
•  Auxiliary (size): The smallest weapon size for mechs, light enough to use alongside larger  
  weapons. When you attack with a weapon, you can also attack with an auxiliary weapon in the  
  same mount for free.  
•  Main (size): A normal sized weapon (for a mech)  
•  Heavy (size): A large, heavier weapon typically used to inflict massive damage.  
•  Superheavy (size): A very large, usually special-class weapon with high power requirements.  
  Can only be fired with the Barrage action.  

•  Type - All weapons have a type, which can be one of the following: CQB, Rifle, Launcher,  
  Cannon, Melee, Nexus. These describe the general effect range and combat function of the  
  weapon.
 

•  Explosive (Damage/ Weapon type): Explosive weapons deal their damage in a single,  
  sudden, and incredibly powerful burst of shrapnel, flame, and/or pressure, blasting in a radius  
  around their point of detonation.   
•  Kinetic (Damage/ Weapon type): Kinetic weapons fire solid projectiles of various calibers and  
  sizes, inert or innervated, that rely on simple collision to deal damage from point-of-impact  
  through to point-of-exit. Kinetic weapons utilize chemical and electronic methods of firing or  
  launching their projectiles, and are commonly fed by belts, boxes, and/or internal or external  
  magazines.   
•  Energy (Damage/ Weapon type): Energy weapons are weapons that project beams, lances,  
  bolts, waves, or cones of different energy to damage and destroy their targets. Commonly  
  powered by external or internal batteries, or hooked directly into a mech’s power core, energy  
  weapons demand tremendous amounts of input to provide tremendous amounts of output. 
 

Weapons might also deal Heat or Burn (damage over time), which have additional effects,  
detailed in the damage section.  
\subsection{Mounts}
                                                 MOUNTS  

Mech FRAMEs are standard-built to mount a limited number of core systems. Too many  
weapons or systems will overtax the reactor or add too much stress on the mech’s structure.
 

Each mech FRAME has differing numbers of the following mount points. You cannot add  
weapons or systems to your mech if you don’t have an available mount to do so. You can add a  
weapon of any smaller size in a space that could take a larger weapon (for example, you could  
add a main or aux weapon to a heavy mount, or add two aux weapons to a main/aux mount).
 

Aux/Aux mount: This mount takes up to 2 auxiliary weapons
 
Main/Aux mount: This mount takes 1 main weapon and 1 aux weapon
 
Main mount: This mount takes 1 main weapon
 
Heavy mount:  This mount takes 1 heavy weapon
 

                                                                                                              


Flexible mount: This mount takes either 1 main weapon or up to 2 auxiliary weapons
 

Superheavy weapons take a heavy mount and one other mount of any size.
 

Some mechs have the following mount:
 

Integrated mount: This mount is part of a mech’s FRAME. It includes the listed weapon by  
default, which cannot be removed or replaced. This mount and weapon cannot be removed,  
modified, or duplicated in any way. 
 

Weapons mounted on a mech don’t necessarily need to be part of its chassis - they could be  
slung in holsters, build into compartments, or held/wielded normally. You can decide which when  
you build your mech - it has no effect in the rules. Mounts represent the tax on your mech’s  
systems more than an actual physical structure.
 
\subsection{System Points}
                                          SYSTEM POINTS
 

Mech FRAMEs also come with a certain number of System Points (SP). System points can be  
spent to add additional systems to your mech, and some weapons or heavier systems will take  
system points to add to your mech in addition to requiring open mounts. As a pilot levels up, you  
can add your grit (1/2 your level) to your total system points. You cannot add systems to your  
mech that would cause you to exceed your system points.
 
\subsection{Core System}
                                             CORE system   
Each FRAME comes with a CORE system. This a powerful ability that is unique to the FRAME,  
can’t be transferred (you have to use the FRAME to use it), and can only be used typically once a  
mission by consuming Core Power.
 

All mech FRAMEs have a reservoir of high efficiency system power that is designed only to work  
for a short period of time. This Core Power is essential to the high-powered systems that many  
mechs utilize in emergency situations or points of heavy action.
 

Your mech either has Core Power or it doesn’t. There’s no way to ‘save’ it up (you either have it  
or you don’t). You always get core power when you start a mission or full repair. You only gain  
core power when taking a full repair, or unless the GM grants you core power during the course  
of a mission.
 
\subsection{Base statistics and improvements}
                         BASE STATISTICS and IMPROVEMENTS  
Every FRAME has different base statistics that it starts with, giving it a unique role.
 

HP: Like your pilot, your mech has HP (hit points). Your mech, however, has 4 structure, and  
isn’t destroyed when reaching 0 HP. Instead, its HP resets when it loses structure and it makes a  

                                                                                                             


critical damage check. When a mech runs out of structure, it goes into the CRITICAL state, and  
makes a critical check each time it takes damage.
 
Repairs: Your mech can heal and repair its systems by spending repairs, like a currency. If it runs  
out of repairs, it can no longer heal or repaired destroyed systems.
 
Speed: How far your mech can move when it moves.
 
Evasion: How hard it is for most ranged and melee attacks to hit your mech.
 
Sensor Range: The distance your mech can use tech systems or attacks. If something’s in your  
sensor range, you know it’s there unless it’s hiding (even if you can’t directly see it).
 
Tech attack: Your mech can add its tech attack instead of grit when conducting electronic  
warfare.
 
E-defense: How hard it is for electronic or guided weapons or systems to hit your mech.
 
Heat Capacity: Your mech can take heat from electronic warfare or its own systems. If it takes  
more than its heat capacity, it overheats and suffers adverse effects.
 
Limited systems: Some weapons only have limited uses between full repairs. Once used up,  
they can’t be used again until replenished.
 

Your pilot’s mech skills can boost some of the statistics of any FRAME you pilot when you build  
it, giving you your final mech. This is your pilot’s unique or personal touch that you add to the  
stock FRAME to customize it.
 

The following features of a FRAME get bonuses from your mech skills. This is where your pilot’s  
personal abilities really kick in.
 

GRIT (1/2 your level) can be added to your mech’s attack bonus, HP, and system points
 

Your HULL skill affects your mech’s HP and durability. 
       •  Each point of hull gives you +2 hp. 
       •  Each 2 points of hull gives your mech +1 repair capacity 
Your AGILITY skill affects your mech’s evasion and speed. 
       •  Each point of agility gives +1 evasion 
       •  Each 2 points of agility gives your mech +1 speed 
Your SYSTEMS skill affects sensor range, tech attack and e-defense. 
       •  You can add your systems to your mechs’ sensor range and electronic defense 
       •  You can add systems to tech attack 
Your ENGINEERING skill affects heat capacity and limited systems 
       •  You can add your engineering skill to heat capacity 
       •  Each 2 points of engineering gives your mech +1 to maximum uses for limited systems 
         and weapons 
\subsection{Core Bonus}
                                              CORE BONUS 

As your pilot acquires more licenses with a particular manufacturer, they will gain manufacturer- 
specific knowledge and skills. In game this is represented by a CORE bonus. Every 3 levels, you 

                                                                                                                 


can apply a new CORE bonus to your CORE. Bonuses are a permanent improvements, and  
apply no matter what FRAME you are using. They are are unique (you cannot take the same  
twice) and can offer interesting ways to customize your mech. You can always choose GMS  
CORE bonuses, but to choose a CORE bonus from a manufacturer, you need 3 points in licenses  
from that manufacturer (any licenses, you can mix and match), for each CORE bonus you have  
from them. For example, to take 1 CORE bonus from IPS-Northstar requires at least 3 points in  
IPS-N licenses, to take 2 would require at least 6.  

\subsection{Talents}
                                                 TALENTS  

Your pilot’s experience and abilities with piloting a mech is tracked by Talents. This is where skill  
and ingenuity can push your mech past its limits. Talents give your character abilities with  
particular types of weapons, systems, or styles of play that help define your character further.
 

Talents, like licenses, go from rank I-III. At level 0, you gain three talent points to spend on  
Talents, but can’t take any past rank I. When you level up, you gain +1 talent point to spend on  
talents. Talents only apply to your pilot’s capabilities when piloting a mech (with a few  
exceptions).
 

You can find the talent list in the compendium.
 

                                   PUTTING IT ALL TOGETHER  
Constructing a mech may seem daunting at first, but is actually a fairly simple process.  

At level 0:  
     1.  Pick a FRAME from the licenses you have available to you. At level 0, this should just be  
         the GMS Everest, and at level 2 you can already pick a new FRAME. Your FRAME gives  
         your mech its armor, SP, mounts, base stats, and a CORE system  
    2.   Add bonuses from mech skills  
                 HULL: + 2HP/point, +1 repair cap/2 points  
                 AGILITY: +1 evasion/point, +1 speed/2points  
                 SYSTEMS: +1 tech attack, sensor range, and e-defense per point  
                 ENGINEERING: +1 heat cap/point, +1 to all limited gear/2 points  
    3.   Pick weapons for your FRAME mounts from those you have access to. You might choose  
         different weapons depending on the size of each mount. At level 0, you have access to  
         only the GMS weapon list.  
    4.   Spend your SP on systems. You cannot spend over the amount your mech has, and any  
         excess or unspent SP are lost. At level 0, you can only choose from GMS systems. At  
         higher levels, you can add your grit to total SP on all mechs you make.  
    5.   A mech can always mount duplicate weapons or systems, unless those weapons or  
         systems have the Unique tag.  
    6.   Allocate your talent points. At level 0, you can have 3 rank I talents.
 
    7.   You’re done!
 

                                                                                                                


                                             MOVING FORWARD  

If you want to skip ahead to creating a mech right away, you can go directly to the compendium.  
Though the choices in the full mech compendium are considerable, at level 0, a pilot only has  
more limited access to the General Massive Systems gear. If you’re new to the game or learning  
the system, it’s recommended you start there. You’re not ‘stuck’ with whatever mech you create,  
as you can make a new one every mission if need be.  

Before creating a mech, you might find it useful to read through the rules on damage, heat, and  
repair in the following section, and the rules for mech combat in the following section.
 

                                                                                                                 
