\section{The Cavalry} 

It’s 5014, and our arm of the galaxy is home to trillions. It is not a safe place,  
but it burns bright, and for some there are gentle lands. 
 

Out from our humble beginnings, humanity has colonized the darkness. We have set empty  
worlds and barren moons alight with civilization, have tamed asteroids and gas giants, have even  
built homes in the hard vacuum of space itself. We have taken root throughout our arm of the  
Milky Way galaxy. In every situation and setting, humanity has made their home; life, however it  
expresses itself, continues. Stories begin and end across the stars, though most never leave the  
worlds they were born to. 
 

But for some, their life is as a river, ever-moving, with the land of their birth left somewhere far  
behind. Traders and smugglers, refugees and immigrants, miners, pirates, scientists, volunteers,  
colonists, soldiers and conscripts: humanity on the move, always. Wars pull hundreds of  
thousands into the current, trade and migration yet more. For every ten stable homes, there is  
one family’s worth that has been uprooted, for good or ill. 
 

Blink gates dot the galaxy. These massive, star-bound stations - gatehouses - tap into an  
unknowably vast and strange plane known as the Blink and facilitate safe, instant travel, opening  
all corners of the Deep to the daring. These portals are open and common to those with the  
permission, money, or clearance to enjoy it: Thousands of ships travel through the Blink each  
standard day for trade, migration, travel, war, or any myriad other purposes, so long as they have  
clearance. 
 

The Omninet connects all of humanity to one another, a decentralized network that links every  
computer, every server, every thing to everything. More than just a way to communicate, more  
than just a way for far-flung worlds to read of the galaxy’s news and listen to the music of the  
spheres, the Omninet facilitates government and industry. Data is the new wealth, and the  
Omninet allows for the sharing of the wealth of all worlds. 
 

Manna, the universal currency, unites all the disparate nations of the human diaspora. A single  
currency — based on a combination of material wealth, labor, intellectual property, experience,  
and data — that any market on any planet will accept as fair currency. When a galaxy’s wealth of  
raw resource is available for exploitation, what becomes valuable is not gold but data. 
 

This vast spread of humanity, these trillions of souls, can only be administered by one body:  
Union, the hegemonic council that rules from Cradle, the ancestral home of the human diaspora.  
Earth and Mars, Mercury and Venus. Saturn, Jupiter, Neptune, and Uranus. Io. Titan. Europa.  
Phobos and Deimos. Sol. These worlds and moons around this warm yellow star form Cradle,  
the seat of Union’s power, and the very heart of humanity. Union controls the triumvirate of  
progress: the Blink Gates, the Omninet, and Manna. Without the triumvirate, without Union, the  
galaxy falls into chaos. 
 

All that being so, Union, Cradle -- and far more so Earth herself -- are things and places of myth  
to the vast majority of humanity, fictionalized in Omninet dramas and novels, dreamt about by  

                                                                                                                    


children and wanderers, idealized as the promised land or damned as the pit from whence we  
came by religions across the galaxy. Few have ever seen a Union administrator, or suffered a  
Union naval campaign. For all its control over human affairs, Union prefers to rule from a  
distance. 
 

The galaxy, despite its interconnectedness, is a dangerous place. Rebellion, insurrection, piracy,  
civil wars -- even wars between worlds -- flare up and burn their way through Union space,  
though only the most desperate or dangerous of conflicts require Union’s direct attention.  
Disputes between Union’s subject states are common enough that there is a need for individual  
militaries and militias: Five major suppliers have permeated the galaxy to offer arms and armor to  
those with Manna enough to afford them and an Omninet connection with enough bandwidth to  
download them. 
 

Into this broad and dangerous environment come the players. You take on the role of a lancer --  
a mech pilot, or simply pilot for short -- in a squadron with your fellow players. Whatever the  
conflict, whatever the scale, you can bet that lancers will be involved; together, you and your  
squadron will run missions as the tip of the spear, fighting in only the most dangerous and  
important engagements. You’re the backbone, the heroes, the knights in shining armor, the  
decorated aces sent in when all hope seems lost and victory must be assured. 
 
In short, you, the players, are the cavalry. 

Your character in the world of LANCER is a mechanized cavalry pilot of particular note -- a  
Lancer. You play as the cavalry. Whatever the role, whatever the terrain, whatever the enemy,  
you are the one who gets called in to break the siege, to hold the line. To save the day. 
 

Your pilot hails from a world and culture of your choice and description, but is human. You might  
call Earth home (or, to be accurate to the setting, Cradle), but to be born on Earth in the age of  
Union is exceedingly rare — in the world of LANCER it’s been millennia since we left Earth, and  
the majority of humanity live among the stars and habitable worlds in our arm of the Milky Way.  
In LANCER, humanity is familiar and strange in equal measure.
 

As a pilot, you represent the end product of heavy technological and capital investment on the  
part of your employers or officers, be they corporate, state, tribal, mercenary, a noble family, or  
military. Through a combination of training, natural skill, battlefield experience, and neural  
augmentation, a mech pilot is the equivalent of a knight of old, a flying ace, or other prestige- 
class warrior. 
 

Mech pilots, they will proudly tell you, are a cut above the rest. 
 

They are not entirely wrong. Recruiting, training, and maintaining a mech pilot involves a  
tremendous amount of capital and time investment compared to your average soldier. To operate  

                                                                                                                   


a mech at peak efficiency, a pilot needs to have extensive physical and mental augmentation --  
or be outfitted with advanced, expensive equipment -- years of virtual training, extensive field  
experience, and rigorous zero-G acclimation. Washout rates are high, as are injury rates as a  
result of the demanding training process, but this high bar is necessary: Once a candidate has  
attained their final certifications and has been shipped out to their first post, they face only the  
most dangerous missions. Mechs and their pilots aren’t sent in to keep the peace: they’re sent in  
when all other options have failed. 
 

You are one such person. A pilot. They’re human, though, and flawed like the rest of us. Pilots  
are heroes and villains, brave souls and cowards, lovers and fighters all. They stand when  
everyone else flees, are the first to run to danger, are the best and the brightest of us. But, they  
too, break under the pressure, fail, and kill when they could have saved or spared. 
 

Pilots come from all walks of life. Station, criminal status, wealth -- once a candidate has been  
identified, there are no disqualifying factors for their recruitment. The galaxy is vast, and  
humanity numbers in the trillions, but there’s only one of you. Whatever the circumstances,  
whatever the road that brought you to where you are, you are a now a pilot. You are a cut above  
the rest. You’re the cavalry, the tip of the spear, humanity’s best hope. 
 

Congratulations. You made it through selection and training. You now have your requisite  
certifications. You have your first post, and you are en route to meet your new squadron. This is  
the last real downtime you’ve got before you start your tour, so acquaint yourself with the rules  
and regulations. Write down a bit about yourself. Figure out a callsign, something in Common so  
that everyone knows your talk. 
 

You’re a pilot now, but you’re still a rookie, a greenhorn, a wet-behind the ears boot with no live  
drops notched on their helm. Maybe you’re gonna make a name for yourself out there on the line,  
or maybe some deck techs will hose you out of what remains of your cockpit. 
 

Either way, as they say, you’re a cut above the rest. 
 

Let’s see what you got. 
 
