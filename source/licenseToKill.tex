\chapter{Licensed to Kill: Characters in Lancer}


Unlike other role playing games, LANCER does not track currency. Your access to mech gear,
upgrades, talents, and other aspects of your character are instead gated by licenses. Licenses
are controlled tightly by the major powers in LANCER and allow holders nearly unlimited access
to the items they contain.


Progression in \textbf{LANCER} is represented through gaining license levels by completing missions,
which allows you to unlock \textbf{Ranks} in \textbf{Mech Licenses}, unlocking more gear and mechs for your
pilot to use. All pilots begin at \textbf{license level 0} and can level to \textbf{level 12}. Higher license level not
only allows you to unlock mech gear but also allows your pilot’s \textbf{skills} to improve, improves both
your pilot and mech’s hit points, gives them more \textbf{talent} points to spend on mech combat
abilities, and allows your pilot to gain extra customization options.


Your license level \textbf{describes both your pilot and your mech} - as you level up, your pilot and
mech will both become stronger and have access to more advanced gear and combat
techniques. The first section of this book deals with narrative play and playing as a pilot, the
second with mech combat and playing as a mech, each going into a lot more detail. You can
\textbf{return to this section later} if you need to refer back to it.

\begin{center}
\textbf{Pilot progression}
\end{center}

\textbf{If you’re new to LANCER}, it’s recommended you start a character at \textbf{Level 0} (you haven’t
completed any missions). At this level, your pilot has access to only a limited pool of basic
licenses available to all pilots.


\textbf{At level 0}, characters have 4 \textbf{pilot skills} at +2 based on their background, some\textbf{specialties}
based on their background, +2 to one \textbf{mech skill} or +1 to two mech skills, \textbf{three rank I talents},
and have access to all \textbf{G.M.S. mech licenses}. These are basic licenses that give mechs and
gear that all qualified pilots can access, regardless of license level. For more information on
mech licenses, building a mech, and a list of gear and mech components, see the section later in
this book


When a character \textbf{completes a mission}, they level up.
\begin{itemize}
\item They get \textbf{+2 to any pilot skill} and \textbf{+1 to a mech skill}
\item They gain \textbf{one talent point} and \textbf{one license point} to spend on mech talents and
      licenses. Spending a talent point acquires a talent or license at rank I. Further
      points can then be spent to take a talent or license to rank II or III.
\end{itemize}


\textbf{Grit:} A character adds 1/2 their level to attack rolls, pilot Hit Points, mech Hit Points, and mech
System Points

\textbf{Advance:} Every \textbf{3 levels (3/6/9/12)}, a character can get a new
pilot \textbf{specialty}, and gets a \textbf{core bonus}, a powerful improvement to all mechs they build

\begin{center}
\textbf{Retrofitting}
\end{center}

Every time you level up, you can \textbf{re-allocate} all the points from one of your talents to any other
talent, \textbf{or} you can re-allocate all the points from one of your licenses to any other license. If this
would cause you to no longer qualify for a CORE bonus, you must replace that bonus with a new
one that you qualify for.

\subimport{./licenseToKill/}{levelingChart}
