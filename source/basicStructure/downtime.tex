\section{Downtime}
                                             DOWNTIME


Pilots are individuals (exceptional individuals, perhaps), with lives to live outside of their mechs.
It’s assumed in LANCER that your pilot has a lot to do outside of a mission. What exactly that is
depends on the GM, the narrative, and the players involved in the game.


Your story might not include much downtime - maybe the character are on a planet under siege
and have little time to pursue other projects. It might include a lot of downtime - maybe there’s a
time skip in between missions of several months, allowing characters many opportunities to
pursue their other goals.


What’s important for the game of LANCER is there must be at least SOME downtime between
missions, even if that’s only a few hours. This allows players to prepare for the next mission and
take actions that will impact the story going forward.


                                           TIME AND DOWNTIME


Downtime is almost purely narrative and doesn’t have to follow strict moment-to-moment action.
You can allow one roll in downtime to cover any amount of time, from an hour to a few months




worth of activity. Generally it is better to let the outcomes of rolls follow the narrative
circumstances, ie the impact of rolls that cover several months of activity should be
consequently more.


                      REWARDING DOWNTIME: FREEFORM VS STRUCTURED


Downtime can be used to accumulate reserves by taking certain actions in order to prepare for
the next mission. During downtime, players can take one downtime action, or two if downtime is
especially long. If the players are under siege, for example, they probably only have time for one.


The purpose of downtime is not merely to prepare for the next mission, however - it’s also to
progress personal stories, advance plots, or flesh out characters. You can do freeform
roleplaying as much as you like during downtime with absolutely no intent (as it is during any
mission), but the GM can still feel free to reward this kind of play with similar rewards if it
feels appropriate. For example, even if Pan and Penny are role-playing a scene of them drinking
at a bar together with no intent to prepare anything, gain reserves, or take downtime actions, the
GM might decide to give them +1 Accuracy on skill checks to help each other during the next
mission because of their bond.


\subsection{Reserves}
                                                 RESERVES

During downtime, you might want to prepare reserves for the next mission. This is a catch all
term for extra supplies, gear, support, bonuses, allies, etc that you accumulate during downtime.
The GM can give you reserves at any time as a result of the story (for example, granting you
reserves as part of a mission). You can also typically ask for reserves by using the power at a
cost tool below, or by successfully taking downtime actions.


You establish reserves you are taking on a mission before you embark on that mission. There’s
no limit to what you can take along other than what the GM and your own actions are willing to
grant you.


Here’s some examples of reserves. Don’t take these examples as set in stone, but as ideas for
the kind of reserves that could be available to you. Abstractly speaking, reserves is anything
you are holding as an advantage for the next mission. The reserves available to characters
should be dependent on the GM, your own actions, and the story. Some of these are relevant to
mechs (See the mech section for more details), some for pilots, and some for the story.


Before a mission, if the GM is using a list of reserves, it can help for it to be visible for all
players so players have a good idea of what is potentially available for them. When rewarding
reserves, the GM doesn’t necessarily have to pick from this list.


TABLE: Narrative Reserves




 d20       Narrative	Reserves

  1-2      Access: Gain a keycard, invite, bribes, or insider access to a particular location

  3-4      Backing: Political support from someone powerful for this mission - you can invoke it
           as leverage
  5-6      Supplies: Cross a hazardous or hostile area without having to make a skill check

  7-8      Disguise: Prepare a disguise or cover identity, allowing you to sneak into a location
           uncontested
 9-10      Diversion: Prepare or arrange for a diversion, giving you time to take action without
           fear of consequence
   11-12   Blackmail: Gain blackmail or sensitive information on a particular person

   13-14   Reputation: Make a good name for yourself for the next mission, starting you off on a
           good position with everyone you meet
   15-16   Safe Harbor: Gain a guaranteed safe location where you can convene, plan, or
           recuperate
   17-18   Tracking: Know the location of important objects or people for this mission

   19-20   Knowledge: Gain important knowledge of local history, customs, or culture etiquette

TABLE: Mech Equipment and gear reserves

 d20       Mech	reserves

  1-2      Ammo: Get extra uses (+1 or +2) to a (limited) weapon or system

  3-4      Rented gear: Get access to a weapon or piece of mech gear you don’t normally have
           access to for the mission only
  5-6      Extra repairs: Start the mission with +2 repairs on your mech.

  7-8      CORE battery: Consume this to gain core power on your mech, allowing it to use its
           most powerful ability again (you can’t get more than 1 core power at a time)
 9-10      Deployable Shield: Gain a 1-use deployable shield generator, a size 1 deployable that
           grants all allied actors in a burst 2 radius around it light cover
   11-12   Redundant repair: 1/mission make the stabilize action as a free action

   13-14   Systems reinforcement: Get +1 accuracy to hull, agility, systems, or engineering
           checks, this mission only (choose one)
   15-16   Smart ammo: All weapons become smart, this mission only

   17-18   Boosted servos: Become immune to the slowed condition, this mission only

   19-20   Jump jets: Your mech can fly when it makes its regular move, this mission only

TABLE: Tactical reserves




 d20       Tactical	Reserves

 1-2       Scouting: Get detailed information on the kinds of mechs and threats you will be
           facing soon, such as number, type, and statistics
 3-4       Vehicle: Gain use of a transport vehicle or starship for this mission only (a tier 1 NPC
           with the vehicle or ship tag)
 5-6       Reinforcements: Call in an NPC mech ally, once per mission (choose a tier 1-3 NPC
           from the NPC section).
 7-8       Environmental Shielding: Ignore a particular battlefield hazard or dangerous terrain,
           such as extreme heat or cold
 9-10      Accuracy: Gain +1 accuracy on a particular skill or action due to training or
           enhancement, this mission only
 11-12     Bombardment: Call in a artillery or orbital bombardment once during mech combat
           (Full Action, range 30 within line of sight, blast 2, 3d6 explosive damage).
 13-14     Extended Harness: Carry an extra pilot weapon and two pieces of pilot gear into this
           mission
 15-16     Ambush: Choose exactly where your next battle will take place, including the terrain
           and cover set up, etc
 17-18     Orbital Drop: Start the mission dropping from orbit into a heavily fortified or hard to
           reach location
 19-20     NHP assistant: Gain an NHP, controlled by the GM, that can give you advice on the
           current situation

Here’s some examples of reserves in play:


Xi Xiaoyun

During downtime, Xi negotiates with the powerful boss of the Red Dog Triad local crime
syndicate. He knows their next mission might take them through their territory. He makes some
checks (and some promises) and gets in the boss’s good graces. The GM tells his player he can
write down the following for reserves:

	        Weaponry: I’ve borrowed a Coldcore sniper rifle from the triad (from the Smith Shimano
Corpro gear catalogue) just for this mission

	        Backing: I have the backing of the triad. If someone messes with me, I can invoke that.

	        Information: I can ask the GM what kind of enemies we might run into on the mission
due to the Triad’s spy networks


Pard Landover

Pard decides to do some scouting during downtime, scavenging a strange, abandoned derelict
in his mech for supplies. After some harrowing exploration and a few skill checks, the GM asks
him to write the following down:

	        Ammo: I’ve recovered extra ammo cases, giving me +1 use to all limited weapons

	        Nanomaterials: I’ve recovered some strange nano materials, giving me +2 repairs to my
mech this next mission



\subsection{Downtime Actions}

                                           Downtime Actions


You can use these story actions below to represent what your pilot gets up to during a stretch of
downtime. You can also use them to fill gaps in the story where you want to speed things up,
‘montage’, or cover a greater stretch of time.


Unlike a regular skill check, these downtime actions have specific outcomes depending on
whether you hit, miss, or excel, and usually ask you to choose from a list. The actual skill used is
dependent on context. For example, the Get a Damn Drink action could be rolled using Survive
(for alcohol tolerance, most likely), or Getting Word on the Streets (charming the locals).


Each outcome is dependent on the roll, but is up to the GM and the player making the action to
establish the details, using the prompts given. You can go into as much detail about the ins and
outs of the action as you want - the outcome is the only thing that’s important.

As a good rule, a pilot can do only one of these during downtime, and one or two if the downtime
is especially long. You can always keep the outcomes of these actions as reserves for the next
mission.

You can easily write your own downtime actions that are more dependent on your particular
narrative or story, using them modeled ones below.

To start, let’s look at the Power at a Cost action.

\subsection{Power at a Cost}
POWER AT A COST

This is a really simple downtime action for gaining rewards, opportunities, or additional
resources (such as reserves). You might want reserves, or something more abstract, like time,
safety, information, allies, support, a base of operations, materials, shelter, food, a damn
pack of cigarettes.


Name what you want. You can always get it, but:

BUT (the GM chooses one or two of the following and fills in details, depending on how
outlandish the request is):

    -    It’s going to take a lot more time than they anticipated
    -    It’s going to be really damn risky
    -    You have to have to give something up or leave something behind (wealth, resources,
         allies)
    -    You’re going to piss someone or something important and powerful off to get it
    -    It’s going to involve going WILDLY off the plan
    -    You’ll need more information to proceed safely
    -    What you put together is going to fall apart damn soon




    -    You’ll need to gather more resources first (you know where to find them, however)
    -    You can’t EXACTLY get what they want, just approximately what you want, a lesser
         version, or less of what you want


You can also use the power at a cost action mid-mission for similar effects (not necessarily
during downtime). The rest of the downtime actions are typically for use only in downtime, but
you could adapt them mid mission if they fit.


BUY SOME TIME
You try and stave off some reckoning, extend your window of opportunity, or merely buy more
time and breathing room for you and your group to act. You might be trying to dodge some heat,
survive stranded in the wilderness, or cause a distraction so another plan can go off. You can use
that distraction or bought time as reserves for the next mission.

On a total result of 9 or lower, you’re out of time, and whatever you’re trying to stall catches up
with you unless drastic measures are taken right now.
On a total result of 10-19, you can buy enough time, but the situation becomes precarious or
desperate. Next time you get this result with the same situation, treat it as a 9 or lower.
On a total result of 20+, you buy just enough time as you need for now, until the next mission. If
you’ve already gotten this result, it becomes a 10-19 for the same situation next time.

GET A DAMN DRINK
You can only make this action where there’s a drink to actually get (in town, a station, a city, or
some other populated area). You blow off some steam, carouse, and generally get into trouble.
You could be doing this to make connections, collect gossip, forge a reputation, or maybe just to
forget what happened on the last mission. Drinks don’t have to be involved, but vice usually is.
There’s usually trouble.

On a total result of 9 or lower, you can decide whether you had good time or not. However, you
wake up in a gutter somewhere with only one of the following (choose!):
    -    Your dignity
    -    Most of your possessions
    -    Your memory
On a total result of 10-19, you get one of the following as reserves and lose one of the following:
    -    A good reputation
    -    A friend or connection
    -    A useful item or piece of information
    -    A convenient opportunity
On a total result of 20+, you get two of the above and don’t lose anything

GET CREATIVE
You tweak something or attempt to make something new, either a physical project, or a piece of
software. It doesn’t have to be something on the a gear list, but it generally can’t be something as
impactful as a piece of mech gear. Once finished, you can use it as reserves.




On a total result of 9 or lower, you don’t make any progress on your project for now. If you already
got this result on the same project, treat it as a 10-19 next time.
On a total result of 10-19, you can make progress on your project, but can’t finish it. You can finish
it next time you have downtime without a roll if you get some things before then (pick 2):
    -    Quality materials
    -    Specific knowledge or techniques
    -    Specialized tools
    -    A good workspace
On a total result of 20+, you can finish your project during this downtime. If it’s complicated, treat
this result as a 10-19, but only choose 1.

GET FOCUSED
You focus on increasing your own skills, training, and self-improvement. You might practice, learn,
meditate, or call on a teacher. Before you make this action, name one thing you’d like to learn or
improve on (a skill, technique, an academic subject or language). This could be something like
starship piloting, cooking, chess, boxing, history, or etiquette. It should generally be a non-combat
action.

The GM will give you a new trigger for a skill, giving you +1 Accuracy when you undertake your
new action (you need to work out which skill to use). For example, the trigger could be +1
Accuracy to roll sharp when playing chess, or +1 Accuracy to roll cool when dancing.

GET ORGANIZED
You start, run, or improve an organization, business, or other venture. Describe it when you start it
up: it’s purpose or goal, mode of operation, and primary business. Track your organization’s
efficiency and influence, from 1-6 (they each start at 1). When you want to make a narrative role
where your organization could help, you can roll your organization’s efficiency or influence like a
skill check. Efficiency is how effective your organization is at what it does (a military organization
with high efficiency would be good at combat, for example). Influence is your organization’s size,
reach, and reputation. An organization with high influence can easily acquire assets or create
opportunities, transport people, or sway public opinion. You can use these advantages as
reserves.

On a total result of 9 or lower, your organization folds immediately unless:
    -    It lowers in efficiency and influence by 1, to a minimum of 1. If it’s already at 1 for both of
         these, you can’t choose this option.
    -    It takes one of the following actions: pay debts, prove worthiness, get bailed out, make an
         aggressive move
On a total result of 10-19, your organization is stable. It gains +2 influence or efficiency, to a
maximum of +6.
On a total result of 20+ your organization gains +2 influence and efficiency, to a maximum of +6

GATHER INFORMATION




You poke your nose around, perhaps where it doesn’t belong. You’re investigating something,
doing research, following up on a mystery, tracking a target, or keeping an eye on something. You
might be doing research in a library, or go undercover in an organization to learn what you can.
Whatever you’re doing, you’re generally trying to gather information on a subject of your choosing.
You can use this information as reserves.

On a total result of 9 or lower, you can choose to get out now, or treat this result as a 10-15. If you
choose the latter, you get your information but you are immediately found out, noticed, captured,
or discovered by an organization that either controls or seeks that information.
On a total result of 10-19, you find the information you’ve looking for. However, (choose one):
    -    You leave clear evidence of your rummaging
    -    You have to dispatch, harm, or implicate someone innocent
On a total result of 20+, you get your information cleanly, no complications

GET CONNECTED
You try and make connections, call upon favors, ask for help, or drum up support for a particular
course of action. You need access to communications or just good old fashioned face to face
conversation to take this action. You can use your connection’s resources or aid as reserves for
the next mission.

On a total result of 9 or lower, you’ve got to do a favor or make good on a promise for your
connection right now. Otherwise, they won’t help you at all. If you take action right away, however,
they’ll go along with what you want.
On a total result of 10-19, your connection will help you, but you’ve got to do a favor or make good
on a promise after they help you. If you don’t, treat any result as a 9 or lower next time with the
same organization.
On a total result of 20+, your connection will help you out, no strings attached.

SCROUNGE AND BARTER
You try and get your hands on some gear or asset for your group by dredging the scrapyard,
chasing down rumors, bartering in the local market, hunting around, or through good old
fashioned force of will. You could try and get some better pilot gear that could help you, a vehicle,
narcotics, goods, or other sundries. It’s got to be something physical that you can acquire, but
doesn’t necessarily have to be on the gear list. If you get it, you can take it on the next mission as
reserves.

On a total result of 9 or lower, you can get what you’re looking for, but (choose one):
    -    It was stolen, probably from someone who’s looking for it
    -    It’s degraded, old, filthy, or malfunctioning
    -    Someone else has it right now and won’t give it up without force or convincing
On a total result of 10-19, you can get what you’re looking for, as long as you trade in a little
(choose 1):
    -    Time
    -    Dignity
    -    Reputation




     -   Health, comfort and wellness
On a total result of 20+, you get what you’re looking for, no problems at all.

\subsection{Putting it all Together}
                                    PUTTING IT ALL TOGETHER

After downtime naturally wraps up, the game should loop right back into the next mission,
following the same structure (Brief, Preparation, Mission, Debrief, Downtime).

Of course, narrative play doesn’t cover everything in a game of LANCER. There’s inevitably
always conflict, and that conflict is probably going to involve mechs, which brings us to the next
section.
