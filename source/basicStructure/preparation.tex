\section{Preparation}
                                                Preparation


During this stage, players choose the mechs and gear they are going to bring with them to start
the mission (if they’re bringing mechs along). This doesn’t necessarily prevent them from
changing gear mid-mission, but only the resources, gear, weapons, etc that they start with.


LANCER does not track currency. Instead, players are assumed to have access to whatever gear,
mech parts, etc that their licenses have allowed them to unlock (usually freely available to print
from a nearby facility). You can always wait to establish what your pilot has on them or what
mech your pilot is bringing until this step.


Mechs: Players can bring one mech with them on a mission. Information on mechs and their
creation is found in the section directly following this one.

Pilot gear: Players can choose clothing, armor, up to two weapons, and up to two other
pieces of gear to embark with, as long as all the gear’s total rarity is equal to or less than their
license level.


You can find pilot gear and mech parts in the compendium (pilot gear is at the beginning of each
manufacturer’s section). Basic General Massive Systems pilot gear is rarity 0 and available to all
players - basic GMS mech parts are available to all pilots from LL0.





Players might not always have entirely free access to gear, depending on where they start their
mission or story circumstances. For example, pilots that are stranded in the middle of an alien
wasteland after crash landing (with the mission: find civilization) might not have the best access
to gear. Pilots can always trade, barter, build, or acquire other gear across the course of a
mission. In such a situation, the GM can use the Power at a Cost tool (see below) to give the
players what they need.


                                                    Reserves


Before a mission starts, in this step, pilots must establish reserves that they’re bringing on this
mission. Reserves is a catch-all term for extra gear, ammunition, support, reinforcements,
information, access, or anything similar that has been prepared during downtime or established
during the course of the mission. You typically acquire reserves during downtime, but you could
also be granted reserves as part of a mission. There’s an example list of typical reserves later in
this section, but it’s more of a general term for anything extra you’re bringing on a mission.
