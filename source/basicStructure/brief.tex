\section{Brief}

The Brief is the very first step before you can start on a mission. This is when your pilots sit down around their consoles, or in their cockpits, or in a board room or a barracks somewhere and figure out what needs to be done. It doesn't have to be an actual brief, and could be an entirely out of character conversation between players and the GM. You need to establish a couple of clear things with the Brief: the mission \textbf{goal,} and the \textbf{stakes.} 

Your mission \textbf{Goal} is what you, the players, hope to accomplish with your mission. This might be
the same goal set out by the GM, or it might be defined entirely by the players. For example, the
GM might set it up so that a powerful NPC general needs the players to hold a checkpoint
against rebels. If the player's goals align, then their goal would also be to defend the checkpoint.
Perhaps the players want to secretly defect and help the rebels take control of the checkpoint
instead (in that case, that would be their goal).

Alternately, the players might come up with a goal without any GM initiative, such as deciding
they want to clear out and secure a pirate-infested asteroid to gain a new base of operations for
their mercenary company and decide to set that as their goal with the GM. Both are valid
missions.

Here's some example goals for inspiration:

\textbf{TABLE: Mission Goals}

 d20       Goal

 1         Escort a VIP from a compromised location to a new safe one

 2         Respond to an SOS from an unknown source, location noted in message.

 3         Retrieve a valued or strategic object, item, or information from a secure, hostile
           location
 4         Investigate a rumor or tip from a valued informant

 5         Escort a long-flight weapon or ordinance to its target

 6         Defend an area expecting an an attack (from pirates, hostile alien fauna, etc)

 7         Explore a long-abandoned derelict for artifacts

 8         Bring down a piece of massive infrastructure (bridge, skyhook, dam, etc)

 9         Go loud to provide cover for a covert mission of utmost importance

  10       Assassinate a VIP, discretely, or in broad daylight, to send a message

  11       Attack a hostile defensive position in order to destroy a key objective

  12       Board a hostile ship or station and take it over; or, destroy it

  13       Be first on the ground on a world hostile to human life; create a beachhead

  14       Smuggle something safely or securely through hostile territory

  15       Hunt down a team of notorious, feared, or respected mech pilots

  16       Provide cover for an evacuation

  17       Rescue and extract a someone from a secure or dangerous location, such as a prison
           or a war zone
  18       Secure a dangerous location

  19       Liberate a people held hostage from their cruel ruler, with Union's backing.

  20       Intervene in a desperate attempt to stop an incoming missile or attack.

Success of a mission depends on completing the mission goal, but mission completion does
not. \textbf{Characters that complete a mission (success or failure) always gain 1 license level.} The
mission goal might also change mid-mission as more information comes to light or the
circumstances or parameters change. This is perfectly normal and can create dynamic and
interesting stories.

You can often easily find the stakes by phrasing them as a question, such as the following:
\begin{itemize}
\item Will the players save the newborn colony on Astrada IV from total destruction by the hands of the White Tiger rebels?
\item Will the players discover who has stolen the Harrison Armory bioweapon before they get the opportunity to use it?
\item Will the players escort the diplomatic envoy safely through the raider riddled Mars Reef, or will the envoy's ship be torn to pieces like so many others?
\item Will the players discover the source of the mysterious artifacts buried beneath the planet's surface before the rival corpro acquisitions team locks it down forever?
\end{itemize}  

And so on and so forth. The actual stakes will often depend on the kind of story the GM is trying to tell. Stakes can be deeply personal or more broad. They can be immediate and brutal, or slow and gradual. They can often be as simple as survival. Generally the \textbf{GM} will establish stakes for a mission, but player actions, history, and background both greatly influence stakes and can play directly into stakes themselves. For example, a player that was a former slave might have a lot more of a personal stake in stopping slave traders.

It's really important to start a mission with both \textbf{goal} and \textbf{stakes} established. Not only does it give clearly defined motivation for the characters to be undertaking a mission, but it also sets up what will potentially happen if they fail, and allows the GM to take harder moves if that should come to pass (you knew what the stakes were!).