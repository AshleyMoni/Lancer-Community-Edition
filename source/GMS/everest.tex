\subsection{GMS Standard Pattern I (“Everest”)}

                                GMS Standard Pattern I (“Everest”)  

Most humans don’t think to ask about the history of the water they drink, the earth they walk, or the air they  

breathe. And yet without water, earth, and air, there would be no place for Humanity to make their home.   

The Everest -- officially, the General Massive Systems Standard Pattern I -- is a plainly designed chassis.  

Simple lines. Functional grace. Sturdy bulk. Its use-name, “Everest”, comes from one of the names of the  
tallest mountain on Cradle. Of the surveyed, named mountains in known space, Mount Everest -- or  
Sagarmatha, or Chomolungma, as it has been called in Old Human tongues -- is neither the most  

prominent peak, nor the even the tallest in Cradle’s local star system; and yet, pilots the galaxy over learned  
to call their GMS-SPi’s by that ancient name. Why?    

The sentimental answer is that the Everest is called the Everest because it is a reminder of what was once  
the limit of human endurance, of what once was the peak of human achievement. To summit Everest was  
to stand atop the world, the culmination of months -- maybe years -- of training, investment, and effort,  

defying death and injury upon your quest to summit.   

The real answer is probably much less intentional. Somewhere along the line, GMS’s plain naming  

convention coupled with Union’s anthropocentric emphasis lead a cadet, having graduated to a full pilot, to  
paint EVEREST across their GMS-SPi’s flank. Maybe it was meant to be their callsign, or maybe it was  
meant to represent their success following grueling training, but either way the name stuck in the insular  

pilot culture. Other pilots adopted the name, and over five centuries it has grown to become the official- 
unofficial designation for all GMS-SPi chassis.   

Veteran pilots may never return to crew an Everest after they’ve moved on to other chassis, but they’ll  
always remember when they reached that first summit, when they proved they were worthy of planting their  
own flag on at the peak of the world.   

The Everest may not be the most specialized chassis in the galaxy, but it is the backbone of the galaxy, and  

humanity steps to the stars from its shoulders. 
 

                                                      EVEREST 

  HP: 10          Evasion: 8                                Speed: 4             Heat Cap: 6          Sensors: 10 

 Armor: 0         E-Defense: 8                              Size: 1              Repair Cap: 4        Tech Attack:  
                                                                                                      +0 

                                                        TRAITS: 

  Initiative: The very first turn the Everest takes in any combat, it can take an extra Quick Action as a free  
 action 

                                                 SYSTEM POINTS: 6 

                                                                                                                       


                                                 MOUNTS: 

Flex Mount                        Main Mount                            Heavy Mount 

                                              CORE system 

                                      GMS Hyperspec Fuel Injector  
Active (requires 1 core power): Power up
 
Protocol
 
This turn only, you can make an additional Full Action as a free action or 2 Quick actions as free  
actions. 