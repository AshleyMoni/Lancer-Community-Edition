\section{Bonuses}
   BONUSES  

Every skill check or attack roll in LANCER has two types of bonuses applied to it: Accuracy or  
Difficulty, or a flat number that comes from a skill rating or Grit. Often you can apply both bonuses!
 
\subsection{Accuracy and Difficulty}
                                ACCURACY AND DIFFICULTY  

Accuracy and Difficulty represent the momentary advantages and disadvantages gained and  
lost during rapid, chaotic moments of action:
 

Opposed offensive and defensive electronic warfare systems bombard each other with viruses  
and counter viruses, spoofing targeting systems and layering desperate firewalls.   

Pilots, matched in skill, duel each other amidst the shifting debris of a shattered frigate, avoiding  
incoming fire and slagged, floating bulkheads as they attempt to land their shots.   

Their mech about to overload, a pilot struggles against an unshackled AI to regain control of their  
machine, pitting their skill against the best efforts of their newly freed system.   

                                                                                                              


These situations (and more!) cause pilots to accrue Accuracy and Difficulty on rolls. 
 

1 Accuracy adds 1d6 to the roll it is applied to.
 
1 Difficulty subtracts 1d6 from the roll it is applied to
 
Accuracy and Difficulty cancel each other out, on a 1 to 1 basis.
 
Accuracy and Difficulty do not stack: instead, the greatest result is chosen and applied to the  
final roll.  
 
       •  An attack roll made with +2 Accuracy would not add the results of those two rolls.  
          Instead, you would pick the greatest result between the two and apply it to your final roll.  
       •  An attack roll made with +1 Accuracy and +1 Difficulty would have no bonus or  
         subtraction applied to it: the single Accuracy die and single Difficulty die would cancel  
          each other out before there is a need to roll.  
       •  An attack roll made with +2 Accuracy and +1 Difficulty would be made as a roll with +1  
         Accuracy  

\subsection{Grit}
                                                      GRIT  

Pilots are skilled and unique individuals, multi-talented and resilient. Even so, a brand new pilot  
cannot measure up to a tempered, battle-hardened veteran when push comes to shove. This is  
represented in game by Grit.  

Grit is a flat number equal to half your character’s license level, rounded up. License levels go  
from 0-12 and track the general resources, skill, and experience of your character. Your grit  
improves as you level up, and can’t typically be increased any other way. It represents your direct  
combat experience, expertise, and will to survive.  

\subsection{Pilot Skills}
                                              PILOT SKILLS  

The part of your pilot related to their personal abilities and experience are your pilot skills. These  
skills are mostly used during narrative play. At level 0, your pilot has 4 skills based on narrative  
triggers, representing different aspects of their character’s training or background. Skills apply a  
flat bonus from +2 to +6 to pilot skill checks, and you also might get an accuracy or difficulty  
bonus depending on your character.
 
\subsection{Mech Skills}
                                              MECH SKILLS  

The part of your pilot directly related to building, piloting, and fighting with mechs are your mech  
skills. There are four, and they go from +0 to +6:  

Your HULL skill describes your ability to build and pilot durable, heavy mechs that can take  
punches and keep going  
Your AGILITY skill describes your ability to build and pilot fast, evasive mechs  

                                                                                                                 


Your SYSTEMS skill describes your ability to build and pilot technical mechs with powerful  
electronic warfare  
Your ENGINEERING skill describes your ability to build and pilot mechs with powerful reactors,  
supplies and support systems  

Mech skills help you directly in turn based combat and when piloting your mech. They also give  
you additional bonuses when building a mech.  

                                              WHICH TO USE? 

When you make a skill check, and use your pilot’s natural skill, experience, or abilities, use your 
pilot skills (1d20 + pilot skill) 
When you make a skill check and rely on your mech’s systems, survivability, or raw power, use 
your mech skills (1d20 + mech skill) 
When you make an attack roll, add Grit. You might add other skills, like Hull or Systems, instead, 
but only when specified. 

You can get extra bonuses on all checks or attack rolls from talents, gear, or pilot backgrounds. 
Most, if not all bonuses take the form of bonus Accuracy or Difficulty. 

\subsection{Time in Lancer: Mission, Downtime, and Scene}         
               TIME IN LANCER: MISSION, DOWNTIME, and SCENE 

Before we dig into pilot and mech play in detail, it might be important to note that a game of 
LANCER is usually split up into missions. A mission might encompass one or several play 
sessions. A mission is a goal or objective that can be completed in a discrete amount of time, 
such as destroying a target, evacuating civilians, uncovering a conspiracy, or holding the line 
against enemy attack. Each time you embark on a mission, you pick the pilot gear you’re 
embarking with and the mech you’re bringing with you. 

If you’re not actively on a mission, you’re in downtime. This is the narrative time between 
missions where the moment-to-moment action doesn’t matter as much, and roleplaying is much 
more important. During downtime you can progress plots, projects, or personal stories. 

Within missions and downtime, play is split up into scenes. A scene is a continuous section of 
play or activity. The word ‘scene’ is used here because it’s helpful to think about it in cinematic 
terms. As long as the ‘camera’ or focus is on their players and their action, a scene is happening. 
When the ‘camera’ cuts away from the current scene, it’s over (this is a lot easier to judge 
naturally than it sounds). A good example of a scene is a single battle or combat. A scene could 
even span many locations or be a montage of action where the moment-to-moment action 
doesn’t matter too much. When the current activity or course of action naturally ends, that’s 
when the scene should end too. 

In keeping somewhat with these terms, the game commonly uses the term ‘actor’ to refer to any 
individual character, player or non-player. 

                                                                                                                 


Completing a mission is the primary way to level up in LANCER. Once a pilot completes a  
mission, they gain a license level.
 

For more information on missions and downtime, see the section further on in this book.
 
 
\subsection{Narrative Play and Mech Combat}
                          NARRATIVE PLAY AND MECH COMBAT  

LANCER makes a distinction between narrative, freeform play, and mech combat, where tracking  
individual turns and actions is important. Most of the game and story will typically be in narrative  
play. During narrative play, players and the GM can take actions naturally and spontaneously as  
they come up. Time and scenes might be faster and individual rolls might count for more.
 
	        During mech combat, players must take turns to act, and are restricted in the number  
and type of actions they can take, making each action much more impactful and tactical.
 
	        Certain types of actions or effects will work differently in narrative play vs mech combat.  
For example, attacking someone as a pilot is typically a skill check in narrative play that could  
use skill triggers like Applying Fists to Faces, Assault, or Taking Someone Out, and narratively  
stretching much further. In the world of mech combat, however, attacking something is typically  
an attack roll, adding grit, and each individual roll accomplishes much less.
 

                                                                           