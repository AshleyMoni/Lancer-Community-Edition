\section{Narrative Play}
                                  NARRATIVE PLAY


Once you’ve made your engagement roll, prepared your gear and supplies, set your goal and
your stakes, you’re officially on a mission. A mission could last one session or several sessions.
You might abandon your original goal in favor of a new one, or encounter a twist in the story that
throws your mission into disarray. Playing a mission out is mostly a matter of the GM - there’s no
strong guidelines here as to how to structure it! However, here’s some tools, advice, and aid for
running a mission.

                                      Narrative play vs. mech combat

The following section deals with narrative play, typically when you’re using your pilot skills. This
is the bit of the mission outside of mech combat, which is a lot more structured. Generally in
narrative play each roll accomplishes much more, scenes can cover large stretches of time, and
the outcome of individual rolls is more important.

Mech combat is turn based, tactical combat. Cutting to mech combat is as simple as declaring
it’s started, drawing a map out, and picking who goes first. When you want each roll to accomplish
more and want to play out turn based, tactical combat, you can swap to mech combat.

These are two different modes of play and the rules work slightly differently for each, especially
combat. If you’re in narrative play and get into combat, you do combat with skill checks, and don’t
make attack rolls. NPCs don’t get their own turns (nobody gets a ‘turn’ in narrative play), but their
actions are narrated by the outcome of player rolls. If you’re doing mech combat, you use turn
based, tactical play, make attack rolls and track hit points, and NPCs will take their own turns.

The rules for mech combat (and the difference from narrative play) are found right after the mech
section.

\subsection{Making Skill Checks}
                                          Making Skill Checks

Skill checks are only required when there is a tense narrative situation or when the check
would move the story forward. You don’t need to make a skill check to open a door, to cook a
meal, or to talk to a superior, unless that situation is tense or would add to the story. You should
generally always succeed on mundane tasks, especially if they relate to your background. A
barroom brawl, a tense escape, decoding an encrypted message, hacking a computer, talking
down a pirate, picking someone’s pocket, distracting a guard, hunting alien wildlife, or flattering
the planetary governor are all situations that have some degree of tension and consequence, and
might require a skill check.


Skill checks can cover as much or as little as the narrative requires. For example, you could have
one skill check cover an entire day’s worth of infiltration into a covert facility if you so desire. Or,




you could instead cover the moment to moment action - sneaking into vents, hacking doors,
disabling guards, etc.

When making a skill check:

    -    First name your goal.

    -    The target number is always 10, and the check is a simple 1d20 roll. You then add any
         accuracy or difficulty from your skills, and then any accuracy or difficulty the GM imposes
         to get the total accuracy or difficulty on the roll.

    -    You should only roll once to accomplish your goal (the GM can’t require extra rolls of
         you), though the GM could tweak the difficulty if you’re asking something very hard or
         complicated, or declare that given your goal or circumstances the roll would be
         impossible.

    -    On a 9 or below, you don’t accomplish your goal. On a 10-19, you accomplish your goal.
         On a 20+, you excel on your goal.

    -    On a 19 or lower, the GM can choose to add complications.


Complications or consequences from failing or succeeding pilot skill checks always follow the
fiction and stakes established.


A failure doesn’t necessarily mean outright failure, but that you don’t directly accomplish your
goal. Additionally, if you fail a check, you cannot attempt the same activity again until you change
the narrative circumstances or approach (it’s a new day, you try something different). For
example, you try to climb back up that cliff bare-handed, but fail. You could only make another
skill check to climb the cliff again if you try it with a grappling hook, or get some other help.


                                                 Complications

On any result less than a 20+, the GM can throw additional complications or costs into the mix,
as established. The GM doesn’t have to throw a complication in every time, and can just let the
action play out - however rolling less than 20+ gives them the ability to do so.


Complications are typically chosen from the categories of Harm, Time, Resources, Collateral,
Position, Effect. This can never nullify your success if you roll a 10+ or cause you to not
accomplish your goal, but can add additional nuance to the outcome of your skill check.


It’s important to note that the GM can only inflict complications if it makes sense to do so - in
other words it must be established clearly before the roll. If you’re trying to take someone out
with a sniper rifle at 500 meters and they have no way to see you or shoot back, you probably
can’t take harm as a complication. If you’re trying to knock out a soldier from hiding that soldier
probably doesn’t have a good way to fight back right away, even if you miss. If that soldier is
alerted and looking for you, however, she might get a shot off.


    -    Harm is damage, injury, or bodily harm, as established. If someone is pointing a gun at
         you, you attempt to take control of that gun and you fail, you will probably take harm.





     -   Time means the activity takes more time than normal

     -   Resources means something must be used up, lost, or temporarily expended. This could
         be something concrete like running out of ammunition, losing a map, or your gun
         jamming, or could be something like political influence.

     -   Collateral means someone or something else takes harm or injury instead of you or your
         intended target, like an innocent bystander, the whole building, your organization or an
         ally

     -   Position means you are put in a worse position through your actions, like right in the line
         of fire, clinging to the edge of a cliff, in the bad graces of the Baron, or under a spotlight

     -   Effect means your action has less effect than you intend. If you were trying to take
         someone out cleanly, you make a lot more noise than you intended. If you try to fix a
         broken door, it will only open for a few people at a time.


Narratively, complications are probably much worse if you fail (since you failed to accomplish
your goal and got a complication).


Example complications:

         	- Harm: A player rolls to knock someone out who just drew a knife on them by applying
         fists to faces. They don’t manage to knock their target out however, and they get a knife
         in the gut for 2 damage.

         - Time: A player rolls to charm the baron into granting them an audience, succeeding.
         The baron lets them stew for a few hours, but gives them the audience.

         	- Resources: A player rolls to patch up an NPC’s wounds, and fails. The NPC not only
         bleeds out, but the player runs out of medical supplies trying to treat them.
         - Collateral: A player rolls to blow up a door and fails. The whole building starts to
         collapse

         	- Position: A player rolls to take out an assassination target in a hidden base with a
         sniper rifle and succeeds. They kill their target, but they have to fire multiple times,
         exposing their position to the entire base.

         	- Effect: A player rolls to wreck a security system. It only shuts it down for 5 minutes,
         however, giving the players limited time to act.


                                                       Excel

If you excel on a skill check (get a 20+), tell the GM how you surpass your initial goal. The GM
can moderate this if it’s not within reason. In addition, the GM can’t throw complications at you -
you did that well.


Examples:

         	- Bruja excels when making a skill check to hack a door control. She suddenly finds
         herself with access to the whole network

         	- Penny excels when making a skill check to threaten a royal guard to stand down. The
         guard not only surrenders, but offers to help her get an audience with the king





         	- Xi excels when making a skill check to get to the extraction point quickly. He decides
         is able to find a shortcut to get the whole party there instead of just himself.

         - Raja excels when making a skill check to get a hold of transport off world for his party.
         He decides that he manages to talk the shuttle pilot to walk off the job and hand the
         whole ship to his party

\subsection{Player Initiative and NPC Action}
                                 Player initiative and NPC action


Players always have initiative when making skill checks or taking action in narrative play. That’s
a fancy way to say that the GM can never ask for a roll unless prompted by the players. Players
must name their goal or aim of their action, then GMs can ask for a roll and set difficulty. When
the roll is made, initiative turns back to the players (probably with a ‘what do you do?’ from the
GM). What this does in practice is let players decide the course of action and make sure that
each roll has clearly established stakes and parameters - it’ll help the game feel more fair and
prevent unnecessary rolling.


If the players don’t take action, stall, or pass off responsibility for action, then they are
effectively turning initiative over to the GM!


In addition, NPCs (non-player characters) don’t take actions or make rolls by themselves. Their
actions are based on player rolls. For example, if a player lies to an NPC, the NPC doesn’t roll to
see if the player is lying. If the player is successful, the NPC doesn’t see through their deception
- if they fail, the NPC sees they are clearly lying. If the GM feels like the particular NPC is astute
or insightful and can easily see through lies, they might add 1-2 difficulty to the roll.


There’s a little more on this in the GM section if you need examples. In practice you probably
won’t even think about this that much.

\subsection{Skills in Detail}
                                               Skills in detail


Each skill has basic triggers that allow you to easily decide which skill to use and which
bonuses from backgrounds or training apply to a roll. You don’t have to track or know all of
them (just the ones which you have +accuracy or +difficulty in from backgrounds or training). If
you’re stuck as to which skill you should be using, you can quickly refer to the triggers to get a
good idea.


Here’s a little more detail on each skill and when they might trigger. There’s intentionally a little
overlap between some of the triggers, and each is designed to be somewhat flexible.
Remember, you don’t need to track all the skills, just the ones you are good or bad at!


Applying fists to faces




Punch someone in the face, or alternately fight in open, brutal unarmed combat, whether it’s a fist
fight, martial arts duel, or a huge brawl. This is probably not the smoothest or cleanest fist fight
and probably causes a lot of noise.

Assault
Take part in or direct an open or pitched battle, like a corridor gunfight, a huge shootout, fighting
your way across a battlefield, or undertaking a boarding action. When you assault, you’re always
assaulting something (a position, a rival pilot, an enemy force, a group of guards), and it’s always
loud, open, direct action.

Blow something up
Use explosives (improvised or otherwise), weapons, or maybe just good old fashion brawn to
totally wreck something or turn it into an enormous fireball (maybe a wall, sensor array, outpost,
reactor core - the good stuff). Whenever you’re totally destroying an object, building, etc, you can
use this. Probably not to be used against people unless they’re incidentally in the way.

Threaten
Use force or threats of force to get someone to do what you want them to do. Name what you
want someone to do and what you’re going to do to them if they don’t listen to you. This could
also be blackmail, leverage, or something similarly nasty. Threatening someone can be very high
risk but very effective if successful. If you threaten someone unsuccessfully, your threats have no
further effect on them unless you change something about the situation (like all other skill checks).

Take control
Use force, violence, presence of will, or direct action to take control of something. This is often
something concrete, like an object someone is holding. You could take control of someone’s gun
or a keycard they have on their person. You can additionally can take control of a situation to force
those present to listen, calm down, stop moving, or stop what they’re doing, though you can’t
necessarily force them to do anything further without threatening them. Taking control is never
subtle and always direct and dangerous.

Survive
Persevere through harsh, hostile, or unforgiving environments, such as the vacuum of space,
frozen tundra, a pirate enclave, a crime-ridden colony, untamed wilderness, or scorching desert.
You most often use survive when you want to take a journey through wilderness environments,
navigate, or avoid natural hazards such as carnivorous wildlife, rockfalls, thin ice, or lava fields.
Alternately you could use it to avoid man-made hazards, such as navigating a city safely, or
avoiding dangerous areas of a space station. You could also use it when testing personal
endurance, such as shaking off poison or alcohol.

Stay cool and collected
Do something that requires concentration, speed, or intense precision under pressure, like picking
a lock, finding the right frequency for your omnihook, carefully disarming an explosive, or
unjamming a gun. If you’ve got to do something complicated in a high stress situation without
messing up (and possibly not even breaking a sweat) this is the action to use.




Take someone out
Kill or disable someone quickly, quietly, effectively or from a distance, probably before they even
notice. This is probably a single person but could be two people relatively close together (any
more is sort of stretching it). If you’re looking down a sniper scope at a target, preparing to nerve
pinch a guard to knock them out instantly, quick-drawing during a gun duel, or dropping from a
ceiling to slit a throat, this is the action to use.

Flash
Do something flashy, cool, or impressive with your weapon other than killing, like shoot a very
small or rapidly moving target, shooting someone’s hat off or their weapon out of their hand,
knocking someone out by throwing a gun at them, performing an acrobatic flourish with a sword,
throwing a spear to pin a fleeing target to the ground, or something similar.

Get somewhere quickly
Get somewhere without complications and with speed, but not necessarily quietly. Climb, swim, or
perform acrobatic maneuvers. Fall safely from a great height. Move gracefully in zero-g. Chase or
flee from, outrun or out pace a target. Get somewhere faster than anyone else. You can also use
this when you want to drive or pilot a vehicle.

Act unseen or unheard
Get somewhere or do something without being detected, but not necessarily with speed. Hide,
sneak, or move quietly. Infiltrate a facility while avoiding security, patrols, or cameras. Perform a
quick action or maneuver without being seen or heard, such as picking a pocket, unholstering
your gun, or cheating at cards. Wear a disguise.

Fix, hack, or wreck
Repair a device or faulty system. Alternately, hack it wide open, or totally wreck, disable or
sabotage it. You can use this for hacking or safeguarding electronic systems, such as electronic
door locks, computer systems, omninet webs, or NHP coffins.

Patch
Apply your medical knowledge to administer medication, bandage, staunch bleeding, suture,
cauterize, neutralize poison, or resuscitate. Alternately, you could use it to diagnose or study
disease, pathogens, or illness.

Invent or create
You need tools and supplies to invent or create something successfully. Use this with many
downtime actions to work on projects. You can also use it in the spur of the moment to invent new
devices, tools, or approaches to something (improvised explosives, gear, disguises, or some
similar).

Read a situation




Look for subtext, motive, or threat in a situation or person, often social situations. Use your
intuition to learn someone’s motivation, who is really in charge, or who is about to do something
rash or stupid. Get a gut feeling about a situation or person. Sense if someone is lying to you.

Spot
Spot hidden or difficult to make out details, objects, or people. Spot ambushes, hidden
compartments, or disguised individuals. Spy on a target from a distance, or make out the details,
shape, and number of objects, vehicles, mechs, or people clearly at a distance. Track people or
vehicles.

Investigate
Research a subject, or look at something in great detail. If you can’t find information directly, you
learn how you can get access to that information. Learn about a subject of historical relevance, or
become well-read on a subject. Investigate a mystery or solve a puzzle. Locate a person or object
through research or investigation.

Charm
To charm, you need a receptive audience, or some kind of promise of leverage (money, power,
personal benefit, etc). You can use it when trying to smooth talk your way past guards, get
someone on your side, sway a potential benefactor, talk someone down, perform diplomacy
between two parties, or blatantly lie to someone. You can also use it when trying to impersonate
someone. Charm won’t work on people that aren’t receptive (such as soldiers you are in a
gunfight with) or that you don’t have leverage over (promises of safety, money, power,
recompense, help, etc). These promises don’t necessarily have to be true but they have to have
some weight with your target.

Pull Rank
Pull rank on a subordinate, getting information, resources, or aid from them, even unwillingly. You
can use this on anyone your social status (noble, celebrity, etc) or military rank would have weight
with. Failing this might be risky and could be seen as abusive. You typically can’t pull rank on
hostile targets. You could also use this to pull rank and pretend you are a higher rank than you
are, but it’s even riskier.

Word on the streets
Get gossip, news, or hearsay from the streets. What you get depends on what ‘streets’ you are
getting word from (high society, low society, hearsay, military chatter, etc). This probably takes a
lot less time than investigating something in detail, but the information might be more qualitative
or colored by opinion (sometimes that might be useful). You can always learn where the
information came from or who to go to next.

Get a hold of something
Acquire useful allies, assets, or connections through wealth or social influence. This could be
permanent (buying it or receiving it) or temporary (renting or borrowing help or supplies, etc), and
might be harder or easier depending on how much you want to use it. This can’t be used for
something that’s normally gated by license level (like mech parts) but could be used for aid,




supplies, information, food materials, soldiers, or anything else that has more narrative impact.
Typically this is acquired by buying it from a market or requisitioning it from a parent organization.

Lead or inspire
Give an inspiring speech, or motivate a group of people into action. Administer or run an
organization efficiently or effectively, such as a company, a ship’s crew, a group of colonists or a
mining venture. Effectively command a platoon of soldiers in battle, or (perhaps) an entire army.

\subsection{Skill Challenges}
                                         SKILL CHALLENGES

A skill challenge is a simple way to test an entire group for a particular activity. Everyone makes a
relevant check, and the success of the challenge depends on the overall result of the skill checks
from the entire group, not just one player. If more players succeed than fail, the challenge is a
success. If equally as many succeed as fail, the challenge has a 50% chance of success (roll a
die or flip a coin), representing the razor’s edge of the situation. If more players fail than succeed,
the challenge is a failure.

Here’s some example challenges:
    -    Sneaking into a guarded facility: All players roll a skill check (example skills that could
         trigger: move unseen, spot the cameras, charm the guards into thinking they are a
         superior). Success means they all get in unnoticed, failure means the guards are alerted.
    -    Gaining the favor of the Baron: All players roll a skill check (example skills that could
         trigger: charm, but could also be to threaten the baron, or perhaps read the situation)
         Success means the players gain a private audience with the Baron, failure means the
         players’ meddling is noticed by rival nobility and they are thrown out.
    -    Traversing the Wastes: All players roll a skill check (example skills that could trigger:
         survive, spot water, or get across the waste quickly). Success means they cross the
         wastes unharmed. Failure means they cross the wastes, but it is a harrowing journey and
         they arrive there with no repairs or supplies left and lacking food and water.

Challenges are good when you want to extend the narrative impact of rolls.

You can also have extended challenges that have 3 rounds of rolling and calculate the outcome
based on rounds ‘won’ by the players. For example, the players may have to gain the favor of the
baron, then plant information in the baron’s castle, and sabotage the gate. They are only truly
successful if the majority (2/3) of these tasks are accomplished.

\subsection{Combat in Narrative Play}
                                      Combat in Narrative Play


When running combat narratively, use the normal rules for skill checks. That means when the
individual actions in combat doesn’t matter that much or you want a combat scene to play out
more like a movie than a tactical game. If there’s not a mech involved (and you’re just playing on
the pilot scale), it’s almost always preferable to use narrative combat.





You don’t need to track turns or make attack rolls, and you might only need to make a few rolls
for the whole combat (one roll for each action or goal, as normal!). Turn-based combat in
LANCER is usually reserved for mech combat.


You can also use a skill challenge to run narrative combat if you want it to be a bit more
structured.


Here’s a couple examples:

	        - Bruja and Penny are negotiating with the Black Star Bandits to try and get them to
release a hostage. The negotiations go sour (Penny fails her skill check to charm the bandit
captain), and the bandits draw on them. Bruja decides to take out the bandits quickly with her
Sidekick. She rolls a skill check, getting a 15. She kills the bandits and the GM decides to add
position as a complication, so the rest of the bandit camp is alerted and they will need to get out
quickly.

	        - Pan is in a pitched battle, on foot. He sees a gun emplacement raining hell down upon
his allies and decides to take control of it. He rolls and gets a 7, failing. The soldiers defending
the emplacement turn the gun on him, preventing him from getting any closer. The GM adds
collateral as a complication. Pan looks behind him and sees some members of his squad get
gunned down in the ill-advised assault.

	        - Raja is commanding a platoon of troops to board an enemy ship and take control of the
command center. He rolls to lead the charge, getting a 22, (he excels). There’s no complications,
and when his group successfully fights their way to the command center, they immediately
surrender and hand control over.


\subsection{Hit Points, Damage, and Injury}

                                 Hit Points, damage, and injury

Pilots normally only care about Hit points, also called HP (how much damage your pilot can take
before they are out of the action!) during mech combat, but they might also take damage as a
result of complications during skill checks.

At level 0, pilots have 6 Hit Points, representing not only bodily health, but also the ability to
duck, dodge, avoid damage, or just sheer luck. At higher levels, they add their grit (1/2 level) to
calculate their base HP (leading to a total of 6 + grit). A pilot that takes damage doesn’t neces-
sarily take bodily harm, but might be using up their stamina, luck, or ability to avoid that incom-
ing damage.

If a consequence deals damage, it’s enough to hurt or kill. Taking things like minor grazes,
bruises, etc doesn’t deal damage but could cause other complications.

Here’s what damage looks like for pilots in narrative play:
Minor damage is 1-2 damage. This could be something like getting shot at by small arms fire,
stabbed, unarmed combat, hit by a flying rock, etc
Major damage is 3-4. This is getting shot at by assault or heavy weapons, a long fall, breathing
in toxic gas, etc
Lethal damage is 6+. This is something like having a mech fall on you, getting hit by a mech
scale weapon, having a grenade blow up right under you or something similar.

Pilots might have 1 or 2 armor. Subtract armor from all damage taken as a pilot, unless that
damage has the armor piercing (ap) tag. Some weapons have the ap tag, but damage outside of
that might also (falling a long distance, being immersed in lava, etc).

                                                 Down and out




If you’re reduced to 0 HP or lower as a pilot roll a 1d6. On a 6, you miraculously shrug off the hit
(or its a close call), returning to 1 HP. If you roll a 1, your luck has run out, and you’re immediately
dead. If you roll a 2-5, you are down and out, at 0 HP. You’re knocked out, pinned, bleeding out,
or otherwise unable to act. Your evasion (how hard it is to hit you in mech combat) becomes 5
and if you take any more damage, you’ll die (if someone comes over and shoots you in the head,
for example).


You can die instead of being down and out, if you choose so. Typically you’d just bleed out and
wake up in prison, a field camp, a hospital, or buried under a pile of dead bodies somewhere.

\subsection{Rests and Full Repair}

                                         Rests and Full Repair

If a character takes an hour and rests with no strenuous activity, they can regain 1/2 their HP, and
recover from Down and Out, coming back to consciousness. When you take at least 10 hours to
rest and full repair, you can recover all your hp.


Rest and repair also help your mech (you can see more in the mech section on damage).


If you’re dead, it might not be the end for you. See the rules on cloning in the Death section in
mech combat.
