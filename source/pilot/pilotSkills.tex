\section{Pilot Skills}

A pilot’s skills refer to a pilot’s general experience, proficiency, personal ability, and aptitude with
particular approaches or courses of action. They don’t have to necessarily describe physical
qualities of your pilot.

They go from \textbf{+0 to +6.} This number is a flat bonus applied to a pilot skill check (a check used to
describe your pilot’s personal abilities or aptitudes) to get the final result when the skill\textbf{triggers.} 

A skill \textbf{trigger} (or just the skill itself) tells you \textbf{when} you can get that bonus. This is pretty simple in
practice. If your pilot gets +2 when \textbf{blowing something up,} if your goal ever involves setting off
explosives, throwing grenades, setting charges, etc, you get that skill bonus.

Your pilot skills are used for narrative actions and generally don’t get used during\textbf{mech combat.} 

\begin{center}
\textbf{Tracking Bonuses}
\end{center}


You don’t actually have to write down skills you don’t have any sort of bonus to. It’s just a flat
1d20 vs 10 roll with situational accuracy or difficulty applied otherwise. For example, at level 0, a
pilot might write down ‘+2 to threaten, get somewhere fast, assault, and spot’. There’s no reason
to write down or track the rest!


\begin{center}
\textbf{Skills}
\end{center}

Here’s the list of skills, which are thematically grouped here to help you choose. Each bolded
word or group of words describes a skill and when that skill would trigger, giving you a bonus on a
skill check. You get \textbf{4 of these at level 0}, each at \textbf{+2.} 

\textbf{Your pilot’s ability to use, resist, and apply direct force, physical or otherwise:}

Get a bonus when: - You’re \textbf{applying fists to faces}
\begin{itemize}
\item - you’re \textbf{assaulting} a position, person, or group of people (in battle or otherwise), hard, fast, and up close
\item - you want to \textbf{blow something up}
\item - you’re forcing someone’s hand by \textbf{threatening} them
\item - You want to \textbf{take control} of something (an object, a captive, the situation)
\end{itemize}  


\textbf{Your pilot’s ability to perform skillfully and accurately under pressure:}

Get a bonus when:
\begin{itemize}
\item You want to stay \textbf{cool and collected} while performing an action that takes skill or precision
\item You want to \textbf{take someone out}, cleanly
\item You want to perform some \textbf{flashy} action, like shooting an apple off someone’s head
\item You want to \textbf{get somewhere fast}, on foot or in a vehicle
\item You want to \textbf{act unseen and unheard}
\end{itemize}  

\textbf{Your pilot’s ability to notice details, think creatively, and prepare:}

Get a bonus when:
\begin{itemize}
\item You want to \textbf{fix, hack, or wreck} a system or device or \textbf{patch} a bleeding wound
\item You want to \textbf{invent or create} something with tools and supplies
\item You want to \textbf{read a situation} for subtext, motive, or threat
\item You want to \textbf{spot} hidden details, track a target, make out distant objects, or observe with an eagle eye
\item You want to \textbf{investigate}, research a subject, or look at something in great detail
\end{itemize}  

\textbf{Your pilot’s ability to talk, lead, change minds, make connections, and requisition resources}

Get a bonus when:
\begin{itemize}
\item You want to \textbf{charm} an audience with your words or actions
\item You want to \textbf{pull rank} on someone
\item You want to get\textbf{word on the streets}
\item You want to \textbf{get a hold of} useful allies, assets, or connections
\item You want to \textbf{lead or inspire} allies, troops, or a whole organization
\end{itemize}  

There’s a little more detail on each of these uses in the \textbf{Narrative Play} section below