\section{MOVEMENT}

Characters move a number of spaces equal to their speed value. They can freely move through (but not end their turn in) the space of friendly targets, but treat hostile targets as obstructions. 

\subsection{Engagement, Size, and Movement}

All characters have to worry about \textbf{engagement}. If you move adjacent to a hostile target, you become engaged. Being engaged gives penalties to ranged attacks (+1 difficulty), and if you become engaged with a target the same size or larger than you, you must stop moving and lose any additional movement you have left. Targets smaller than you cannot stop you from moving, so larger mechs can move around more easily.

An actor that is larger than another actor does not count the smaller actor as an obstruction and ignores engagement from that smaller actor. For example, a size 1 mech can freely pass through the space of a size ½ human, and a size 2 mech can freely pass through the space of that size 1 mech. Mechs can always move through pilots or human NPCs on foot.

Typical sizes:\\
½: A human, a hard suit, small mechs, an EVA suit\\
1: Typical light mechs, assault mechs, line mechs\\
2: Battle tanks, many vehicles, heavy mechs\\
3: Siege mechs, heavy vehicles\\
4-5: Titanic mechs, mech-oriented flyers

\subsection{Disengage}
A character can spend a full action to disengage, allowing them to ignore engagement when moving and allowing their movement to not trigger reactions (such as overwatch).

\subsection{Involuntary movement}
Some actions or attacks push, pull, or shove a character in a certain direction. Involuntary movement such as pushing, pulling, or knocking a character forces them to move in a direct line in a direction specified by the triggering action or attack. Mechs that are pushed, pulled, or knocked around do not provoke reactions and ignore engagement with their movement, though they must still obey obstructions.

\subsection{TRAVERSAL AND THE ENVIRONMENT}

Mech combat takes place on many types of worlds with many hostile and hazardous environments. Here’s what you need to worry about:


\textbf{Difficult Terrain} reduces a mech’s speed. 1 space of movement through Difficult Terrain costs 2 spaces worth of movement speed. Difficult terrain could be rough, marshy or swampy ground, icy landscape, or treacherous, rocky scree. What constitutes difficult terrain for pilots and mechs might be different.

\textbf{Dangerous Terrain} prompts an engineering check to navigate the first time on a turn an actor enters it on their turn, or if they start their turn there. Should a player fail that check, they take 5 kinetic, energy, explosive, or burn damage on failure, depending on the hazard. Intense radiation, boiling gases, lava, or falling rocks are good examples of dangerous terrain. An actor only needs to make one check a turn for dangerous terrain.

\textbf{Obstructions} block passage. Obstructions are typically environmental, but can include NPCs and other players. \textbf{Obstacles smaller than the moving object do not block movement}, and can be passed through freely. \textbf{Friendly NPCs or allied players never cause obstruction}, but you can’t end your movement in their space.


\textbf{Lifting and Dragging}\\
A mech can \textbf{drag} another character or item up to \textbf{2x} its size (but is Slowed while doing so), and \textbf{lift} a character or item overhead that’s its size or smaller, but remain immobile. While dragging or lifting another object or character, a mech cannot take reactions. Pilots follow the same rules but cannot drag or lift objects larger than size 1/2.

\textbf{Climbing} like difficult terrain, costs 2 spaces of movement for every space moved. Climbing especially difficult surfaces might require a successful hull or agility skill check not to fall.


A mech can \textbf{jump} half its speed horizontally in a straight line, ignoring ground based obstacles that it could jump over (such as pits, gaps, etc), and cannot jump higher than its size (so a size 1 mech can jump up to 1 space high during that movement).

\textbf{Falling} causes damage if a character falls 3 or more spaces and cannot recover before it hits the ground. A character takes 3 AP kinetic damage for each 3 spaces it falls, up to a maximum of 12 AP kinetic damage. Typically a character falls about 10 spaces a round, but a mech cannot fall in a zero-g environment (or even a low-g environment) and speeds may differ depending on where you are.

\subsection{COVER}

Cover is obscurement from observation or gunfire. In narrative terms, cover refers to smoke screens, hard cover (a building, a wall, a bulkhead, etc) between the attacker and the target, soft cover (trees, earthen mounds, etc) between the attacker and the target, obscured vision, electronic countermeasures, or any other obstruction physical, mental, electronic, etc, between an attacker and their target.

Smoke, foliage, trees, blinding light, dust clouds, low hills, low walls, etc are all examples of \textbf{light cover}. Light cover is typically \textbf{not solid} enough to reliably block fire, but causes enough visual interference or reduces profile enough to make aiming difficulty.

Tall walls of buildings, ruined buildings, bulkheads, reinforced emplacements, destroyed mechs or vehicles, spacecraft, etc are all examples of \textbf{heavy cover}. Heavy cover is \textbf{solid} enough to block shots and hide behind.

\textbf{Light Cover} adds +1 Difficulty to an attacker’s roll to hit for ranged weapon attacks.\\
\textbf{Heavy Cover} adds +2 Difficulty to an attacker’s roll to hit for ranged weapon attacks.

If a character has a better form of cover, it is not superseded by a weaker form of cover unless specifically mentioned. For example, a mech gains light cover from a talent. If a mech fires at that mech in heavy cover, they will still treat that target as in heavy cover, as that cover is better than the light cover granted to that target by their talent.


\subsection{Splitting up movement and action}

A character may take its \textbf{actions} at any point during its movement, and complete that movement after that action completes. However, each action itself cannot be split into several parts.

For example, a mech with 3 weapons and 6 movement can move 3 spaces, then attack, then move 3 more spaces. However, if that mech takes the barrage action, its action must complete before it can move further, i.e. it must fire all weapons at once (it can’t move 2 spaces, fire a weapon, move two spaces, fire a weapon, etc).

Actions and reactions themselves cannot be split into parts, and each action must resolve before the next takes effect. For example, a reaction typically interrupts and resolves before the action that triggered it resolves.


\subsection{TELEPORT}

Some experimental mechs have the ability to teleport. When a mech teleports, it instantly moves to a point within the specified range (it needs free space that will fit the whole of its body to be able to do so). Teleporting does not provoke reactions and ignores engagement. It ignores obstructions entirely and ignores line of sight. A mech can attempt to teleport to a space it can’t see, but if that space is already occupied, the teleport fails. The mech loses their action and takes 5 AP kinetic damage.

\subsection{FLIGHT}

Some characters have the ability to \textbf{Fly}. When you Fly:

\begin{itemize}
\item You ignore ground-based terrain, and you can totally ignore obstruction from ground based targets while flying. You only become engaged with targets if you move physically adjacent to them while you’re flying.
\item You ignore obstructions while flying at all points while you’re flying. If you need to pass over a a size 3 obstacle to get to the other side, feel free to do that (it’s assumed you just juke around). However, you can only ignore obstructions if it’s physically possible for you to do so (you can’t go right through a wall).
\item You can end your movement anywhere within a vertical or horizontal range of you equal to your fly speed, in any combination. For example, a mech with a fly speed of 6 could end its movement anywhere within 6 spaces of its location, up to 6 spaces high.
\item You cannot be knocked prone while flying
\end{itemize}

Flying also has some downsides:
\begin{itemize}
\item Flight movement must start and end along a \textbf{straight line}, though the direction can be changed for each separate movement. For example, a flying mech could move in one direction, then boost in another. 
\item If a flying character is ever immobilized, stunned, shut down, or otherwise cannot move, it falls.
\item If a flying mech takes structure damage, it must pass an agility check or fall.
\end{itemize}



\begin{center}
     \textbf{Hover Flight}
\end{center}

Some very advanced mechs have \textbf{Hover}. Hover mechs do not need to move in a straight line, and can remain still while airborne (they don’t have to move on their turn and can move any distance).