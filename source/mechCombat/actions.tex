ac\section{ACTIONS}

\begin{itemize}
    \item Players can take two \textbf{quick actions} or one \textbf{full action} on their turns
    \item You \textbf{cannot make duplicate actions} unless you make them as a free action or reaction. For example, you can only boost 1/turn, but you can boost again if you have a free action that allows you to boost, or if you overcharge to do so.
\end{itemize}

\begin{center}
    \textbf{BASIC QUICK ACTIONS}
\end{center}
\textbf{SKIRMISH} - Attack with one weapon from your mech.\\
\textbf{BOOST} - Move your speed\\
\textbf{RAM} - Attempt to knock down or knock back your target\\
\textbf{GRAPPLE} - Attempt to grab on your target, potentially immobilizing it or riding it.\\
\textbf{QUICK TECH} - Perform quick electronic warfare or systems-boosting activities\\
\textbf{HIDE} - Attempt to hide \\
\textbf{SEARCH} - Look for a hidden target

\begin{center}
    \textbf{BASIC FULL ACTIONS}
\end{center}
\textbf{BARRAGE} - Attack with two weapons, or attack with a single superheavy weapon.\\
\textbf{FULL TECH} - Choose and perform two options from the tech list\\
\textbf{IMPROVISED ATTACK} - Attack with a fist, rifle butt, or improvised weapon in melee.\\
\textbf{STABILIZE} - Heal and cool down your mech, reload, or attempt to end conditions affecting it\\
\textbf{DISENGAGE} - Move safely, avoiding reactions and engagement

\begin{center}
    \textbf{OTHER ACTIONS}
\end{center}
\textbf{ACTIVATE} - Activate a system or piece of gear that uses a quick or full action\\
\textbf{SHUT DOWN} - Shut down your mech as a desperate measure, to end system attacks, regain control of AI, and cool your mech\\
\textbf{BOOT UP} - Fire up your mech from Shut Down\\
\textbf{MOUNT/DISMOUNT} - Get in or out of your mech\\
\textbf{SELF DESTRUCT} - As a last ditch measure, set your reactor to go critical and explode\\
\textbf{PREPARE} - Hold a quick action for a specified trigger\\
\textbf{OVERCHARGE} - Push your mech past its limits, gaining an extra quick action as a free action at the cost of heat

\begin{center}
    \textbf{REACTIONS}
\end{center}
\textbf{BRACE} - Brace your mech for impact, reducing damage at the cost of your next turn’s actions\\
\textbf{OVERWATCH} - Attack a close-by target attempting to move

\begin{center}
    \textbf{PILOT ACTIONS}
\end{center}
Pilots on foot can take the following actions: \textbf{BOOST}, \textbf{HIDE}, \textbf{SEARCH}, \textbf{ACTIVATE}, \textbf{DISENGAGE}, \textbf{PREPARE}, \textbf{OVERWATCH}, \textbf{BOOT UP}, \textbf{SHUT DOWN}, \textbf{MOUNT/DISMOUNT}

They also get the following unique actions:

\textbf{FIGHT} - Attack with one weapon as a pilot \\
\textbf{JOCKEY} - Attempt to attack a mech on foot as a pilot



\subsection{BASIC QUICK ACTIONS}
                       
\subsubsection{SKIRMISH}

When you take the skirmish action, you attack with a \textbf{single weapon from your mech.} 
\begin{itemize}
    \item You can also make an attack with another auxiliary weapon from the \textbf{same mount}. That weapon can’t deal bonus damage. Auxiliary weapons are light and can be used to make quick, numerous attacks.
    \item Superheavy weapons are too cumbersome to be fired with a skirmish action and must be fired as part of a barrage action.
\end{itemize}

\subsubsection{BOOST}
When you take the \textbf{boost} action, you can move your speed. \textbf{Boosting} allows you to move again, in addition to taking a move action on the same turn. Certain talents and systems only activate on boosts (not regular movement).

\subsubsection{RAM}
\textbf{Ramming} is a \textbf{melee attack} made against an adjacent target with the aim of knocking down or back an enemy mech. 
If your attack is successful, your target is knocked \textbf{Prone} and you may also knock your target back up to 1 space directly away from you.

\subsubsection{GRAPPLE}
When you \textbf{Grapple}, you attempt to grab hold of an enemy mech and overpower it, disarming, subduing, or damaging it so that it cannot do the same to you. 

In order to perform a Grapple, choose an adjacent target and make a \textbf{melee attack}. On hit:
\begin{itemize}
    \item Both parties are \textbf{engaged}
    \item While grappled or grappling, neither party can \textbf{boost} or \textbf{take reactions}
    \item The smaller party is \textbf{immobilized}, but moves when the larger party moves, mirroring their movement. If both parties are the same size, they can make a contested hull check when they attempt to move, counting as the large party for their turn if they win.
    \item The grapple breaks if either target breaks adjacency (is knocked back for example)
    \item The attacker can end the grapple as a Free Action, and the defender can end the grapple as a quick action by making a successful hull or agility check.
    \item If there are multiple parties involved in a grapple, the same rules apply, but when counting size, count up all opponents of a side in a grapple. For example, if my all and I are both size 1 and grappling a size 2 target together, we would count our total size (2) and could attempt to drag our target around. 
\end{itemize}

\subsubsection{QUICK TECH}
The \textbf{Quick Tech} actions cover electronic warfare, countermeasures, and other actions that can be taken by a pilot, often aided by their mech’s powerful computing and simulation cores. Many pilots choose NHP (non-human person) assistants or more conventional comp/con units to help them with these tasks. All mechs have access to the basic tech actions. Further tech actions can be enhanced by taking systems that upgrade them. 


Some tech actions are \textbf{attacks} (often called \textbf{tech attacks}) and benefit from generic bonuses to attack rolls. All tech actions must choose a target within \textbf{Sensor Range} to be effective, and roll \textbf{systems vs. e-defense}. To use a tech action, choose a target in your sensor range (including yourself) and choose one of the following options:

\textbf{Bolster}\\
You use the formidable core processing power of your mech’s systems to boost one other target’s systems. That target can take +2 Accuracy on its next skill check of any kind before the end of its next turn. A mech can only benefit from bolster once at a time.

\textbf{Scan}\\
You can use your mech’s powerful internal systems to deep scan your enemies.
To \textbf{Scan}, make a \textbf{tech attack} against a target in your sensor range. On a successful attack, ask your GM to reveal one of the two to you:
\begin{itemize}
    \item Your target’s full statistics (HP, Speed, Evasion, Armor, HASE, etc), weapons, and systems
    \item Hidden information about the target, such as information caches it is carrying, current mission, pilot ID, etc. 
\end{itemize}
This information is only current when you receive it (for example, if the target loses HP again, your information won’t update).

\textbf{Lock On}\\
Make a \textbf{tech attack} against a target in range. On hit, the target suffers from the Lock On condition, enabling some systems and talents. Any attacker can end Lock On on a target when they attack that target to gain +1 Accuracy on their very next attack roll against that target.

\textbf{Invade}\\
Make a \textbf{tech attack} against a target in range. On success, your target takes \textbf{1d3 heat} and you may choose one of the following options:
\begin{itemize}
\item \textbf{Fragment Signal/Feed Misinformation:} You feed false information, obscene messages, or phantom signals to your target’s core computer, inflicting the Impaired Condition on your target until the end of their next turn.
\item \textbf{Aggressive Code:} You attack your target’s servos and engines, inflicting the Slowed condition on your target until the end of their next turn.
\item \textbf{Attack systems:} You go for the throat, the core computer. Inflict an additional 1d3 heat on your target
\end{itemize}


\subsubsection{HIDE}

In order to perform the \textbf{Hide} action, you need cover or concealment. The cover needs to be large enough to totally conceal your mech (such as a smoke cloud or building) or you won’t be able to hide. Lack of line of sight is always sufficient, and if you’re invisible, you can always attempt to hide.

\textbf{Hiding} is always successful. After you hide, you gain the \textbf{hidden} condition. A hidden target can’t be directly targeted by attacks or hostile actions, but can still be incidentally hit by attacks that target an area. NPCs cannot perfectly locate a hidden target but only know their approximate location. 

Performing any attack (melee, ranged, or tech), the boost action, or taking a reaction will break hiding. You can take other actions as normal. You must end your turn in cover to keep hidden. You automatically lose hidden if you end your turn in a place where you wouldn’t benefit from cover (ie, a mech comes around a wall and can now draw unbroken line of sight to you), your cover is destroyed, or you move from cover. If you’re hiding and invisible, you also lose hidden if you lose invisibility.


\subsubsection{SEARCH}

To detect a hidden target takes a quick action and makes a contested check.\\
\qquad\textbf{Mech:} The searching party needs you to be in their sensor range and makes a \textbf{systems} check. A hidden mech makes an \textbf{agility} check.\\
\qquad\textbf{Pilot:} The searching party needs you to be in range 5 and makes a pilot skill check, using skills such as \textbf{notice}. A hidden pilot makes a skill check and can use bonuses such as \textbf{infiltrate}.

Once a hidden target is detected, it loses the hidden condition.


\subsection{Basic Full Actions}BASIC FULL ACTIONS
\subsubsection{BARRAGE}

When you take the barrage action, you can attack with \textbf{two weapons}, or \textbf{one superheavy} weapon. You can choose the same target or different targets when you make these attacks.
\begin{itemize}
    \item When you attack with a weapon, you can also attack with another auxiliary weapon in the same mount. That weapon can’t deal bonus damage.
    \item The barrage action takes your mech’s full attention and the engagement of all its systems, so it requires a \textbf{Full Action} to use. 
    \item \textbf{Superheavy weapons} can only be fired as part of a barrage action, as they require the full attention of your mech’s systems.
\end{itemize}

\subsubsection{FULL TECH}
Choose and perform two options from the Quick Tech list (or choose from other systems that would take a quick tech action to use). You can repeat options, but must choose different targets for each option.

Alternately, use a system or tech option that takes a Full Tech action to activate.

\subsubsection{IMPROVISED ATTACK}

If your mech is unarmed or does not have a melee weapon, it can use an action to make an improvised attack action with a rifle butt, fist, or other improvised melee weapon against a target in melee. You may use the butt of a weapon, a slab of concrete, a length of hull plating, etc, to complete this improvised attack. How you flavor the attack is up to you!

An improvised attack costs a full action by itself to perform, and is separate to the skirmish or barrage actions above. It counts as a melee attack. An improvised attack is a \textbf{melee attack} that deals 1d6 kinetic damage on hit.

\subsubsection{STABILIZE}
During a heated battle or prolonged mission, it may become necessary to enact emergency protocols in order to purge your mech‘s systems of excess heat, to repair your chassis where you can, and/or buy your system time to eliminate hostile code.

To that end, a pilot may spend a Full Action to \textbf{Stabilize} and do \textbf{one} of the following:
\begin{itemize}
    \item Cool your mech, \textbf{resetting the heat gauge}
    \item Spend \textbf{1 Repair} to refill HP to maximum.
\end{itemize}
And \textbf{one} of the following:
\begin{itemize}
    \item \textbf{Reload} all weapons with the Loading Tag
    \item \textbf{End all Burn} currently affecting your mech
    \item \textbf{Perform an Engineering Check.} If successful, end one of the following conditions on yourself or an adjacent ally.
    \item \textbf{Jammed}
    \item \textbf{Impaired}
    \item \textbf{Lock On}
    \item \textbf{Immobilized}
    \item \textbf{Slowed}
\end{itemize}

\subsubsection{DISENGAGE}
When you disengage, you attempt to move safely. Until the end of your turn, your movement ignores engagement and does not provoke reactions, such as overwatch.

\subsection{OTHER ACTIONS}
\subsubsection{ACTIVATE}
Some systems or pieces of gear take either a \textbf{quick} or \textbf{full action} to use or activate. Such systems are marked with the quick or full action tags.

\subsubsection{SHUT DOWN}                                
\textbf{Shutting Down} your mech is a risky move, though one that is sometimes necessary to prevent potentially catastrophic systemic overload or AI unshackling. 

You can shut down as a \textbf{quick action}. When you take the \textbf{Shut Down} action, your mech powers completely off and enters the \textbf{Shut Down} state. While \textbf{Shut Down}:
\begin{itemize}
    \item Your mech is stunned. However, you can still take the \textbf{Boot} action to reboot your mech.
    \item Your mech is immune to all tech actions or attacks and can’t benefit from friendly tech actions. Any tech effects or conditions caused by a tech action (such as lock on, etc) affecting the mech immediately end.
    \item Your evasion becomes 5
    \item Your mech immediately cools (get rid of all heat)
    \item Any unshackled AI you have installed are re-shackled.
\end{itemize}


\subsubsection{BOOT UP}
You can power up a \textbf{shut down} mech as a \textbf{full action}, ending the \textbf{shut down} condition on it. Mechs that are powered off must be powered on with a boot up action before becoming active. You must be piloting a mech to boot it up.

\subsubsection{MOUNT OR DISMOUNT }
\textbf{Mounting} or \textbf{Dismounting} a mech is a turn of phrase commonly used by pilots. You don‘t “get in“ or “climb aboard“, you \textit{mount}. You‘re the cavalry, after all. It takes a \textbf{quick action} to mount or dismount. You must be adjacent to your mech to \textbf{Mount} it, and when you \textbf{Dismount} your mech, you are placed adjacent to it. If there’s no free space, you cannot dismount your mech.

If you want to \textbf{Eject} when you dismount your mech, you can do so, flying 6 in a direction of your choice. However, it’s a one-way system meant to be used in case of emergency, and leaves your mech permanently \textbf{impaired} until you full repair (and the eject system can’t be used again until you take a full repair).


\subsubsection{SELF DESTRUCT}
\textbf{Self-destructing} by overloading your reactor is a final, catastrophic play a pilot can trigger. You can initiate self destruct as quick action, causing your reactor to start melting down. At the end of your next turn, or up to two turns after (you choose), your mech will explode as though it suffered a reactor meltdown, annihilating it, killing any pilot inside, and causing a burst 2 explosion for 4d6 explosive damage around it. Characters caught in the explosion can pass an agility check to halve this damage.

\subsubsection{PREPARE}
\textbf{When you prepare an action}, you’re holding in preparation for a specific time or trigger (a more advantageous shot, for example). You can only prepare a \textbf{quick action}, and it costs a quick action to prepare. This counts as that action’s duplicate, so you can’t, for example, skirmish and then prepare a skirmish action.

Until the start of your next turn, you can take the prepared action as a reaction. You must set a trigger for this reaction phrased as a ‘When X, then Y’ sentence. X must be an enemy or allied reaction, action, or movement. For example: “When my ally moves adjacent to me, I want to throw a smoke grenade,” or “When an enemy moves adjacent to me, I want to ram them”.

It is apparent to a casual observer when you are preparing an action (you are clearly taking aim, cycling up systems, etc). You can’t take reactions while you’re holding a prepared action, but can take them normally afterwards. If you want to take a reaction and drop your prepared reaction, you can also do so. If the trigger doesn’t activate, you lose your prepared action.


\subsubsection{OVERCHARGE}
It is possible for skilled pilots to push their mech beyond factory specifications for a short period of time in order to gain a tactical advantage. Moments of hyperspec action won‘t tax your mech‘s systems too much, but sustained action beyond prescribed limits will take its toll. 

You may \textbf{Overcharge} your mech only once per turn. \textbf{Overcharging} incurs \textbf{1 heat}. The next time you overcharge before you make a full repair, this cost increases to \textbf{1d3 heat}. The next time, the cost increases to \textbf{1d6 heat}, and thereafter to \textbf{1d6+3 heat}. Taking a full repair resets this counter.

Overcharging immediately allows you to make \textbf{any quick action of your choice as a free action}, even one you already made this turn.


\subsection{REACTIONS}
Reactions are special moves that can be made out of turn order in response to incoming triggers such as attacks or movement. Upon use, reactions are, unless specified otherwise, expended until the \textbf{beginning of your next turn}. You can only make \textbf{one reaction per turn} (your turn or another actor’s), but any number per round, as long as you have unspent reactions to perform. All mechs can use the \textbf{Brace} and \textbf{Overwatch} reactions once per round by default.

\subsubsection{BRACE}
Once per round, you can choose to brace your mech’s systems in response to incoming fire. You can choose to brace against an attack after the attack hits you and you learn what the damage is.

If you choose to \textbf{Brace}, you gain resistance to all damage from the triggering attack (damage is halved, rounding up) and all other attacks against you are made at +1 difficulty until the end of your next turn. However, the stress of bracing means until the end of your next turn you cannot take reactions and on that turn you can only make one quick action (no regular move, no full actions, no free actions, and no overcharge).

\subsubsection{OVERWATCH}
All mechs are able to perform \textbf{Overwatch}. Overwatch represents your mech’s ability to control and defend the space around it from enemy incursion, whether through pilot skill, reflex, or finely tuned sub-systems. By default, a mech can make 1 overwatch reaction per round.

If any enemy starts any movement (move, boost, etc) inside the \textbf{threat} of one or more of your weapons, you can immediately make a \textbf{skirmish} action as a reaction against that target using that weapon and any others on the same mount that count the target inside their threat.

Threat is 1 by default for all weapons unless listed otherwise.

\subsection{Free Actions}
\textbf{Free Actions} are actions often granted by systems, talents, gear, or overcharge. Characters may perform any number of \textbf{Free Actions} on their turn, but only on their turn, and only those granted to them. The most common type of Free Action is a \textbf{protocol}, which can be activated or deactivated only at the start of a turn.

\subsection{PILOT ACTIONS}
Pilots can take the following actions: \textbf{BOOST, HIDE, SEARCH, ACTIVATE, DISENGAGE,  PREPARE, OVERWATCH, BOOT UP, SHUT DOWN, MOUNT/DISMOUNT}\\
These are the same as the mech actions.

They also get the following actions:

\subsubsection{FIGHT (FULL ACTION)}
Make a \textbf{melee} or \textbf{ranged} attack with one weapon against a target in line of sight and range.

\subsubsection{JOCKEY (FULL ACTION)}
It is possible (though very foolhardy) to aggressively attack an enemy mech while on foot. To \textbf{jockey} a mech as a pilot, you must be adjacent to it. You must make a contested check with the mech, using GRIT (your GM could allow another skill, such as maneuver, flash, or brawl if you argue for it). The mech contests with hull. If you win the contest, you’re now riding the mech, sharing its space and moving when it moves. The mech you’re riding can shake you off by repeating the contest successfully as an action, and you can jump off as part of your movement any time. 

Attempting to jockey takes a full action. The turn you successfully jockey, you can choose one of the below for free, and repeat one each turn after that you continue to jockey as a full action.

\textbf{Distract:} You inflict the \textbf{Impaired} or \textbf{Slowed} condition on your target until the start of your next turn.\\
\textbf{Shred:} Deal \textbf{2 heat} to your target by ripping at wiring, paneling, etc\\
\textbf{Damage:} Deal \textbf{4 kinetic damage} to that mech by firing or slashing at joints, hatches, etc