\section{Actions}
 ACTIONS

•   Players can take two quick actions or one full action on their turns
•   You cannot make duplicate actions unless you make them as a free action or reaction. For
    example, you can only boost 1/turn, but you can boost again if you have a free action that
    allows you to boost, or if you overcharge to do so

\subsection{Basic Quick Actions}

                       BASIC QUICK ACTIONS
\subsubsection{Skirmish}

                                             SKIRMISH

When you take the skirmish action, you attack with a single weapon from your mech.
        - You can also make an attack with another auxiliary weapon from the same mount. That
        weapon can’t deal bonus damage. Auxiliary weapons are light and can be used to make
        quick, numerous attacks.

        	- Superheavy weapons are too cumbersome to be fired with a skirmish action and must
        be fired as part of a barrage action.

\subsubsection{Boost}

                                                BOOST

When you take the boost action, you can move your speed. Boosting allows you to move again,
in addition to taking a move action on the same turn. Certain talents and systems only activate
on boosts (not regular movement).

\subsubsection{Ram}

                                                 RAM
Ramming is a melee attack made against an adjacent target with the aim of knocking down or
back an enemy mech.
If your attack is successful, your target is knocked Prone and you may also knock your target
back up to 1 space directly away from you.

\subsubsection{Grapple}

                                              GRAPPLE
When you Grapple, you attempt to grab hold of an enemy mech and overpower it, disarming,
subduing, or damaging it so that it cannot do the same to you.

In order to perform a Grapple, choose an adjacent target and make a melee attack. On hit:
    -   Both parties are engaged
    -   While grappled or grappling, neither party can boost or take reactions
    -   The smaller party is immobilized, but moves when the larger party moves, mirroring their
        movement. If both parties are the same size, they can make a contested hull check when
        they attempt to move, counting as the large party for their turn if they win.
    -   The grapple breaks if either target breaks adjacency (is knocked back for example)
    -   The attacker can end the grapple as a Free Action, and the defender can end the grapple
        as a quick action by making a successful hull or agility check.




    -    If there are multiple parties involved in a grapple, the same rules apply, but when counting
         size, count up all opponents of a side in a grapple. For example, if my all and I are both
         size 1 and grappling a size 2 target together, we would count our total size (2) and could
         attempt to drag our target around.

\subsubsection{Quick Tech}

                                              QUICK TECH

The Quick Tech actions cover electronic warfare, countermeasures, and other actions that can
be taken by a pilot, often aided by their mech’s powerful computing and simulation cores. Many
pilots choose NHP (non-human person) assistants or more conventional comp/con units to help
them with these tasks. All mechs have access to the basic tech actions. Further tech actions can
be enhanced by taking systems that upgrade them.


Some tech actions are attacks (often called tech attacks) and benefit from generic bonuses to
attack rolls. All tech actions must choose a target within Sensor Range to be effective, and roll
systems vs. e-defense. To use a tech action, choose a target in your sensor range (including
yourself) and choose one of the following options:


Bolster
You use the formidable core processing power of your mech’s systems to boost one other
target’s systems. That target can take +2 Accuracy on its next skill check of any kind before the
end of its next turn. A mech can only benefit from bolster once at a time.


Scan

You can use your mech’s powerful internal systems to deep scan your enemies.
To Scan, make a tech attack against a target in your sensor range. On a successful attack, ask
your GM to reveal one of the two to you:

             -   Your target’s full statistics (HP, Speed, Evasion, Armor, HASE, etc), weapons, and
                 systems

             -   Hidden information about the target, such as information caches it is carrying,
                 current mission, pilot ID, etc.

This information is only current when you receive it (for example, if the target loses HP again,
your information won’t update).


Lock On
Make a tech attack against a target in range. On hit, the target suffers from the Lock On
condition, enabling some systems and talents. Any attacker can end Lock On on a target when
they attack that target to gain +1 Accuracy on their very next attack roll against that target.


Invade

Make a tech attack against a target in range. On success, your target takes 1d3 heat and you
may choose one of the following options:





         Fragment Signal/Feed Misinformation: You feed false information, obscene messages,
         or phantom signals to your target’s core computer, inflicting the Impaired Condition on
         your target until the end of their next turn.


         Aggressive Code: You attack your target’s servos and engines, inflicting the Slowed
         condition on your target until the end of their next turn.


         Attack systems: You go for the throat, the core computer. Inflict an additional 1d3 heat
         on your target

\subsubsection{Hide}

                                                      HIDE

In order to perform the Hide action, you need cover or concealment. The cover needs to be large
enough to totally conceal your mech (such as a smoke cloud or building) or you won’t be able to
hide. Lack of line of sight is always sufficient, and if you’re invisible, you can always attempt to
hide.


Hiding is always successful. After you hide, you gain the hidden condition. A hidden target can’t
be directly targeted by attacks or hostile actions, but can still be incidentally hit by attacks that
target an area. NPCs cannot perfectly locate a hidden target but only know their approximate
location.


Performing any attack (melee, ranged, or tech), the boost action, or taking a reaction will break
hiding. You can take other actions as normal. You must end your turn in cover to keep hidden.
You automatically lose hidden if you end your turn in a place where you wouldn’t benefit from
cover (ie, a mech comes around a wall and can now draw unbroken line of sight to you), your
cover is destroyed, or you move from cover. If you’re hiding and invisible, you also lose hidden if
you lose invisibility.

\subsubsection{Search}
                                                   SEARCH

To detect a hidden target takes a quick action and makes a contested check.
         Mech: The searching party needs you to be in their sensor range and makes a systems
         check. A hidden mech makes an agility check.

         Pilot: The searching party needs you to be in range 5 and makes a pilot skill check, using
         skills such as notice. A hidden pilot makes a skill check and can use bonuses such as
         infiltrate.

Once a hidden target is detected, it loses the hidden condition.

\subsection{Basic Full Actions}BASIC FULL ACTIONS
\subsubsection{Barrage}
                                               BARRAGE

When you take the barrage action, you can attack with two weapons, or one superheavy
weapon. You can choose the same target or different targets when you make these attacks.

        	- When you attack with a weapon, you can also attack with another auxiliary weapon in
        the same mount. That weapon can’t deal bonus damage.

        	- The barrage action takes your mech’s full attention and the engagement of all its
        systems, so it requires Full Action to use.

        	- Superheavy weapons can only be fired as part of a barrage action, as they require the
        full attention of your mech’s systems.

\subsubsection{Full Tech}
                                              FULL TECH

Choose and perform two options from the Quick Tech list (or choose from other systems that
would take a quick tech action to use). You can repeat options, but must choose different targets
for each option.


Alternately, use a system or tech option that takes a Full Tech action to activate.

\subsubsection{Improvised Attack}

                                      IMPROVISED ATTACK

If your mech is unarmed or does not have a melee weapon, it can use an action to make an
improvised attack action with a rifle butt, fist, or other improvised melee weapon against a target
in melee. You may use the butt of a weapon, a slab of concrete, a length of hull plating, etc, to
complete this improvised attack. How you flavor the attack is up to you!


An improvised attack costs a full action by itself to perform, and is separate to the skirmish or
barrage actions above. It counts as a melee attack. An improvised attack is a melee attack that
deals 1d6 kinetic damage on hit.

\subsubsection{Stabilize}
                                              STABILIZE
During a heated battle or prolonged mission, it may become necessary to enact emergency
protocols in order to purge your mech‘s systems of excess heat, to repair your chassis where you
can, and/or buy your system time to eliminate hostile code.

To that end, a pilot may spend a Full Action to Stabilize and do one of the following:
•  Cool your mech, resetting the heat gauge
•  Spend 1 Repair to refill HP to maximum.

And one of the following:
•  Reload all weapons with the Loading Tag




•  End all Burn currently affecting your mech
•  Perform an Engineering Check. If successful, end one of the following conditions on yourself
  or an adjacent ally.
             •  Jammed
             •  Impaired
             •  Lock On
             •  Immobilized
             •  Slowed

\subsubsection{Disengage}
                                            DISENGAGE

When you disengage, you attempt to move safely. Until the end of your turn, your movement
ignores engagement and does not provoke reactions, such as overwatch.

\subsection{Other Actions} OTHER ACTIONS
\subsubsection{Activate}
                                             ACTIVATE
Some systems or pieces of gear take either a quick or full action to use or activate. Such
systems are marked with the quick or full action tags.

\subsubsection{Shut Down}
                                           SHUT DOWN

Shutting Down your mech is a risky move, though one that is sometimes necessary to prevent
potentially catastrophic systemic overload or AI unshackling.

You can shut down as a quick action. When you take the Shut Down action, your mech powers
completely off and enters the Shut Down state. While Shut Down:

       •  Your mech is stunned. However, you can still take the Boot action to reboot your mech.
       •  Your mech is immune to all tech actions or attacks and can’t benefit from friendly tech
         actions. Any tech effects or conditions caused by a tech action (such as lock on, etc)
         affecting the mech immediately end.

       •  Your evasion becomes 5
       •  Your mech immediately cools (get rid of all heat)
       • Any unshackled AI you have installed are re-shackled.

\subsubsection{Boot Up}
                                              BOOT UP

You can power up a shut down mech as a full action, ending the shut down condition on it.
Mechs that are powered off must be powered on with a boot up action before becoming active.
You must be piloting a mech to boot it up.

\subsubsection{Mount or Dismount}
                                     MOUNT OR DISMOUNT

Mounting or Dismounting a mech is a turn of phrase commonly used by pilots. You don‘t “get
in“ or “climb aboard“, you mount. You‘re the cavalry, after all. It takes a quick action to mount or
dismount. You must be adjacent to your mech to Mount it, and when you Dismount your mech,
you are placed adjacent to it. If there’s no free space, you cannot dismount your mech.


If you want to Eject when you dismount your mech, you can do so, flying 6 in a direction of your
choice. However, its a one-way system meant to be used in case of emergency, and leaves your
mech permanently impaired until you full repair (and the eject system can’t be used again until
you take a full repair).

\subsubsection{Self Destruct}
                                          SELF DESTRUCT

Self-destructing by overloading your reactor is a final, catastrophic play a pilot can trigger. You
can initiate self destruct as quick action, causing your reactor to start melting down. At the end
of your next turn, or up to two turns after (you choose), your mech will explode as though it
suffered a reactor meltdown, annihilating it, killing any pilot inside, and causing a burst 2
explosion for 4d6 explosive damage around it. Characters caught in the explosion can pass an
agility check to halve this damage.

\subsubsection{Prepare}
                                                PREPARE

When you prepare an action, you’re holding in preparation for a specific time or trigger (a more
advantageous shot, for example). You can only prepare a quick action, and it costs a quick
action to prepare. This counts as that action’s duplicate, so you can’t, for example, skirmish and
then prepare a skirmish action.


Until the start of your next turn, you can take the prepared action as a reaction. You must set a
trigger for this reaction phrased as a ‘When X, then Y’ sentence. X must be an enemy or allied
reaction, action, or movement. For example: “When my ally moves adjacent to me, I want to
throw a smoke grenade,” or “When an enemy moves adjacent to me, I want to ram them”.


It is apparent to a casual observer when you are preparing an action (you are clearly taking aim,
cycling up systems, etc). You can’t take reactions while you’re holding a prepared action, but can
take them normally afterwards. If you want to take a reaction and drop your prepared reaction,
you can also do so. If the trigger doesn’t activate, you lose your prepared action.

\subsection{Overcharge}
                                            OVERCHARGE
It is possible for skilled pilots to push their mech beyond factory specifications for a short period
of time in order to gain a tactical advantage. Moments of hyperspec action won‘t tax your
mech‘s systems too much, but sustained action beyond prescribed limits will take its toll.





You may Overcharge your mech only once per turn. Overcharging incurs 1 heat. The next time
you overcharge before you make a full repair, this cost increases to 1d3 heat. The next time, the
cost increases to 1d6 heat, and thereafter to 1d6+3 heat. Taking a full repair resets this counter.


Overcharging immediately allows you to make any quick action of your choice as a free
action, even one you already made this turn.

\subsection{Reactions}
                                        REACTIONS

Reactions are special moves that can be made out of turn order in response to incoming triggers
such as attacks or movement. Upon use, reactions are, unless specified otherwise, expended
until the beginning of your next turn. You can only make one reaction per turn (your turn or
another actor’s), but any number per round, as long as you have unspent reactions to perform.
All mechs can use the Brace and Overwatch reactions once per round by default.

\subsubsection{Brace}
                                                  BRACE

Once per round, you can choose to brace your mech’s systems in response to incoming fire. You
can choose to brace against an attack after the attack hits you and you learn what the damage
is.


If you choose to Brace, you gain resistance to all damage from the triggering attack (damage is
halved, rounding up) and all other attacks against you are made at +1 difficulty until the end of
your next turn. However, the stress of bracing means until the end of your next turn you cannot
take reactions and on that turn you can only make one quick action (no regular move, no full
actions, no free actions, and no overcharge).

\subsubsection{Overwatch}

                                              OVERWATCH

All mechs are able to perform Overwatch. Overwatch represents your mech’s ability to control
and defend the space around it from enemy incursion, whether through pilot skill, reflex, or finely
tuned sub-systems. By default, a mech can make 1 overwatch reaction per round.


If any enemy starts any movement (move, boost, etc) inside the threat of one or more of your
weapons, you can immediately make a skirmish action as a reaction against that target using
that weapon and any others on the same mount that count the target inside their threat.


Threat is 1 by default for all weapons unless listed otherwise.

\subsection{Free Actions}
   Free Actions

Free Actions are actions often granted by systems, talents, gear, or overcharge. Characters may
perform any number of Free Actions on their turn, but only on their turn, and only those granted
to them. The most common type of Free Action is a protocol, which can be activated or
deactivated only at the start of a turn.

\subsection{Pilot Actions}
   PILOT ACTIONS

Pilots can take the following actions : BOOST, HIDE, SEARCH, ACTIVATE, DISENGAGE,
PREPARE, OVERWATCH, BOOT UP, SHUT DOWN, MOUNT/DISMOUNT
These are the same as the mech actions.


They also get the following actions:

\subsubsection{Fight}
                                      FIGHT (FULL ACTION)
Make a melee or ranged attack with one weapon against a target in line of sight and range.

\subsubsection{Jockey}

                                    JOCKEY (FULL ACTION)
It is possible (though very foolhardy) to aggressively attack an enemy mech while on foot. To
jockey a mech as a pilot, you must be adjacent to it. You must make a contested check with the
mech, using GRIT (your GM could allow another skill, such as maneuver, flash, or brawl if you
argue for it). The mech contests with hull. If you win the contest, you’re now riding the mech,
sharing its space and moving when it moves. The mech you’re riding can shake you off by
repeating the contest successfully as an action, and you can jump off as part of your movement
any time.


Attempting to jockey takes a full action. The turn you successfully jockey, you can choose one of
the below for free, and repeat one each turn after that you continue to jockey as a full action.


Distract: You inflict the Impaired or Slowed condition on your target until the start of your next
turn.

Shred: Deal 2 heat to your target by ripping at wiring, paneling, etc

Damage: Deal 4 kinetic damage to that mech by firing or slashing at joints, hatches, etc

