\subsection{The Turn}
During combat, \textbf{players always take the very first turn.} One player or friendly NPC (nominated by all players) gets to act first. If the players can't agree, the GM chooses. After that player finishes their turn, the GM may \textbf{activate} a \textbf{hostile, GM-controlled NPC}, allowing them to take a turn. Each NPC can usually only be activated once, unless they have special traits. \textbf{The player that previously acted then nominates a player or friendly NPC to act next}, and so on. Each actor gets 1 turn in a round, alternating between players and hostile NPCs, with players each choosing the next player or friendly NPC to act.

If there are only actors of once side left, the remaining actors take their turns in any order. After all actors have completed a turn, this constitutes \textbf{1 round.} The round then begins again, alternating, so if one side ended the last round, the other side starts the new round. This may, for example, mean that hostile NPCs take the first turn in the new round if the players outnumber them.

On a turn, players and NPCs can perform a \textbf{move}, and either two \textbf{quick actions} or one \textbf{full action}, with no duplicate actions allowed. Players can \textbf{overcharge} their mechs to gain an extra quick action at the cost of heat, and all actors can also take any number of \textbf{Free Actions} or \textbf{reactions}.

\textbf{MOVE} - A player can move their character up to their full movement speed.\\
\textbf{QUICK ACTION} - A quick action represents an action that takes a few moments, such as quickly firing a weapon, using a system, or moving a little further\\
\textbf{FULL ACTION} - A full action represents an action that takes your full attention, such as a sustained barrage of fire, or field repairing your mech\\
\textbf{FREE ACTION} - A free action can be made at any point during your turn, but only on your turn. It doesn't count as a quick or full action, so you can still make those as normal. Free actions can also be used to make a duplicate action (for example, a free action could allow you to boost if you have already made that action). You only get free actions if some part of your character grants you them.\\
\textbf{REACTION} - Reactions are special moves that can be made out of turn order in response to incoming attacks, movement, or other prompts. You can make each reaction only a specified number of times per round, but take as many overall as you want. By default, mechs have two reactions they can take once a round: \textbf{brace}, and \textbf{overwatch} but they may gain more from systems or talents. Reactions resolve \textit{before} the triggering action completes by default, but some may resolve after.

\begin{center}
    \textbf{PILOTS}    
\end{center}
On foot, a pilot has the following statistics in mech combat:\\
\quad\quad\textbf{HP:} 6 + grit\\
\quad\quad\textbf{Evasion:} 10\\
\quad\quad\textbf{E-defense:} 10\\
\quad\quad\textbf{Armor:} 0\\
\quad\quad\textbf{Size:} 1/2\\
\quad\quad\textbf{Speed:} 4

These statistics might change depending on the gear and armor a pilot brings with them.

Pilot weapons and armor are at a scale that they can't be relied on to take down mechs - and mech weapons are at a scale that they normally completely pulverize a pilot-scale foe. The following rules apply to pilots (some of these refer to mech rules later in this section):
\begin{itemize}
\item Pilots have the \textbf{biological} tag. They are immune to Tech actions (even beneficial ones), though they can still be targeted by electronic systems such as drones or smart weapons. If a pilot would take Heat, they instead take an equivalent amount of energy damage.
\item When a pilot is called on to make a mech skill check, they use \textbf{Grit} instead of the required statistic.
\item Pilots can't aid a mech, give, or receive any bonuses that would apply to mech-sized weapons (such as from Talents)
\item Pilots and pilot weapons and gear don't benefit from Talents
\item Pilots can't cause a mech to become engaged and don't provide obstructions to mechs no matter the size.
\end{itemize}

It is possible for a pilot, with enough experience, to gain enough technology and experience to become capable of fighting on nearly even terms with some mechs, but such pilots are usually stuff of legend.

\begin{center}
    \textbf{Pilot, Mech, and AI}    
\end{center}

As components of the same character, pilots and mechs share the same move and actions. You can split them up if you so choose. If you want to use a quick action to skirmish with your mech, use another quick action to dismount, then use your move to run to cover as your pilot, you can absolutely do so.

A mech needs to be piloted for you to take actions with it, with the pilot physically present inside the cockpit, unless that mech has the AI property. If your mech has the AI property, at the start of your turn you can choose to turn your controls over to your AI. If you do so, your pilot can no longer take actions or reactions with your mech until the start of your next turn, but your mech gets its own set of actions and reactions, freeing you up to take normal action as a pilot. However, your AI cannot benefit from any of your talents while it pilots your mech.