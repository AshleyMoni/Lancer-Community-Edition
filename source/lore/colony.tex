\subsection{Core and Colony}

\textit{Lancer}'s canon universe is, essentially, post-scarcity -- that is, resources are not only plentiful,
but accessible for most people at little to no cost.

In practice, \textit{Lancer}'s post-scarcity golden age only exists for some: well-developed Core worlds
adjacent to blink gates are fantastically wealthy, rich with technology and cultural capital. Their
citizens, generally speaking, do not want for anything: they're afforded a base level of housing,
education, healthcare, and food, localized to their state.

Wealth and capital are not common constructs on these worlds, as currency tends to be
restricted to a generalized requisition ability; only when you venture outside the bounds of Core
space do you run into a need for money.

Core worlds are varied in appearance and urbanization. So long as they fit the following criteria,
they are considered a ``Core'' world:

\begin{enumerate}
\item Global distribution of population, or the capability to distribute its population
\item A reliable method of transitioning people and goods from the world to orbit, or the
         capability to do so.
\item A central government or other unipolar governing body with demonstrated adherence to
         Union's edicts.
\end{enumerate}

Worlds that fit these criteria are considered to be Core worlds of Union; thanks to the immediacy
of blinkspace, outside of Cradle and her neighboring systems there is no real need for stellar
proximity to be considered a ``Core'' world.

Worlds that do not fit the criteria for Core designation have a myriad names, designations,
classifications, and so on -- a good catch-all term is ``Colony'' world. On these colony worlds, life
is less secure, and their populations often want for food, medicine, etc. The colony designation
can encompass everything from initial, small-team settlements, to worlds with populations in the
millions.

These two classifications of worlds, Core and Colony, are tied into Union's larger economic
system, extant as a consequence of the development of blinkspace travel and the trade that
followed.

In order to participate in intergalactic commerce, worlds translate their currency to Manna, a
universal unit of value assigned and administered by Union's Economic Bureau. Core and colony
worlds that participate in intergalactic trade use Manna to effect trade outside their borders. It is
common for these worlds to have a primary economy and a secondary, manna-based economy.
Manna is \textit{incredibly} valuable compared to local currencies.\footnote{For example, a person who commands,
say, an account with 1,000 Manna would be fantastically wealthy on any given colony world, able to finance
their day-to-day life for decades without need to earn more.}

Post-scarcity in practice means that, on a Core world, players will have access to most
unrestricted consumer and raw goods. Specialized items might require certain licenses, available
through purchase or qualifications (in game terms, as rewards given by the GM), but are readily
available (i.e. they can get them within the day, delivered or picked up as convenient).

On Colony worlds, true post-scarcity availability diminishes the farther you get from the nearest
Core worlds, or as a result of shortages, resource-hoarding, or loss. Players will have access to
necessary goods (unless there is a shortage, rationing, etc) and wide access to raw materials;
specialized items may be difficult to obtain due to any number of reasons: they're limited in
number and kept under lock and key by the colonial governor, they're in the hold of a downed
ship on the other side of the world, they're of limited number after the last supply ship took off,
and so on.

\subsection{Colonies and Worlds: Planting A Flag}

The process of settling a world differs in specifics, but generally a private Core-to-Colony
settlement mission proceeds as follows:

First, a group of people form a Colonial Venture, a loose, temporary corpro meant to pool manna
and licenses in order to petition the owner of the destination system for a colony charter. The
system owner is \textit{typically} Union, as few other interstellar states have the resources to ID and flag
habitable worlds.

After a Colonial Venture secures a colony charter, they lobby local (or intergalactic, depending on
proximity) colony firms for supplies, infrastructure, and materiel that they cannot supply for
themselves.

Colony firms offer realtime-tiered packages in exchange for a cut of the colony world's raw
resource output. These packages typically feature a settlement concierge unit, a 100k+
genebank, a tier 1 printer, and a colony ship packed with a bundle of prefabricated habitation
pods, heavy drones, medical benches, pan-biome seed libraries, and other necessary colonial
infrastructure.

Not everyone who is party to a Colonial Venture departs with the colony ship.

A typical colony ship is hundreds of meters to a kilometer long: the vast majority of that space is
devoted to prefabricated supplies. The live crew onboard the colony ship will be the first settlers
of the new colony world: a small team of engineers, scientists, and specialists numbering in the
dozens. They will -- with help from the colony's concierge unit and its attendant drones, heavy
drones, and subalterns -- make planetfall, and begin the long work of establishing a colony
footprint. In the meantime, the first native generation is incubated, birthed, tended, and raised by
the concierge unit and assigned natal/educational colonists.

Fifteen to twenty years after landfall is made, the first generation of native-born colonists is at
population-viable levels (usually in the thousands, though depending on colony scale this can be
a larger number) and select members of the landfall team takes formal control of the colony's
development from the colony concierge. The first generation begins to work to improve the
colony and explore their new world, building out both the colony footprint and beginning work on
new secondary and tertiary sites.

Concurrent to the first generation's development, an additional first (1.5) generation is grown
from distinct reserve genetic material. This second generation comes of age a year or two at
most from the first, to provide some genetic variance and further establish a stable, viable
population.

Assuming all variables to be nominal, the colonial settlement is now established, and further
development occurs organically.