\chapter{A Curious Alchemy, A Mundane Miracle}
A Curious Alchemy, A Mundane Miracle  

Printing is a ubiquitous term for matter processor/fabricator systems found throughout the  
galaxy thanks in large part to paracausal scientific advances made post-Deimos Event. Printers  
range in size from handheld units fed by back-worn matter processors, to hanger-sized, fully  
self-contained printing facilities. 
 

Printers range in time and efficiency. The larger and or more complex an item is, the longer it  
takes to print. Generally speaking, most print protocols involve some assembly after constituent  
parts have been fabricated. 
 

All printers function in the same basic manner: raw matter is processed -- the more pure the  
element, the higher quality the result -- and fabricated into the requested item (or its constituent  
parts). Handheld printer operators craft items and objects in augmented reality; larger printers  
are automated. 
 

You cannot print a printer. Union tightly controls access to printer plans and licenses, and does  
not allow them to be distributed. 
 

You cannot print food beyond basic protein reconstruction: a mealy, grainy loaf of compressed  
edible matter that is unsatisfying, but sufficient to survive. Food is still an important luxury,  
cultural, and prestige item, and a given person on a given world will prefer “real” food to  
synthetic food. 
 

Printing a size .5 mech chassis will take about six to eight hours with a hanger-sized printer.  
Printing a size .5 mech chassis with a handheld printer will take about a week. 
 

Printing a personal defense weapon with an unscheduled tabletop printer takes about fifteen  
seconds; printing a PDW with an unscheduled handheld printer takes about five minutes. 
 

Despite the presence of printers and other processor/fabricator systems, the majority of  
construction across human occupied space -- and certainly outside the galactic Core -- is still  
performed the “old” way: through sourcing raw materials, refining, fabrication, and assembly. 
 

Printers of all classes are valuable. What follows is a basic list of available printers and their  
general capacity.
 

    $\bullet$    Sub- or Unscheduled Printer
 
             $\circ$     Subscheduled or Unscheduled printers are handheld, pack-mounted, or table- 
                 mounted printers common among cheap, mass-goods merchants, fabricators,  
                 sculptors, and other private individuals. 
 

                                                                                                           


         $\circ$     While the printers themselves are black box items, the license for obtaining one is  
             not particularly hard to get. It does require training, certification, and yearly check- 
             ins for continued licensing in order to operate. Additionally, your print orders must  
             pass a review process, and are filed to a publicly-viewable omninet database by  
             category. 
 
         $\circ$     Sub and Unscheduled printers are most common among personal businesses and  
             are best for making person-portable hard goods and repairing hard objects. 
 
$\bullet$    Schedule 1 Printer
 
         $\circ$     Typically hangar sized, schedule 1 printers can handle vehicles and structures up  
             to a size 1 mech.
 
         $\circ$     Of the massive-size printers, these are the most portable, and usually the size  
             found installed on dedicated logistics ships, orbitals, and long-term lander  
             packages. 
 
         $\circ$     Schedule 1 printers, like all scheduled printers, are not the best for fabricating  
             objects smaller than a .5 mech: they can be fitted with precision attachments, but  
             unless the object is of simple geometry, you’re better off filing an order from a  
             artisan sculptor with an sub-schedule printer or true goods merchant.  
 
              
 
$\bullet$    Schedule 2 printer
 
         $\circ$     But for rare cases, schedule 2 printers are built-in place and are not portable.  
             Filling 4-5 stories in height, schedule 2 printers can handle a single requisition up  
             to size 2, or process multiple size .5 requisitions at the same time. 
 
         $\circ$     As with S1 printers, unless an S2 printer is outfitted with a specialized suite, for  
             small items it is best to look elsewhere.   
 
$\bullet$    Schedule 3 printer 
 
         $\circ$     The largest printers, generally reserved for Union, corprostate, and municipal  
             uses. Built in place, not portable, standing at least 8 to 10 stories tall, schedule 3  
             printers can fabricate anything size 3 or smaller, and can handle multiple size 1  
             fabrications running in parallel. 
 
         $\circ$     Schedule 3 printers are usually built as self-contained buildings themselves, with  
             multiple on-site suites for printing smaller-than size .5 items. These are separate  
             from the main fabrication chambers, and can run in parallel without taxing the  
             main system. 
 
         $\circ$     Schedule 3 printers can be converted to operate in micro/null gravity as self- 
             contained ships, often accompanying battlegroups as a rear echelon logistic  
             support element tasked with fabricating fast hull repairs and bulk orders.  
 
$\bullet$    Schedule 4 and up
 
         $\circ$   Truly massive machines, materiel, and construction project may employ  
             networked suites of Schedule 3 printers, an informal construct classified by Union  
             as a Schedule 4 configuration. These only operate in microgravity, due to their  
             bulk and the size of the projects they work on. 
 
         $\circ$     Massive engineering and construction projects, despite the prevalence of printing  
             technology, more often than not use conventional super-engineering and  
             construction methods; printers are reserved for refining raw material and  
             producing inert constituent parts -- beams, panels, wiring, and so on -- which they  
             can churn out at a far more rapid and reliable rate than more complex structures. 
 

                                                                                                               

\section{Manna}
Manna  

Union is not motivated by currency, and neither are its subjects. The hegemon’s society is  
structured around a galvanizing mission: ensuring the survival of the human species through  
implementing the edicts of the Central Committee (which, in turn, is implementing the best-fit  
plan dictated to them by Forecast/GALSIM, though none but the Central Committee and  
Forecast/GALSIM know this). 
 

Union is post-scarcity and does not function as a market-based economy. An “economy” in  
Union is only understood as a historical or antiquated term, as your average Terran views capital  
and the exchange of currency for goods as a relic of an unsustainable past, one that led to a  
collapse that plunged humanity into thousands of years of self-inflicted darkness, violence, and  
misery. 
 

However, Union recognizes that not all of its client states have progressed to a post-capital  
society. In order to foster fair galactic trade and build a shared consciousness  -- rather than  
violently suppress monetization --  Union’s Central Committee recognized early on the need for a  
galaxy-wide standardized currency: this they call Manna. 
 

To create Manna, Union extracts an abstracted unit of value from its subject states through  
complex treaties and client-facing economic structures. Data, raw materials, human potential --  
tens of thousands of factors go into the creation of a single omni-digital unit of Manna.  
 

Manna’s exchange rate is relative to the currency for which it is being exchanged, or to the  
currency that is being exchanged for it. Wealthy, developed worlds are rich in Manna due to their  
data output, their raw human potential, and other factors. Small colony worlds also benefit from  
Manna’s formulae: their control over raw materials, projected development, and so on all  
contribute to a beneficial exchange value. 
 

Cosmopolitans trade in Manna, as do states and any other entity that engages in trade across  
solar systems. Since the vast bulk of humanity still is bound to their home worlds, stations,  
moons, etc, the vast bulk of humanity still uses whatever their world’s currency is, and will only  
encounter Manna if they do business off-world (or with entities that are off-world). 
 