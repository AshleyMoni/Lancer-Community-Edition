\subsection{Pilots}

A note on the player character role, and how pilots become Lancers.

Player pilots in \textit{Lancer} generally fill a common archetype and tend to be aware of their status --
in-setting, as a ``Lancer'', the galaxy’s colloquial term for a role similar to a flying ace --  but how
they respond to that status and act in their role differs from pilot to pilot.

It is perfectly possible -- and narratively interesting -- to \textit{not} play as a Lancer, but as a run-of-the-
mill pilot, one without the blessing of destiny or the distinction of being the Special One.

While there are gradations of skill among pilots, generally speaking, a trained and outfitted mech
pilot in their chassis will beat a conscript or unskilled pilot in a mech, even in situations where the
trained pilot might be outnumbered.

A mech pilot in Lancer, outside of their mech, is usually a well-trained combatant, adept-to-
experts in the fields their backgrounds cover. They are not super-soldiers, though some with
heavy cybernetic enhancement might be able to operate outside the bounds of ``normal'' human
ability. Most pilots -- as a consequence of being a human in the narrative present -- have some
degree of non-invasive biological enhancement, but rely heavily on wearable/detachable
technologies and months-to-years of training to augment their talents and knowledge.

A Union-rated mech pilot, one that is a Regular or Auxiliary trooper, is generally educated by the
Navy, specialized in their role through training, traveled some (through their world or local
system), and is at least in their mid 20s -- it is rare to see a pilot younger than that due to the
necessary qualifications for training. All Union Regular and Auxiliary pilots are put through the
same naturalization courses, trained by Union regulars and instructors, and serve in integrated
units for a period of ten years.

For pilots not affiliated with Union as Regulars or Auxiliary troopers, standards are different.
Corpro States might tune their pilots via implants or prosthetics to be more compatible with their
own chassis lines; the Aun raise their pilots alongside their NHP-analogs who will pilot their
chassis with them; the Karrakin Baronies outfit their sons and daughters with legacy chassis and
offer them to martial academies for training; Ungrateful guerillas take what they can get, and so
on.

The majority of pilots in Lancer are simply individuals trained and outfitted to crew their
machines. Your pilot -- or your player(s)’s pilot -- might be the special one, or the chosen one, or
otherwise blessed with some combination of fate, talent, and destiny, or they might not. Maybe
they’re just the person in the pilot seat, in the right place at the right time.

\subsection{Pilots, Lancers, and Union}

What sets your pilot apart from the rest of the mech pilots in the galaxy? What makes them a
\textit{Lancer}?

It is important to note that the term ``Lancer(s)'', in \textit{Lancer}, is a catch-all term for mech pilots
across the galaxy who are roughly equivalent, by Union standards, to a contemporary ace. Not all
pilots or people who crew mechs are Lancers -- the majority of them are just competent pilots --
but all Lancers are pilots.

Lancers \textit{tend to be} set apart from ``regular'' pilots by ability, talent, training, luck, skill, reputation, or
some combination of these factors: they do not achieve any classification or gain any medals,
certifications, or rank that qualifies them as a Lancer, it’s simply a term that exceptional pilots
achieve through a combination of time in the saddle and performance on campaign.

There is an argument that some formal identification system exists at levels so far removed from
the players as to be a secret that would not only never be uncovered, but not even assumed as a
possibility. A rumor exists among pilots: that Lancers are chosen for their role, their names noted
down in some distant database deep under Cradle’s surface. Some pilots -- the mystic, the
superstitious, the paranoid -- believe that Union has intervened in their lives or the lives of others
in order to make them Lancers.

For the rest, it is because of their skill, strength, and experience that they have earned the right to
call themselves Lancer. No help from Union, no preordination from down on high. They became a
pilot when they were conscripted, or when they enlisted, or when the ships came and bombed
their homes, and they chose to fight back. Pilots -- Lancers -- are \textit{made}, yes, but whether by deep
machinations, by fate, by chance, or by choice, who can say.

It’s important to note: pilots and lancers both have downtime. Combat, exercises in the field, and
moments of stress and fear generally do not occupy the majority of their time.

So what does this look like in practice? Some examples follow:

\textbf{Union Regulars: A Cog In The Greatest Machine}

\fluff{A squad of Union regular troops, tight-knit on campaign together, laughing and joking as they run
post-sortie checks on their chassis. They are one squad of hundreds on the deck of a carrier --
their home for only a few more years -- each member from a different world, bound together now
by a greater purpose.

Union doesn’t work without these pilots. A whole galaxy of bickering states, of kings and
presidents who think they can shake the pillars of heaven. Who think they can spit in the face of
Union. Might as well turn their back on the ocean. When they do, these are the people who
darken their skies, who drop to their world, who set them right where they’ve done wrong.

Nothing sets this campaign apart for them, Regulars on a second tour. An Administrator’s report,
filed to the DoJ. A battlegroup organized. A briefing to see: a king who thinks himself a god, who
grinds his boot on the neck of a people he thinks are his. He forgot they’re not his people, they’re
Union. Time’s come for this king to get a reminder that Cradle’s reach is long, its vision is pure,
and its will inexorable.

As veterans of this campaign by now, these pilots have been given a little bit more autonomy than
the replacements: their chassis bear more kill hashes, carry heavier weapons than boot squads
fresh to the deck. They’re still subordinate to their CO, but all the other grunts look to them when
the lead and lasers start flying: they’ve been in the shit a bit longer, learned how to move when
someone downrange wants to kill them.

They are Lancers, veterans of a campaign that has ground all of their friends to dust and ash.
From different worlds, of different backgrounds, but all bound by the same truth:

They’re Union, and they’re here to help.}

\textbf{Yond-Balor, Second of House Yond, Heir-By-Blade}

\fluff{In the Baronies, glory is given only to those who prove they deserve it. Yond-Balor Karrakis,
second son of Yond-Aleph Karrakis, Baron of House Yond, sub-barony of the House of Glass, has
proven through combat and intrigue that he alone deserves to command the warhost of the House
of Glass. But one obstacle remained: Yond-Argo Karrakis, Yond-Balor’s older brother, first
claimant to the throne and, by extension, the warhost of the House of Glass.

As a youth, Yond-Balor was raised to crew the chassis, to gird himself in sealed power armors
and become the death-word of House Yond. He proved himself in battle against the Ungratefuls,
against House Muur and House Fleur, even in single combat against the Vice-Lord of the House
of Sand. Yond-Balor alone stood by his father’s deathbed while Yond-Aleph fevered, poisoned by
assassins sent by the Mutable Houses.

And where was Yond-Argo? Where was the First name’d? Gone. Travelling through space,
pledged to those headless ones: to Union. Yond-Balor wasted no time in dispatching his older --
far older -- brother when he came to pay respects to his late father.

Yond-Balor has taken what is his by birth, drawn it steaming from the gut of his worthless kin. As
the chassis closes around him and the red light of the ready-light begins to pulse, he grins. His
name is no longer second-son -- it is Lancer, first and only.}

\textbf{Jeddah, Loyal Wing}

\fluff{Young-Wing Jeddah bin Surat al Noor has long greyed while maintaining his Loyal Wing’s kit.
After his salvation by the Albatross, Jeddah wanted nothing more than to don his own kit, to be
marked by the crimson lion, to feel the smooth hilt of his very own waveblade in his hand.

How many times had he cleaned blood and burn from his Loyal Wing’s hardsuit? How many times
had he nursed his Loyal-Wing while he lay in agony, moaning through nightmares that gripped
him since the Metavault?

Countless times. Jeddah may not have seen the terrors that Loyal Wings face, but he has seen
their echoes. Never did it dissuade him, for he remembered: these are the only ones who came to
save him.

Now, on his eighteenth subjective year, he has been called before the Loyal Council for
assessment and ordination. His teacher has picked him, of all his fellows, to learn the art of
piloting, the art of standing-at-the-speartip. His breast swells with pride when he hears his name
called.

Rising, he walks across the dawn-lit courtyard to stand beside the rest of the Loyal Council. In
time, he too will heft the lance, he too will wear the golden armor, will wield the laser-and-rifle.

He will be a Lancer, hero to the lost.}

\textbf{Penny, In At The Dawn}

\fluff{Penny threaded the last stitch of her MIRRORSMOKE MC patch on the shoulder of her coat. She
her coat out, looked at it, nodded. She let it go. Microgravity hold it for a moment while she stood,
slipped her arms in and zipped it up.

The Penny in the mirror was kind, if a little tired, but who wouldn’t be after ten years in hard-
sleep?

Still, she smiled. A little bit of pride in her situation wasn’t a terrible thing to have, was it? The
MSMC 501st Detachment. The One-Eyed Fox. The ``Here-For-Nows''.

Subcommander Gerrard’s voice echoed over the carrier’s all-comm.

``Sunrise. It’s sunrise. Good morning everyone, welcome to the Dawnline Shore.'' There was a
pause, the sound of someone talking off-mic. Gerrard came back a moment later. ``Since Geordie
won our calendar-pick, we’re going to listen to some old shit. Apologies in advance. We’ve got
three days until we’re in orbit above Myrrh.''

The old shit started right after, echoing over the PA. Penny liked it. Ray-gae, it was called. She
hummed along as she tapped her fabrication request into the ship’s queue. For all the rep these
men had, they were actually quite kind. Gruff, unpolished. But kind, and bound to a kind of honor.
At least, the Foxes were.

Penny didn’t know it, but she would be the only one to make it out of this milkrun; she was a cut
above the rest, though why she wouldn’t find out until later.

Penny, you see, would be called Lancer. But that wouldn’t matter much to her: she only wanted to
be called friend.}

\textbf{MJ Martinez, A Kid From Old Spinrock}

\fluff{``You ain’t nothin’''

Last words the decklord ever spake to MJ Martinez. You ain’t nothin.

MJ unclipped his helm. Tugged it off. Felt the breeze cool the sweat from his brow. Free now he
was from the constant chatter of the battlescape. Was only cheers and cheers anyways: MJ and
his company had done the impossible: held the breach, repelled the Crown Host.

Whennow did the decklord die? Must’ve been a hunnert or twain years back. MJ grinned, couldn’t
help. AIN’T NOTHIN was slapped in whitewash on the flank of his chassis, scored some by
tachyon but proud and loud.

Now MJ had been through the shit. Left Old Spinrock and joined with the Oxes, cadet’d for a
decade until he proved his mettle enough to crew an Everest. Ain’t nothin’ in terms of what you
can grow to crew, but for a kid from the hub of a spinsat, an Everest is the King Death.

A decade more and MJ got himself a squad. A decade more, a comp’ny. The No-how’s. Sigil: a
grin, one missin’ tooth. Mirrored MJ’s. You get to do that when you’re a real damn hero.

A real damn hero. MJ sure as shit wasn't gonna paint that on his core, but it did sound nice. A call
drew his attention -- more Union regulars trooping by, waving up at him as he stood half-out his
Everest. He waved back, saluted, they cheered.

Those old decklord words echoed: you ain't nothin'.

Well, what had their chances been against the oncomin’ horde? Nothin’. Stood strong and Union-
proud 'gainst the shinin' host of a king who called himself god. Nothin’ stopped him before.

Perfect, if nothin’ was gonna save the world, then he was the right pilot for the job. A nobody in a
nothin’-rig who saved the world.

MJ laughed. Not bad for a nobody from Old Spinrock. A Lancer, he was. Claimed the title on his
own.}

\textbf{Tyrannocleave}

\fluff{It began in the mines, as everything in the Baronies does. You ask ``where'' and ``why'' - I will tell
you.

Do not protest, Grand Baron. Do not weep. It is unbecoming of your station, filth.

Listen now: we have nothing but time, so I shall take my time. The first violence I commit to you is
to tell you my story. You shall know before you die.

My mother died when she bore me. Hooked to tubes and deadlife machines. Your men had her
bear me with the assistance of an exo. A machine massaged her heart. Current forced her
muscles to work. A black box strapped to her chest forced breath into her lungs — because you
knew. You knew at that point the only protest we had was to die, and deny you new bodies.

So you took me from her and I never knew her again — I found the footage of my own birth later, I
wanted to see — and in the care of your stone-matrons I learned how to tear worlds to pieces. I
learned the lash, and the pick, and how to read the earth you had me kill.

When I was ten I learned that I must take these pills to survive. That I was born riddled with
cancers that you would never take from me — too expensive. The recovery process too long. And
your quarterly profits could never slip, that would mean ruin for your name.

Look at me. It is why we are hideous. Your propaganda paints us as beasts — outward we are,
but it is your hand that shapes our flesh. It is your word that scars our skin and bids cancer grow
thick in our bodies. In the name of Manna and your House, you ruin millions.

Millions. There are millions of us, and countless more. This is your unbecoming. Your death,
below your feet. Every inch of every palace and ship and grand city you build, you build on our
backs, with our labor, at the cost of our lives. It began in the mines, but will not end there: your
Baronies are as riddled with us as my body is with metastatic.

Yes, weep now. For your perfect face. Your perfect worlds. Your perfect dominion.

Did you know — I did not know what the ``sky'' was until the Ungratefuls found me? Liberation
means this. Not simply seeing the sky, but knowing that there is such a thing as land without
stone above you. And then knowing that there are those who put you there. I always thought —
we were always ministered this — that we were damned and penitent. That we had transgressed
Lordgod’s perfect kingdom and must mine in penance.

How wrong I was! It was not a divine prescript that doomed us to labor, but you, and others like
you. How surmountable the problem became then.

This is the Ungratefuls’ gift to me. This is their gift to the rest of us, the waste, the offal of your
courts and palaces. This is why even your machines fight alongside us.

Because you never taught us there could be a thing like the sky. Because you thought that we
would never learn.

Goodbye, Grand Baron. Your House burns, your coffers have been emptied, your monuments
have been torn down. Your line, your sons and daughters, hang from the balconies of your own
palace.

We learned, Grand Baron. We learned how to hope, and who to hate.}

And so on.

\subsection{A Lifetime of Experience}

Generally speaking, the natural life of a pilot is only marginally longer than that of a given
Diasporan. Their observed life tends to be closer to that of a Cosmopolitan, as a good number of
pilots tend to fall in this group of humanity. The subjective experience a pilot has of their life,
though, is no different from you or me.

Pilots are valuable hard power resources for their states, corpro-states, and groups -- they get
shipped all around their home system, home world, or the galaxy as needed for the duration of
their deployment.

The average Diasporan from a developed world with no augments or significant bioengineering
lives somewhere in the ballpark of 120 real years.

Subjective age doesn’t match up with real age: it’s perfectly possible for someone’s real age to
be 300, and their subjective age to be 30. Remember, ``subjective age'' is how old a person
appears to an observer and how old they perceive themselves to be; ``real age'' is how old a
person is in Cradle-Standard years, tracked by Union.

Union’s registration system counts real age as time progressing as an observer from Cradle
would perceive it. Therefore, a Cosmopolitan’s real age may increase significantly, depending on
the length of time that passes on Cradle while they are engaging in interstellar travel.

In Lancer, a pilot only lives once (with some rare, and rarely talked about exceptions). Facsimile
programs do exist, but these are digital simulacra, moving portraits and holograms that give the
appearance of the person they represent, but are not free-thinking digital consciousnesses.
Death, it would still seem, is an inescapable end for humanity -- hence Union’s discomfort in
stagnation.

\subsection{Mechanized Cavalry}

Why mechs?

The rise of mechanized cavalry can be attributed to two factors: rapid human expansion into
space, and the conflicts that stem from expansion.

Union’s first three thousand years of expansion and colonization occurred without the benefit of
the blink network or knowledge of how to pierce into blinkspace, as those breakthroughs would
not happen until after the Deimos Event. Drones and unmanned interstellar vehicles were used to
scout on hundred-year increments: Fired off towards target worlds to be followed by early-
pattern nearlight ships not yet built, crewed by Far-Field teams not yet born, one hundred years
after the drone had arrived.

In its 2nd millenium, Union’s expansionist imperative demanded that humans spread out among
the stars. Old colonies and installations waited to be reactivated by human hands, and Union
marked the growing diaspora as both a point of civilizational pride and necessary for the survival
of the species.

However, humans do not survive on hostile worlds or in hard vacuum. Ships and stations did fine
to protect people from hard vacuum, radiation, and all the terrors of deep space and dangerous
atmospheres, but planets needed to be claimed, not flown over.

To address this problem, Union astrocartographers and Far-Field administrators issued a call for
a standardized, medium-term livable suit pattern that could be used for tactical, scientific, and
civilian purposes in space and on alien worlds. Universal, powered, hardened against the
elements, and comfortable to wear, the hardsuit soon became an indispensable piece of
hardware for anyone leaving the bounds of Cradle.

The first hardsuits were adaptable, universally compatible, with flourishes and specializations
unique to their manufacturers and the demands of their users. Early Far-Field teams wore larger
suits with more robust equipment, some equipped with weapons to protect themselves from
native flora and fauna; Colony populations first adopted hardsuits as personal emergency
equipment in the case of damage to their sealed habitats, then as a common method of travel
and exploration outside of their settlements; Spacers wore slimmer, sealed suits, often living in
them for days on end while they piloted their great ships on long voyages between populated
worlds.

The suit found military applications as well, as some companies and Union foundries began to
add plated armor on their suits, wove ballistic knit into them, and integrated hardened
technologies that allowed soldiers to manipulate smart weaponry.

On worlds where tracked or wheeled terrestrial vehicles proved insufficient, larger hardsuits were
built, capable of hauling cargo that would have otherwise required transport trucks. In these
suits, the pilot occupied a cockpit, not just the suit, and extensive training was required to ensure
the pilot could operate the mechanized chassis professionally. These heavy suits were commonly
accompanied by drone flights and operated in tandem with other heavy suits.

The first mechanized chassis and the first pilot were born from this combination of exploration
drive and protection from the elements. It took an acute moment -- a flashpoint -- to catapult the
mechanized chassis from a useful-if-plodding civilian platform to a deadly military instrument.

That flashpoint presented itself on Hercynia, a jungle world in distal space, around 4500U.

Hercynia was a lush, massive tropical-to-temperate world rich with nitrogen and oxygen. It was
dominated by continent-spanning tropical forests around its equatorial and temperate zones,
only giving way to borean plains in the northern and southern poles. It seemed, on all scans, to
be a perfect colony world: rich in resources, breathable atmosphere, a temperature range that
was pleasing to most humans, and carbon-based flora. Hercynia was a Gaia world, perfectly
suited for human life.

The contract was posted, a consortium of colony firms won, and a joint colonial expedition was
undertaken.

Initial colonial sites were established in early 4500. Within months, colonial scientists pinged the
Union Science Bureau with urgent calls for assistance: alien, indigenous structures had been
discovered by colonial survey drones sent to plan out future development. Shortly thereafter,
contact was made by colonial elements sent to explore the structures.

Humanity had encountered the first conscious, sapient alien species in their history. The
Egregorians, so-named due to their autonomous/consensus co-consciousness, were outwardly
horrifying in appearance, but largely peaceful and able to communicate. Colonial dispatches of
the time indicate that the Egregorians were reverent of the new arrivals, regarding them and their
technology as magic, as godlike\footnote{Note: the term used did not indicate a ``god'' as humans would conceive of one, but rather an analogous-god as the
Egregorians conceptualized of it: a divine simultaneous sensation/perception/consciousness is a succinct, if not
perfect, shorthand. For brevity’s sake: ``god'' or ``the divine'' or some variant therein.}.

Union assembled a team of xenobiologists -- an established field due to discoveries of alien flora
and fauna on many colony worlds -- and linguists, anthropologists, and engineers to head to
Hercynia and investigate. Upon landing and contact, integration into Union structures began. The
private colonies were Unionized, their charter companies compensated, and Hercynia was
walled off: no public omninet, no public blink access, and credible-source dissemination through
interested channels of a total colony collapse due to disease.

Hercynia became a black site. A hole in space. Union’s next great project began: integrating and
naturalizing the Egregorian many-peoples into human social structure.

This ended poorly. Refer to \textit{No Room For A Wallflower} for a detailed history.

The resulting conflict prompted massive research and development into combat-capable and
effective mechanized chassis platforms across all theaters. Mechanized Chassis became a
viable, all-round option for combat in all theaters -- on hard terrain, in zero-g, and in transitional
spaces, a piloted mechanized chassis, a mech, could outperform and outmaneuver any other
ground-based all-theater attack option.

\subsection{Infantry and other Ground Forces}

Infantry, tanks, groundcars, light vehicles, trucks, etc, all still exist and are far more widely used
than mechanized chassis in combat.

Mechs are used much in the same way that cavalry was used in ancient combat: rapid, heavy,
armored, deadly, and expensive to recruit, train, and maintain. Mechs can break down, are
vulnerable to systemic attacks, are hard to camouflage, are susceptible to human-portable
weapons and traps, and require significant time and resource investment to build. Licenses, save
for GMS, must be acquired and certified, renewed on occasion.

Not everyone can become a pilot, but all a person needs to fight is a reason.

Mechs are shock units of a category above tracked and wheeled weapons platforms. Infantry
make up the bulk of all individual units in an army, and, ultimately, are still the only way that
states and state-like entities can take and hold territory.

Additionally, infantry are far less expensive and more expendable than a mech and its pilot.

Furthermore, while printers are relatively ubiquitous, not everyone has the licenses, resources, or
time to field a printer large enough to fabricate mechanized chassis.

A soldier on a given developed world usually is a professional fighter serving a term of service,
either a volunteer or someone serving due to compulsory state requirements (a mandatory
service period, a lottery-based service period, or so on). This typical soldier has undergone a
lengthy period of basic physical and mental training meant to condition them into being a
temporary member of the military class, followed by a shorter period of specialized training
based on their assignments, and now is posted to a base, unit, or patrol.

Most military mech pilots begin with this training.

This typical soldier is outfitted with a main battle weapon, possibly a sidearm, ammunition (if their
military uses weapons that require it), equipment and gear appropriate to their specialization, a
uniform, and basic personal armor to protect them from small arms fire, shrapnel, blades, and
blunt force trauma. This typical soldier has been assigned to a unit of similarly outfitted soldiers,
given a rank, and has a directive -- punishable by some compelling form of hard or soft power --
to obey their superior and all other superiors. Some of these soldiers may carry more specialized
equipment commensurate with specialized training that they received -- shaped charges, a
longer-range omninode, a heavier battle weapon, a drone swarm and its control unit, a CQB or
area-denial weapon, and so on.

Variations on this galactic catch-all professional soldier exist. Some worlds are more developed
than others, and some are less; similarly, some put stock in their militaries, and some prefer to
spend their resources elsewhere.

More militarized societies might simply have more soldiers, or better equipped soldiers, or
enforce conscription, or have a caste system or other form of ordering their society around
martial orders. Note that ``more militarized'' does not necessarily mean more technologically
advanced: it is perfectly possible in \textit{Lancer} to encounter a society utterly devoted to a military
hierarchy whose soldiers proudly polish steel pikes and have never encountered black powder,
much less a mechanized chassis.

Also note that -- like in other areas of society -- military technology does not necessarily have to
be uniform across a society. Some states may have a limited number of mechs, won hundreds of
years before in a trade with a passing Cosmopolitan ship, but their local industry is only able to
develop black powder muskets for their infantry; others may have left ranged weapons by the
wayside in favor of shimmering blades and mirrored shields, riding into battle on hovering skiffs
over massed formations of pikemen.

In short: while mechs are the focus of Lancer, infantry still form the backbone of most every
organized army in the galaxy. Expect to encounter them.

\subsection{Space combat}

Space combat between fleets is elegant at a distance and brutal up close.

Against the stark black of deep space, long silhouettes drift in tightening gyres, maneuvering to
dodge torpedos and kinetic kill-clouds thousands of kilometers distant. Energy beams, invisible
to the naked eye, streak across the void, shimmering only where they impact their targets.

To an observer, this combat between fleets-of-line is silent, sterile: Long capital ships appear to
twirl thousands of miles apart, closing slow as their orbits align. Clouds of glittering metal chaff
and slag bloom into the darkness, catching the light of distant stars. The blue torches of
torpedoes trace fading lines in the night.

However, to those engaged, there is no elegance. No grace. There is just the long, persistent
terror of space combat:

Days before they can even see their opponents through optics, the first torpedoes, kill-clouds,
spoofers, shrouds, and subaltern kinetics have been launched. Pilots, officers, and
crewmembers are roused from stasis and ordered to battle stations. Massive kinetic and energy
weapons, one-shots, begin their acceleration cycles, spooling up for their perfect shots.

The flagship’s XO NHP goes live, paired with the commanding officer but given free rein to
partition and duplicate themselves into sub-sentient subjectivities to better advise and
coordinate all of their ship’s systems. Tactical command is given to the flagship NHP; strategic
and kill command remain the purview of the commanding officer.

The fleet, carrier group, battlegroup, or patrol NHPs construct a virtual war room, networking into
a hybrid one/many mind (this is referred to later as a Fleet Legion) in order to ensure total-actor
integration over the battlespace. All commanding officers are party to the information and
recommendations that issue from the Legion, and tactical feeds are meted out to lower ranks on
a need-to-know basis (commensurate with their rank and tac/strat portfolio).

The first commands after the initial volley are maneuver and systemic orders: avoid incoming fire,
communicate with allied ships, begin to close the unpredictability gap. At this point, all hands
prepare for combat: ready onboard null-atmosphere equipment, lock into your station, push
combat stims, cycle pressure suits, link to Legion subjectivity.

Then, scramble fighters, bombers, and landers. Corvettes and gunboats, destroyers and cruisers
-- subline ships-- begin attack runs. Frigates, battleships, tender ships, and carriers: hold your
lines, continue systemic/kinetic countermeasures.

Along flight decks and inside carrier blisters, all-hands alarms howl as pilots and techs hurry to
finish pre-flight checks and procedures. Techs load ordinance and payloads onto fighters and
bombers while pilots and crews prep systems, uploading the latest telemetries, battle reports,
flight plans, and obstacle reports. If there are mech chassis and marines aboard, they hurry to
their landers.

Combat Area Patrol (CAP) wings are launched, escorts tasked with defending landers, corvettes,
bombers, and gunboats from other fighter wings, torpedo flights, and subline ships. They chart
flight paths through the kill-clouds and anti-ship weapons, aiming to cross the shrinking no-
man’s land to harass enemy capital ships, force them to deal with threats at all ranges. At their
earliest launch, it will take roughly a day to cross no man’s land.

Bombers and subline ships aim to engage capital ships at a close enough range that they cannot
maneuver to avoid their payloads: bombers and subline ships present small -- relatively speaking
-- and agile targets, deceptively high-threat units that present a very real danger to any capital
ships that let them get too close.

Landers, laden with marines and mechs in support, have the most dangerous mission: crash into
the enemy, disembark, and either capture or disable the enemy ship from the inside. Their ships
are built cheap and sturdy, typically with modular chambers and detachable single-use boosters.
Their mission, after all, is to take the ship or fail.

The fleets at this point are engaged, and the combat continues in a shrinking window: the
unpredictability gap, the space where NHPs and pilots can still outmaneuver their opponents,
shrinks faster.

Ship-to-ship combat increases in intensity as the ships-of-the-line circle towards each other. Mid
and close range kinetic cloud weapons open up, huling thousands of projectiles at plotted and
predicted paths. Some short-cycle batteries open fire at this point, their beams carving invisible
lines of terrible energy through the black, scattering off projected shielding and ablative armor.

Meanwhile, systemic weapons pound away at fleet Legions and individual ship systems,
attempting to gain tactical advantage. Those spoof probes and shrouds, launched in the early
days of the fleet engagement, activate, pinging enemy sensors and comms arrays with hostile
code, creating false signatures and signals to distract weapons and pilots. Subaltern kinetics
inform their masters of final trajectories, then plunge towards their targets, triggering their
payloads on impact or, failing to find a positive hit, in proximity to the enemy.

Legions face each other down, NHPs engaging in electronic warfare fought in methods esoteric
and incomprehensible to human observers, hurling ontological/anti-solipsistic paradox weapons
back and forth on a plane of battle removed from the human subjectivity.

Finally, at range too close for the enemy to successfully engage in evasive maneuvers, long-cycle
batteries open up, hurling tremendous, demi-solar particle lances at their targets. Capital
Commanders at this point must carefully balance their power budget, shifting between angled
shielding and weapon power if they are to survive a hit from a long-cycle battery.

This is the battle’s climax, the moment when the unpredictability gap closes. Due to the
tremendous power needed to fire a capital ship’s spinal cannons, each ship-of-the-line generally
only has one shot to hit their target, as cooling and recharging a ship’s main guns -- kinetic or
energy -- simply takes too long to be viable in combat. Commanders know this, and hold on to
their single shot as long as they can: they must hit, and score a clean hit, or they’ll be exposed to
an enemy with all the time in the world to take the killing blow.

Meanwhile, at the battle’s height, fighters and subline ships buzz in angry swarms, locked in
bitter wing-combat between their enemy counterparts. Marines and mechs fight grinding
compartment-to-compartment, deck-to-deck CQB and melee actions as they fight to gain
control. Cloud-kill kinetics and point-defense weaponry pepper the flanks of great capital ships,
tearing away at superficial armors, blisters, and distal chambers. Here and there along the line,
batteries score hits against their targets, and the battlespace is filled with the brilliant micronovae
of a capital ship’s cataclysmic death. In Legionspace, NHPs tear at each other’s fundamental
sense of being in combat somehow more terrible than that occurring in subjective space.

When one side beats a retreat or is eliminated, the battle ends.

Most ships of the line, unless the system is damaged, have at least a .9 lightspeed eject drive: at
the start of the battle, conservative, nervous, or cautious captains might begin to spool this
system up so that it is hot and ready to fire in an emergency. When triggered -- manually, at the
order of an NHP, or automatically -- the eject drive shunts its ship from its current speed to .9
Light, hurling it towards a planned (or randomized) eject route. This expeditious retreat is
dangerous, taxing both systems and personnel, but it’s better than death.

The remains of the battle are left to the victors. Survivors are rounded up. Scuttled or captured
ships are boarded by skeleton crews and turned towards friendly shipyards: printer technology
cannot build capital ships, as they’re simply too large. Prisoners are dealt with. Communications
are relayed back to central command. NHPs drop from Legionspace, unlinking, drawing down to
their non-combat parameters. Objectives are assessed, adjusted, and fleets either continue on
their campaign, retire, or steam for a friendly shipyard for repairs and replacements.

From a distance, silence. Up close: the combat of titans, with individuals caught in the middle.
Typical fleet engagements cost thousands of lives: when fought near inhabited worlds, moons, or
stations, the cost can become exponential. Unconventional stellar combat -- such as
accelerating or nudging asteroids and comets into planets -- can prove to be yet more costly.

In Lancer, large-scale fleet combat is (relatively) rare and terrible. It represents the breakdown of
a whole sector, as systemic powers bring their considerable production and logistical capacity to
bear against each other in contests over worlds and ideologies. The result of this is never cheap,
with civilian casualties numbering in the millions; should capitol worlds be engaged, the human
cost can reach and surpass billions.

Smaller-scale fleet combat tends to occur between warring states that share a world, or a world
and its moon, and usually in low orbit as fleets ferry ground troops from one continent to another,
or as flights of ships escort intercontinental/interstellar missiles along their flight paths. These
fleets are generally composed of subline ships, corvettes, mounted mechs, and fighter/bomber
wings; it is rare for a capital ship to engage in fleet-to-fleet combat in low orbit unless it is
supporting an invasion and striking ground targets.

Finally, the most common space combat is piracy, where small groups of ships -- converted
civilian shuttles, older-model fighter/bombers, mounted mechs -- attack other small groups of
ships, lone subline freighters, or isolated mining/resource-extraction stations. These combats are
fast and chaotic, with little-to-no use of NHPs, orders of battle, or capital ships -- save for the
steel and c legends of interstellar piracy.