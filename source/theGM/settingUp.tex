\section{Setting up a Game}

LANCER is \textbf{best played with a group of 3-5 players} (excluding the GM). You can play with more
or less players, but people will get differing amounts of time in the spotlight. Combat (by far the
most rules heavy part of the game) can be tricky or lengthy to run with so many players, so the
more players you have, the more time it will take to add it into your narrative.

Each player will need a \textbf{character sheet, pencil}, a \textbf{d20} (20 sided dice) and a \textbf{large number of
d6s}. Accuracy and Difficulty rarely stack past about +4 or so if you want to keep that in mind.

A typical play session of LANCER can take anywhere between 2 and 5 hours (especially with
more combat), so keep that in mind when planning your start time. It’s old GM advice, but it can
be useful to make sure people have drinks and snacks on hand, and taking breaks after a couple
hours of play can do wonders to keep people alert and attentive.

\textbf{We strongly recommend using a hex or grid-style map to keep track of actors during
combat}. If playing online this is relatively simple to set up, there are a number of apps that will
do this for you (including the very popular Roll20 app). If playing offline you can use an erasable
battle map (fairly easy to acquire), miniatures, or even just a plain sheet of graph paper (a large
size one will be easier).

It can be very useful for you, as the GM, to keep notes while you play for later reference so we
like to keep a notepad on hand. You might also ask your players to take notes if the story or
political situation gets a little complicated. Sometimes people forget everything about a narrative
when they have weeks between sessions.

\subsection{The First Session}

During the very first session, there’s a couple of important steps you can take that will drastically
help the rest of your games. If you’re playing with a group of friends who have all played before

this advice might be more or less useful but if you’re playing with people you’re meeting for the
first time, these steps can be extremely helpful.

\begin{enumerate}
\item \textbf{Make sure everyone gets an equal opportunity to introduce themselves}. This
sounds very self-evident but quieter or less bombastic players can often be
overlooked. Get everyone to introduce themselves as a player and also introduce
their character a little bit.
\item \textbf{Set expectations for the game}. This is a \textbf{really important} step. Let your players
know the kind of game you’re planning on running. Is it going to be more combat
focused, with very little story? A story of political intrigue, with very little mech
combat? A mix of both? What rules are you using? Are you using any homebrew
(self created rules) in your game? Clearly setting expectations can let players
know immediately whether your game will be the kind of game they are going to
enjoy, or to modify their own expectations of your game so they can get
enjoyment out of it. It is sometimes impossible to please everyone - it’s better to
let a player find a game that fits them rather than try to accommodate every player
at once. The reason you do this in the first session is so you don’t find out three
sessions in when conflicts of play-style start arising. Even if something comes up
later, clearly letting players know what your game is going to be about when that
time comes.
\item\textbf{Set up a second session}. If you’re just doing a one shot or a casual game, don’t
worry about this step, but once you have everyone gathered at once, it’s very
helpful to compare everyone’s schedules.
\end{enumerate}

These steps are more or less useful for any role-playing game you might run (not just LANCER).
Here’s a couple pieces of advice specifically for LANCER though:

\begin{enumerate}
\item If you’re new to the game, start at \textbf{License Level 0}. The options may be limited at this
level, but it makes jumping in for new players much easier, and the game does not suffer
at the tactical level, even with just basic GMS gear available.
\item If it’s the first session, \textbf{skip over downtime} and cut right to mission goals, stakes, and
preparation.
\end{enumerate}

\subsection{Where and How To Start Your Narrative}

So, you have your players, you have your dice, you have your table reserved (or cleaned!), your
pens, pencils, and paper stocked, your snacks and drinks ready to go.

What’s missing? What’s the most important part of a \textit{Lancer} session?

The \textbf{narrative!}

By \textit{narrative} we don’t solely mean the story that you write. Some GMs like to go by the skin of
their teeth and write nearly nothing for story, relying on improvisation, and other GMs like to go
as far as to write dialogue for NPCs. You can find what works for you, but the narrative here
means both \textbf{the story you create and the way your players interact with that story}.

Your narrative is the most important part of \textit{Lancer}. The section that follows details some
common ideas for how to start your narrative, but they are only suggestions, not prescriptions;
the best narratives come from you, and from the stories you write with your players. As the GM,
you’re going to have to do some work to set up the world(s) that your narrative takes place
across.

The best starting place for this is a simple question:

\textit{Always ask yourself}  \textbf{“Why does this matter?”}

When writing villains, encounters, moments, and whole narratives, the most important editorial
question you can hold on to is “why?”

\textit{Why} should your players want to embark on this narrative? \textit{Why} should they want to deal with
that character, rather than ignore them? \textit{Why} should they care about the stakes you’ve set up in
your narrative?

Always ask yourself “Why?”, and be prepared with an answer: this is how you craft a story that
engages players.

\textit{Why} should players want to embark on this narrative? Well, because it’s their home world under
threat. \textit{Why} should they want to deal with that character, rather than ignore them? Because that
character is their commanding officer, or a hero of theirs, or a long-lost friend. \textit{Why} should they
care about the stakes you’ve set in your narrative? Because your players’ families, friends,
ideals, etc, are threatened by the antagonist, or because your players are intrigued by the
mystery of it, or because your players have been implanted with subcutaneous hypercaloric
immolator webbing that will trigger a febrile self-immolation response if they fail.

Additionally, remember that players have needs separate from their characters, though they will
usually choose to play characters that align with those needs. Try and strike a balance, and don’t
be afraid of editing on the go: if you see half your players losing interest because you’re in a
roleplay heavy session rather than combat, adjust. And vice versa.

For quick reference: When writing your narrative, always have an answer (or an idea how how to
answer!) the “Why?” Question. Also, be aware that your players may have competing interests,
and try to balance your sessions accordingly if necessary.

\subsection{Mission Hooks}

Generally the writers of this book believe writing a whole story arc out for your players (with
beats, NPCs, and decisions all pre-determined) is an inadvisable approach to creating a good
narrative. Players often disconnect from stories in which they feel they have little or no agency,
and it often conflicts with the principles outlined in the \hyperlink{GMAgenda}{GM Agenda} (trying not to say ‘no’, trying
to give your players time in the spotlight).


The easiest way to craft a compelling story without restricting player choice too much is to start
everything with a hook. \textbf{A hook is an interesting or compelling scenario that answers the
question “Why does this matter?”}. Players in Lancer embark on missions, campaigns, and
encounters because a compelling narrative hook draws them in. Hooks are common across
tabletop roleplaying games: indeed, as a GM you probably have a stable of hooks ready to adapt
to any new system you come across.


\textbf{The most important part of the hook is that it provides a compelling reason for the players
to investigate}. This can be out of a sense of duty, monetary reward, personal power, or
something connected to their character’s background.


You should try and write your own where possible, but to start (and for inspiration and examples)
\textbf{twenty example hooks} unique to Lancer are listed below. They are very useful as catalyzing
moments to begin a story focused on one location, around one event, or one character. You can
fill in details as you desire.


If you’re feeling adventurous, you could roll a d20 and choose one of these hooks if you want to
randomly determine them.


\begin{center}
-1-
\end{center}
\fluff{Travelers on a public shuttle or space elevator in transit between a world’s surface and a
geostationary orbital port. The ride is a long one, with nothing else to do but watch the world
below grow larger in the shuttle’s portholes and ride out the nauseating gravity changes. Who are
the other passengers on this elevator to the planet’s surface? Perhaps unbeknownst to the
players, one of the people on that shuttle is on a mission of utmost importance to the future of
the planet below. Before the shuttle touches down, the characters, willingly or not, will become
intimately involved in that mission\dots}

\begin{center}
-2-
\end{center}
\fluff{You walk along the vast, gently curving, false-sky concourse of the blink station. Neon light spills
from the bars and clubs along the top floor, the low sound of distant music mixing with the
polyglot chorus of languages as Cosmopolitans mingle with Diasporans-in-transit. The vast
spread of humanity is here, an endless stream of stories and potential adventures. Above, the
arrivals and departures board rattles ceaselessly. The characters have come here for business.
Their contact is deep in the station, but finding them won’t be an easy task, and they are soon to
find out if the risk involved in the gig is worth its weight in manna\dots}

\begin{center}
-3-
\end{center}
\fluff{Kilometers outside the walls of a lonely colony, a quiet homestead seems a place out of time,
where a small community of farmers keep the land and live humble lives. The woods beyond are
dark and deep, unexplored, haunted; the settlers speak of dim light that dances between the
close-packed trunks. They are a superstitious people, and dare not venture out into the woods.
You, however, are not, and you’re not so sure those lights are only harmless ghosts\dots}

\begin{center}
-4-
\end{center}
\fluff{A derelict mining ship hangs in orbit above a massive, roiling gas giant. Abandoned due to a
deadly gas leak, the ship is still rich with rare and dangerous raw materials. The news has spread
across the local omni, and fortune-seekers, ace pilots, and tyrannical reavers all swarm to claim
their part of the prize. The race is on, may the first crew claim the riches\dots}

\begin{center}
-5-
\end{center}
\fluff{A Union DOJ/HR emancipation team’s shuttle has been grounded by anti-air batteries installed by
a slaver-state’s military. Your squad is en route to assist the emancipation team. There may be a
diplomatic solution, but liberation always comes with a cost\dots}

\begin{center}
-6-
\end{center}
\fluff{A Corpro-State is attempting a hostile takeover of a distal system. The colonists there have sent
out an SOS asking anyone who can hear to come and help -- your team is in transit and in range,
but as you near touchdown, to your teams’ surprise, you see familiar livery. Your chassis are all
licensed from the CS that is committing the hostile takeover\dots}

\begin{center}
-7-
\end{center}
\fluff{In the bowels of a massive, planet-sized metropolis, a pirate lord makes their home. The cityworld
government has petitioned Union for assistance in removing the pirate lord from power; Union
has tapped your team to go in and get the job done. The pay is better than anything you’ve seen,
but the job will be long and fraught with danger - a descent into a literal underworld\dots}

\begin{center}
-8-
\end{center}
\fluff{A strange, solitary figure who calls themselves “Administrator” has appeared on your world. They
demand that you take them to your leader, and though they appear human, they display abilities
beyond your wildest imagining; magic, like nothing you’ve seen. You agree to escort them, but
the road ahead is long and fraught with danger, and you’re not so sure about the ghost they carry
with them\dots}

\begin{center}
-9-
\end{center}
\fluff{The Great Leader has died, and now a scramble for power begins between his heirs. The noble
who rules your continent has put out a call for all able-bodied persons to arm themselves and
report for muster. You and your friends have only just become of age, and you are blessed to live
in interesting times. An adventure abroad awaits in war, but your grandfather who remembers the
last succession war has words of warning for you before you go: “beware the iron titans, and
should they ask you to become one, do not lose your humanity\dots”}

\begin{center}
-10-
\end{center}
\fluff{All over the world, the oceans are rising. It is the Swell, the time once every ten thousand years
when the oceans rise and swallow all islands but the largest. Administrator-Steward Tault has
prayed to God Union for assistance, but in the meantime they have begat to you and your fellows
great suits of armor to don, so that you may become heralds, travelling all the world to save those
who cannot save themselves\dots}

\begin{center}
-11-
\end{center}
\fluff{You are a Union marine on a peacekeeping mission, a boring assignment on a miserable
backwater mud-world where no one has ever heard of the omninet, much less Union. The
population has been restless lately, as a plague burns through their crowded tenement cities. A
popular figure, Speaker, is rousing the masses, blaming a small minority of Cosmopolitan
missionaries for the plague\dots}

\begin{center}
-12-
\end{center}
\fluff{A tenuous peace has been negotiated between warring factions on a world petitioning for Core
status. An Administrator is on their way, but the talks are beginning to fray between the ancient
houses of nobility on this world. You lead a team of negotiators in the capital who are desperately
working to hold the handshake peace agreement together long enough for the Administrator to
arrive. If all else fails, your role may change from diplomats to bodyguards in the blink of an eye.}


\begin{center}
-13-
\end{center}
\fluff{On a lonely desert world, a colonial survey team has discovered a stone monolith. It is ancient,
pre-dating even Old Humanity. Upon further inspection, the survey team discovers that what they
at first thought were weathering marks are actually the eroded remains of pictographs: a written
language. A Union Science Bureau Far-Field team is dispatched to investigate amid further
reports that the entrance to a subterranean complex has also been discovered, with mummified
human remains in strange space suits collapsed inside\dots}

\begin{center}
-14-
\end{center}
\fluff{In a fertile system crowded with Terran worlds and moons, a world newly unified under the
banner of an ambitious young king celebrates victory. The king’s opponents have retreated to one
of the other worlds in the system: there, they begin construction on great engines and rockets
with which they will direct one of the world’s own moons into the unified world. Little do they
know that the king and his war-minds have plotted their own strike, and that even now plans are
in motion to finish the war once and for all\dots}

\begin{center}
-15-
\end{center}
\fluff{On the glittering surface of a freshwater ocean world, a peaceful nation scattered across
constellations of bucolic islands welcome tired Cosmopolitans from across the galaxy to their
new home. This tropical paradise is where Cosmopolitans often choose to end their lives — new
arrivals land at the world’s only spaceport every day, embarking on slow sail boats across the
warm, shallow oceans to nameless islands where they can live in still peace among tight-knit
communities of like temporality. All is well, until a shadow falls over the world: corpro-state
privateer ships, hungry for freshwater to resupply their empty holds, begin draining the ocean.
The elderly Cosmopolita who have retired to this world cannot fight the privateers on their own,
but they do know a small band of adventurers they can call for help\dots}

\begin{center}
-16-
\end{center}
\fluff{A Union Navy battlegroup on patrol in a proximal system encounters the derelict remains of a pre-
collapse colony ship in orbit around an uncharted world, and a team is dispatched to explore its
bulk. En route to the derelict the world below — thought to be empty due to its lack of any
appreciable artificial signatures — opens fire, destroying the shuttle and crippling a nearby frigate.
The battlegroup scrambles to prepare a response, tapping a your strike team to rescue the crew
of the Slowed frigate in the meantime. And the world below, silent, waits for the first landing
teams to arrive\dots}

\begin{center}
-17-
\end{center}
\fluff{A world under siege by its stellar neighbor has surrendered, but the invading army has not yet
relented — reports crowd the omninet of mass killings, enslavements. Union has decided it will
step in, and is marshaling its forces at the system’s blink gate. The invaders seem undeterred:
already a battlegroup is hurtling towards the system’s blink station in a bid to destroy it and
prevent Union from counterattacking. Union has a small presence on the besieged world: a team
of mechanized cavalry pilots have been fighting a rearguard action to buy evacuating shuttles
more time. With a whole system in the balance, however, their mission might change\dots}

\begin{center}
-18-
\end{center}
\fluff{The parliamentary delegation of a core world is en route to a watershed interstellar conference,
the culmination of a generations-long diplomatic process that will, at long last, create peace in a
cluster of previously warring systems. This should be cause for celebration, but some actors do
not want a unified cluster of systems: as the delegation makes its way to a neutral moon where
the diplomatic conference is set to take place, agents hidden among the diplomats,
parliamentarians, and their retainers, gather to disrupt the meeting, as a cloaked fleet hurtles on
an intercept course towards the delegation convoy. The only people between peace and disaster
is one small team of pilots, outnumbered, outgunned, but not yet out of time\dots}

\begin{center}
-19-
\end{center}
\fluff{Your world  is vast and gold and proud, alone in a sea of night and stars. Until, one day, a strange
silver ship arrived from the pale blue sky, streaming lines of vapor behind. A dark man in a grey
suit, flanked by thin metal golems emerged from the belly of the sky-ship and, with a word, was
whisked away to the Godhead. This burns in your belly like a coal, a jealous fire: how could the
Godhead have picked this sky-man, this dark alien, and not you! Your kin! After a cycle of change,
of the Godhead bowing before this dark alien, of a new idol — “Union” — being raised in its
place, it is time to strike back at these heretics. You and your bond brothers volunteered to join
the sky-man’s armies, trained with their weapons and their armor. And now, invited to a grand
parade at the False Godhead’s city-temple, it is time to free your people, to return your world
back to its true place in the stars\dots}

\begin{center}
-20-
\end{center}
\fluff{Battle plans rattle through your subdermals. The station’s blueprints, flash-memorized to your
short term memory, are fresh. Your chassis is cycled, loaded, and nominal, your Comp/Con’s
voice a reassuring murmur in your aural. Your wingmates, are secure in the lander on your flanks.
You’ve dropped into combat hundreds of times before, so why are you this nervous? RA, the
name a curse and a ghost that you can’t even shake. RA and its demons wait behind those
doors\dots}

--

And so on.

