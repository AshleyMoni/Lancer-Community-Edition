\section{The Session}

Here’s some good basic rules, terms, and things to keep in mind during a typical session:

\begin{center}
     \textbf{Rests and Full Repairs}
\end{center}

Players can take a \textbf{rest} whenever (it takes about an hour) as long as they have the time and
space. During a rest their can repair their mechs by spending repairs, repair destroyed weapons
or systems, and clear all heat and statuses from their mech. Some talents, systems, etc, only
activate on a rest, like the Grease Monkey’s talents.

Players can only \textbf{full repair} by taking ten hours. Imagine a full repair like a total reset - they can
re-build their mech, refresh their repair cap, clear their critical and heat gauges, clear all
conditions on their mech, heal to full, gain all [limited] systems and weapons back, and regain
Core Power.

Full repairs are more under your control, and so access to them will set the pace for your game.
Remember though that the GM agenda is not to punish the players - if they need to full repair
badly, give them a spot they can do it, or else offer Power at a Cost (the downtime action).

\begin{center}
     \textbf{Core Power}
\end{center}                                               

Character’s mechs have a \textbf{core power} section, which is a box they can check. They either have
it or they don’t (you can’t ‘save it up’). All mechs regain core power when they full repair. Core
power can be spent to activate the very powerful CORE systems, which are a ‘one and done’
sort of deal, only typically activated once per mission.

\textbf{If players want more core power you can use it as a reward or grant it to them as a boon in
certain situations.} It’s always up to you as a GM, except when players take a full repair (they
always get it back). Granting players more core power lets them use their CORE systems again
(very powerful abilities), so keep that in mind.
                                    
\begin{center}
     \textbf{Balancing fights}
\end{center}

Generally players should be able to complete \textbf{one or two encounters} before needing to rest and
repair, and \textbf{3-4 encounters} before needing to full repair. This is assuming that encounters are
reasonably challenging for the players, and don’t take this number as a hard, inflexible number.
You should always prepare combats for players with the expectation that things might either go
very well or very badly for them and your plans (or the character’s plans!) might need to change.

Don’t withhold the opportunity to full repair or rest over the idea of verisimilitude.

\begin{center}
     \textbf{Mech Combat Length}
\end{center}

Mech Combat starts when \textbf{hostile action} is taken by any player or non-player character. It’s
played out according to the turn/round based combat rules found in the main section of this
book, and it \textbf{ends} when one or all opposing sides are subdued, surrender, flee, or are completely
destroyed.

If players have overwhelmingly won a combat and there is little remaining threat (for example,
there is only one weak enemy left for four players) it is very possible to simple declare combat
ended and decide the outcome of the remaining enemies narratively.

Certain FRAME systems and modules remain active \textbf{‘until the end of the current scene’}. All this
means is the module remains active until the scene in which they were activated is completely
over. Otherwise, if activated outside of combat (or if you need a narrative timer), FRAME systems
are very taxing on a mech’s power systems, and typically only remain active for about 15-30
minutes.

\begin{center}
\textbf{Leveling Up and Rewarding Players.}
\end{center}

Players should generally level up (get one license level) once per mission, after completing that
mission. You can tweak that however you wish, especially if the mission is very long or odious.

\textbf{By default, LANCER deals with player rewards entirely through the leveling system}. When a
player levels up, it is assumed they have amassed enough currency, reputation, connections, etc
to buy access to the next license level that they get on leveling. Anything else a pilot could buy
or get their hands on, they should generally be able to just buy it outright (no need to track
currency), or else make some pilot skill checks to get their hands on it through graft, negotiate,
connections, or bartering.

However, the following section presents rewards you could give out to players as incentives for
part of a mission, completing certain tasks, or satisfying certain requirements. It’s up to you how
heavily you want to use these in your game or lean on them to hook your players.

\textbf{1. Use Manna}

You can use the Manna system (see the ‘Changing Core Assumptions’ section), which adds a
currency system to the game. You can track manna for items, or even use it to replace the

leveling system, in which case players no longer gain License Points when leveling up, but must
buy them.

\fluff{You check your slate again, not sure you read the glowing number correctly, sure that you added
another zero on accident. No. It’s all there, all those commas and zeros. You’re rich in Manna,
fabulously rich. In the zero-G of realspace travel, your stomach turning is both a physical and
mental thing. You whoop, your cry of joy mingling with the cheers of your squadmates, as their
slates and subdermals ping, notifying them of a successful transfer of funds.

With this Manna, maybe you can finally get Boss Kozta’s goons off your back. Maybe you can
even take back what he took from you. How much did a proper set of STAMPEDE cannons cost
on the Horus-net again?...}

\textbf{2. Grant Reserves}

Grant players pilot gear, vehicles, or other useful material that they can use for \textbf{reserves}. For
example, players acquire a useful vehicle, an enormous drill, blackmail on a politician, insider
information on a rebel general, or a new hardsuit. They might become friendly with the local rebel
group, or the hard-bitten mercenary at the bar, or the socialite who controls the cash flow on the
space station.

\_\_\_\_\_\_\_\_\_\_\_\_\_

\fluff{The Administrator, as she promised, returned. You and your small band greet her at the makeshift
spaceport, an old marble quarry with a rickety scaffold tower overlooking it to sight ships
approaching the recessed landing zone.

“We’ve waited years,” you say, speaking first. You’re decades older now, but the Administrator
doesn’t look a day older than when she left. Your heart, your soul. You think of your children’s
mother, out even now in the timberfields.

“The ship is yours,” the Administrator says. She tosses you her slate. “Access, flight plans,
transponder codes. It’s all on there. The NHP is tuned to you, already. I’ve been teaching it.”

“The ship is mine,” you repeat. A reward, of a kind…}

\_\_\_\_\_\_\_\_\_\_

\fluff{Over your chassis’ omni, a cracking voice.

“That did it! The hardlight wall is down! All units, push forward -- Green squadron, Red squadron,
lay some fire down!”

You lay back in your crash couch, the gimballed cockpit of your chassis adjusting for the move.
You did it. Your squad keys in over the local band, cheering.

You made a breach. Already, over the wide band, the battlescape was alight. Reinforcements
were pouring in through the breach. Somehow, impossibly, the battle had turned in your favor.

“Gold squadron,” the Legion’s level voice.

“Go ahead, Command.”

“Good work, Gold squadron. Report back to the waypoint marked on your HUD. Your job is done
for the day: all scenario probabilities report total success from this point on.”}

\textbf{3. Grant Skill Points}

You can directly grant players pilot skill points to spend on pilot skills (+2 at a time). A player that
has been learning to pilot a starship could easily be rewarded +2 to Get Somewhere Fast after a
mission to represent their diligence and study. Doing so increases player power levels, so use
this reward carefully.

\textbf{4. Reward a Unique or Restricted system or weapon}

Rewarding your players with items that are unique, exotic, or otherwise restricted from their usual
requisition pool is the closest thing in Lancer to magical or wondrous items typically found in
fantasy tabletop RPGs.

The easiest option is to \textbf{reward players with a weapon or system from a license they do not
or cannot have access to}. The weapon or system can only be used for \textbf{one mission} (think of it
like a ‘rental’), then they lose access to it.

Fatigued like you’ve never known, you crash down into your bunk, not even bothering to get all
the way out of your flight suit. You kick your boots off, toss your insulating hood onto the floor of
your cabin. You’ll get it later, firs you need to rest.

\fluff{“Hey flyboy, Cap’s got something for you.”

The crewman’s bark wakes you not minutes later. You sit up, groaning, and see with a start that
the crewman is accompanied by the ship’s XO and head motor pool engineer. You snap a salute,
which they wave off.

“You did good out there. Still more work to do. Motor?” The XO says in his characteristic gruff
voice. The ship’s head engineer steps forward and presses his personal slate into your hand.

“Anything you want, kid. Just learn it first before I have to hoze you out of your cockpit.”

You scroll through the list, previously locked licenses unlocked and waiting your requisition. The
fatigue disappears, replaced only by excitement…}

\textbf{5. Grant \hypertarget{Exotic}{Exotic} Tech}

Drawing up exotic or truly unique systems or weapons is a bit more of a process. We
recommend adapting your \textbf{exotic} system or weapon to the narrative you’re running. We will
eventually include a table of exotic weapon/system types here to get you started; official Lancer
narratives will feature their own exotic weapons and systems.

Exotic tech refers to a particular type of mech system or weapon which is typically unlicensed,
unsanctioned, experimental, or non-human in origin. Due to its nature, exotic tech cannot be re-
printed when a mech is destroyed, and is lost permanently unless the weapon or system itself
can be salvaged.

Exotic tech can be a way for GMs to offer physical rewards to players without directly giving
them more license or talent points.

It follows the following rules and conventions:
\begin{itemize}
\item[--]Installing or uninstalling a system or weapon with the Exotic tag requires you take a full repair
\item[--]\Exotic tech is typically more powerful than comparable tech
\item[--]A weapon or system with the exotic tag \textbf{cannot be re-printed} with your mech should it be destroyed, but must be physically re-acquired
\end{itemize}

Here’s a couple examples of Exotic tech for your use. We’ll include a short table in a future
update. These are not particularly balanced in any way, but might give you a general idea of what
to look for.

\gearBox
[name = {Miniaturized Nuclear Missile},
fluff = {Your mech is equipped with the latest in thermonuclear technology, typically reserved for ship-to-ship combat.},
template = {\Superheavy \Exotic \Launcher \\
Range 50 \\
\Limited{1} \\
\Blast{20} \\
10d6 \explosive damage + 10 heat},
rules = {Mechs caught in a \Blast{40} zone centered on the impact point must pass a systems skill check
with 2 \Difficulty or be immediately shut down. This missile can never be replenished once used.}]

\gearBox
[name = {Living Metal},
fluff = {Your mech has partly biological components of alien origin that automatically crawl over damaged parts of your mech and knit them back together, wire by wire.},
template = {2 SP \\
\Unique, Biological, \Exotic},
rules = {Your repair cap increases by 4. Each round, you may spend 1 repair once to heal as an end-of-round action.}]

\fluff{The Chosen of Aun fell, its golden chassis trailing a greasy pall of smoke from its shattered
cockpit.

You step forward, you chassis moving as an extension of your own form, ceramoferrous plating
ticking and cooling as you vent your chassis’ heat tax. The battle has moved on, ignoring the end
of your desperate, decisive single combat.

“No signs of life,” your NHP whispers in your aural. “I see incredible tachyon bleedout,
ontological stuttering.” She pauses. “There’s something else in there, sir. Be careful. I cannot see
it. Raise your shield.”

You follow her suggestion, hefting your stasis wall.

The shattered Chosen twitches, its tons of ruined machine-mass rattling in death. A light burns
from the belching smoke.

“That is it, there, in the void I cannot see. What is it?” Your NHP whispers.

A steady wind tugs the smoke away, and you see it.

A golden disc, broad and hammered, unadorned. A light like the sun streams from behind it no
matter which way you view it from.

“It… is perfect,” you whisper back. You reach out a delicate manipulator, grab the disc and pull it
towards your chassis. You feel the sudden attunement, the connection. Yours, so long as you
keep it.

But what does it do?...}

\textbf{6. Reward Talent Points}

Talent points can be directly awarded to players (as they are not necessarily locked to level) and
spent as normal. The world of LANCER grants easy explanation for this sudden burst of
instantaneous talent - there are a great number of neurological implants available for purchase
from military and civilian sources.

Granting players increased numbers of talent points can be very powerful, so you should use this
option sparingly.