\section{Running the Game}
RUNNING THE GAME

The following section includes some advice and clarification for helping you actually run the
game. Generally as the GM it’s not actually your responsibility to know all the rules (that’s what
this book is for!) but there are some conceits that can be helpful for you to keep in mind.


                                            The Golden Rule

Here it is again for your convenience: When referring to the rules in this book, specific
statements override general statements. Armor normally reduces all incoming damage, but
certain tags (AP) and certain weapons or mods (paracausal ammo) can go right through it.


                                       A couple good principles

A good principle to follow as a GM is to try as much as you can to play in reaction to player
action. Player rolls do ‘double duty’ for you - they determine if a character is successful but also
give you clues on how to move the story forward. Try to require or ask for rolls in response to
player initiative, rather than straight up asking for certain rolls. This naturally creates stakes and
consequences connected to the player in question.


This can be a little tricky with quiet or less proactive groups, in which case it can be useful to
elicit responses from your group.


Eliciting responses isn’t really as complicated as it sounds. It’s very useful for GMs that have
trouble keeping player attention, or have players that are more hesitant to take action. It can be
very helpful when a game is stalling or stagnating. Here’s a couple things you can do to get a
game moving and elicit action from your players:

            -   Ask questions. A really simple one. Here’s a couple good examples:
                     -   What do you think you’re going to do next?
                     -   How do you/how does your character feel about this?
                     -   Who’s feeling suspicious here?
                     -   What do you think is really going on?
                     -   What’s the way forward from here?
            -   Address characters, not players. For example, address Chandler, the mech
                 pilot, instead of Jeff, the player, when you’re talking to them.
            -    Be descriptive, and try not to describe things in terms of game mechanics first.
                Ask your players if they want to ‘try climbing that cliff’ instead of ‘make a skill
                check to climb that cliff’
            -    Keep things ‘in character’. When players ask NPCs questions or talk to them, try
                to respond as that NPC and not as yourself. Ask players to try and address each
                other as their characters as much as possible. Keeping things ‘in-fiction’ will help
                 keep the game immersive and engaging.


