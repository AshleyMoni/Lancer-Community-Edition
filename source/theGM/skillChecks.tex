\section{Skill Checks}

Skill checks in LANCER at always a 10+ (whether they are pilot or mech skill checks). NPCs
introduce some ways to scale up the difficulty of their skill checks in their profiles, but during the
course of narrative play, you can also tweak skill checks to offer more or less of a challenge.

\textbf{The first and most important question to ask when deciding whether a skill check is
required is whether that check is even necessary at all}. When the outcome is \textit{uncertain,
important, has clear or relevant stakes, or would lead to an interesting situation} (success or fail),
it’s generally correct to make a skill check. If the outcome is not important, let the players
automatically succeed.

This is especially relevant for tasks that you think might be plot important, simple to accomplish,
or might stall the plot. You shouldn’t have players make a pilot skill check to see if they know
plot-important information - just give it to them.

For another example, the players come across a heavy boulder blocking the road. One of the
players decides to use their mech to push the boulder aside. If doing so (or bypassing the
boulder) is not actually that important, pushing the boulder isn’t that dangerous or risky, or it’s
not a particularly big boulder, then don’t even roll - the player just does it. However, it’s probably
easy to think of a number of reasons a skill check might be required in this situation - the boulder
could fall and potentially damage a mech, the boulder could roll aside and cause collateral
damage, or maybe the mech in question is moving the boulder in the middle of an ambush.

\textbf{The second question is} what the purpose or goal of the skill check is. This will help set up
fictional consequences, establish stakes, and make sure you don’t make repeated rolls to
accomplish the same goal. Players only need to make \textbf{one roll} to accomplish their stated, direct
goal.

\textbf{If you need to make a skill check}, most of the time, the roll should be made \textit{without} any
additional accuracy or difficulty required (only that added by the player’s backgrounds, traits, or
talents, etc), just a flat 10+. However:

\begin{itemize}
\item If you want an \textbf{easy} skill check, you could have the player roll with +1 or +2 bonus accuracy. If
you’re going to have players make an easy check, you should really ask if they need to even
make a check at all.

\item If you want a \textbf{harder} than normal skill check, you could have the player roll with +1 or +2
difficulty.

\item If you want a \textbf{very hard} check, you could have the player roll with +2 or +3 more additional
\Difficulty.
\end{itemize}

If the check is so hard that would need to add +4 difficulty or more, just tell the player what
they’re doing isn’t possible with the approach they are taking and offer them a different
approach. Maybe that boulder is too big to move alone - they need some leverage, or for
multiple mechs to move it.

\begin{center}
\textbf{Helping on skill checks}
\end{center}

If one or more players want to help on a skill check outside of combat, grant the player making
the check \textbf{+1 \Accuracy} (regardless of number of players helping). The players helping also share
in the consequences of success or failure.

\begin{center}
\textbf{Failing forward}
\end{center}

Failing forward simply means that narrative should not be predicated on the outcome of skill
checks, but rather pushed forward by them. You can use \textbf{complications} to push the narrative
forward even when players fail.

For example, if your players fail to hack a door, not only do they fail to hack the door, but the
guards are now alerted to the system and are on their way. Or perhaps they instead spy a vent
they can enter - a more dangerous but reliable way of getting to their goal. Perhaps they need to
come back with a piece of specialized equipment that will open the door for sure. They can find
nearby, but it’s guarded. Or perhaps the door DOES open, but onto an entire corridor full of
guards.

\begin{center}
\textbf{Making repeated checks}
\end{center}

If a task can’t be completed in one roll, set a skill challenge (see the GM tool). Never make more
than one skill check for the same goal.

You can stretch rolls narratively as much as you like. If you want to ‘montage’ or speed through a
scene, this is a really useful tool. Climbing an entire mountain could be a process of only one hull
check, for example (if the details aren’t that important). The outcome is what’s important in the
long run.

\textbf{Remember that if players fail a skill check, they can’t repeat it until they change the
narrative circumstances.} If they fail at lifting the boulder, they can’t do it the same way again -
they need a different approach.

\begin{center}
\textbf{Clearly communicate stakes, and commit}
\end{center}

It’s very important to \textbf{clearly communicate what’s at stake} when making skill checks. You can
do this naturalistically.

\quad“Hey Chandler, I see you’re going to use your mech to lift this boulder. Just know, the
boulder is really heavy and dropping it could probably do a whole lot of damage to whatever’s
underneath it.”

\textbf{Allow players to ‘back out’ of rolls} once the consequences are made clear to them. It’s totally
fine for players to change their minds once they see how risky things will be. That way, the roll
feels fair and you can easily renegotiate the roll with your player if they want an easier or less
risky approach.

\textbf{Commit to the consequences of rolls.} If Chandler drops that boulder, you can be damned sure
his mech is taking damage. Consistency is important, and if you’ve already clearly
communicated that he might take damage, then commit to it.

