\hypertarget{GMAgenda}{}
\section{THE GM AGENDA}

What is the role of a Game Master?

A lot of players, RPG fans, and game designers alike will all have differing opinions on what
makes a good game master. FRAME systemly, the golden rule to go by is \textbf{whatever works for
you and your players}. A role playing game is FRAME systemly meant to be a fun social activity -
if you're not having fun, then that's always a cue to try to figure out what isn't working. However,
figuring out the specifics can often be tricky. For this reason, here's some principles to stick by

that we think are applicable to almost all situations. If you try to adhere to these principles we're
of the strong belief it will often improve your game and the storytelling therein.

\textbf{As the Game Master, your job is to facilitate, arbitrate, and make rulings} and to adapt to the
choices your players make; \textbf{your job is \textit{not} to defeat your players}. Think about yourself as the
lead architect, director, writer, and editor. You stand on top of the hill and shove your players off
of it, time and again, and write like mad to make sure they land on their feet.

However, ultimately, \textbf{the story you tell will \textit{not} be the one that you outlined}. Your players will
kill important NPCs before they become important. Your players will not go to that colony that
has that important data log. Your players may not bite the hook you want them to bite. Your
player characters, with their backstories that tie perfectly (or good enough!) into your campaign,
might die, forcing re-writes.

All that is ok. All that is part of playing Lancer, as it is any roleplaying game worth its price.

\textbf{As the game master, you should try to say ``no'' as little as possible}. There will obviously be
situations in which the rules, your judgement, or common sense dictate that a player cannot
accomplish the impossible. But in most situations it's almost always better to say ``Yes, and…'',
``Yes, but….'', or ``Yes, however…''. Rather than outright denial, give players a different option,
offer them a weaker outcome, give them another (maybe more difficult) way to accomplish their
goal, or let them attempt it anyway (even if it's nearly impossible). Most of the time the outcome
will be same, but by turning the choice back to the player it becomes both empowering and
rewarding to players and keeps the story moving.

\textbf{As the GM, you should try to make sure everyone at the table gets a chance to be the hero},
that everyone at the table gets the opportunity to feel important and contribute in a way that they
want. Your players might want to smash and grab. Other players might play quiet but for rolling
into combat situations to test their build. Still other players might do their best work in no-
combat sessions where contracts, treaties, and court intrigue is negotiated.

As a GM, all of that is ok. Your job is to balance the needs and desires of your party with the
story that you want to tell with them.

At its heart,\textbf{ \textit{Lancer} is a collaborative storytelling game}. You should want your players
laughing, crying, serious, and silly. You should want them doodling their characters on someone
else's turn, or ordering takeout to eat over their character sheets. \textbf{As GM, you're \textit{not} the reason
your players show up}: they show up for their characters and your world. Without players to take
on the role of protagonist in your drama, you have no game, no story.

So, be kind to the players. Be fair. Be flexible. But be firm when you need to be. Sometimes, a
roll of a 1 is a roll of a 1, and even if it blows up your story, it cannot be changed.

Your role? You sit at the head of the table, you write the world, but you lead alongside your
players: remember, this is their story as much as it is yours. What follows in this guide is an
outline, of sorts, of the aspects of a campaign or session you as a Lancer GM should be
concerned with. What does the setting look like? How do your players get around the world or
the galaxy? Who are the actors in your story? How do the lights stay on?

As GM, your role is to be prepared, and to keep the game moving when your players need you
to.

It's not as easy as it sounds, but that's what this guide is for.