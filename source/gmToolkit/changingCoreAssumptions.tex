\section{Changing Core Assumptions}
                             CHANGING CORE ASSUMPTIONS

The mechanics of LANCER assume a couple of things that might not be present in your
campaign. If you want to tweak these things it’s entirely up to you. The following tools can help
you change around some core conceits of the game.


                           PILOTS DON’T HAVE ACCESS TO A PRINTER


It’s assumed that pilots generally have access to a printer to create new mechs. This might not
be the case in your campaign or even your setting, however! Maybe players are outlaws or
renegades, with limited access to resources. Maybe the printer in their area is broken or
damaged. Maybe they are operating on the fringes of civilization, where any kind of technology is
hard to come by.





Printing a mech allows a player to get back in the game, so to speak, when their mech is
destroyed. Remember that players can repair and rebuild their mech completely, as long as it is
at least partly intact and they have access to it, whenever they take a full repair, regardless of
whether they have a printer or not.


If pilots don’t have a printer and their mech is destroyed (or they can’t access the mech), use the
Power at a Cost tool at the beginning of this section (the goal: I want to rebuild my mech) to get
access to people, materials, a workshop, etc where mechs can be manually built or repaired.
Building a mech can also be a downtime activity (see the section above).


                                          DEATH IS MORE LIKELY


Here’s an optional rule you can use if you want to slightly tweak LANCER’s default ‘heroic death’
rules:

             -    If you take more than your maximum HP in a single hit (after armor) as a pilot,
                 you’re dead, no matter what.

             -    If you’re dead, that’s it. No cloning or revivification.

             -    If you take two points of structure in a single hit, your mech is destroyed, no
                  matter what.

Make sure you know what kind of game your players and you are playing before adding this sort
of rule in.


                                    I WANT TO SIMULATE CURRENCY

LANCER does away with currency management like in other RPGs in favor of tying everything to
the leveling system. If your players want to buy something, they can just buy it (unless it’s
expensive or rare, then do some role playing or use Power at a Cost). It’s assumed pilots are
still paid (in manna, currency etc), you just don’t track it.

If you don’t like that system, want something more granular, or want something to replace the
License Level system, you can track Manna instead. Maybe your pilots don’t have benefactors or
access to a market where they can freely buy mech licenses, for example.


                                                     MANNA

Manna is a universal currency in the canon of LANCER promoted by Union to integrate client
states and regulate business, in common use in certain parts of the galaxy.


Manna is represented by a capital M preceding the denomination, like so: M1, M2, M3, M100,
M500, and so on. Manna is a digital currency, though it has been localized in some areas as a
physical currency. There are also fractions of M1: M.75, M.50, M.25, M.10, and M.05.


Here’s what certain things typically cost at average purchasing power in Manna:





M.01: A cup of black coffee. Beans were grown in zero-g so it doesn’t taste the best.

M.05: A beer. Probably artificial but the spacers like it that way.

M.10: A decent, hot meal.

M.25: A night’s stay in a station capsule, pretty damn cramped and noisy

M.50: Standard bribe to gatesec

M1: Ticket into an exclusive offworlder nightclub

M2.5: Assault rifle, lightly used, sights are slightly crooked

M10: Personal kinetic shielding, generally reliable

M100: A military grade hard suit

M1000: A one-seater starship

M10,000: A full starship, crew of 5

M1,000,000: A freighter or warship

Anything higher than M1000 is usually difficult to get your hands on.


                                       USING MANNA TO LEVEL

If you want to set a cost on mech parts or licenses, you can set a manna cost instead of using
license level for certain licenses. Doing so effectively changes the leveling system to be based on
manna, so keep that in mind.


To rent (use) a piece of equipment from a license for one mission costs M100 for rank I, 300
for rank II, and 900 for rank III.

To buy a piece of equipment costs M250 for rank I, 500 for rank II, and 1000 for rank III.


If you rent a piece of equipment, it’s gone after one mission. If you buy a piece of equipment, it’s
not re-printed if your mech is destroyed. Renting or buying a weapon doesn’t level up a player
(they don’t get the FRAME unlocks). You can’t rent or buy a mech FRAME, you only get them by
permanently unlocking them (as if you’d leveled up normally).


To permanently unlock a rank of a license, it costs 1500 (no matter the rank). If you
permanently unlock a license, you level up (using the same leveling rules, getting 1 core point, 1
talent point, and possible targeting bonuses, system points, or core mounts). You get access to
all the gear from that license permanently. You can re-print anything you’ve permanently
unlocked. Permanently unlocking a license is the same as ‘buying’ a mech so if players want to
‘buy’ a mech FRAME, tell them it’s going to cost about 1500 to get the rank I license to access it.


Manna rewards could vary per mission, but if you want to keep the same leveling pace, you
should award players about 1500 mana per mission (with more or less at your discretion).

