\section{Faction Tracker}
                                       FACTION TRACKER
There are many factions in the world of LANCER, many of which are outlined in the official canon
below. You may, during the course of your game, find it relevant to keep track of factions in your
game (or may run entire games revolving around factions). If you want to codify things a bit, you
can use this tool.

Factions can be tracked simply by Power and Hold. Power is the wealth, force, and influence of
the faction, simply put. Hold is how resilient that faction is (strong, normal, or weak), how well it
holds on to that position.

Power has the following ranks (it’s not linear), as well as some examples.

            Power      Scale             Examples

            -3         Sub-local        A gang, small militia, or militant group

            -1         Local            A small colony, a huge bandit gang, a small military

            0          State            A ruler, warlord, or king; a pirate lord, a mercenary company,
                                        a large colony, a reaver pilgrimage

            1          Planetary        A unified planetary government, a god-king, a small
                                         planetary-state, a pirate haven

            3          System           A major shipping company, a trade collective, a minor corpo-
                                        state

            5          Multiple         A major corpo-state such as Harrison Armory, a Karrakin
                       Systems          trade barony, Aun Ecumene, a pre-collapse civilization

            10         Galaxy            Union

            15         Metaphysical      RA

The players are probably a faction with power -3 to -1.




If a faction undertakes a major project that does not bring them into direct conflict with another
faction, such as exploration, mining, trade, construction, research, expansion, etc roll 2d6+ that
faction’s power to see how it goes. On double 1s, the action fails no matter what. On a total result
of 2-6, the action is still in progress, or a failure. On a 7-9 the action is successful, but might take
more time or resources than expected. On a 10+, the action is flatly successful.
If two factions perform an action that would bring them into conflict with each other (such
as a war), each rolls 2d6 and adds their power. On double 1s, the faction loses no matter what,
otherwise the faction with the higher result scores victory (however that is defined). This doesn’t
have to be direct conflict but could be a trade war, bidding contest, bid for influence, race for
resources, etc.

If a faction has strong hold, they roll 3d6 and pick the highest. If a faction has weak hold, they roll
3d6 and pick the lowest. If a faction has normal hold, they roll as normal. Hold depends on how
well-entrenched a faction is. A government such as Union or a planetary government usually has
strong hold. A collapsing state, chaotic bandit gang, or unorganized military fleet has weak hold.
Any other faction has normal hold. Any faction that goes to war immediately goes to weak hold. If
a faction is insurgent (a rebellion, secret operation, etc), they always have strong hold, but lose
that hold if they rise to power 1 or above.

You can use this tool to check how well factions undertake certain actions to provide a sense of a
living world for the players, or even allow the players to influence the outcome of events. If a
faction has some major advantage or disadvantage that the players grant them, you could give
them strong or weak hold, depending on the player’s actions. It might be possible for factions at
weak hold, if they lose a conflict, to go down a rung on the power ranking (from planetary to state,
for example).

You might find it useful to track the faction’s attitudes towards the players as the story progresses.

