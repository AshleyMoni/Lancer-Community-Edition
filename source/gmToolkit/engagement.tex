\chapter{Engagement}
                                         ENGAGEMENT


If you want to mix things up in a mission where the starting situation on the ground is unclear (a
hot drop, an invasion, a foray into enemy territory), you can use the engagement rule.





Engagement happens right before the Boots on the Ground step of mission. Make an
engagement roll where everyone can see. This is a simple d6 roll. Roll it and consult the
following chart to establish what the situation is like the moment players get there.


 D6                                                    Starting Situation

 6                                                     Situation normal, no complications other than
                                                       expected
 4-5                                                   Minor complications or unwelcome surprises

 2-3                                                   Major complications or unwelcome surprises

 1                                                     Situation FUBAR

Engagement cuts out unnecessary planning or stalling and cuts right to when the players arrive
on the scene. When we make an engagement roll, we immediately establish a situation and put
the players in that situation, ready to take action and respond.


This doesn’t have to throw the players right into combat (and probably shouldn’t the majority of
the time). For an example, let’s say the players have embarked on a mission to escort a refugee
caravan through a heavily guarded checkpoint manned by local partisans. The GM decides the
moment players get boots on the ground is when they meet up with the caravan outside of the
checkpoint. Based on the engagement roll, it could go the following ways.


6 - No major issues, the caravan is unmolested and ready to move

4-5 - The caravan is far larger than the players initially expected. It will move slowly and become
hard to guard.

2-3 - The caravan is delayed and the players will have to track it down or wait under threat of
bandit attack

1 - The caravan is under direct bandit attack the moment players arrive on the scene.


                                    Changing the engagement roll


The engagement roll can be increased by adding extra d6s (and choose the highest). It can also
be decreased by subtracting dice. If the total pool is 0 or lower, roll two d6s and choose the
lowest.


Check the chart below for ideas on how to modify the roll. This is mostly qualitative, based on
the nature of the mission, but if the roll’s going to be adjusted, it should be fairly obvious for both
the players and the GM (it can’t be arbitrarily changed).


Most of the time engagement should just be a straight roll (1d6).

                                         Engagement modifiers





 Situation                                                                             Effect

 The mission is in an exceptionally safe or stable location                            +1d6

 The mission is in a notably unstable, dangerous, or distant location                  -1d6

 The characters have good scouting, information, or details about the                  +1d6
 mission
 The characters have exceptionally poor information about the mission                  -1d6

 Powerful forces are contesting or helping the players on their mission                -1d6/+1d6

 The mission is routine                                                                +1d6

 The mission is an emergency, impromptu, or rushed                                     -1d6

