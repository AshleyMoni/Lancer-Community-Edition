\section{IPS-N Nelson}

                                              IPS-N NELSON

The IPS-N NELSON brings the close-quarters doctrine espoused by ISP-N to its most pure form. The
NELSON is built to brawl in environments too volatile for firearms or when ordnance has been exhausted.

With its functional size, the NELSON can attack fast while remaining a difficult target to track. Layers of
fractal-fold BULWARK plating allows for ceramic-analogous carbon flaking, effectively nulling the impact of
incoming solid-state fire by dispersing kinetic energy across a rounded hull. This null-k plating protects the

pilot from impact trauma, allowing for sustained combat efficacy in high-trade scenarios.

The NELSON is an iconic IPS-N chassis, known across the galaxy as the FRAME of choice for the

Albatross, the Cosmopolitan interstellar anti-piracy agency. Their distinctive white, gold, and red livery and
mastery of the war pike -- as well as seeming agelessness due to time dilation -- has won both the
Albatross and the NELSON a venerated place in Diasporan lore -- and secured the Albatross an

endorsement contract with IPS-N in perpetuity.

                                                     License:

I. War Pike, Bulwark Mods

II. NELSON FRAME, Thermal Charge, Armor Lock System

III. Power knuckles, RAMJET


                                                    NELSON

  HP: 8           Evasion: 10                            Speed: 5            Heat Cap: 6        Sensors: 10

  Armor: 0        E-Defense: 8                           Size: 1             Repair Cap: 5      Tech Attack:
                                                                                                +0

                                                     TRAITS:

  Momentum: After making the boost action, the next melee attack from the Nelson deals +1d6 bonus
  damage on hit

  Skirmisher: After making any attack, the Nelson can move 1 in any direction. This movement doesn’t
  provoke reactions, ignores engagement, and doesn’t count against its movement for the turn. It can’t
  make this move if immobilized or slowed.

                                               SYSTEM POINTS: 6

                                                    MOUNTS:

  Flex Mount                          Main/Aux Mount

                                                  CORE system




                                             Perpetual Momentum Drive

  IPS-N’s PMD exploits fighter-tier nearlight spooling to conserve and sustain a passive .000001LS
  charge, able to be dumped into extant boost systems at the pilot’s command. The chassis fielding this
  system must be heavily adapted through strengthening joints, limbs, and installing a k-comp crash
  couch to protect the pilot from sudden g force and shear.

  Active (requires 1 Core Power): Spool up PMD
   Protocol
   Once activated, this system remains active until the rest of the current scene. While its active, the free
   movement from the Nelson’s Skirmisher trait increases to 4.

War Pike

A War Pike is a simple weapon. A long haft, topped with a dense, slim point, meant to puncture armor.
Derivative of a mining pylon, the modern war pike is a sturdy, balanced, and reliable weapon, perfect for a
charge.

Main Melee

Thrown 5, Threat 3, Knockback 1

1d6 kinetic damage


Bulwark Mods

A mark of pride for IPS-N, all proprietary mech cores feature IPS-N’s QuickMod system, a modular, legacy-
compatible system of joints, hardpoints, and internal slots that make installing upgrades simple.

1 SP
Your mech has extended or armored arms or legs, redundant motor systems, or is otherwise
reinforced for harsh terrain. Your mech ignores difficult terrain.


Armor Lock System

IPS-N’s Armor Lock System is a total-body modification for a mech core that provides additional chassis
stability when pilots are faced with a situation that puts their core under greater-than-anticipated stress.

1 SP, Unique

2 heat (self)
When you take the Brace reaction, you can activate this system. Until the end of your following
turn, enemy attacks targeting you are made with 1 additional Difficulty, you can’t fail agility or hull
checks, be knocked back, grappled, knocked prone, or moved by any external force smaller
than size 5. You end any grapples currently affecting you.


Thermal Charge

Pilots have long made this popular modification to their pikes. Now, IPS-N is offering these pilots’
modifications as a licensed and quality-tested suite for pan-galactic printing. A pike modified with a charge

-- often colloquially called a “Fire Pike”-- is a simple plasma projector integrated into war pike, tuned to
project a plasma sheath over the pike’s head.




2 SP
Mod

Limited (3)
This mod can only be applied to a melee weapon. As a free action when you hit with any melee
attack, you may spend a charge of this system to activate the shaped charge on your weapon
and deal +1d6 bonus explosive damage.


Power Knuckles

A simple weapon system, IPS-N’s power knuckles are a popular modification for pilots of CQB mech cores.
Whether as shaped studs, hyperdense knuckles, or a series of magnetically-accelerated micro-rams, power

knuckles amplify the already incredible hitting power of a mech core.

Auxiliary Melee

Threat 1

1d3+1 explosive damage

On a Critical Hit, your target must pass a hull check or be knocked prone


RAMJET
Air. Air and momentum. There’s a threshold that veteran Nelson pilots know well, the Point of Endless
Momentum. When you get moving fast enough, in the right atmosphere, the air itself feeds into auxiliary
ports on the chassis, compressing, howling out like a demon’s angry scream. The Point of Endless
Momentum is a giant’s hand on your chest and a god’s chariot under your feet and you feel like you can
outrun light itself and there’s nothing else like it.

3 SP, Unique

Protocol

2 heat (self)

Until the start of your next turn, your mech gains +2 speed when boosting and its melee attacks
(including rams, grapples, etc) gain knock back +2. However, your mech must move its
maximum speed each time it moves and can only move in straight lines (it can stop if it would
collide with an obstacle or enemy, and it can change direction between movements).

