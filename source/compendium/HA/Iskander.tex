\subsection{HA Iskander}

\begin{mech}{HA}{Iskander}

\fluff{The ISKANDER is a specialist‘s chassis, designed to provide area-denial and breach capability to squads in which it is a member of. A large chassis, the ISKANDER commonly sports weapons meant to ensure dominance in all close-quarters situations, as well as increased blast shielding to protect its pilot from deadly concussive forces.}

\begin{license}
\item Stub cannon, Repulsor Field
\item \textbf{ISKANDER FRAME}, Assault Launcher, Gravity Gun
\item Sticky Bombs, Grounding Charge
\end{license}

\frameBox
[hp = 8,
evasion = 8,
speed = 3,
heat cap = 7,
sensors = 15,
armor = 1,
e-defense = 10,
size = 2,
repair cap = 4,
tech attack = 1,
traits = {\textbf{Mine Deployers:} When the ISKANDER uses a quick action to plant a mine, it can plant up to 2 other mines in free adjacent spaces as a free action.

\textbf{Skeleton Key:} The ISKANDER never activates mines or other systems that activate by proximity unless it chooses to (allied or enemy).},
sp = 6,
mount one = flex mount,
mount two = heavy mount,
core system name = Broad-Sweep Seeder,
core system text = {The ISKANDER’s broad-sweep area-denial/countermeasure Seeder is a HA proprietary system developed during the Orrugi Occupation, where HA Acquisition Legionnaires encountered an embittered, recalcitrant local guerilla movement. IEDs, VBEDs, and D/SABEDs were common weapons employed by the local resistance: to counter this threat, the Armory developed a triple-use system to scan, ID, and eliminate explosive threats in proximity to stationary units. This system proved successful and, with minor adaptation, was tuned for use on Armory chassis.

The Broad-Sweep Seeder projects a hivecone of excited LIDAR that flags potential targets: mag-accelerated, dull-coat flechettes then disable that target. This system may also load explosive-backed hivemines.},
core active name = Death Cloud,
core active text = {Action,

As an action, your mech fires an enormous expanding cloud of micro-mines across the whole battlefield. Any mech of your choice on the whole battlefield (or a roughly 50x50x50 area), that voluntarily moves more than 1 space takes 3 AP explosive damage. If a mech boosts, it takes 3 AP explosive damage immediately. Characters are aware of the presence of the mines. They last until the end of 3 rounds, counting this one, then deactivate.

You do not activate mines from this system.}]

\gearBox
[name = {Stub Cannon},
fluff={A supercompact rotary pistol, short range, but able to be integrated into hardpoints or held in a manipulator.},
template = {\Auxiliary \Cannon \\
\Limited{6}, \Knockback{1}\\
\Range{5}\\
3 \explosive damage}]

\gearBox
[name={Repulsor Field},
template={1 SP, \Unique\\
\QuickAction},
rules={You can use this system as a quick action to emit a \Burst{2} pulse around your mech. Targets caught in the area (allied or enemy) must pass a hull check or be knocked back directly away from you 1 space. The pulse then detonates any mines or explosives caught in the field}
]

\gearBox
[name = {Assault Launcher},
fluff = {Assault launchers are universal launchers. Ammunition is loaded first into a comparable sabot, then electromagnetically accelerated either directly or indirectly towards its target. The sabot shatters upon firing, releasing the projectile to perform as designed at range far greater than factory limits.},
template={2 SP},
rules={You can fire and deploy mines and grenades up to any point in \Range{15} and line of sight instead of their regular range.}
]

\gearBox
[name={Gravity Gun},
fluff={The complex negotiations of gravity and time, shattered in an instant by a machine that can pluck a waves like a player strums a guitar string.\\
We’ve weaponized the thing that holds all things in its embrace. What could go wrong?},
template={\Heavy \Rifle\\
\Range{8}, \Blast{3}},
rules={When you attack with this rifle, all targets caught in the area must pass a hull check or take 1d6 \energy damage and be pulled as close to the center of the blast as possible.}]

\gearBox
[name ={Sticky Bomb Launcher},
fluff={Sticky bombs attach to ferrous metals by means of burnout electromagnetic generators, triggered in proximity after firing or manually by the user.},
template={\Main \Launcher\\
\Arcing\\
\Range{15}},
rules = {To fire this weapon, you can attack a point on the environment within range without rolling, or make a regular ranged attack roll against a target within range. On a hit, the target or area does not take damage, but instead has a sticky bomb attached. It takes a quick action and a successful engineering check to remove all sticky bombs from an area or mech.\\
As a free action, you can detonate all sticky bombs fired by this weapon to deal 1d6 \explosive damage in \Burst{1} area centered on all targeted mechs or areas. Mechs caught in the blast area can pass an agility check to halve this damage, but mechs ‘stuck’ by this weapon fail their check automatically.}]

\gearBox
[name = {Grounding Charge},
fluff={Grounding charges take the pulse/wave principle of the thumper and applies a second component: gravitic generation. When triggered, the initial pulse wave acts similarly to the Thumper, but immediately after the wave dissipates, the grounding charge triggers a gravity well that pulls all destabilized materiel towards it. A potent anti-positional weapon, grounding charges are commonly used to disrupt prepared positions and pull enemies from cover.},
template = {2 SP, \Limited{1}\\
\Mine},
rules = {This charge can also be detonated with a \QuickAction. Once detonated, targets in a \Burst{6} area centered on the charge must make a successful hull check or be knocked prone and pulled as far as possible towards the charge as though they moved normally (starting with the closest target). The charge also pulls any flying mechs or vehicles within \Range{6} above the area that fail the check to the ground, taking damage as if they fell.}]

\end{mech}