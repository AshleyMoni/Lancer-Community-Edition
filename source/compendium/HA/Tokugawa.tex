\subsection{HA Tokugawa}
\begin{mech}{HA}{Tokugawa}

\fluff{HA’s TOKUGAWA chassis is a relative newcomer on the market, popular in core systems for security and CQB/ breach applications. The TOKUGAWA is a large, imposing mech, a sturdy platform from which the recommended kit can draw the necessary power it needs in order to perform within optimum parameters.}

\begin{license}
\item External batteries, Annihilator
\item \textbf{TOKUGAWA FRAME}, Experimental heat sink, Plasma Lash
\item Torch, AMATERASU class NHP
\end{license}

\frameBox
[hp = 8,
evasion = 8,
speed = 4,
heat cap =8,
sensors = 10,
armor = 1,
e-defense = 6,
size = 1,
repair cap = 4,
tech attack = -1,
traits = {\textbf{Reactor Flare:} The Tokugawa’s energy weapon attacks deal +1d6 bonus damage if it has 2 or less reactor stress remaining

\textbf{Plasma Sheathe:} While the Tokugawa is in the danger zone, it’s energy weapon attacks deal all bonus damage as Burn},
sp = 6,
mount one = flex mount,
mount two = main mount,
mount three = main mount,
core system name = Superheated Reactor Feed,
core system text = {A certain breed of pilot rides the very edge of catastrophe, swinging between an equal chance of success and failure each moment. Tokugawa pilots are familiar with the howl of their chassis’s heat warning, the warbling siren a song of destruction: with a superheated reactor feed, Tok pilots ramp their heat debt to the max in order to supercharge their energy weapons. This allows them to churn out damage and make no friends in the engineering bay, should they not melt into a ball of slag before they make it back from the line.},
core active name = Radiance,
core active text = {\Protocol

Choose 1 energy weapon your mech is wielding. If it is a ranged weapon, its range increases by 5, if it is a melee weapon, its threat increases by +1. For the rest of this combat, this weapon also deals +1d6 Burn damage (roll on each attack). However, each time you fire this weapon, you gain +3 heat.}]

\gearBox
[name = {External Batteries},
fluff = {External batteries are by no means a Harrison Armory exclusive, but HA literature will ensure you that HA-Brand POWERALL cells are the longest lasting, fastest cycling and highest capacity solid state cells available. A consequence of their high capacity is a proportionate increase in volatility if the system should ever be damaged, but pilots looking to utilize HA technology agree through continued use to absolve HA from all liability.},
template = {2 SP, \Unique}
rules = {Your ranged weapons that deal \energy damage gain +5 range, and your melee weapons that deal \energy damage gain +1 threat. If you take structure damage, this system explodes and is destroyed, dealing 3 AP \explosive damage to your mech. This damage can’t be prevented in any way.}]

\gearBox
[name = {Annihilator},
fluff = {HA specializes in conventional and unconventional arms development; solutions to tactical problems are designed both in the lab and in the field, often the latter outperforming the former in combat situations. The Annihilator takes its name from pilots’ slang for a field-rigged weapon developed during the Bradbury Rebellion, when desperate resistance pilots machined a way of shunting the incredible waste heat of their core’s reactor into a directed blast.},
template={\Main \CQB\\
\AP, \HeatSelf{2}\\
\Range{5}, \Threat{3}
1d3+1 \energy damage},
rules = {This weapon creates an energy pulse for \Burst{1} around its target’s location on successful hit. Actors caught in the area must pass an engineering check or take 1d3+1 AP \energy damage.}]

\gearBox
[name = {Experimental Heat Sink},
fluff = {The Harrison Armory DEEP WELL system is a part of their VANGUARD line of equipment available to licensed HA beta testers. Though a complicated and delicate weave of heat exchangers, Harrison Armor’s DEEP WELL system attempts to recycle the heat generated by a chassis’ systems into useable energy. While the system works well, the delicate nature of the exchange renders the DEEP WELL highly volatile.},
template = {4 SP, \Unique},
rules = {While your mech is in the Danger Zone, it has resistance to heat.}]

\gearBox
[name = {Plasma Lash},
fluff = {A threat at multiple ranges, the Lash was designed following the Armory’s review of data collected from the Hercynian Crisis. After-action reports noted the efficacy of thermal and thermobaric weapons against hardened targets, but made constant reference to the limitations placed upon conventional thermal weapons: the need for atmosphere-specific fuel the impact of prevailing weather conditions, the necessity of atmosphere.\\
Enter the Plasma Lash, an early development into plasma weaponry. Firing an excited toroid of plasma, the Lash can be used as a ranged weapon with a delayed-detonation, or, essentially, a point-blank melee weapon; if used at point blank range, the toroid is detonated inside of its launcher, directing an intense thermal blast directly at its target.},
template = {\Main \Melee or \Main \CQB\\
\HeatSelf{2}\\
\Range{5}, \Threat{3},
1 \energy Damage + \Burn{3}},
rules = {This weapon can be used with either profile, but not both in the same round.}]

\gearBox
[name = {Torch},
fluff = {The Armory’s TORCH is a backbone core weapon: a heavy, two-handed, dual crescent-bladed plasma torch. The melee weapon is powered by its wielder’s reactor, connected by both powerlines and inert cabling; it can be separated into two torch-axes, and its plasma blades are capable of being tuned into new shapes. A common sight in CQB situations, the torch has of late become a status symbol among pilot officers, many preferring to carry them alongside a smaller auxiliary weapon.},
template = {\Main \Melee\\
\HeatSelf{1}\\
\Threat{1}\\
1d6 \energy damage + \Burn{2}}]

\gearBox
[name ={AMATERASU-class NHP},
fluff = {AMATERASU came to prominence in the Armory NHP think tank after its repeated victories in war thought-games. AMATERASU is characterized by its brash, enthusiastic personality, often expressing frustration with timid pilots who have it in their employ; however, this bombastic personality hides a calculating, brilliant tactical mind that feeds information to pilots often faster then they can process it. AMATERASU’s combat doctrine demands action and impetus, a chaotic blend of reckless maneuvering and aggressive offense that keeps defenders beleaguered and unable to respond with any great efficacy. Pilots willing to partner with AMATERASU should be aware that this attack style often leaves their cores vulnerable to counterattack, and that this NHP enjoys what it calls “good-natured ribbing”.},
template = {3 SP, \Unique\\
\AI},
rules = {Your mech gains the AI property and the AMATERASU protocol\\
AMATERASU protocol\\
\Protocol\\
\HeatSelf{1d3+3}\\
Increase the bonus damage on hit of your next ranged or melee attack this turn by your current heat after activating this protocol. The chosen weapon must deal at least partly \energy damage.}]
\end{mech}