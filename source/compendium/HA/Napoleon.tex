\subsection{HA Napoleon}

\begin{mech}{HA}{Napoleon}

\fluff{Perhaps in a tongue-in-cheek nod to its namesake, the NAPOLEON is a squat silhouette when fielded next to other Harrison Armory chassis. But packed into its compact frame are marvels of Armory engineering, technology that demands the NAPOLEON be piloted only by the best and the brightest. Stasis technology is the very cutting edge of gravitic manipulation technology, only now hitting the commercial market for those with the requisite licenses. The NAPOLEON incorporates a mix of gravitic manipulation technology, proven anti-kinetic/energy shielding, and superpositional force multiplication to dominate enemies -- earning its namesake through battlefield success as well as stature.}

\begin{license}
\item Phasing Weapon, Stasis Barrier
\item \textbf{NAPOLEON FRAME}, Stasis Mine, Dispersal Shield
\item H.A. Blackshield, Displacer
\end{license}

\frameBox
[hp = 6,
evasion = 8,
speed = 4,
heat cap = 8,
sensors = 5,
armor = 2,
e-defense = 8,
size = 1/2,
repair cap = 3,
tech attack = 0,
traits = {\textbf{Well shielded:} If the NAPOLEON would take half damage from an effect (weapon, system, explosion, etc) on a successful check of any kind, it instead takes 0 damage.

\textbf{Flash Aegis:} When the NAPOLEON Braces, it reduces damage to 0 instead of gaining resistance},
sp = 7,
mount one = main/aux mount,
core system name = HA Vantablack Aegis,
core system text = {The Armory’s VANTAblack AEGIS system is a breakthrough in personal shielding developed by HA’s Think Tank. In line with other NHP-derived technologies, the VANTAblack is a so-called “black-box” technology; pilots with requisition power to obtain a VANTAblack system are typically of high rank or standing within HA, and the inner workings of the system are not known to the public at large. On outward appearance, the VANTAblack system has been described by Cosmopolitan pilots as similar to the void/blindness one has when looking out at blinkspace; it is safe to assume that the system utilizes unstable blinkfield technology to manifest a thin blinkspace bubble within defined parameters around the system -- the blinkfield can only sustain for a brief moment, but can be flickered to create an essential-total blinkspace dome.},
core active name = Activate Aegis,
core active text ={\QuickAction

A shimmering, utterly black field envelops your mech, covering it like a second skin, and taking only a few moments to activate. While this field is active, your mech reduces all damage from any source to 1 (after armor as normal), though it can still be grabbed, knocked back, pinned, thrown, and affected by other mechs.

While this shield is active, your mech can only move and make the grapple, improvised attack, ram, and boost actions. It cannot take free actions, reactions, or overcharge, and cannot use any systems or benefit from Flight (if it’s already flying, it falls). It cannot make or be the target of any tech actions (including lock on, etc), cannot benefit from beneficial tech actions (such as lock on, bolster, etc) and cannot communicate or receive communications with or from anyone except the GM (though hand signals are still possible).

It can still be affected by statuses, grappled, and take heat. It can otherwise interact normally with the world, such as picking up or dragging items, etc.

The shield lasts until the end of the current challenge, or about ten minutes otherwise.

Damage that goes through reduction (such as paracausal ammo) can still harm the Napoleon while its aegis is up.},
]

\gearBox
[name = {Phasing weapon},
fluff= {Phase-Ready ammunition, as first described after its incorporation into the civil hostilities on Luna de Oro, is the “devil‘s round“: each round contains a nanoprocessor suite networked with its firing weapon that, ideally, calculates and translates the specific nature of that round‘s superpositional relation with its doppelgänger in the immediate space before its intended target. To wit, Phase Ready ammunition, when fired, exists in two places at once: exiting the barrel of the weapon it was fired from, and at the moment of impact into its target. The prime round may never hit its target, but as it already exists at the moment of impact, its doppelgänger round will hit its target. The fuzzy nature of such spooky action occurs in a way not fully understood save for in the faltering explanations of Harrison Armory‘s NHP Think Tank; as such, the action is not perfect, but falls within acceptable parameters for licensed production.},
template ={2 SP\\
\Mod},
rules = {Choose 1 weapon. This weapon can totally ignore line of sight and cover as long as you roughly know your target’s location when you attack, but your target counts as having invisibility if you attack this way. This weapon can attack through solid walls or obstacles, as long as its target’s location is known and they are in range.}]

\gearBox
[name = {Stasis Barrier},
fluff = {Stasis Barriers are the result of Harrison Armory‘s interest in gravitic manipulation and superpositional negotiation. Contained within a solid-state generator/projector, a Stasis Barrier is a deplorable wall of antigravity, contained by its power supply, that interdicts and denies most all incoming kinetic and energy-based weaponry. Another of HA‘s NHP Think Tank development, the Stasis Barrier is now a mainstay of the Armory‘s personal and materiel defense line and a common enough sight on all Armory Depot/Development worlds. By manipulating local gravitational forces, the Barrier rejects projectiles and energy lances, denying particles and waves both on a molecular level; matter that impacts a Stasis Barrier simply ceases to exist, save for anomalous fluctuations that cause some projectiles to break through. As the Barrier is technology from the Armory‘s Think Tank line, some of its fuzzy nature is not fully understood, but rest assured failsafes have been installed to force a regular cessation of projection to ensure the device remains operating within established safe parameters.},
template = {2 SP, \Limited{1},\\
\Shield, \Deployable},
rules = {This module deploys as a 4 space long piece of size 2 cover that lasts until the end of the current challenge. While behind the barrier, a target counts as having heavy cover and has resistance to all damage from blast, line, and cone attacks. At the end of the challenge, it deactivates and is used up. The cover itself is immune to all damage.}]

\gearBox
[name = {Stasis Mine},
fluff = {Stasis Mines developed by the Armory‘s NHP Think Tank, are portable, unit-specific versions of Stasis Barriers. Initially pegged as a potential personal shielding device, early tests proved that stasis is as-yet detrimental to the individual inside a projected field. Think Tank suggests that the cognitive hazards of sudden and total pause of temporal/gravitic/positional existence without preparation -- however long the stasis session lasts -- is irrevocably traumatic.},
template = {1 SP,  \Limited{1}\\
\Mine},
rules = {You can detonate this mine with a quick action once planted (or it activates normally). Once detonated, this mine creates a \Burst{4} area around it. Affected targets may make an agility check with 1 difficulty to escape if on the edge, otherwise they are trapped inside. The area inside is locked from the normal flow of space time, creating an impermeable barrier around its edge. Effects, mechs, and pilots inside are stunned and removed from play until the end of next round, and all other effects cannot penetrate into the area. Time does not flow normally for targets inside the area (it stops completely), and is separate to the outside world. Active effects, attacks, modules, and other individuals and actions inside the area pause. At the end of the the next round (after all characters have acted), this area returns and resumes play as normal.}
]

\gearBox
[name = {Dispersal Shield},
fluff = {Dispersal Shielding is a milder form of stasis projection that manipulates only gravity, adjusting the perceived mass of its user so that projectiles and excited particles bend and warp around and through them. Hostile fire does not quite “miss“ so much as they undergo atomic shuffling, disincorporating on the atomic level so that they pass through their targets without colliding.},
template = {3 SP, \Unique\\
\Reaction, \Shield},
rules={1/round you can force any attack that misses you to be re-rolled against a target of your choice within your attacker’s range (even a target allied to them).}]

\gearBox
[name = {Harrison Armory Blackshield},
fluff = {The Armory Blackshield leans into the fuzzy nature of quantum manipulation characteristic of Think Tank research and development. The Blackshield operates in similar fashion to blinkspace gates, generating a pulse of spherical energy that allows its operator to pierce perceived space/time and exist, for a moment, in the null-environment of blinkspace. Blinkspace, described by early test pilots and their NHP companions, is a void, a space outside of human perception that it at once infinite and without form, blank and cacophony. NHPs that accompanied those first pilots have since been retired, their handlers citing recursive ontological tail-chasing and paracausal obsession; since then, NHP protocols have been updated to include a sense-exposure doctrine, allowing them to do as corporeal, sapient pilots do and simply accept the unreality of blinkspace without going mad. Think Tank NHP‘s and their counterpart engineers acknowledge the tactical benefits of (non)momentary (non)existence in blinkspace, but they caution pilots against repeated exposure without sufficient pre-and-post exposure conditioning and counseling.},
template = {2 SP,\\
\Shield, \Unique\\
\FullAction\\
\HeatSelf{4}},
rules = {As a full action, this system can be activated to generate a \Burst{4} area centered on user. While active, the flow of time is altered drastically in a small sliver of space in a bubble around the user. Nothing, not even light, can enter or exit the shield. It is impermeable and invulnerable. When the shield is activated, mechs caught on the edge must make an agility check to choose which side they end up on, otherwise the user chooses. To those inside the shield, the world outside the shield goes totally black and the inverse happens from outside. No action or effect can enter or exit the shield while it is active or draw line of sight though (even those that normally ignore it), though time passes normally on both sides. The shield drops automatically at the end of the user’s next turn.}]

\gearBox
[name = {Displacer},
fluff = {The Displacer is the result of ongoing blinkspace exposure tests in Think Tank‘s R\&D department and miniaturization of commonly employed interstellar travel methods. The Displacer itself is conventional in appearance but requires a massive secondary, dorsal-mounted core in order to power: when fired, the Displacer identifies a bubble of local space (size and location determined by the firing pilot) and snaps it into blinkspace. Where the contents of that bubble go is unknown, but the effect is dramatic: anything inside the projected bubble simply ceases to exist in this dimension, transported somewhere else in the void of blinkspace. The Displacer makes no sound when fired, but the sudden and necessary venting of its power supply is tremendous; similarly, the heat wave of its backblast is deadly to any unshielded personnel exposed to it.},
template ={\Main \Rifle\\
\Unique, \Loading, \AP, \HeatSelf{10}\\
\Range{10}, \Blast{1}\\
10 \energy damage}]
\end{mech}