\section{Cygnus Spur, Orion Arm: Humanity Across the Stars}

Cygnus Spur, Orion Arm: Humanity  
Across the Stars  
\subsection{Cosmopolitans and Diasporans}
Cosmopolitans and Diasporans  

In the narrative present of Lancer, humanity is vast and polyglot. We live in a golden age  
for most: the human race is spread out across habitable and uninhabitable worlds, enjoys the  
fruits of robust scientific, political, and cultural advancements, and has access to such a gross  
amount of resources that society on Core worlds is, essentially, post-scarcity. 
 

Life on a given developed Core world, is stable, safe, and without want. People who make their  
homes on terrestrial worlds, large orbital stations, and habitable moons are Diasporan humans --   
their homes are in a specific location on a specific world or station, and their lives are concerned  
with events that occur around, in, or to that location.  
 

However, outside of the developed Core worlds there is still a raw, dangerous edge to human life.  
The frontier waits both away from the bright core of the galaxy, and towards it. Colonies need to  
be built, settlements developed, and old ghosts put to sleep.
 

The humans who embark on missions or pilgrimages to these frontiers are called Cosmopolitans.  
Their homes are, typically, their ships or the world they left behind. 
 

Interstellar travel comes at a cost: time. Life for a Cosmopolitan human is split into subjective  
and real time, or, time as a Cosmopolitan experiences it, and time as the rest of the galaxy  
experiences it. 
 

Cosmopolitans leave their old lives behind as the effects of relativistic interstellar travel splits  
them from the “real” time of the rest of the galaxy. They trade the permanence and normality of a  
terrestrial life for the vast life, the uprooted life, a life lived in the wind. 
 

On Core worlds, humanity expresses itself in many faiths, cultural practices, genders, and social  
structures. We create art, shape the land, build glittering cities, construct great works of  
engineering. We write, we cook, we drink, we play sports, we journey, we wander. We populate  
our arm of the galaxy as a roiling people, often contradictory, often myopic, but ever-learning,  
ever-growing.  
 

Humanity is no less fragile and no more disposable now that we have crossed the stars and  
number in the trillions: the great purpose of human life on the galactic scale is to defend the  

                                                                                                              


collective while defining the individual. To journey beyond, to discover ourselves, to not simply  
exist but to become. 
 

It is a good thing, some of Union’s philosophers think, that we are alone in this great venture. 
 

The galaxy that humanity has stepped again into is, seemingly, empty. There are few truths held  
by the majority of humanity -- our numbers are simply too vast -- but the ones that are universal  
are fundamental, seen as core tenets of what it means to be human: 
 

First, to be a human being in Union is to be afforded the decency of a life lived with your basic  
needs seen to: your state will make available to you food, water, shelter, and labor, and will never  
deny you those rights. To do so is to violate the most basic of social contracts. 
 

Second, no walls shall stand between worlds. The void of interstellar space is deep and cold,  
utterly hostile to life. Any civilian world, station, or moon not deemed to be of significant private  
or restricted interest must allow all who petition for access to feel firm ground beneath their feet,  
breathe clean air, and feel again the light of a life-giving sun. 
 

Finally, no human shall be held in bondage through force, labor, or debt. Scarcity of natural  
resources on a Core world is a false premise, a myth, a tool used by the few to oppress the many  
while enriching themselves; this the same for colony worlds. The dignity of human life is  
paramount among the Diaspora, and to use hard or soft power to exploit a person, their labor,  
and deny them just compensation is abhorrent.  
 

All that being said, humanity exists on a spectrum of development. While the majority of the  
developed galaxy might hold these to be self-evident truths, guaranteeing them is never a  
“finished” project — “most” of the developed galaxy only represents a plurality, not a majority, of  
the populated galaxy. 
 

Remember: power never gives up power. Power is only ever taken from the powerful and  
redistributed to the people, where it must constantly be cultivated, regulated, and maintained —  
this is the dream some worlds have realized, and the project that Union, humanity’s core  
organization, works to accomplish. 
 

Union is many things, but above anything else it is an incomplete project, human to its core.  
Despite every wonder of technology, every miracle at their disposal, Union still needs people to  
make the right decisions, to be brave, and to make it work. 
 
\subsection{Why we Fight}
Why We Fight  

Notice the hedging -- “usually”, “most”, “the majority”, “by and large” -- this is to mark the gaps  
in the golden age, the places where the galaxy does not conform to Union’s standards of  
development.
 

                                                                                                           


War still plagues humanity. No system so large and so varied will be completely harmonious, but  
Union and its proxies are working to make it so. Note again that “harmonious” in this context  
means that which is agreeable to Union -- it is absolutely possible that your PCs will reject  
Union’s hegemony.  
 

Areas of major conflict, interest, importance, or intrigue in Lancer are called flashpoints. Some  
are detailed in this core rulebook under the section A Golden Age, of A Kind . 
 

In Lancer, humans fight over two things: territory rights and ideology. 
 

Resource-rich and strategically placed terrestrial worlds and moons -- habitable or not -- are  
common regional flashpoints. The rise of destabilizing actors like Corpro-States has led to a run  
on planets rich with rare elements necessary to fabricate the fantastic technologies the galaxy  
has come to view as normal. This produces conflict between corpro-states. Some hire  
mercenaries to accompany their claims teams, others raise and field their own standing armies. 
 

Ideology tends to be the seed of conflict that drives states to fight states, or for states to fracture  
and fight civil wars. These conflicts are protracted and bloody, as neither side is likely to  
surrender unless they give in to attrition. 
 

States and Corpro-States engage one another over territory and labor. Most of these conflicts  
occur in small engagements over important persons, locations, and facilities as small teams of  
highly trained agents fight expensive corpo-state mercenaries in clandestine firefights.  
Occasionally, a Copro-State’s acquisition of a world will turn hostile, and conflict will rage across  
and above it as the state attempts to secure its holdings. 
 
\subsection{Pilots}
Pilots  

A note on the player character role, and how pilots become Lancers.  
 

Player pilots in Lancer generally fill a common archetype and tend to be aware of their status --  
in-setting, as a “Lancer”, the galaxy’s colloquial term for a role similar to a flying ace --  but how  
they respond to that status and act in their role differs from pilot to pilot.
 

It is perfectly possible -- and narratively interesting -- to not play as a Lancer, but as a run-of-the- 
mill pilot, one without the blessing of destiny or the distinction of being the Special One. 
 

While there are gradations of skill among pilots, generally speaking, a trained and outfitted mech  
pilot in their chassis will beat a conscript or unskilled pilot in a mech, even in situations where the  
trained pilot might be outnumbered. 
 

A mech pilot in Lancer, outside of their mech, is usually a well-trained combatant, adept-to- 
experts in the fields their backgrounds cover. They are not super-soldiers, though some with  
heavy cybernetic enhancement might be able to operate outside the bounds of “normal” human  
ability. Most pilots -- as a consequence of being a human in the narrative present -- have some  

                                                                                                           


degree of non-invasive biological enhancement, but rely heavily on wearable/detachable  
technologies and months-to-years of training to augment their talents and knowledge. 
 

A Union-rated mech pilot, one that is a Regular or Auxiliary trooper, is generally educated by the  
Navy, specialized in their role through training, traveled some (through their world or local  
system), and is at least in their mid 20s -- it is rare to see a pilot younger than that due to the  
necessary qualifications for training. All Union Regular and Auxiliary pilots are put through the  
same naturalization courses, trained by Union regulars and instructors, and serve in integrated  
units for a period of ten years.
 

For pilots not affiliated with Union as Regulars or Auxiliary troopers, standards are different.  
Corpro States might tune their pilots via implants or prosthetics to be more compatible with their  
own chassis lines; the Aun raise their pilots alongside their NHP-analogs who will pilot their  
chassis with them; the Karrakin Baronies outfit their sons and daughters with legacy chassis and  
offer them to martial academies for training; Ungrateful guerillas take what they can get, and so  
on.  
 

The majority of pilots in Lancer are simply individuals trained and outfitted to crew their  
machines. Your pilot -- or your player(s)’s pilot -- might be the special one, or the chosen one, or  
otherwise blessed with some combination of fate, talent, and destiny, or they might not. Maybe  
they’re just the person in the pilot seat, in the right place at the right time.   
\subsection{Pilots, Lancers, and Union}
Pilots, Lancers, and Union  
   
What sets your pilot apart from the rest of the mech pilots in the galaxy? What makes them a  
Lancer?  

It is important to note that the term “Lancer(s)”, in Lancer, is a catch-all term for mech pilots  
across the galaxy who are roughly equivalent, by Union standards, to a contemporary ace. Not all  
pilots or people who crew mechs are Lancers -- the majority of them are just competent pilots --  
but all Lancers are pilots.   

Lancers tend to be set apart from “regular” pilots by ability, talent, training, luck, skill, reputation, or  
some combination of these factors: they do not achieve any classification or gain any medals,  
certifications, or rank that qualifies them as a Lancer, it’s simply a term that exceptional pilots  
achieve through a combination of time in the saddle and performance on campaign.   

There is an argument that some formal identification system exists at levels so far removed from  
the players as to be a secret that would not only never be uncovered, but not even assumed as a  
possibility. A rumor exists among pilots: that Lancers are chosen for their role, their names noted  
down in some distant database deep under Cradle’s surface. Some pilots -- the mystic, the  
superstitious, the paranoid -- believe that Union has intervened in their lives or the lives of others  
in order to make them Lancers.  

For the rest, it is because of their skill, strength, and experience that they have earned the right to  
call themselves Lancer. No help from Union, no preordination from down on high. They became a  

                                                                                                                      


pilot when they were conscripted, or when they enlisted, or when the ships came and bombed  
their homes, and they chose to fight back. Pilots -- Lancers -- are made, yes, but whether by deep  
machinations, by fate, by chance, or by choice, who can say.   

It’s important to note: pilots and lancers both have downtime. Combat, exercises in the field, and  
moments of stress and fear generally do not occupy the majority of their time.    

So what does this look like in practice? Some examples follow  

Union Regulars: A Cog In The Greatest Machine  

A squad of Union regular troops, tight-knit on campaign together, laughing and joking as they run  
post-sortie checks on their chassis. They are one squad of hundreds on the deck of a carrier --  
their home for only a few more years -- each member from a different world, bound together now  
by a greater purpose.   

Union doesn’t work without these pilots. A whole galaxy of bickering states, of kings and  
presidents who think they can shake the pillars of heaven. Who think they can spit in the face of  
Union. Might as well turn their back on the ocean. When they do, these are the people who  
darken their skies, who drop to their world, who set them right where they’ve done wrong.   

Nothing sets this campaign apart for them, Regulars on a second tour. An Administrator’s report,  
filed to the DoJ. A battlegroup organized. A briefing to see: a king who thinks himself a god, who  
grinds his boot on the neck of a people he thinks are his. He forgot they’re not his people, they’re  
Union. Time’s come for this king to get a reminder that Cradle’s reach is long, its vision is pure,  
and its will inexorable.    

As veterans of this campaign by now, these pilots have been given a little bit more autonomy than  
the replacements: their chassis bear more kill hashes, carry heavier weapons than boot squads  
fresh to the deck. They’re still subordinate to their CO, but all the other grunts look to them when  
the lead and lasers start flying: they’ve been in the shit a bit longer, learned how to move when  
someone downrange wants to kill them.   

They are Lancers, veterans of a campaign that has ground all of their friends to dust and ash.  
From different worlds, of different backgrounds, but all bound by the same truth:   

They’re Union, and they’re here to help.   

Yond-Balor, Second of House Yond, Heir-By-Blade  

In the Baronies, glory is given only to those who prove they deserve it. Yond-Balor Karrakis,  
second son of Yond-Aleph Karrakis, Baron of House Yond, sub-barony of the House of Glass, has  
proven through combat and intrigue that he alone deserves to command the warhost of the House  
of Glass. But one obstacle remained: Yond-Argo Karrakis, Yond-Balor’s older brother, first  
claimant to the throne and, by extension, the warhost of the House of Glass.    

                                                                                                           


As a youth, Yond-Balor was raised to crew the chassis, to gird himself in sealed power armors  
and become the death-word of House Yond. He proved himself in battle against the Ungratefuls,  
against House Muur and House Fleur, even in single combat against the Vice-Lord of the House  
of Sand. Yond-Balor alone stood by his father’s deathbed while Yond-Aleph fevered, poisoned by  
assassins sent by the Mutable Houses.   

And where was Yond-Argo? Where was the First name’d? Gone. Travelling through space,  
pledged to those headless ones: to Union. Yond-Balor wasted no time in dispatching his older --  
far older -- brother when he came to pay respects to his late father.    

Yond-Balor has taken what is his by birth, drawn it steaming from the gut of his worthless kin. As  
the chassis closes around him and the red light of the ready-light begins to pulse, he grins. His  
name is no longer second-son -- it is Lancer, first and only.   

Jeddah, Loyal Wing   

Young-Wing Jeddah bin Surat al Noor has long greyed while maintaining his Loyal Wing’s kit.  
After his salvation by the Albatross, Jeddah wanted nothing more than to don his own kit, to be  
marked by the crimson lion, to feel the smooth hilt of his very own waveblade in his hand.   

How many times had he cleaned blood and burn from his Loyal Wing’s hardsuit? How many times  
had he nursed his Loyal-Wing while he lay in agony, moaning through nightmares that gripped  
him since the Metavault?   

Countless times. Jeddah may not have seen the terrors that Loyal Wings face, but he has seen  
their echoes. Never did it dissuade him, for he remembered: these are the only ones who came to  
save him.   

Now, on his eighteenth subjective year, he has been called before the Loyal Council for  
assessment and ordination. His teacher has picked him, of all his fellows, to learn the art of  
piloting, the art of standing-at-the-speartip. His breast swells with pride when he hears his name  
called.   

Rising, he walks across the dawn-lit courtyard to stand beside the rest of the Loyal Council. In  
time, he too will heft the lance, he too will wear the golden armor, will wield the laser-and-rifle.   

He will be a Lancer, hero to the lost.    

Penny, In At The Dawn  

Penny threaded the last stitch of her MIRRORSMOKE MC patch on the shoulder of her coat. She  
her coat out, looked at it, nodded. She let it go. Microgravity hold it for a moment while she stood,  
slipped her arms in and zipped it up.   

                                                                                                             


The Penny in the mirror was kind, if a little tired, but who wouldn’t be after ten years in hard- 
sleep?  

Still, she smiled. A little bit of pride in her situation wasn’t a terrible thing to have, was it? The  
MSMC 501st Detachment. The One-Eyed Fox. The “Here-For-Nows”   

Subcommander Gerrard’s voice echoed over the carrier’s all-comm.   

“Sunrise. It’s sunrise. Good morning everyone, welcome to the Dawnline Shore.” There was a  
pause, the sound of someone talking off-mic. Gerrard came back a moment later. “Since Geordie  
won our calendar-pick, we’re going to listen to some old shit. Apologies in advance. We’ve got  
three days until we’re in orbit above Myrrh.”  

The old shit started right after, echoing over the PA. Penny liked it. Ray-gae, it was called. She  
hummed along as she tapped her fabrication request into the ship’s queue. For all the rep these  
men had, they were actually quite kind. Gruff, unpolished. But kind, and bound to a kind of honor.  
At least, the Foxes were.   

Penny didn’t know it, but she would be the only one to make it out of this milkrun; she was a cut  
above the rest, though why she wouldn’t find out until later.   

Penny, you see, would be called Lancer. But that wouldn’t matter much to her: she only wanted to  
be called friend.   

MJ Martinez, A Kid From Old Spinrock  

“You ain’t nothin’”   

Last words the decklord ever spake to MJ Martinez. You ain’t nothin.   

MJ unclipped his helm. Tugged it off. Felt the breeze cool the sweat from his brow. Free now he  
was from the constant chatter of the battlescape. Was only cheers and cheers anyways: MJ and  
his company had done the impossible: held the breach, repelled the Crown Host.  

Whennow did the decklord die? Must’ve been a hunnert or twain years back. MJ grinned, couldn’t  
help. AIN’T NOTHIN was slapped in whitewash on the flank of his chassis, scored some by  
tachyon but proud and loud.  

Now MJ had been through the shit. Left Old Spinrock and joined with the Oxes, cadet’d for a  
decade until he proved his mettle enough to crew an Everest. Ain’t nothin’ in terms of what you  
can grow to crew, but for a kid from the hub of a spinsat, an Everest is the King Death.   

A decade more and MJ got himself a squad. A decade more, a comp’ny. The No-how’s. Sigil: a  
grin, one missin’ tooth. Mirrored MJ’s. You get to do that when you’re a real damn hero.   

                                                                                                            


A real damn hero. MJ sure as shit wasn’t gonna paint that on his core, but it did sound nice. A call  
drew his attention -- more Union regulars trooping by, waving up at him as he stood half-out his  
Everest. He waved back, saluted, they cheered.    

Those old decklord words echoed: you ain’t nothin’.   

Well, what had their chances been against the oncomin’ horde? Nothin’. Stood strong and Union- 
proud ‘gainst the shinin’ host of a king who called himself god. Nothin’ stopped him before.   

Perfect, if nothin’ was gonna save the world, then he was the right pilot for the job. A nobody in a  
nothin’-rig who saved the world.   

MJ laughed. Not bad for a nobody from Old Spinrock. A Lancer, he was. Claimed the title on his  
own.   

Tyrannocleave    

It began in the mines, as everything in the Baronies does. You ask “where” and “why” - I will tell  
you.   

Do not protest, Grand Baron. Do not weep. It is unbecoming of your station, filth.     

Listen now: we have nothing but time, so I shall take my time. The first violence I commit to you is  
to tell you my story. You shall know before you die.   

My mother died when she bore me. Hooked to tubes and deadlife machines. Your men had her  
bear me with the assistance of an exo. A machine massaged her heart. Current forced her  
muscles to work. A black box strapped to her chest forced breath into her lungs — because you  
knew. You knew at that point the only protest we had was to die, and deny you new bodies.  

So you took me from her and I never knew her again — I found the footage of my own birth later, I  
wanted to see — and in the care of your stone-matrons I learned how to tear worlds to pieces. I  
learned the lash, and the pick, and how to read the earth you had me kill.   

When I was ten I learned that I must take these pills to survive. That I was born riddled with  
cancers that you would never take from me — too expensive. The recovery process too long. And  
your quarterly profits could never slip, that would mean ruin for your name.   

Look at me. It is why we are hideous. Your propaganda paints us as beasts — outward we are,  
but it is your hand that shapes our flesh. It is your word that scars our skin and bids cancer grow  
thick in our bodies. In the name of Manna and your House, you ruin millions.   

Millions. There are millions of us, and countless more. This is your unbecoming. Your death,  
below your feet. Every inch of every palace and ship and grand city you build, you build on our  
backs, with our labor, at the cost of our lives. It began in the mines, but will not end there: your  
Baronies are as riddled with us as my body is with metastatic.   

                                                                                                           


Yes, weep now. For your perfect face. Your perfect worlds. Your perfect dominion.   

Did you know — I did not know what the “sky” was until the Ungratefuls found me? Liberation  
means this. Not simply seeing the sky, but knowing that there is such a thing as land without  
stone above you. And then knowing that there are those who put you there. I always thought —  
we were always ministered this — that we were damned and penitent. That we had transgressed  
Lordgod’s perfect kingdom and must mine in penance.   

How wrong I was! It was not a divine prescript that doomed us to labor, but you, and others like  
you. How surmountable the problem became then.  

This is the Ungratefuls’ gift to me. This is their gift to the rest of us, the waste, the offal of your  
courts and palaces. This is why even your machines fight alongside us.    

Because you never taught us there could be a thing like the sky. Because you thought that we  
would never learn.   

Goodbye, Grand Baron. Your House burns, your coffers have been emptied, your monuments  
have been torn down. Your line, your sons and daughters, hang from the balconies of your own  
palace.   

We learned, Grand Baron. We learned how to hope, and who to hate.  

And so on.   
\subsection{A Lifetime of Experience}
A Lifetime of Experience  

Generally speaking, the natural life of a pilot is only marginally longer than that of a given  
Diasporan. Their observed life tends to be closer to that of a Cosmopolitan, as a good number of  
pilots tend to fall in this group of humanity. The subjective experience a pilot has of their life,  
though, is no different from you or me. 
 

Pilots are valuable hard power resources for their states, corpro-states, and groups -- they get  
shipped all around their home system, home world, or the galaxy as needed for the duration of  
their deployment. 
 

The average Diasporan from a developed world with no augments or significant bioengineering  
lives somewhere in the ballpark of 120 real years. 
 

Subjective age doesn’t match up with real age: it’s perfectly possible for someone’s real age to  
be 300, and their subjective age to be 30. Remember, “subjective age” is how old a person  
appears to an observer and how old they perceive themselves to be; “real age” is how old a  
person is in Cradle-Standard years, tracked by Union. 
 

                                                                                                           


Union’s registration system counts real age as time progressing as an observer from Cradle  
would perceive it. Therefore, a Cosmopolitan’s real age may increase significantly, depending on  
the length of time that passes on Cradle while they are engaging in interstellar travel.  
 

In Lancer, a pilot only lives once (with some rare, and rarely talked about exceptions). Facsimile  
programs do exist, but these are digital simulacra, moving portraits and holograms that give the  
appearance of the person they represent, but are not free-thinking digital consciousnesses.  
Death, it would still seem, is an inescapable end for humanity -- hence Union’s discomfort in  
stagnation. 
 
\subsection{Mechanized Cavalry}
Mechanized Cavalry  

Why mechs? 
 

The rise of mechanized cavalry can be attributed to two factors: rapid human expansion into  
space, and the conflicts that stem from expansion. 
 

Union’s first three thousand years of expansion and colonization occurred without the benefit of  
the blink network or knowledge of how to pierce into blinkspace, as those breakthroughs would  
not happen until after the Deimos Event. Drones and unmanned interstellar vehicles were used to  
scout on hundred-year increments: Fired off towards target worlds to be followed by early- 
pattern nearlight ships not yet built, crewed by Far-Field teams not yet born, one hundred years  
after the drone had arrived.
 

In its 2nd millenium, Union’s expansionist imperative demanded that humans spread out among  
the stars. Old colonies and installations waited to be reactivated by human hands, and Union  
marked the growing diaspora as both a point of civilizational pride and necessary for the survival  
of the species. 
 

However, humans do not survive on hostile worlds or in hard vacuum. Ships and stations did fine  
to protect people from hard vacuum, radiation, and all the terrors of deep space and dangerous  
atmospheres, but planets needed to be claimed, not flown over.
 

To address this problem, Union astrocartographers and Far-Field administrators issued a call for  
a standardized, medium-term livable suit pattern that could be used for tactical, scientific, and  
civilian purposes in space and on alien worlds. Universal, powered, hardened against the  
elements, and comfortable to wear, the hardsuit soon became an indispensable piece of  
hardware for anyone leaving the bounds of Cradle. 
 

The first hardsuits were adaptable, universally compatible, with flourishes and specializations  
unique to their manufacturers and the demands of their users. Early Far-Field teams wore larger  
suits with more robust equipment, some equipped with weapons to protect themselves from  
native flora and fauna; Colony populations first adopted hardsuits as personal emergency  
equipment in the case of damage to their sealed habitats, then as a common method of travel  
and exploration outside of their settlements; Spacers wore slimmer, sealed suits, often living in  

                                                                                                           


them for days on end while they piloted their great ships on long voyages between populated  
worlds. 
 

The suit found military applications as well, as some companies and Union foundries began to  
add plated armor on their suits, wove ballistic knit into them, and integrated hardened  
technologies that allowed soldiers to manipulate smart weaponry.    
 

On worlds where tracked or wheeled terrestrial vehicles proved insufficient, larger hardsuits were  
built, capable of hauling cargo that would have otherwise required transport trucks. In these  
suits, the pilot occupied a cockpit, not just the suit, and extensive training was required to ensure  
the pilot could operate the mechanized chassis professionally. These heavy suits were commonly  
accompanied by drone flights and operated in tandem with other heavy suits.  
 

The first mechanized chassis and the first pilot were born from this combination of exploration  
drive and protection from the elements. It took an acute moment -- a flashpoint -- to catapult the  
mechanized chassis from a useful-if-plodding civilian platform to a deadly military instrument. 
 

That flashpoint presented itself on Hercynia, a jungle world in distal space, around 4500U.    
 

Hercynia was a lush, massive tropical-to-temperate world rich with nitrogen and oxygen. It was  
dominated by continent-spanning tropical forests around its equatorial and temperate zones,  
only giving way to borean plains in the northern and southern poles. It seemed, on all scans, to  
be a perfect colony world: rich in resources, breathable atmosphere, a temperature range that  
was pleasing to most humans, and carbon-based flora. Hercynia was a Gaia world, perfectly  
suited for human life.
 

The contract was posted, a consortium of colony firms won, and a joint colonial expedition was  
undertaken. 
 

Initial colonial sites were established in early 4500. Within months, colonial scientists pinged the  
Union Science Bureau with urgent calls for assistance: alien, indigenous structures had been  
discovered by colonial survey drones sent to plan out future development. Shortly thereafter,  
contact was made by colonial elements sent to explore the structures. 
 

Humanity had encountered the first conscious, sapient alien species in their history. The  
Egregorians, so-named due to their autonomous/consensus co-consciousness, were outwardly  
horrifying in appearance, but largely peaceful and able to communicate. Colonial dispatches of  
the time indicate that the Egregorians were reverent of the new arrivals, regarding them and their  
                                        1 
technology as magic, as godlike  .   

Union assembled a team of xenobiologists -- an established field due to discoveries of alien flora  
and fauna on many colony worlds -- and linguists, anthropologists, and engineers to head to  
Hercynia and investigate. Upon landing and contact, integration into Union structures began. The  

1 Note: the term used did not indicate a “god” as humans would conceive of one, but rather an analogous-god as the  

Egregorians conceptualized of it: a divine simultaneous sensation/perception/consciousness is a succinct, if not  
perfect, shorthand. For brevity’s sake: “god” or “the divine” or some variant therein. 

                                                                                                                     


private colonies were Unionized, their charter companies compensated, and Hercynia was  
walled off: no public omninet, no public blink access, and credible-source dissemination through  
interested channels of a total colony collapse due to disease. 
 

Hercynia became a black site. A hole in space. Union’s next great project began: integrating and  
naturalizing the Egregorian many-peoples into human social structure. 
 

This ended poorly. Refer to No Room For A Wallflower for a detailed history.
 

The resulting conflict prompted massive research and development into combat-capable and  
effective mechanized chassis platforms across all theaters. Mechanized Chassis became a  
viable, all-round option for combat in all theaters -- on hard terrain, in zero-g, and in transitional  
spaces, a piloted mechanized chassis, a mech, could outperform and outmaneuver any other  
ground-based all-theater attack option.   
\subsection{Infantry and other Ground Forces}
Infantry and other Ground Forces  

Infantry, tanks, groundcars, light vehicles, trucks, etc, all still exist and are far more widely used  
than mechanized chassis in combat. 
 

Mechs are used much in the same way that cavalry was used in ancient combat: rapid, heavy,  
armored, deadly, and expensive to recruit, train, and maintain. Mechs can break down, are  
vulnerable to systemic attacks, are hard to camouflage, are susceptible to human-portable  
weapons and traps, and require significant time and resource investment to build. Licenses, save  
for GMS, must be acquired and certified, renewed on occasion. 
 

Not everyone can become a pilot, but all a person needs to fight is a reason.   
 

Mechs are shock units of a category above tracked and wheeled weapons platforms. Infantry  
make up the bulk of all individual units in an army, and, ultimately, are still the only way that  
states and state-like entities can take and hold territory. 
 

Additionally, infantry are far less expensive and more expendable than a mech and its pilot. 
 

Furthermore, while printers are relatively ubiquitous, not everyone has the licenses, resources, or  
time to field a printer large enough to fabricate mechanized chassis.
 

A soldier on a given developed world usually is a professional fighter serving a term of service,  
either a volunteer or someone serving due to compulsory state requirements (a mandatory  
service period, a lottery-based service period, or so on). This typical soldier has undergone a  
lengthy period of basic physical and mental training meant to condition them into being a  
temporary member of the military class, followed by a shorter period of specialized training  
based on their assignments, and now is posted to a base, unit, or patrol. 
 

Most military mech pilots begin with this training. 
 

                                                                                                            


This typical soldier is outfitted with a main battle weapon, possibly a sidearm, ammunition (if their  
military uses weapons that require it), equipment and gear appropriate to their specialization, a  
uniform, and basic personal armor to protect them from small arms fire, shrapnel, blades, and  
blunt force trauma. This typical soldier has been assigned to a unit of similarly outfitted soldiers,  
given a rank, and has a directive -- punishable by some compelling form of hard or soft power --  
to obey their superior and all other superiors. Some of these soldiers may carry more specialized  
equipment commensurate with specialized training that they received -- shaped charges, a  
longer-range omninode, a heavier battle weapon, a drone swarm and its control unit, a CQB or  
area-denial weapon, and so on. 
 

Variations on this galactic catch-all professional soldier exist. Some worlds are more developed  
than others, and some are less; similarly, some put stock in their militaries, and some prefer to  
spend their resources elsewhere. 
 

More militarized societies might simply have more soldiers, or better equipped soldiers, or  
enforce conscription, or have a caste system or other form of ordering their society around  
martial orders. Note that “more militarized” does not necessarily mean more technologically  
advanced: it is perfectly possible in Lancer to encounter a society utterly devoted to a military  
hierarchy whose soldiers proudly polish steel pikes and have never encountered black powder,  
much less a mechanized chassis. 
 

Also note that -- like in other areas of society -- military technology does not necessarily have to  
be uniform across a society. Some states may have a limited number of mechs, won hundreds of  
years before in a trade with a passing Cosmopolitan ship, but their local industry is only able to  
develop black powder muskets for their infantry; others may have left ranged weapons by the  
wayside in favor of shimmering blades and mirrored shields, riding into battle on hovering skiffs  
over massed formations of pikemen.
 

In short: while mechs are the focus of Lancer, infantry still form the backbone of most every  
organized army in the galaxy. Expect to encounter them.   
 
\subsection{Space combat}
Space Combat  

Space combat between fleets is elegant at a distance and brutal up close.
 

Against the stark black of deep space, long silhouettes drift in tightening gyres, maneuvering to  
dodge torpedos and kinetic kill-clouds thousands of kilometers distant. Energy beams, invisible  
to the naked eye, streak across the void, shimmering only where they impact their targets. 
 

To an observer, this combat between fleets-of-line is silent, sterile: Long capital ships appear to  
twirl thousands of miles apart, closing slow as their orbits align. Clouds of glittering metal chaff  
and slag bloom into the darkness, catching the light of distant stars. The blue torches of  
torpedoes trace fading lines in the night. 
 

                                                                                                           


However, to those engaged, there is no elegance. No grace. There is just the long, persistent  
terror of space combat:
 

Days before they can even see their opponents through optics, the first torpedoes, kill-clouds,  
spoofers, shrouds, and subaltern kinetics have been launched. Pilots, officers, and  
crewmembers are roused from stasis and ordered to battle stations. Massive kinetic and energy  
weapons, one-shots, begin their acceleration cycles, spooling up for their perfect shots. 
 

The flagship’s XO NHP goes live, paired with the commanding officer but given free rein to  
partition and duplicate themselves into sub-sentient subjectivities to better advise and  
coordinate all of their ship’s systems. Tactical command is given to the flagship NHP; strategic  
and kill command remain the purview of the commanding officer. 
 

The fleet, carrier group, battlegroup, or patrol NHPs construct a virtual war room, networking into  
a hybrid one/many mind (this is referred to later as a Fleet Legion) in order to ensure total-actor  
integration over the battlespace. All commanding officers are party to the information and  
recommendations that issue from the Legion, and tactical feeds are meted out to lower ranks on  
a need-to-know basis (commensurate with their rank and tac/strat portfolio).
 

The first commands after the initial volley are maneuver and systemic orders: avoid incoming fire,  
communicate with allied ships, begin to close the unpredictability gap. At this point, all hands  
prepare for combat: ready onboard null-atmosphere equipment, lock into your station, push  
combat stims, cycle pressure suits, link to Legion subjectivity.
 

Then, scramble fighters, bombers, and landers. Corvettes and gunboats, destroyers and cruisers  
-- subline ships-- begin attack runs. Frigates, battleships, tender ships, and carriers: hold your  
lines, continue systemic/kinetic countermeasures.   
 

Along flight decks and inside carrier blisters, all-hands alarms howl as pilots and techs hurry to  
finish pre-flight checks and procedures. Techs load ordinance and payloads onto fighters and  
bombers while pilots and crews prep systems, uploading the latest telemetries, battle reports,  
flight plans, and obstacle reports. If there are mech chassis and marines aboard, they hurry to  
their landers. 
 

Combat Area Patrol (CAP) wings are launched, escorts tasked with defending landers, corvettes,  
bombers, and gunboats from other fighter wings, torpedo flights, and subline ships. They chart  
flight paths through the kill-clouds and anti-ship weapons, aiming to cross the shrinking no- 
man’s land to harass enemy capital ships, force them to deal with threats at all ranges. At their  
earliest launch, it will take roughly a day to cross no man’s land. 
 

Bombers and subline ships aim to engage capital ships at a close enough range that they cannot  
maneuver to avoid their payloads: bombers and subline ships present small -- relatively speaking  
-- and agile targets, deceptively high-threat units that present a very real danger to any capital  
ships that let them get too close.
 

Landers, laden with marines and mechs in support, have the most dangerous mission: crash into  
the enemy, disembark, and either capture or disable the enemy ship from the inside. Their ships  

                                                                                                           


are built cheap and sturdy, typically with modular chambers and detachable single-use boosters.  
Their mission, after all, is to take the ship or fail.
 

The fleets at this point are engaged, and the combat continues in a shrinking window: the  
unpredictability gap, the space where NHPs and pilots can still outmaneuver their opponents,  
shrinks faster.
 

Ship-to-ship combat increases in intensity as the ships-of-the-line circle towards each other. Mid  
and close range kinetic cloud weapons open up, huling thousands of projectiles at plotted and  
predicted paths. Some short-cycle batteries open fire at this point, their beams carving invisible  
lines of terrible energy through the black, scattering off projected shielding and ablative armor. 
 

Meanwhile, systemic weapons pound away at fleet Legions and individual ship systems,  
attempting to gain tactical advantage. Those spoof probes and shrouds, launched in the early  
days of the fleet engagement, activate, pinging enemy sensors and comms arrays with hostile  
code, creating false signatures and signals to distract weapons and pilots. Subaltern kinetics  
inform their masters of final trajectories, then plunge towards their targets, triggering their  
payloads on impact or, failing to find a positive hit, in proximity to the enemy.   
 

Legions face each other down, NHPs engaging in electronic warfare fought in methods esoteric  
and incomprehensible to human observers, hurling ontological/anti-solipsistic paradox weapons  
back and forth on a plane of battle removed from the human subjectivity. 
 

Finally, at range too close for the enemy to successfully engage in evasive maneuvers, long-cycle  
batteries open up, hurling tremendous, demi-solar particle lances at their targets. Capital  
Commanders at this point must carefully balance their power budget, shifting between angled  
shielding and weapon power if they are to survive a hit from a long-cycle battery. 
 

This is the battle’s climax, the moment when the unpredictability gap closes. Due to the  
tremendous power needed to fire a capital ship’s spinal cannons, each ship-of-the-line generally  
only has one shot to hit their target, as cooling and recharging a ship’s main guns -- kinetic or  
energy -- simply takes too long to be viable in combat. Commanders know this, and hold on to  
their single shot as long as they can: they must hit, and score a clean hit, or they’ll be exposed to  
an enemy with all the time in the world to take the killing blow. 
 

Meanwhile, at the battle’s height, fighters and subline ships buzz in angry swarms, locked in  
bitter wing-combat between their enemy counterparts. Marines and mechs fight grinding  
compartment-to-compartment, deck-to-deck CQB and melee actions as they fight to gain  
control. Cloud-kill kinetics and point-defense weaponry pepper the flanks of great capital ships,  
tearing away at superficial armors, blisters, and distal chambers. Here and there along the line,  
batteries score hits against their targets, and the battlespace is filled with the brilliant micronovae  
of a capital ship’s cataclysmic death. In Legionspace, NHPs tear at each other’s fundamental  
sense of being in combat somehow more terrible than that occurring in subjective space.  
 

When one side beats a retreat or is eliminated, the battle ends. 
 

                                                                                                           


Most ships of the line, unless the system is damaged, have at least a .9 lightspeed eject drive: at  
the start of the battle, conservative, nervous, or cautious captains might begin to spool this  
system up so that it is hot and ready to fire in an emergency. When triggered -- manually, at the  
order of an NHP, or automatically -- the eject drive shunts its ship from its current speed to .9  
Light, hurling it towards a planned (or randomized) eject route. This expeditious retreat is  
dangerous, taxing both systems and personnel, but it’s better than death. 
 

The remains of the battle are left to the victors. Survivors are rounded up. Scuttled or captured  
ships are boarded by skeleton crews and turned towards friendly shipyards: printer technology  
cannot build capital ships, as they’re simply too large. Prisoners are dealt with. Communications  
are relayed back to central command. NHPs drop from Legionspace, unlinking, drawing down to  
their non-combat parameters. Objectives are assessed, adjusted, and fleets either continue on  
their campaign, retire, or steam for a friendly shipyard for repairs and replacements. 
 

From a distance, silence. Up close: the combat of titans, with individuals caught in the middle.  
Typical fleet engagements cost thousands of lives: when fought near inhabited worlds, moons, or  
stations, the cost can become exponential. Unconventional stellar combat -- such as  
accelerating or nudging asteroids and comets into planets -- can prove to be yet more costly.  
 

In Lancer, large-scale fleet combat is (relatively) rare and terrible. It represents the breakdown of  
a whole sector, as systemic powers bring their considerable production and logistical capacity to  
bear against each other in contests over worlds and ideologies. The result of this is never cheap,  
with civilian casualties numbering in the millions; should capitol worlds be engaged, the human  
cost can reach and surpass billions. 
 

Smaller-scale fleet combat tends to occur between warring states that share a world, or a world  
and its moon, and usually in low orbit as fleets ferry ground troops from one continent to another,  
or as flights of ships escort intercontinental/interstellar missiles along their flight paths. These  
fleets are generally composed of subline ships, corvettes, mounted mechs, and fighter/bomber  
wings; it is rare for a capital ship to engage in fleet-to-fleet combat in low orbit unless it is  
supporting an invasion and striking ground targets. 
 

Finally, the most common space combat is piracy, where small groups of ships -- converted  
civilian shuttles, older-model fighter/bombers, mounted mechs -- attack other small groups of  
ships, lone subline freighters, or isolated mining/resource-extraction stations. These combats are  
fast and chaotic, with little-to-no use of NHPs, orders of battle, or capital ships -- save for the  
steel and c legends of interstellar piracy.  
 
\subsection{Core and Colony}
Core and Colony  

Lancer’s canon universe is, essentially, post-scarcity -- that is, resources are not only plentiful,  
but accessible for most people at little to no cost. 
 

In practice, Lancer’s post-scarcity golden age only exists for some: well-developed Core worlds  
adjacent to blink gates are fantastically wealthy, rich with technology and cultural capital. Their  

                                                                                                           


citizens, generally speaking, do not want for anything: they’re afforded a base level of housing,  
education, healthcare, and food, localized to their state. 
 

Wealth and capital are not common constructs on these worlds, as currency tends to be  
restricted to a generalized requisition ability; only when you venture outside the bounds of Core  
space do you run into a need for money. 
 

Core worlds are varied in appearance and urbanization. So long as they fit the following criteria,  
they are considered a “Core” world: 
 

     1.  Global distribution of population, or the capability to distribute its population
 
    2.   A reliable method of transitioning people and goods from the world to orbit, or the  
         capability to do so. 
 
     3.  A central government or other unipolar governing body with demonstrated adherence to  
         Union’s edicts. 
 

Worlds that fit these criteria are considered to be Core worlds of Union; thanks to the immediacy  
of blinkspace, outside of Cradle and her neighboring systems there is no real need for stellar  
proximity to be considered a “Core” world. 
 

Worlds that do not fit the criteria for Core designation have a myriad names, designations,  
classifications, and so on -- a good catch-all term is “Colony” world. On these colony worlds, life  
is less secure, and their populations often want for food, medicine, etc. The colony designation  
can encompass everything from initial, small-team settlements, to worlds with populations in the  
millions.  
 

These two classifications of worlds, Core and Colony, are tied into Union’s larger economic  
system, extant as a consequence of the development of blinkspace travel and the trade that  
followed.  
 

In order to participate in intergalactic commerce, worlds translate their currency to Manna, a  
universal unit of value assigned and administered by Union’s Economic Bureau. Core and colony  
worlds that participate in intergalactic trade use Manna to effect trade outside their borders. It is  
common for these worlds to have a primary economy and a secondary, manna-based economy.  

                                                                       2 
Manna is incredibly valuable compared to local currencies. 
 

Post-scarcity in practice means that, on a Core world, players will have access to most  
unrestricted consumer and raw goods. Specialized items might require certain licenses, available  
through purchase or qualifications (in game terms, as rewards given by the GM), but are readily  
available (i.e. they can get them within the day, delivered or picked up as convenient).
 

On Colony worlds, true post-scarcity availability diminishes the farther you get from the nearest  
Core worlds, or as a result of shortages, resource-hoarding, or loss. Players will have access to  
necessary goods (unless there is a shortage, rationing, etc) and wide access to raw materials;  
specialized items may be difficult to obtain due to any number of reasons: they’re limited in  

2 For example, a person who commands, say, an account with 1,000 Manna would be fantastically wealthy  
on any given colony world, able to finance their day-to-day life for decades without need to earn more.  

                                                                                                                  


number and kept under lock and key by the colonial governor, they’re in the hold of a downed  
ship on the other side of the world, they’re of limited number after the last supply ship took off,  
and so on.      
 
\subsection{Colonies and Worlds: Planting A Flag}
Colonies and Worlds: Planting A Flag   

The process of settling a world differs in specifics, but generally a private Core-to-Colony  
settlement mission proceeds as follows:
 

First, a group of people form a Colonial Venture, a loose, temporary corpro meant to pool manna  
and licenses in order to petition the owner of the destination system for a colony charter. The  
system owner is typically Union, as few other interstellar states have the resources to ID and flag  
habitable worlds.  
 

After a Colonial Venture secures a colony charter, they lobby local (or intergalactic, depending on  
proximity) colony firms for supplies, infrastructure, and materiel that they cannot supply for  
themselves. 
 

Colony firms offer realtime-tiered packages in exchange for a cut of the colony world’s raw  
resource output. These packages typically feature a settlement concierge unit, a 100k+  
genebank, a tier 1 printer, and a colony ship packed with a bundle of prefabricated habitation  
pods, heavy drones, medical benches, pan-biome seed libraries, and other necessary colonial  
infrastructure. 
 

Not everyone who is party to a Colonial Venture departs with the colony ship. 
 

A typical colony ship is hundreds of meters to a kilometer long: the vast majority of that space is  
devoted to prefabricated supplies. The live crew onboard the colony ship will be the first settlers  
of the new colony world: a small team of engineers, scientists, and specialists numbering in the  
dozens. They will -- with help from the colony’s concierge unit and its attendant drones, heavy  
drones, and subalterns -- make planetfall, and begin the long work of establishing a colony  
footprint. In the meantime, the first native generation is incubated, birthed, tended, and raised by  
the concierge unit and assigned natal/educational colonists. 
 

Fifteen to twenty years after landfall is made, the first generation of native-born colonists is at  
population-viable levels (usually in the thousands, though depending on colony scale this can be  
a larger number) and select members of the landfall team takes formal control of the colony’s  
development from the colony concierge. The first generation begins to work to improve the  
colony and explore their new world, building out both the colony footprint and beginning work on  
new secondary and tertiary sites. 
 

Concurrent to the first generation’s development, an additional first (1.5) generation is grown  
from distinct reserve genetic material. This second generation comes of age a year or two at  
most from the first, to provide some genetic variance and further establish a stable, viable  
population. 
 

                                                                                                            


Assuming all variables to be nominal, the colonial settlement is now established, and further  
development occurs organically.  
 
\subsection{A Curious Alchemy, A Mundane Miracle}
A Curious Alchemy, A Mundane Miracle  

Printing is a ubiquitous term for matter processor/fabricator systems found throughout the  
galaxy thanks in large part to paracausal scientific advances made post-Deimos Event. Printers  
range in size from handheld units fed by back-worn matter processors, to hanger-sized, fully  
self-contained printing facilities. 
 

Printers range in time and efficiency. The larger and or more complex an item is, the longer it  
takes to print. Generally speaking, most print protocols involve some assembly after constituent  
parts have been fabricated. 
 

All printers function in the same basic manner: raw matter is processed -- the more pure the  
element, the higher quality the result -- and fabricated into the requested item (or its constituent  
parts). Handheld printer operators craft items and objects in augmented reality; larger printers  
are automated. 
 

You cannot print a printer. Union tightly controls access to printer plans and licenses, and does  
not allow them to be distributed. 
 

You cannot print food beyond basic protein reconstruction: a mealy, grainy loaf of compressed  
edible matter that is unsatisfying, but sufficient to survive. Food is still an important luxury,  
cultural, and prestige item, and a given person on a given world will prefer “real” food to  
synthetic food. 
 

Printing a size .5 mech chassis will take about six to eight hours with a hanger-sized printer.  
Printing a size .5 mech chassis with a handheld printer will take about a week. 
 

Printing a personal defense weapon with an unscheduled tabletop printer takes about fifteen  
seconds; printing a PDW with an unscheduled handheld printer takes about five minutes. 
 

Despite the presence of printers and other processor/fabricator systems, the majority of  
construction across human occupied space -- and certainly outside the galactic Core -- is still  
performed the “old” way: through sourcing raw materials, refining, fabrication, and assembly. 
 

Printers of all classes are valuable. What follows is a basic list of available printers and their  
general capacity.
 

    $\bullet$    Sub- or Unscheduled Printer
 
             $\circ$     Subscheduled or Unscheduled printers are handheld, pack-mounted, or table- 
                 mounted printers common among cheap, mass-goods merchants, fabricators,  
                 sculptors, and other private individuals. 
 

                                                                                                           


         $\circ$     While the printers themselves are black box items, the license for obtaining one is  
             not particularly hard to get. It does require training, certification, and yearly check- 
             ins for continued licensing in order to operate. Additionally, your print orders must  
             pass a review process, and are filed to a publicly-viewable omninet database by  
             category. 
 
         $\circ$     Sub and Unscheduled printers are most common among personal businesses and  
             are best for making person-portable hard goods and repairing hard objects. 
 
$\bullet$    Schedule 1 Printer
 
         $\circ$     Typically hangar sized, schedule 1 printers can handle vehicles and structures up  
             to a size 1 mech.
 
         $\circ$     Of the massive-size printers, these are the most portable, and usually the size  
             found installed on dedicated logistics ships, orbitals, and long-term lander  
             packages. 
 
         $\circ$     Schedule 1 printers, like all scheduled printers, are not the best for fabricating  
             objects smaller than a .5 mech: they can be fitted with precision attachments, but  
             unless the object is of simple geometry, you’re better off filing an order from a  
             artisan sculptor with an sub-schedule printer or true goods merchant.  
 
              
 
$\bullet$    Schedule 2 printer
 
         $\circ$     But for rare cases, schedule 2 printers are built-in place and are not portable.  
             Filling 4-5 stories in height, schedule 2 printers can handle a single requisition up  
             to size 2, or process multiple size .5 requisitions at the same time. 
 
         $\circ$     As with S1 printers, unless an S2 printer is outfitted with a specialized suite, for  
             small items it is best to look elsewhere.   
 
$\bullet$    Schedule 3 printer 
 
         $\circ$     The largest printers, generally reserved for Union, corprostate, and municipal  
             uses. Built in place, not portable, standing at least 8 to 10 stories tall, schedule 3  
             printers can fabricate anything size 3 or smaller, and can handle multiple size 1  
             fabrications running in parallel. 
 
         $\circ$     Schedule 3 printers are usually built as self-contained buildings themselves, with  
             multiple on-site suites for printing smaller-than size .5 items. These are separate  
             from the main fabrication chambers, and can run in parallel without taxing the  
             main system. 
 
         $\circ$     Schedule 3 printers can be converted to operate in micro/null gravity as self- 
             contained ships, often accompanying battlegroups as a rear echelon logistic  
             support element tasked with fabricating fast hull repairs and bulk orders.  
 
$\bullet$    Schedule 4 and up
 
         $\circ$   Truly massive machines, materiel, and construction project may employ  
             networked suites of Schedule 3 printers, an informal construct classified by Union  
             as a Schedule 4 configuration. These only operate in microgravity, due to their  
             bulk and the size of the projects they work on. 
 
         $\circ$     Massive engineering and construction projects, despite the prevalence of printing  
             technology, more often than not use conventional super-engineering and  
             construction methods; printers are reserved for refining raw material and  
             producing inert constituent parts -- beams, panels, wiring, and so on -- which they  
             can churn out at a far more rapid and reliable rate than more complex structures. 
 

                                                                                                               

\subsection{Manna}
Manna  

Union is not motivated by currency, and neither are its subjects. The hegemon’s society is  
structured around a galvanizing mission: ensuring the survival of the human species through  
implementing the edicts of the Central Committee (which, in turn, is implementing the best-fit  
plan dictated to them by Forecast/GALSIM, though none but the Central Committee and  
Forecast/GALSIM know this). 
 

Union is post-scarcity and does not function as a market-based economy. An “economy” in  
Union is only understood as a historical or antiquated term, as your average Terran views capital  
and the exchange of currency for goods as a relic of an unsustainable past, one that led to a  
collapse that plunged humanity into thousands of years of self-inflicted darkness, violence, and  
misery. 
 

However, Union recognizes that not all of its client states have progressed to a post-capital  
society. In order to foster fair galactic trade and build a shared consciousness  -- rather than  
violently suppress monetization --  Union’s Central Committee recognized early on the need for a  
galaxy-wide standardized currency: this they call Manna. 
 

To create Manna, Union extracts an abstracted unit of value from its subject states through  
complex treaties and client-facing economic structures. Data, raw materials, human potential --  
tens of thousands of factors go into the creation of a single omni-digital unit of Manna.  
 

Manna’s exchange rate is relative to the currency for which it is being exchanged, or to the  
currency that is being exchanged for it. Wealthy, developed worlds are rich in Manna due to their  
data output, their raw human potential, and other factors. Small colony worlds also benefit from  
Manna’s formulae: their control over raw materials, projected development, and so on all  
contribute to a beneficial exchange value. 
 

Cosmopolitans trade in Manna, as do states and any other entity that engages in trade across  
solar systems. Since the vast bulk of humanity still is bound to their home worlds, stations,  
moons, etc, the vast bulk of humanity still uses whatever their world’s currency is, and will only  
encounter Manna if they do business off-world (or with entities that are off-world). 
 
\subsection{Union Galactic Organization}
Union Galactic Organization   

The galaxy is vast, and humanity contains multitudes. Organizing the galaxy into something  
resembling a state is a full-time task that requires a broad, complex bureaucracy: the Union  
Administrative Department (UAD). 
 

The UAD has classified humanity into two major categories: Diasporans, or humans who live in  
“real” time on terrestrial worlds, moons, and/or space stations; and Cosmopolitans, humans who  
spend the majority of their lives in the “subjective” time accrued by interstellar travel. 
 

                                                                                                            


Larger stations in stable locations or in orbit around uninhabitable worlds generally operate  
independently of any terrestrial government, acting as their own states with their own territory of  
influence. Generally speaking, stations exist because they were built for a purpose: mining, gas  
extraction, a dry dock platform, shipyards, civilian scientific research, and so on. The populations  
that live onboard them largely work to support the mission of the station and/or the people who  
do that work. Civilian stations usually have permanent populations, the largest of them  
numbering in the tens of thousands. 
 

Blink gates are special cases — while they exist outside of the state boundaries of Cradle, they  
are integral to Union’s control over the populated galaxy. As such, a blink gate is managed by a  
Union Governor, policed by Union security forces, and administered by Union bureaucrats and  
personnel. Blink stations serve as gates, points that hold open stable holes into blinkspace.
 

Military stations are smaller and do not have permanent populations born and raised onboard.  
Union claims jurisdiction over all stations, though only maintains a Union-flagged presence on  
strategically important stations. 
 

For simple top-down galactic orientation, Union has adopted a concentric ring system of territory  
classification centered on the Cradle system. Each ring bears the name of a mountain range  
found on Cradle, increasing in length the closer you get to the Cradle system. Blink stations are  
named after peaks in the mountain range their ring is named for. 
 

The ring-naming convention is a shorthand system, used in simple civilian and governmental  
maps. Think how neighborhoods are named -- there are generally accurate markers, but people  
will refer to exact addresses when looking for specific locations.   

Rings are concentric: the farther out you get from Cradle, the larger the rings are.   

And remember, naming stations after peaks is simply a convention -- there are not yet stations  
built for every peak in a given range. If it winds up that there are not enough peak names, Union  
will simply make up new ones. 
 

Cradle: ANDES  Line -- Aconcagua Station, Cerro Bonete Station, Galan Station, etc
 

Ring 1: ROCKY MOUNTAIN Line -- Elbert Station, Lincoln Station, Castle Station, etc
 

Ring 2: KUNLUN Line -- Kongur Tagh Station, Karakoram Station, Mayakovsky Station, etc 
 

Ring 3: URAL Line -- Manaraga Station, Elbrus Station, Iremel Station, etc
 

Ring 4: ATLAS  Line -- Toubkal Station, Ouanoukrim Station, M’Goun Station, etc
 

Ring 5: HIMALAYA Line -- Everest Station, Kanchenjunga Station, Annapurna Station, etc
 

Ring 6: ALTAI  Line -- Belukha Station, Nairamdal Station, Kharkhiraa Station, etc 
 

                                                                                                           


Ring 7: CARPATHIAN Line -- Gerlachovsky Station, L’adovy Station, Moldoveanu Station, etc
 

Ring 8: SIERRA MADRE Line -- Mohinora Station, Peak Station, Bridger Station, etc
 

Ring 9: VINDHYA Line -- Kalumar Station, Dhupgarh Station, Mahendragiri Station, etc
 

Ring 10: CASCADE Line -- Rainier Station, Adams Station, Hood Station, etc
 

Ring 11: ANNAMITE  Line -- Phou Bia Station, Phu Xai Lai Leng Station, Ngoc Linh Station, etc. 
 

The Annamite Line is the current “edge” of Union space. Beyond that line is uncharted territory.  
 
\subsection{Non Human Life}
Non-Human Life  
Non-human life is common in Lancer; non-human sentient life (i.e. alien civilization) is unique in  
how rare it is. 
 

Contents of the module No Room For A Wallflower dives into this topic more, but generally  
speaking, outside of one world there is no non-human alien civilization. 
 

Forecast/GALSIM commonly simulates hostile alien life, though it is widely assumed that such  
simulations are low confidence at best. 
 
\subsection{Shelter}
Shelter  

Lodging and shelter is important -- in Lancer, specific types of lodging are numberless, as myriad  
as the inhabited worlds in the galaxy demand. Here, though, are some possible examples of  
shelter: 
 

Temporary Camp - Temporary camps are common in the field. Any settlement, lodging, etc, that  
is organized for single night or a few nights is a temporary camp. Usually a collection of pitched  
tents and a sketched-out perimeter organized around a fire or heating element. A place to rest  
your head for the night while on the march or on the run.  
 

Base of Operations - A more permanent encampment, but not necessarily a permanent  
settlement, a base of operations is a longer term encampment typically employed by military or  
scientific teams who need a home in the field that can provide necessary shelter and supplies to  
complete a long-term mission. Bases like this tend to be supplied at regular intervals by states or  
private entities, run by crews working to effect a specific goal. You won’t find many tents at  
bases: most structures are sealed buildings, prefabricated and alien to the environment. Bases  
tend to have a defined perimeter, a working crew (skeleton or heavily populated), and be located  
in remote or hostile environments -- in order to stay at one, you need to be posted, be a prisoner,  

                                                                                                              


or have clearance. Science, mining, signal, and military stations and orbitals fall under this  
category of lodging.   
 

Landfall Settlement - Landfall settlements are the first incursions a colony group makes onto a  
claimed world. A typical landfall settlement is more of a base or project than a colonial town or  
city: like a base of operations, the population of a landfall settlement is task-oriented, small in  
number (6-10 engineers), and doesn’t expect visitors. A clonal Companion/Concierge unit  
manages the settlement and acts as a personal assistant to the live crew there. There is usually  
an environmentally sealed main base -- a low-lying collection of self-contained habitation and  
science pods -- and a growing, unoccupied collection of buildings constructed on a planned  
layout for a future colonial settlement. The organic population is small, but the Comp/Con unit  
controls a large group of subalterns and heavy drones, tasked with gathering raw material for  
construction, sowing agrarian land, and doing the bulk of the manual labor. They might have a  
spare bunk, but the population at a landfall settlement typically does not expect visitors.       
 

Colonial Settlement - A colonial settlement can range from the first generation to populate a  
wave 1 landfall settlement, to a thriving city on the eve of signing its articles of independence  
from its Comp/Con administrator. A colonial settlement has beds available, most are utilitarian  
and made from native materials, though there might be one or two notable luxury lodgings  
available. 
 
 
 
Sleeping Tube - Sleeping tubes are common on stations, in commuter interstellar ships, and in  
highly developed metropolises. Cheap and functional, sleeping tubes are coffin-sized, warmly lit  
and padded micro-rooms meant to provide a place to sleep for a night or a layover. They really  
only fit one person; that person’s gear or baggage is stowed in a locker in the lobby of the  
establishment that maintains the sleeping tubes. 
 

Rented Room - Room rentals are available anywhere there is a need in a developed city or  
station. Rooms can range from a space just larger than a sleeping tube, to opulent suites in  
hotel-stations, suspended above pearl worlds. 
 
 
 
Apartment  - Apartments are domiciles ranging from small studio apartments to floor-spanning  
penthouses. Available to rent or to own on stations and inhabited worlds, apartments are built  
into buildings, one unit among many.   
 

Freestanding House - Houses can be found on any world, and are freestanding buildings that are  
occupied (typically) by a single family. 
 

Omni Address - In a galaxy connected by a massive internet, an omni address is a viable mental  
living space. These spaces are virtual, unique to their owner, and networked. Privacy separators  
are a common practice, to the point where keeping your omni address public would likely be  
viewed as an odd thing among most people with an address. Omni addresses are unique,  
galaxy-and-temporal identification signatures and virtual spaces both; these are commonly  
employed by Core world Diasporans and Cosmopolitans, far less so for colonial Diasporans. 
 

                                                                                                                     

\subsection{Gravity}
Gravity  

There is no stable artificial gravity in Lancer. Gravity on stations is spin gravity. Gravity on ships is  
tied to directional acceleration, magnetized grip pads, or spin gravity. 
 

Artificial gravity can be generated, but in the narrative present it is unstable and is the result of  
massive energy expenditure. Unstable artificial gravity can be created in safe containment as a  
momentary impulse; as a result, it has been weaponized, and studies into stabilization are  
ongoing. 
 

Nonlethal artificial impulse gravity is a necessary component to make nearlight ejections and  
bolts survivable for organic life.   
\subsection{How to get There}
How To Get There  

Not everyone has a ship, knows how to fly, or has the right licenses to get around. Player  
characters, typically pilots and their mechs, will need to travel in order to get to where they need  
to be. 
 

Generally speaking, if the pilots in your narrative operate under the orders of a state or state-like  
entity, they’ll have some kind of transportation afforded to them. Usually this is a temporary  
convenience: a transport/tender ship that drops their party off where they need to be, a billet  
aboard a passing capital ship, and so on. 
 

So, what are common (and uncommon!) ways of getting around? 
 

Best to arrange modes of transport into the theaters they operate in: global, space, and  
interstellar. 
 

Global travel is any travel that takes place on a single world, which includes terrestrial, aquatic,  
and atmospheric travel. Terrestrial travel can take the form of trains, caravans, convoys,  
landships, long marches, cars, and so on. Aquatic travel can take place above or below the  
water, on ships and submarines. Atmospheric travel takes the form of airplanes, airships, low- 
orbit vehicles, and the liminal zone of transit on space elevators, sky hooks, etc. 
 

Space travel is common in Lancer, encompassing the zone between a world’s high atmosphere  
out to the nearest blink gate. Ships of all size and classification transit in this zone, running  
shuttle routes between worlds, making supply runs out to local colonies and settlements, and  
engaging in general transit and business. 
 

Some spaceships are rated to operate in a world’s gravity well, and can transit between  
atmospheric flight and travel in hard vacuum. These tend to have a low gross weight --  fighters,  
bombers, and some smaller corvettes fit into this category of atmosphere-rated ships. Capital  
ships are not able to fly for extended periods of time, though should one find itself in  
atmosphere, it could burn hard for lateral movement as it plummets to a catastrophic end.  
 

                                                                                                                  


Interstellar travel is common for certain classes of people in Lancer: military personnel,  
diplomats, explorers, merchants, Union personnel, colonists, migrants, scientists, and so on.  
Many people have many reasons to engage in interstellar travel, but the equipment necessary is  
difficult to obtain. Generally speaking, unless you’re military on deployment, an official on Union  
business, in the employ of a company, university, or benefactor, or fantastically wealthy, you’ll  
need to take a public blinkship. Public blinkships transit between blink stations, which are the  
massive, cosmopolitan melting pots of the galaxy, where people mingle, do business, and pass  
the time until their scheduled blink ship arrives and they can board.   
 

As a shorthand, the modes of stellar travel can be described as follows:
 

Blink Travel
 

Nothing seems out of the ordinary or that much different from normal stellar travel. Your ship is  
underway at a comfortable G, you can walk around in plain clothes (if your ship is large enough  
and pressurized), eat food, drink, sleep, exercise, etc. Blinkspace is perceived as a blindness, if  
you look out through a porthole. Through a screen, blinkspace looks black. 
 

Actual blinkspace travel takes only a moment, and if you were to not look outside of the ship as it  
is underway, you would notice nothing out of the ordinary, simply a black void. 
 

Prolonged exposure can lead to complications. NHPs exposed to blinkspace report existential  
complications. 
 

Nearlight Bolt  

A nearlight bolt (or nearlight ejection, in a combat or emergency scenario) is a sudden, often  
traumatic acceleration to .8 or .99 lightspeed. When prepared for the bolt, you are usually  
strapped into a pressurized crash couch, medicated for it, secure. If not, there is a very real (to  
certain) chance of being pulverized by the sudden movement of the ship. 
 

A nearlight ejection is not a common method of travel: it is an emergency acceleration that  
serves to extract/ disengage a ship from a situation that would otherwise be deadly. A nearlight  
bolt is an uncomfortable, but often necessary form of travel common among military,  
government, and emergency entities. 
 

A nearlight ejection/bolt is dangerous when you’re prepared for it (but normal enough that it is a  
combat tactic) and deadly when you are not. Ships equipped for nearlight jumps are equipped  
with crash couches that fire corresponding opposing bursts of contained artificial gravity in order  
to counteract the G forces that would otherwise pulverize its crew. 
 

Normal Spaceflight  

                                                                                                           


You can walk, talk, eat, and drink, in addition to any other activity you could do on a .1 to 1G  
world. This speed is achieved over slow, comfortable acceleration.    
 

Gravity might get a little uncomfortable at peak speed (if that speed is higher than your  
native gravity), and always pulls in the opposite direction of travel: “Behind” is always  
“down.” 
 
\subsection{Time and its Passage}
Time, And Its Passage  

Time, in Lancer, is a rare thing: a commodity that can be exhausted. Whether you’re on a core  
world or out in the farthest reach of distal space, time is the one resource that is truly scarce in  
that, when your time runs out, you can never find more of it.  

That said, there are some methods for extending one’s presence in real time, and the most  
common among them is conventional interstellar travel. To venture between worlds without the  
blink, one must board a ship that uses some conventional means  -- i.e., non-paracausal system  
of movement, like a solar sail, ion drive, rockets, and so on -- to propel itself across space,  
generally at a steady pace, to reach its destination. This conventional travel, unlike traversing the  
blink with the aid of a blink gate, may best be measured in how much time it will take to reach  
their destination, as distance is (to a layperson Cosmopolitan or Diasporan) a more abstract  
measurement:  

2 Cradle Distance Units (CDU) is so large a number as to be meaningless. The same journey  
written as time rather than distance is 10 standard years realtime/ 6 months subjective, a much  
more apparent cost.   

This stretching of time as a result of travel at great speed is called time dilation. It occurs --  
generally speaking -- either because of the difference in velocity between two observers, or  
because of a significant difference in their location to a significant gravitational field (like,  
say, proximity to a black hole).  

For interstellar space travel, velocity causes time dilation. Time, to the person on the ship,  
would seem to progress as normal, but to an observer back on the world, the ship carrying  
their loved one would appear to slow, to slow, to slow until it stopped. Time on the world, to  
the worldbound observer, would of course remain normal.   

So what happens to the person on the ship? And what happens to the person on the world?  
Both would, from their points of view, experience time as normal. But the fact of the matter  
is that time is not equal for the both of them. As the person on the ship accelerates, their  
relative velocity to their home world increases; they closer they get to the speed of light,  
time back home for them slows, slows, slows, until it all but stops.   

But to the people back home, the subjective experience of time progresses as normal. The  
family of the interstellar traveller, their friends, their lovers and rivals, could look skyward  

                                                                                                           


and see, distant though it may be, the ship their loved one travels upon. They could walk  
out into a field with a telescope, or to an uplift station’s memorial observation decks, and  
view their loved one’s ship, frozen, all but preserved in time.   

But this frozen state is only due to perception. On the ship, as on the world it left, time  
continues as normal. As they rocket ahead, closer and closer to the speed of light, the  
difference in time grows, stretches, dilates. It is possible, depending on speed, that for  
every year a ship’s passenger experienced, ten would pass back at their home.     

States could fail. Disasters could wipe out lands. Wars, famine, disease. And the opposite:  
fortunes could change, families grow, businesses prosper. And through it all, frozen in time  
to the worldbound observer, the interstellar ship would hang in the sky, a distant light.   

If that traveller ever returned home, they would find themselves in their future, having “time  
travelled” ahead as a cost of venturing between stars. A journey from their homeworld to a  
system-local blink gate, then from the destination blink gate to their target world, and back,  
could cost a traveller three years subjective time -- time as they experience it -- and thirty  
years real time -- time as people back home experience it.   

When your players or NPCs embark on an interstellar journey, they will need to take into account  
the amount of time that it will cost them -- to venture between the stars comes at great cost to an  
individual; by exposing themselves to time dilation, they will necessarily catapult themselves some  
number of years into the future -- a future they may not want to live in without the familiar comforts  
of friends, family, and the states, culture, and environment they’re familiar with.   

For that matter, matters of politics, economics, or great importance that cast your players between  
stellar systems will, necessarily, encounter the complication of time dilation. That urgent distress  
call from a neighboring system? The fastest you can respond to it is ten years real, six months  
subjective. That promise to fetch a valued item or VIP, and return? Twenty real, two subjective.  
And so on.   

The blink shortens this steep cost, but the cost is still present: most interstellar travel accrues time  
in the transit from a port of origin to a local blink gate, then from the exit blink gate to the intended  
destination; without the blink and the gates to pass between, humanity would have to return to  
conventional causality, and travel across the vast distances of interstellar space. Instead of  
sacrificing ten, twenty, or thirty years real time on a journey, it would be in the range of one, two,  
or three thousand, and without any kind of stasis to preserve them, the ship’s crews would die of  
old age aboard their ship.   

Mechanically, what does the reality of interstellar space travel mean for your players? Either a  
narrative defining mechanic, or nothing important, depending on their backstories, characters’  
attachments, and the overarching narrative. If your players travel in a mercenary company, camp  
followers in tow, then it is of no great concern that time passes differently for them -- their friends  
and families travel with them. However, if your players travel alone, their families left behind, then  
after even one trip there is a chance that their homes will be vastly different when they return than  
when they left.   

                                                                                                           


We recommend a mix of hard cost and narrative liberty. While time dilation certainly is a factor  
even in systemic travel (travel between worlds and stellar bodies of the same solar system), it  
likely won’t be a tool of massive dramatic import. We recommend slipping a few days here and  
there, no more than a month at most of difference between when they left and when they  
returned.   

For larger distances -- say, moving between solar systems connected via blink gates, then the  
cost of interstellar travel would come into play. We recommend a simple formula, more narrative  
than hard math: for a distant frontier system, travel to and from blink gate would take at least a  
     3 
year  subjective, five real. For a developed, heavily populated Core system, travel to a blink gate  
would take three months subjective, nine months real. Crossing blinkspace is, essentially,  
instantaneous.     

You may, if you want, choose to adopt a far more hard science fiction approach to interstellar  
travel; if so, we recommend utilizing online tools to keep fidelity to your constraints.  

   

           
\subsection{Common societies}
Common Societies  

The inhabited galaxy in Lancer is complex and, seemingly, contradictory: it is both a polyglot mix  
of many different cultures, states, religions, and economies, and a single entity managed by a  
centralized power with near-total control over galactic travel, communication, and economy.
 

The galaxy as administered by Union is a single entity: Union is the hegemonic power, and all  
human societies are constituents of Union. Without those states, there is no Union; without  
Union, there is no unified human race. States adjacent to Union’s home system, Cradle, proudly  
fly both Union’s flag and their own, but few other states display such dual loyalties. 
 

Most states in the populated galaxy (“states” in this context refers to any organized social  
structure, from colonial settlement to interstellar nation) assent to Union’s control in order to  
effectively manage their territory. Without the blink network, manna, or omninet, the vast  
distances of space would make communication, travel, trade, migration, etc, essentially  
impossible. Humanity would be isolated without Union. 
 

That being said, Union is by and large a background entity at best for the vast majority of people  
in the galaxy. While heads of state, church, tribe, commune, etc, might meet with Union  
representatives (usually an Administrator and their subaltern), your average Diasporan human will  
never see a Union flag in person. 
 

3 To keep sane, we recommend using measurements of time equal to or simple orders of magnitude more  
or less than standard Earth measurements. I.e one year in Lancer’s narrative present is the same as one  
year on Earth at the time of publication.  

                                                                                                                       


Player characters and non-player characters (NPCs) who call a common society home are  
members of the Diaspora or the Cosmopolita. They are the galactic “normal”, though individual  
expressions of normal may vary 
 

Examples of a common society in Lancer are listed below.   

 Democratic Republic                                   A consensus government, usually  
                                                       constitutional, of representatives elected by  
                                                       their constituents. 

 Ruling Council - Labor                                A council of labor leaders, typically union  
                                                       bosses, that plan for the greater good.  

 Ruling Council - Meritorious                          A council of elected representatives, the  
                                                       “best of the best”, that plan for the greater  
                                                       good.  

 Ruling Council - Dynastic                             A council of nobles who inherit their position  
                                                       by birth, marriage, or death, that plan for the  
                                                       perpetuation of their dynasty and state.  

 Technocracy - Liberal                                 A centrally administered state that weighs,  
                                                       assigns value to, and tracks metrics of its  
                                                       constituents.  

 Technocracy - Machine Rule                            A state run by an NHP, cycled regularly, to  
                                                       enforce an impartial government that  
                                                       prioritizes efficiency. 

 Socialist Collective                                  A true socialist state, run by a central  
                                                       committee. Its relationship to manna may be  
                                                       unknown by its general population.  

 Corpro-State                                          A new form of government, but one growing  
                                                       in popularity thanks to the aggressive political  
                                                       maneuverings of Union Economic Bureau  
                                                       agents. A private entity organized with a  
                                                       single goal: maximize profit.  

 Commune - Small Band Coalition                        Smaller population of environmentally- 
                                                       conscious bands, typically organized around  
                                                       small commune-villages.   

 Commune - Single State                                A global government of stewards, run in  
                                                       accordance and harmony with the ecology of  
                                                       the world on which they live.  

 Commune - Filial Constellation                        As the previous two entries, but with strong  
                                                       family ties between the communes.  

 Caste-Ordered State - Gene Stock                      A state ordered along genetic markers -- SSC  
                                                       is an example of a blended COS/GS and CS. 

                                                                                                           


 Caste-Ordered State - Labor Structured                Caste-ordered labor states are divided along  
                                                       lines of work: you are born into your caste  
                                                       and can only associate with others in your  
                                                       caste (note: this also includes military caste  
                                                       states) 

 Tyrant State - Patronizing                            A single, tyrannical ruler, who casts  
                                                       themselves as a benevolent leader while  
                                                       holding total power over the government and  
                                                       its people. Can be hereditary or dynastic.  

 Tyrant State - Fascist Central Power                  A centrally controlled state, often run by a  
                                                       demagogue, with strong nationalist rhetoric  
                                                       and and emphasis on the supremacy of its  
                                                       people over others.   

 Monarchy - Divine Right                               A classical kingdom, with a hereditary ruler  
                                                       deemed to be a representative of god made  
                                                       flesh. Tends to have a wide nobility.  

 Monarchy - Constitutional                             A kingdom ruled by a monarch, but  
                                                       administered by an elected parliament.  

 Monarchy - Dynastic                                   A kingdom ruled by a single family, with a  
                                                       developed nobility and a church in support  
                                                       (though helmed by a powerful papal figure) 

 Religious - Holy See                                  A state ruled by a church, with a single figure  
                                                       deemed to be the sole voice of god at the  
                                                       head. May be hereditary, may be chosen by  
                                                       council, the people, or signs.  

 Religious - High Priest Council                       A state ruled by a church, which is lead by a  
                                                       council of religious leaders.  
\subsection{Uncommon Societies}
Uncommon Societies  

Uncommon societies in Lancer are states, groups, and territories organized in ways that are rare  
in the galaxy. States that do not participate in galactic politics, states that have little interaction  
with the Cosmopolita, states that have been lost or otherwise hidden, and states that do not  
have a permanent home are examples of some uncommon societies. 
 

Players who hail from uncommon societies often have views on Union and galactic politics that  
skew far from common political leanings, complicating their relationships with other players and  
NPCs that they encounter during your campaign. 
 

Fledgling societies have populations in the hundreds to the thousands. Developing societies  
from the thousands to the hundreds of thousands. Developed societies in the hundreds of  
thousands to billions.  
 

                                                                                                           


More examples are listed below:
 

             Hermit State                         A purposefully isolated, developed state that rejects  
                                                  diplomatic advances (or only extends diplomatic  
                                                  offers rarely).  

             Lost Colony - Young                  A colony world recently lost in the administrative  
                                                  shuffle, as a result of an omninet blackout, or other  
                                                  catastrophe.  

             Lost Colony - Established            A colony world lost long ago, developed in isolation  
                                                  from (and usually without the knowledge of) other  
                                                  civilizations.  

             Nomadic State - Terrestrial          Terrestrial nomadic states tend to be small, as they  
                                                  travel across the world they inhabit, usually moving  
                                                  from habitable zone to habitable zone.  

             Nomadic State - Stellar              Stellar nomadic states are similar to terrestrial  
                                                  nomadic states, only they travel through space  
                                                  instead of a single world.  

             Failed State/ Anarchic World         A failed state is a temporary state; it exists only in the  
                                                  fall.  

             Monastic World                       Monastic worlds are peaceful worlds, places of  
                                                  contemplation with relatively small populations. They  
                                                  may be devoted to study of a text, of a fighting style,  
                                                  or the stars.  

             Band - Pirate                        Similar to nomadic states, but focused on raiding  
                                                  interstellar shipping lanes.  

             Band  - Slaver                       As a pirate state, but with the aim of stealing people.  

             Reservation -  Pre-Industrial        A reservation world is a quiet world, set aside to  
                                                  emulate a “simpler” time. Knowledge of Union is  
                                                  limited -- if people know about Union at all--  and  
                                                  people live in a manner like old humanity before the  
                                                  information age.  

             Reservation - Arcadian               Similar to a pre-industrial reservation world, but taken  
                                                  even farther back. Union tends to observe these  
                                                  worlds, but not interfere. Arcadian reservations worlds  
                                                  are the pre-pastoral, hunter-gatherer idyll, and tend to  
                                                  be excluded from Union’s tithes.  
