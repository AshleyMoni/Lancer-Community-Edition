
\section{The Basic Structure of Lancer: Narrative Play}
          The Basic Structure of LANCER:  
                               Narrative Play  

 LANCER as a game is focused on the mission. The most important thing about a mission is  
there are some stakes involved. There’s something that needs doing, and probably needs doing  

                                                                                           


fast! There’s natural tension in the story that needs to be resolved through player action, and  
without the players intervening, the outcome will be radically different (often for the worse!). If  
you’re not on a mission, you’re in downtime. During downtime, the moment-to-moment action is  
probably not as important, and things can ‘montage’ or jump from scene to scene quite easily.  
There’s often less tension or time sensitivity (but not necessarily none at all).   

Any given story of LANCER always begins in downtime, unless it’s the first session. If it’s the first  
session, see the section below!
 

During downtime, players do the following:
 
     -   Take downtime actions to work on personal projects to progress the story or gain  
         reserves (more on that shortly)
 
     -   Play out any other freeform scenes between the players or NPCs
 

Downtime lasts until the next mission. Before a mission starts, the following steps take place:
 
         	- Brief - The players or the GM establish what the goal of the mission is, and the GM  
         establishes the stakes.  
         - Preparation - Players choose the mech they are going to start with for this mission,  
         pick pilot gear, establish reserves, and make any other preparations
 
         	- Boots on the Ground - We cut to the players right as they arrive on the scene.
 

When the mission is over (completed or abandoned or resolved in one way or another -even  
negatively!), players can debrief, level up, and return to downtime. The loop then continues.
 

The following sections will explain each step in a little more detail. We’ll go in the order you’d  
usually go during your first session
 

                                                The first session
 

During the first session, it’s recommended to skip right over downtime and go right to the brief  
and onwards. You’ll also want to take the extra step of establishing who we are with the players,  
if you haven’t already. Some GMs and players already have a good idea of what this is before  
coming to the table, but it’s perfectly fine to start the first session without having anything firmly  
established. Many groups even play through a first session in order to establish who their group  
is and don’t want to establish it before they start- that’s fine too. 
 

However, establishing a common goal or purpose before the end of the first session will definitely  
help with explaining character motivation and cohesion. If you need inspiration you can roll on  
the tables below or establish it with your group.
 

TABLE: Who are we?
 

                                                                                                                   


 d20       Identity 

 1         An infamous private military corporation 

 2         Glory-seeking warriors 

 3         Union Regulars, career soldiers 

 4         Union Auxiliaries, recruited from a local world 

 5         Elites of the Planetary Defense Force 

 6         Enforcers of the Law 

 7         Criminals, Thieves, and Swindlers 

 8         Acolytes of an ancient martial order 

 9         Devotees of a higher power 

  10       Guardians of an ancient royal lineage 

  11       Corporate security, asset protection 

  12       Explorers of the unknown 

  13       Pirate scum 

  14       Defenders of the homeland 

  15       The forefront of the rebellion 

  16       Saviors of the weak and helpless 

  17       Hungry travelers, in it for the money 

  18       Inventors, engineers, and test subjects 

  19       Inheritors of a famous legacy 

  20       The only ones who can stop what’s coming 

If your players have a patron or a parent organization, you can establish that here
 

TABLE: Who gives us orders?
 

 d20       Patron 

 1-2       Anyone who pays us 

 3-4       Our commanding officer 

                                                                                                              


 d20       Patron 

 5-6       The Hierophant or high priest 

 7-8       A corporate patron or sponsor 

 9-10      Our ancient martial code or law, our duty 

 11-12     Our mentor and founder 

 13-14     Our local Union Administrator and high command 

 15-16     The whisperings of a long-dead monolith 

 17-18     Our liege lord or king 

 19-20     The elders of our organization 

Finally, if you want to establish some history or relationships between the players, you can do  
that quickly and easily. Go around the table and ask each player to choose exactly one or two  
other players and ask them to establish one quick fact or experience that the two characters  
have between them. If you like, and you have time, you can play out a scene or two to get a  
good feeling of the characters.
 

You can use the questions on the table below as prompts. A player can ask one or two of these  
to the table in general and write down answers. Nobody has to answer any question, especially if  
they don’t know the answer yet or the question makes them uncomfortable. Remember to be  
respectful of your fellow players!
 

TABLE: Personal History  

 d20       Personal	History 

 1         Which of you did I grow up with? 

 2         Which of you almost killed me once? 

 3         Which of you was I in love with (or still am)? 

 4         Which have you have I served with for some time? 

 5         Which of you distrusts me? 

 6         Which of you have I gotten drunk with more than once? 

 7         Which of you sees me as a mentor? 

 8         Which of you taught me all I know about building mechs? 

 9         Which of you was I marooned with on a hostile planet for some time? 

                                                                                                                


 d20       Personal	History 

  10       Which of you took me on my first mission? 

  11       Which of you is most likely to ask me for advice? 

  12       Which of you knows a deep secret of mine? What is it? 

  13       Which of you thinks they have me all figured out? 

  14       Which of you finds me completely incomprehensible? 

  15       Which of you is the most curious about me? 

  16       Which of you finds me attractive? 

  17       Which of you thinks they can teach me a thing or two? 

  18       Which of you never expected to see me again? 

  19       Which of you will support and stand by me, no matter what? 

  20       Which of you calls me a friend? 

Adding personal history between the character adds hooks and relationships that can influence  
how characters treat each other and creates fun roleplaying opportunities. It’s perfectly fine to  
start without any history between characters if that’s how you prefer to play your game.
 
\subsection{Brief}
                                                     BRIEF
 

The Brief is the very first step before you can start on a mission. This is when your pilots sit down  
around their consoles, or in their cockpits, or in a board room or a barracks somewhere and  
figure out what needs to be done. It doesn’t have to be an actual brief, and could be an entirely  
out of character conversation between players and the GM. You need to establish a couple of  
clear things with the Brief: the mission goal, and the stakes.
 

Your mission Goal is what you, the players, hope to accomplish with your mission. This might be  
the same goal set out by the GM, or it might be defined entirely by the players. For example, the  
GM might set it up so that a powerful NPC general needs the players to hold a checkpoint  
against rebels. If the player’s goals align, then their goal would also be to defend the checkpoint.  
Perhaps the players want to secretly defect and help the rebels take control of the checkpoint  
instead (in that case, that would be their goal).
 

Alternately, the players might come up with a goal without any GM initiative, such as deciding  
they want to clear out and secure a pirate-infested asteroid to gain a new base of operations for  
their mercenary company and decide to set that as their goal with the GM. Both are valid  
missions.
 

                                                                                                                 


Here’s some example goals for inspiration:
 

TABLE: Mission Goals
 

 d20       Goal 

 1         Escort a VIP from a compromised location to a new safe one 

 2         Respond to an SOS from an unknown source, location noted in message.  

 3         Retrieve a valued or strategic object, item, or information from a secure, hostile  
           location 
 4         Investigate a rumor or tip from a valued informant  

 5         Escort a long-flight weapon or ordinance to its target  

 6         Defend an area expecting an an attack (from pirates, hostile alien fauna, etc) 

 7         Explore a long-abandoned derelict for artifacts 

 8         Bring down a piece of massive infrastructure (bridge, skyhook, dam, etc) 

 9         Go loud to provide cover for a covert mission of utmost importance 

  10       Assassinate a VIP, discretely, or in broad daylight, to send a message 

  11       Attack a hostile defensive position in order to destroy a key objective 

  12       Board a hostile ship or station and take it over; or, destroy it 

  13       Be first on the ground on a world hostile to human life; create a beachhead 

  14       Smuggle something safely or securely through hostile territory 

  15       Hunt down a team of notorious, feared, or respected mech pilots 

  16       Provide cover for an evacuation 

  17       Rescue and extract a someone from a secure or dangerous location, such as a prison  
           or a war zone 
  18       Secure a dangerous location 

  19       Liberate a people held hostage from their cruel ruler, with Union’s backing. 

  20       Intervene in a desperate attempt to stop an incoming missile or attack. 

Success of a mission depends on completing the mission goal, but mission completion does  
not. Characters that complete a mission (success or failure) always gain 1 license level. The  
mission goal might also change mid-mission as more information comes to light or the  
circumstances or parameters change. This is perfectly normal and can create dynamic and  
interesting stories.
 

                                                                                                                 


You can often easily find the stakes by phrasing them as a question, such as the following:
 
     -   Will the players save the newborn colony on Astrada IV from total destruction by the  
         hands of the White Tiger rebels?
 
     -   Will the players discover who has stolen the Harrison Armory bioweapon before they get  
         the opportunity to use it?
 
     -   Will the players escort the diplomatic envoy safely through the raider riddled Mars Reef,  
         or will the envoy’s ship be torn to pieces like so many others?
 
     -   Will the players discover the source of the mysterious artifacts buried beneath the  
         planet’s surface before the rival corpro acquisitions team locks it down forever?
 

And so on and so forth. The actual stakes will often depend on the kind of story the GM is trying  
to tell. Stakes can be deeply personal or more broad. They can be immediate and brutal, or slow  
and gradual. They can often be as simple as survival. Generally the GM will establish stakes for a  
mission, but player actions, history, and background both greatly influence stakes and can play  
directly into stakes themselves. For example, a player that was a former slave might have a lot  
more of a personal stake in stopping slave traders.
 

It’s really important to start a mission with both goal and stakes established. Not only does it  
give clearly defined motivation for the characters to be undertaking a mission, but it also sets up  
what will potentially happen if they fail, and allows the GM to take harder moves if that should  
come to pass (you knew what the stakes were!).
 
\subsection{Preparation}
                                                Preparation
 

During this stage, players choose the mechs and gear they are going to bring with them to start  
the mission (if they’re bringing mechs along). This doesn’t necessarily prevent them from  
changing gear mid-mission, but only the resources, gear, weapons, etc that they start with.
 

LANCER does not track currency. Instead, players are assumed to have access to whatever gear,  
mech parts, etc that their licenses have allowed them to unlock (usually freely available to print  
from a nearby facility). You can always wait to establish what your pilot has on them or what  
mech your pilot is bringing until this step.
 

Mechs: Players can bring one mech with them on a mission. Information on mechs and their  
creation is found in the section directly following this one.
 
Pilot gear: Players can choose clothing, armor, up to two weapons, and up to two other  
pieces of gear to embark with, as long as all the gear’s total rarity is equal to or less than their  
license level.
 

You can find pilot gear and mech parts in the compendium (pilot gear is at the beginning of each  
manufacturer’s section). Basic General Massive Systems pilot gear is rarity 0 and available to all  
players - basic GMS mech parts are available to all pilots from LL0.
 

                                                                                                                   


Players might not always have entirely free access to gear, depending on where they start their  
mission or story circumstances. For example, pilots that are stranded in the middle of an alien  
wasteland after crash landing (with the mission: find civilization) might not have the best access  
to gear. Pilots can always trade, barter, build, or acquire other gear across the course of a  
mission. In such a situation, the GM can use the Power at a Cost tool (see below) to give the  
players what they need.
 

                                                    Reserves
 

Before a mission starts, in this step, pilots must establish reserves that they’re bringing on this  
mission. Reserves is a catch-all term for extra gear, ammunition, support, reinforcements,  
information, access, or anything similar that has been prepared during downtime or established  
during the course of the mission. You typically acquire reserves during downtime, but you could  
also be granted reserves as part of a mission. There’s an example list of typical reserves later in  
this section, but it’s more of a general term for anything extra you’re bringing on a mission.
 
\subsection{Boots on the ground}
                                  BOOTS ON THE GROUND
 

This step cuts out unnecessary planning or stalling and cuts right to when the players arrive on  
the scene. Boots on the ground means immediately establishing a situation and put the players  
in that situation, ready to take action and respond. 
 

This doesn’t have to throw the players right into combat (and probably shouldn’t the majority of  
the time). For an example, let’s say the players have embarked on a mission to escort a refugee  
caravan through a heavily guarded checkpoint manned by local partisans. The GM decides the  
moment players get boots on the ground is when they meet up with the caravan outside of the  
checkpoint.
 

Here’s some other examples.
 
         	- The mission is to infiltrate a guarded facility. The mission starts as players are creeping  
         up to the facility, lit eerily from below
 
         	- The mission is to scout for missing colonists in a newly founded colony planet. The  
         mission starts as players are making their way through the jungle and hear unearthly  
         howls rising through the trees in the distance
 
         	- The mission is to help guard a diplomatic summit. The mission starts as the players see  
         a suspicious figure dart aware from the crowd during the opening ceremony.
 