\part{The Basic Structure of Lancer: Narrative Play}

          The Basic Structure of LANCER:
                               Narrative Play

 LANCER as a game is focused on the mission. The most important thing about a mission is
there are some stakes involved. There’s something that needs doing, and probably needs doing




fast! There’s natural tension in the story that needs to be resolved through player action, and
without the players intervening, the outcome will be radically different (often for the worse!). If
you’re not on a mission, you’re in downtime. During downtime, the moment-to-moment action is
probably not as important, and things can ‘montage’ or jump from scene to scene quite easily.
There’s often less tension or time sensitivity (but not necessarily none at all).

Any given story of LANCER always begins in downtime, unless it’s the first session. If it’s the first
session, see the section below!


During downtime, players do the following:

     -   Take downtime actions to work on personal projects to progress the story or gain
         reserves (more on that shortly)

     -   Play out any other freeform scenes between the players or NPCs


Downtime lasts until the next mission. Before a mission starts, the following steps take place:

         	- Brief - The players or the GM establish what the goal of the mission is, and the GM
         establishes the stakes.
         - Preparation - Players choose the mech they are going to start with for this mission,
         pick pilot gear, establish reserves, and make any other preparations

         	- Boots on the Ground - We cut to the players right as they arrive on the scene.


When the mission is over (completed or abandoned or resolved in one way or another -even
negatively!), players can debrief, level up, and return to downtime. The loop then continues.


The following sections will explain each step in a little more detail. We’ll go in the order you’d
usually go during your first session


                                                The first session


During the first session, it’s recommended to skip right over downtime and go right to the brief
and onwards. You’ll also want to take the extra step of establishing who we are with the players,
if you haven’t already. Some GMs and players already have a good idea of what this is before
coming to the table, but it’s perfectly fine to start the first session without having anything firmly
established. Many groups even play through a first session in order to establish who their group
is and don’t want to establish it before they start- that’s fine too.


However, establishing a common goal or purpose before the end of the first session will definitely
help with explaining character motivation and cohesion. If you need inspiration you can roll on
the tables below or establish it with your group.


TABLE: Who are we?





 d20       Identity

 1         An infamous private military corporation

 2         Glory-seeking warriors

 3         Union Regulars, career soldiers

 4         Union Auxiliaries, recruited from a local world

 5         Elites of the Planetary Defense Force

 6         Enforcers of the Law

 7         Criminals, Thieves, and Swindlers

 8         Acolytes of an ancient martial order

 9         Devotees of a higher power

  10       Guardians of an ancient royal lineage

  11       Corporate security, asset protection

  12       Explorers of the unknown

  13       Pirate scum

  14       Defenders of the homeland

  15       The forefront of the rebellion

  16       Saviors of the weak and helpless

  17       Hungry travelers, in it for the money

  18       Inventors, engineers, and test subjects

  19       Inheritors of a famous legacy

  20       The only ones who can stop what’s coming

If your players have a patron or a parent organization, you can establish that here


TABLE: Who gives us orders?


 d20       Patron

 1-2       Anyone who pays us

 3-4       Our commanding officer




 d20       Patron

 5-6       The Hierophant or high priest

 7-8       A corporate patron or sponsor

 9-10      Our ancient martial code or law, our duty

 11-12     Our mentor and founder

 13-14     Our local Union Administrator and high command

 15-16     The whisperings of a long-dead monolith

 17-18     Our liege lord or king

 19-20     The elders of our organization

Finally, if you want to establish some history or relationships between the players, you can do
that quickly and easily. Go around the table and ask each player to choose exactly one or two
other players and ask them to establish one quick fact or experience that the two characters
have between them. If you like, and you have time, you can play out a scene or two to get a
good feeling of the characters.


You can use the questions on the table below as prompts. A player can ask one or two of these
to the table in general and write down answers. Nobody has to answer any question, especially if
they don’t know the answer yet or the question makes them uncomfortable. Remember to be
respectful of your fellow players!


TABLE: Personal History

 d20       Personal	History

 1         Which of you did I grow up with?

 2         Which of you almost killed me once?

 3         Which of you was I in love with (or still am)?

 4         Which have you have I served with for some time?

 5         Which of you distrusts me?

 6         Which of you have I gotten drunk with more than once?

 7         Which of you sees me as a mentor?

 8         Which of you taught me all I know about building mechs?

 9         Which of you was I marooned with on a hostile planet for some time?




 d20       Personal	History

  10       Which of you took me on my first mission?

  11       Which of you is most likely to ask me for advice?

  12       Which of you knows a deep secret of mine? What is it?

  13       Which of you thinks they have me all figured out?

  14       Which of you finds me completely incomprehensible?

  15       Which of you is the most curious about me?

  16       Which of you finds me attractive?

  17       Which of you thinks they can teach me a thing or two?

  18       Which of you never expected to see me again?

  19       Which of you will support and stand by me, no matter what?

  20       Which of you calls me a friend?

Adding personal history between the character adds hooks and relationships that can influence
how characters treat each other and creates fun roleplaying opportunities. It’s perfectly fine to
start without any history between characters if that’s how you prefer to play your game.

\chapter{Brief}
                                                     BRIEF


The Brief is the very first step before you can start on a mission. This is when your pilots sit down
around their consoles, or in their cockpits, or in a board room or a barracks somewhere and
figure out what needs to be done. It doesn’t have to be an actual brief, and could be an entirely
out of character conversation between players and the GM. You need to establish a couple of
clear things with the Brief: the mission goal, and the stakes.


Your mission Goal is what you, the players, hope to accomplish with your mission. This might be
the same goal set out by the GM, or it might be defined entirely by the players. For example, the
GM might set it up so that a powerful NPC general needs the players to hold a checkpoint
against rebels. If the player’s goals align, then their goal would also be to defend the checkpoint.
Perhaps the players want to secretly defect and help the rebels take control of the checkpoint
instead (in that case, that would be their goal).


Alternately, the players might come up with a goal without any GM initiative, such as deciding
they want to clear out and secure a pirate-infested asteroid to gain a new base of operations for
their mercenary company and decide to set that as their goal with the GM. Both are valid
missions.





Here’s some example goals for inspiration:


TABLE: Mission Goals


 d20       Goal

 1         Escort a VIP from a compromised location to a new safe one

 2         Respond to an SOS from an unknown source, location noted in message.

 3         Retrieve a valued or strategic object, item, or information from a secure, hostile
           location
 4         Investigate a rumor or tip from a valued informant

 5         Escort a long-flight weapon or ordinance to its target

 6         Defend an area expecting an an attack (from pirates, hostile alien fauna, etc)

 7         Explore a long-abandoned derelict for artifacts

 8         Bring down a piece of massive infrastructure (bridge, skyhook, dam, etc)

 9         Go loud to provide cover for a covert mission of utmost importance

  10       Assassinate a VIP, discretely, or in broad daylight, to send a message

  11       Attack a hostile defensive position in order to destroy a key objective

  12       Board a hostile ship or station and take it over; or, destroy it

  13       Be first on the ground on a world hostile to human life; create a beachhead

  14       Smuggle something safely or securely through hostile territory

  15       Hunt down a team of notorious, feared, or respected mech pilots

  16       Provide cover for an evacuation

  17       Rescue and extract a someone from a secure or dangerous location, such as a prison
           or a war zone
  18       Secure a dangerous location

  19       Liberate a people held hostage from their cruel ruler, with Union’s backing.

  20       Intervene in a desperate attempt to stop an incoming missile or attack.

Success of a mission depends on completing the mission goal, but mission completion does
not. Characters that complete a mission (success or failure) always gain 1 license level. The
mission goal might also change mid-mission as more information comes to light or the
circumstances or parameters change. This is perfectly normal and can create dynamic and
interesting stories.





You can often easily find the stakes by phrasing them as a question, such as the following:

     -   Will the players save the newborn colony on Astrada IV from total destruction by the
         hands of the White Tiger rebels?

     -   Will the players discover who has stolen the Harrison Armory bioweapon before they get
         the opportunity to use it?

     -   Will the players escort the diplomatic envoy safely through the raider riddled Mars Reef,
         or will the envoy’s ship be torn to pieces like so many others?

     -   Will the players discover the source of the mysterious artifacts buried beneath the
         planet’s surface before the rival corpro acquisitions team locks it down forever?


And so on and so forth. The actual stakes will often depend on the kind of story the GM is trying
to tell. Stakes can be deeply personal or more broad. They can be immediate and brutal, or slow
and gradual. They can often be as simple as survival. Generally the GM will establish stakes for a
mission, but player actions, history, and background both greatly influence stakes and can play
directly into stakes themselves. For example, a player that was a former slave might have a lot
more of a personal stake in stopping slave traders.


It’s really important to start a mission with both goal and stakes established. Not only does it
give clearly defined motivation for the characters to be undertaking a mission, but it also sets up
what will potentially happen if they fail, and allows the GM to take harder moves if that should
come to pass (you knew what the stakes were!).

\chapter{Preparation}
                                                Preparation


During this stage, players choose the mechs and gear they are going to bring with them to start
the mission (if they’re bringing mechs along). This doesn’t necessarily prevent them from
changing gear mid-mission, but only the resources, gear, weapons, etc that they start with.


LANCER does not track currency. Instead, players are assumed to have access to whatever gear,
mech parts, etc that their licenses have allowed them to unlock (usually freely available to print
from a nearby facility). You can always wait to establish what your pilot has on them or what
mech your pilot is bringing until this step.


Mechs: Players can bring one mech with them on a mission. Information on mechs and their
creation is found in the section directly following this one.

Pilot gear: Players can choose clothing, armor, up to two weapons, and up to two other
pieces of gear to embark with, as long as all the gear’s total rarity is equal to or less than their
license level.


You can find pilot gear and mech parts in the compendium (pilot gear is at the beginning of each
manufacturer’s section). Basic General Massive Systems pilot gear is rarity 0 and available to all
players - basic GMS mech parts are available to all pilots from LL0.





Players might not always have entirely free access to gear, depending on where they start their
mission or story circumstances. For example, pilots that are stranded in the middle of an alien
wasteland after crash landing (with the mission: find civilization) might not have the best access
to gear. Pilots can always trade, barter, build, or acquire other gear across the course of a
mission. In such a situation, the GM can use the Power at a Cost tool (see below) to give the
players what they need.


                                                    Reserves


Before a mission starts, in this step, pilots must establish reserves that they’re bringing on this
mission. Reserves is a catch-all term for extra gear, ammunition, support, reinforcements,
information, access, or anything similar that has been prepared during downtime or established
during the course of the mission. You typically acquire reserves during downtime, but you could
also be granted reserves as part of a mission. There’s an example list of typical reserves later in
this section, but it’s more of a general term for anything extra you’re bringing on a mission.

\chapter{Boots on the ground}
                                  BOOTS ON THE GROUND


This step cuts out unnecessary planning or stalling and cuts right to when the players arrive on
the scene. Boots on the ground means immediately establishing a situation and put the players
in that situation, ready to take action and respond.


This doesn’t have to throw the players right into combat (and probably shouldn’t the majority of
the time). For an example, let’s say the players have embarked on a mission to escort a refugee
caravan through a heavily guarded checkpoint manned by local partisans. The GM decides the
moment players get boots on the ground is when they meet up with the caravan outside of the
checkpoint.


Here’s some other examples.

         	- The mission is to infiltrate a guarded facility. The mission starts as players are creeping
         up to the facility, lit eerily from below

         	- The mission is to scout for missing colonists in a newly founded colony planet. The
         mission starts as players are making their way through the jungle and hear unearthly
         howls rising through the trees in the distance

         	- The mission is to help guard a diplomatic summit. The mission starts as the players see
         a suspicious figure dart aware from the crowd during the opening ceremony.

\chapter{Narrative Play}
                                  NARRATIVE PLAY


Once you’ve made your engagement roll, prepared your gear and supplies, set your goal and
your stakes, you’re officially on a mission. A mission could last one session or several sessions.
You might abandon your original goal in favor of a new one, or encounter a twist in the story that
throws your mission into disarray. Playing a mission out is mostly a matter of the GM - there’s no
strong guidelines here as to how to structure it! However, here’s some tools, advice, and aid for
running a mission.

                                      Narrative play vs. mech combat

The following section deals with narrative play, typically when you’re using your pilot skills. This
is the bit of the mission outside of mech combat, which is a lot more structured. Generally in
narrative play each roll accomplishes much more, scenes can cover large stretches of time, and
the outcome of individual rolls is more important.

Mech combat is turn based, tactical combat. Cutting to mech combat is as simple as declaring
it’s started, drawing a map out, and picking who goes first. When you want each roll to accomplish
more and want to play out turn based, tactical combat, you can swap to mech combat.

These are two different modes of play and the rules work slightly differently for each, especially
combat. If you’re in narrative play and get into combat, you do combat with skill checks, and don’t
make attack rolls. NPCs don’t get their own turns (nobody gets a ‘turn’ in narrative play), but their
actions are narrated by the outcome of player rolls. If you’re doing mech combat, you use turn
based, tactical play, make attack rolls and track hit points, and NPCs will take their own turns.

The rules for mech combat (and the difference from narrative play) are found right after the mech
section.

\section{Making Skill Checks}
                                          Making Skill Checks

Skill checks are only required when there is a tense narrative situation or when the check
would move the story forward. You don’t need to make a skill check to open a door, to cook a
meal, or to talk to a superior, unless that situation is tense or would add to the story. You should
generally always succeed on mundane tasks, especially if they relate to your background. A
barroom brawl, a tense escape, decoding an encrypted message, hacking a computer, talking
down a pirate, picking someone’s pocket, distracting a guard, hunting alien wildlife, or flattering
the planetary governor are all situations that have some degree of tension and consequence, and
might require a skill check.


Skill checks can cover as much or as little as the narrative requires. For example, you could have
one skill check cover an entire day’s worth of infiltration into a covert facility if you so desire. Or,




you could instead cover the moment to moment action - sneaking into vents, hacking doors,
disabling guards, etc.

When making a skill check:

    -    First name your goal.

    -    The target number is always 10, and the check is a simple 1d20 roll. You then add any
         accuracy or difficulty from your skills, and then any accuracy or difficulty the GM imposes
         to get the total accuracy or difficulty on the roll.

    -    You should only roll once to accomplish your goal (the GM can’t require extra rolls of
         you), though the GM could tweak the difficulty if you’re asking something very hard or
         complicated, or declare that given your goal or circumstances the roll would be
         impossible.

    -    On a 9 or below, you don’t accomplish your goal. On a 10-19, you accomplish your goal.
         On a 20+, you excel on your goal.

    -    On a 19 or lower, the GM can choose to add complications.


Complications or consequences from failing or succeeding pilot skill checks always follow the
fiction and stakes established.


A failure doesn’t necessarily mean outright failure, but that you don’t directly accomplish your
goal. Additionally, if you fail a check, you cannot attempt the same activity again until you change
the narrative circumstances or approach (it’s a new day, you try something different). For
example, you try to climb back up that cliff bare-handed, but fail. You could only make another
skill check to climb the cliff again if you try it with a grappling hook, or get some other help.


                                                 Complications

On any result less than a 20+, the GM can throw additional complications or costs into the mix,
as established. The GM doesn’t have to throw a complication in every time, and can just let the
action play out - however rolling less than 20+ gives them the ability to do so.


Complications are typically chosen from the categories of Harm, Time, Resources, Collateral,
Position, Effect. This can never nullify your success if you roll a 10+ or cause you to not
accomplish your goal, but can add additional nuance to the outcome of your skill check.


It’s important to note that the GM can only inflict complications if it makes sense to do so - in
other words it must be established clearly before the roll. If you’re trying to take someone out
with a sniper rifle at 500 meters and they have no way to see you or shoot back, you probably
can’t take harm as a complication. If you’re trying to knock out a soldier from hiding that soldier
probably doesn’t have a good way to fight back right away, even if you miss. If that soldier is
alerted and looking for you, however, she might get a shot off.


    -    Harm is damage, injury, or bodily harm, as established. If someone is pointing a gun at
         you, you attempt to take control of that gun and you fail, you will probably take harm.





     -   Time means the activity takes more time than normal

     -   Resources means something must be used up, lost, or temporarily expended. This could
         be something concrete like running out of ammunition, losing a map, or your gun
         jamming, or could be something like political influence.

     -   Collateral means someone or something else takes harm or injury instead of you or your
         intended target, like an innocent bystander, the whole building, your organization or an
         ally

     -   Position means you are put in a worse position through your actions, like right in the line
         of fire, clinging to the edge of a cliff, in the bad graces of the Baron, or under a spotlight

     -   Effect means your action has less effect than you intend. If you were trying to take
         someone out cleanly, you make a lot more noise than you intended. If you try to fix a
         broken door, it will only open for a few people at a time.


Narratively, complications are probably much worse if you fail (since you failed to accomplish
your goal and got a complication).


Example complications:

         	- Harm: A player rolls to knock someone out who just drew a knife on them by applying
         fists to faces. They don’t manage to knock their target out however, and they get a knife
         in the gut for 2 damage.

         - Time: A player rolls to charm the baron into granting them an audience, succeeding.
         The baron lets them stew for a few hours, but gives them the audience.

         	- Resources: A player rolls to patch up an NPC’s wounds, and fails. The NPC not only
         bleeds out, but the player runs out of medical supplies trying to treat them.
         - Collateral: A player rolls to blow up a door and fails. The whole building starts to
         collapse

         	- Position: A player rolls to take out an assassination target in a hidden base with a
         sniper rifle and succeeds. They kill their target, but they have to fire multiple times,
         exposing their position to the entire base.

         	- Effect: A player rolls to wreck a security system. It only shuts it down for 5 minutes,
         however, giving the players limited time to act.


                                                       Excel

If you excel on a skill check (get a 20+), tell the GM how you surpass your initial goal. The GM
can moderate this if it’s not within reason. In addition, the GM can’t throw complications at you -
you did that well.


Examples:

         	- Bruja excels when making a skill check to hack a door control. She suddenly finds
         herself with access to the whole network

         	- Penny excels when making a skill check to threaten a royal guard to stand down. The
         guard not only surrenders, but offers to help her get an audience with the king





         	- Xi excels when making a skill check to get to the extraction point quickly. He decides
         is able to find a shortcut to get the whole party there instead of just himself.

         - Raja excels when making a skill check to get a hold of transport off world for his party.
         He decides that he manages to talk the shuttle pilot to walk off the job and hand the
         whole ship to his party

\subsection{Player Initiative and NPC Action}
                                 Player initiative and NPC action


Players always have initiative when making skill checks or taking action in narrative play. That’s
a fancy way to say that the GM can never ask for a roll unless prompted by the players. Players
must name their goal or aim of their action, then GMs can ask for a roll and set difficulty. When
the roll is made, initiative turns back to the players (probably with a ‘what do you do?’ from the
GM). What this does in practice is let players decide the course of action and make sure that
each roll has clearly established stakes and parameters - it’ll help the game feel more fair and
prevent unnecessary rolling.


If the players don’t take action, stall, or pass off responsibility for action, then they are
effectively turning initiative over to the GM!


In addition, NPCs (non-player characters) don’t take actions or make rolls by themselves. Their
actions are based on player rolls. For example, if a player lies to an NPC, the NPC doesn’t roll to
see if the player is lying. If the player is successful, the NPC doesn’t see through their deception
- if they fail, the NPC sees they are clearly lying. If the GM feels like the particular NPC is astute
or insightful and can easily see through lies, they might add 1-2 difficulty to the roll.


There’s a little more on this in the GM section if you need examples. In practice you probably
won’t even think about this that much.

\section{Skills in Detail}
                                               Skills in detail


Each skill has basic triggers that allow you to easily decide which skill to use and which
bonuses from backgrounds or training apply to a roll. You don’t have to track or know all of
them (just the ones which you have +accuracy or +difficulty in from backgrounds or training). If
you’re stuck as to which skill you should be using, you can quickly refer to the triggers to get a
good idea.


Here’s a little more detail on each skill and when they might trigger. There’s intentionally a little
overlap between some of the triggers, and each is designed to be somewhat flexible.
Remember, you don’t need to track all the skills, just the ones you are good or bad at!


Applying fists to faces




Punch someone in the face, or alternately fight in open, brutal unarmed combat, whether it’s a fist
fight, martial arts duel, or a huge brawl. This is probably not the smoothest or cleanest fist fight
and probably causes a lot of noise.

Assault
Take part in or direct an open or pitched battle, like a corridor gunfight, a huge shootout, fighting
your way across a battlefield, or undertaking a boarding action. When you assault, you’re always
assaulting something (a position, a rival pilot, an enemy force, a group of guards), and it’s always
loud, open, direct action.

Blow something up
Use explosives (improvised or otherwise), weapons, or maybe just good old fashion brawn to
totally wreck something or turn it into an enormous fireball (maybe a wall, sensor array, outpost,
reactor core - the good stuff). Whenever you’re totally destroying an object, building, etc, you can
use this. Probably not to be used against people unless they’re incidentally in the way.

Threaten
Use force or threats of force to get someone to do what you want them to do. Name what you
want someone to do and what you’re going to do to them if they don’t listen to you. This could
also be blackmail, leverage, or something similarly nasty. Threatening someone can be very high
risk but very effective if successful. If you threaten someone unsuccessfully, your threats have no
further effect on them unless you change something about the situation (like all other skill checks).

Take control
Use force, violence, presence of will, or direct action to take control of something. This is often
something concrete, like an object someone is holding. You could take control of someone’s gun
or a keycard they have on their person. You can additionally can take control of a situation to force
those present to listen, calm down, stop moving, or stop what they’re doing, though you can’t
necessarily force them to do anything further without threatening them. Taking control is never
subtle and always direct and dangerous.

Survive
Persevere through harsh, hostile, or unforgiving environments, such as the vacuum of space,
frozen tundra, a pirate enclave, a crime-ridden colony, untamed wilderness, or scorching desert.
You most often use survive when you want to take a journey through wilderness environments,
navigate, or avoid natural hazards such as carnivorous wildlife, rockfalls, thin ice, or lava fields.
Alternately you could use it to avoid man-made hazards, such as navigating a city safely, or
avoiding dangerous areas of a space station. You could also use it when testing personal
endurance, such as shaking off poison or alcohol.

Stay cool and collected
Do something that requires concentration, speed, or intense precision under pressure, like picking
a lock, finding the right frequency for your omnihook, carefully disarming an explosive, or
unjamming a gun. If you’ve got to do something complicated in a high stress situation without
messing up (and possibly not even breaking a sweat) this is the action to use.




Take someone out
Kill or disable someone quickly, quietly, effectively or from a distance, probably before they even
notice. This is probably a single person but could be two people relatively close together (any
more is sort of stretching it). If you’re looking down a sniper scope at a target, preparing to nerve
pinch a guard to knock them out instantly, quick-drawing during a gun duel, or dropping from a
ceiling to slit a throat, this is the action to use.

Flash
Do something flashy, cool, or impressive with your weapon other than killing, like shoot a very
small or rapidly moving target, shooting someone’s hat off or their weapon out of their hand,
knocking someone out by throwing a gun at them, performing an acrobatic flourish with a sword,
throwing a spear to pin a fleeing target to the ground, or something similar.

Get somewhere quickly
Get somewhere without complications and with speed, but not necessarily quietly. Climb, swim, or
perform acrobatic maneuvers. Fall safely from a great height. Move gracefully in zero-g. Chase or
flee from, outrun or out pace a target. Get somewhere faster than anyone else. You can also use
this when you want to drive or pilot a vehicle.

Act unseen or unheard
Get somewhere or do something without being detected, but not necessarily with speed. Hide,
sneak, or move quietly. Infiltrate a facility while avoiding security, patrols, or cameras. Perform a
quick action or maneuver without being seen or heard, such as picking a pocket, unholstering
your gun, or cheating at cards. Wear a disguise.

Fix, hack, or wreck
Repair a device or faulty system. Alternately, hack it wide open, or totally wreck, disable or
sabotage it. You can use this for hacking or safeguarding electronic systems, such as electronic
door locks, computer systems, omninet webs, or NHP coffins.

Patch
Apply your medical knowledge to administer medication, bandage, staunch bleeding, suture,
cauterize, neutralize poison, or resuscitate. Alternately, you could use it to diagnose or study
disease, pathogens, or illness.

Invent or create
You need tools and supplies to invent or create something successfully. Use this with many
downtime actions to work on projects. You can also use it in the spur of the moment to invent new
devices, tools, or approaches to something (improvised explosives, gear, disguises, or some
similar).

Read a situation




Look for subtext, motive, or threat in a situation or person, often social situations. Use your
intuition to learn someone’s motivation, who is really in charge, or who is about to do something
rash or stupid. Get a gut feeling about a situation or person. Sense if someone is lying to you.

Spot
Spot hidden or difficult to make out details, objects, or people. Spot ambushes, hidden
compartments, or disguised individuals. Spy on a target from a distance, or make out the details,
shape, and number of objects, vehicles, mechs, or people clearly at a distance. Track people or
vehicles.

Investigate
Research a subject, or look at something in great detail. If you can’t find information directly, you
learn how you can get access to that information. Learn about a subject of historical relevance, or
become well-read on a subject. Investigate a mystery or solve a puzzle. Locate a person or object
through research or investigation.

Charm
To charm, you need a receptive audience, or some kind of promise of leverage (money, power,
personal benefit, etc). You can use it when trying to smooth talk your way past guards, get
someone on your side, sway a potential benefactor, talk someone down, perform diplomacy
between two parties, or blatantly lie to someone. You can also use it when trying to impersonate
someone. Charm won’t work on people that aren’t receptive (such as soldiers you are in a
gunfight with) or that you don’t have leverage over (promises of safety, money, power,
recompense, help, etc). These promises don’t necessarily have to be true but they have to have
some weight with your target.

Pull Rank
Pull rank on a subordinate, getting information, resources, or aid from them, even unwillingly. You
can use this on anyone your social status (noble, celebrity, etc) or military rank would have weight
with. Failing this might be risky and could be seen as abusive. You typically can’t pull rank on
hostile targets. You could also use this to pull rank and pretend you are a higher rank than you
are, but it’s even riskier.

Word on the streets
Get gossip, news, or hearsay from the streets. What you get depends on what ‘streets’ you are
getting word from (high society, low society, hearsay, military chatter, etc). This probably takes a
lot less time than investigating something in detail, but the information might be more qualitative
or colored by opinion (sometimes that might be useful). You can always learn where the
information came from or who to go to next.

Get a hold of something
Acquire useful allies, assets, or connections through wealth or social influence. This could be
permanent (buying it or receiving it) or temporary (renting or borrowing help or supplies, etc), and
might be harder or easier depending on how much you want to use it. This can’t be used for
something that’s normally gated by license level (like mech parts) but could be used for aid,




supplies, information, food materials, soldiers, or anything else that has more narrative impact.
Typically this is acquired by buying it from a market or requisitioning it from a parent organization.

Lead or inspire
Give an inspiring speech, or motivate a group of people into action. Administer or run an
organization efficiently or effectively, such as a company, a ship’s crew, a group of colonists or a
mining venture. Effectively command a platoon of soldiers in battle, or (perhaps) an entire army.

\subsection{Skill Challenges}
                                         SKILL CHALLENGES

A skill challenge is a simple way to test an entire group for a particular activity. Everyone makes a
relevant check, and the success of the challenge depends on the overall result of the skill checks
from the entire group, not just one player. If more players succeed than fail, the challenge is a
success. If equally as many succeed as fail, the challenge has a 50% chance of success (roll a
die or flip a coin), representing the razor’s edge of the situation. If more players fail than succeed,
the challenge is a failure.

Here’s some example challenges:
    -    Sneaking into a guarded facility: All players roll a skill check (example skills that could
         trigger: move unseen, spot the cameras, charm the guards into thinking they are a
         superior). Success means they all get in unnoticed, failure means the guards are alerted.
    -    Gaining the favor of the Baron: All players roll a skill check (example skills that could
         trigger: charm, but could also be to threaten the baron, or perhaps read the situation)
         Success means the players gain a private audience with the Baron, failure means the
         players’ meddling is noticed by rival nobility and they are thrown out.
    -    Traversing the Wastes: All players roll a skill check (example skills that could trigger:
         survive, spot water, or get across the waste quickly). Success means they cross the
         wastes unharmed. Failure means they cross the wastes, but it is a harrowing journey and
         they arrive there with no repairs or supplies left and lacking food and water.

Challenges are good when you want to extend the narrative impact of rolls.

You can also have extended challenges that have 3 rounds of rolling and calculate the outcome
based on rounds ‘won’ by the players. For example, the players may have to gain the favor of the
baron, then plant information in the baron’s castle, and sabotage the gate. They are only truly
successful if the majority (2/3) of these tasks are accomplished.

\section{Combat in Narrative Play}
                                      Combat in Narrative Play


When running combat narratively, use the normal rules for skill checks. That means when the
individual actions in combat doesn’t matter that much or you want a combat scene to play out
more like a movie than a tactical game. If there’s not a mech involved (and you’re just playing on
the pilot scale), it’s almost always preferable to use narrative combat.





You don’t need to track turns or make attack rolls, and you might only need to make a few rolls
for the whole combat (one roll for each action or goal, as normal!). Turn-based combat in
LANCER is usually reserved for mech combat.


You can also use a skill challenge to run narrative combat if you want it to be a bit more
structured.


Here’s a couple examples:

	        - Bruja and Penny are negotiating with the Black Star Bandits to try and get them to
release a hostage. The negotiations go sour (Penny fails her skill check to charm the bandit
captain), and the bandits draw on them. Bruja decides to take out the bandits quickly with her
Sidekick. She rolls a skill check, getting a 15. She kills the bandits and the GM decides to add
position as a complication, so the rest of the bandit camp is alerted and they will need to get out
quickly.

	        - Pan is in a pitched battle, on foot. He sees a gun emplacement raining hell down upon
his allies and decides to take control of it. He rolls and gets a 7, failing. The soldiers defending
the emplacement turn the gun on him, preventing him from getting any closer. The GM adds
collateral as a complication. Pan looks behind him and sees some members of his squad get
gunned down in the ill-advised assault.

	        - Raja is commanding a platoon of troops to board an enemy ship and take control of the
command center. He rolls to lead the charge, getting a 22, (he excels). There’s no complications,
and when his group successfully fights their way to the command center, they immediately
surrender and hand control over.


\subsection{Hit Points, Damage, and Injury}

                                 Hit Points, damage, and injury

Pilots normally only care about Hit points, also called HP (how much damage your pilot can take
before they are out of the action!) during mech combat, but they might also take damage as a
result of complications during skill checks.

At level 0, pilots have 6 Hit Points, representing not only bodily health, but also the ability to
duck, dodge, avoid damage, or just sheer luck. At higher levels, they add their grit (1/2 level) to
calculate their base HP (leading to a total of 6 + grit). A pilot that takes damage doesn’t neces-
sarily take bodily harm, but might be using up their stamina, luck, or ability to avoid that incom-
ing damage.

If a consequence deals damage, it’s enough to hurt or kill. Taking things like minor grazes,
bruises, etc doesn’t deal damage but could cause other complications.

Here’s what damage looks like for pilots in narrative play:
Minor damage is 1-2 damage. This could be something like getting shot at by small arms fire,
stabbed, unarmed combat, hit by a flying rock, etc
Major damage is 3-4. This is getting shot at by assault or heavy weapons, a long fall, breathing
in toxic gas, etc
Lethal damage is 6+. This is something like having a mech fall on you, getting hit by a mech
scale weapon, having a grenade blow up right under you or something similar.

Pilots might have 1 or 2 armor. Subtract armor from all damage taken as a pilot, unless that
damage has the armor piercing (ap) tag. Some weapons have the ap tag, but damage outside of
that might also (falling a long distance, being immersed in lava, etc).

                                                 Down and out




If you’re reduced to 0 HP or lower as a pilot roll a 1d6. On a 6, you miraculously shrug off the hit
(or its a close call), returning to 1 HP. If you roll a 1, your luck has run out, and you’re immediately
dead. If you roll a 2-5, you are down and out, at 0 HP. You’re knocked out, pinned, bleeding out,
or otherwise unable to act. Your evasion (how hard it is to hit you in mech combat) becomes 5
and if you take any more damage, you’ll die (if someone comes over and shoots you in the head,
for example).


You can die instead of being down and out, if you choose so. Typically you’d just bleed out and
wake up in prison, a field camp, a hospital, or buried under a pile of dead bodies somewhere.

\subsection{Rests and Full Repair}

                                         Rests and Full Repair

If a character takes an hour and rests with no strenuous activity, they can regain 1/2 their HP, and
recover from Down and Out, coming back to consciousness. When you take at least 10 hours to
rest and full repair, you can recover all your hp.


Rest and repair also help your mech (you can see more in the mech section on damage).


If you’re dead, it might not be the end for you. See the rules on cloning in the Death section in
mech combat.

\chapter{Debrief and Downtime}

                      DEBRIEF and DOWNTIME


Once a mission is over (success or failure) and the players have time to recuperate, it’s time for
the Debrief. Much like the Brief, this step of a LANCER game doesn’t have to be an actual
meeting in-character, and could just be an out-of-character conversation amongst the people
playing at the table.


A mission being over or completed should be a natural stopping point in the narrative (it could
even be mid-mission if the objectives have changed). Players don’t necessarily need to have
accomplished their goal, but something needs to have changed enough about the mission that
the mission itself is functionally over.


During the debrief:

	        1. Level up - Characters gain 1 license level, representing their access to resources,
income, or clout. Characters get more access to mech gear and increase their skills. You can find
more information on leveling up at the end of the mech section, but pilots gain +1 to any skill and
every 3 levels can gain additional training, allowing them to get +1 accuracy on certain skill
triggers as if they had a background in it.

	        2. Talk about the mission. This is not a necessary step but can be helpful to think about
what worked and what didn’t during the session. If there were any notable moments that were
fun, interesting, or exciting it can be helpful to talk about them here as well. Not only is it good
feedback for the GM but also can help validate your fellow players. If you’re going to take this
step, remember to be respectful.


\chapter{Downtime}
                                             DOWNTIME


Pilots are individuals (exceptional individuals, perhaps), with lives to live outside of their mechs.
It’s assumed in LANCER that your pilot has a lot to do outside of a mission. What exactly that is
depends on the GM, the narrative, and the players involved in the game.


Your story might not include much downtime - maybe the character are on a planet under siege
and have little time to pursue other projects. It might include a lot of downtime - maybe there’s a
time skip in between missions of several months, allowing characters many opportunities to
pursue their other goals.


What’s important for the game of LANCER is there must be at least SOME downtime between
missions, even if that’s only a few hours. This allows players to prepare for the next mission and
take actions that will impact the story going forward.


                                           TIME AND DOWNTIME


Downtime is almost purely narrative and doesn’t have to follow strict moment-to-moment action.
You can allow one roll in downtime to cover any amount of time, from an hour to a few months




worth of activity. Generally it is better to let the outcomes of rolls follow the narrative
circumstances, ie the impact of rolls that cover several months of activity should be
consequently more.


                      REWARDING DOWNTIME: FREEFORM VS STRUCTURED


Downtime can be used to accumulate reserves by taking certain actions in order to prepare for
the next mission. During downtime, players can take one downtime action, or two if downtime is
especially long. If the players are under siege, for example, they probably only have time for one.


The purpose of downtime is not merely to prepare for the next mission, however - it’s also to
progress personal stories, advance plots, or flesh out characters. You can do freeform
roleplaying as much as you like during downtime with absolutely no intent (as it is during any
mission), but the GM can still feel free to reward this kind of play with similar rewards if it
feels appropriate. For example, even if Pan and Penny are role-playing a scene of them drinking
at a bar together with no intent to prepare anything, gain reserves, or take downtime actions, the
GM might decide to give them +1 Accuracy on skill checks to help each other during the next
mission because of their bond.


\section{Reserves}
                                                 RESERVES

During downtime, you might want to prepare reserves for the next mission. This is a catch all
term for extra supplies, gear, support, bonuses, allies, etc that you accumulate during downtime.
The GM can give you reserves at any time as a result of the story (for example, granting you
reserves as part of a mission). You can also typically ask for reserves by using the power at a
cost tool below, or by successfully taking downtime actions.


You establish reserves you are taking on a mission before you embark on that mission. There’s
no limit to what you can take along other than what the GM and your own actions are willing to
grant you.


Here’s some examples of reserves. Don’t take these examples as set in stone, but as ideas for
the kind of reserves that could be available to you. Abstractly speaking, reserves is anything
you are holding as an advantage for the next mission. The reserves available to characters
should be dependent on the GM, your own actions, and the story. Some of these are relevant to
mechs (See the mech section for more details), some for pilots, and some for the story.


Before a mission, if the GM is using a list of reserves, it can help for it to be visible for all
players so players have a good idea of what is potentially available for them. When rewarding
reserves, the GM doesn’t necessarily have to pick from this list.


TABLE: Narrative Reserves




 d20       Narrative	Reserves

  1-2      Access: Gain a keycard, invite, bribes, or insider access to a particular location

  3-4      Backing: Political support from someone powerful for this mission - you can invoke it
           as leverage
  5-6      Supplies: Cross a hazardous or hostile area without having to make a skill check

  7-8      Disguise: Prepare a disguise or cover identity, allowing you to sneak into a location
           uncontested
 9-10      Diversion: Prepare or arrange for a diversion, giving you time to take action without
           fear of consequence
   11-12   Blackmail: Gain blackmail or sensitive information on a particular person

   13-14   Reputation: Make a good name for yourself for the next mission, starting you off on a
           good position with everyone you meet
   15-16   Safe Harbor: Gain a guaranteed safe location where you can convene, plan, or
           recuperate
   17-18   Tracking: Know the location of important objects or people for this mission

   19-20   Knowledge: Gain important knowledge of local history, customs, or culture etiquette

TABLE: Mech Equipment and gear reserves

 d20       Mech	reserves

  1-2      Ammo: Get extra uses (+1 or +2) to a (limited) weapon or system

  3-4      Rented gear: Get access to a weapon or piece of mech gear you don’t normally have
           access to for the mission only
  5-6      Extra repairs: Start the mission with +2 repairs on your mech.

  7-8      CORE battery: Consume this to gain core power on your mech, allowing it to use its
           most powerful ability again (you can’t get more than 1 core power at a time)
 9-10      Deployable Shield: Gain a 1-use deployable shield generator, a size 1 deployable that
           grants all allied actors in a burst 2 radius around it light cover
   11-12   Redundant repair: 1/mission make the stabilize action as a free action

   13-14   Systems reinforcement: Get +1 accuracy to hull, agility, systems, or engineering
           checks, this mission only (choose one)
   15-16   Smart ammo: All weapons become smart, this mission only

   17-18   Boosted servos: Become immune to the slowed condition, this mission only

   19-20   Jump jets: Your mech can fly when it makes its regular move, this mission only

TABLE: Tactical reserves




 d20       Tactical	Reserves

 1-2       Scouting: Get detailed information on the kinds of mechs and threats you will be
           facing soon, such as number, type, and statistics
 3-4       Vehicle: Gain use of a transport vehicle or starship for this mission only (a tier 1 NPC
           with the vehicle or ship tag)
 5-6       Reinforcements: Call in an NPC mech ally, once per mission (choose a tier 1-3 NPC
           from the NPC section).
 7-8       Environmental Shielding: Ignore a particular battlefield hazard or dangerous terrain,
           such as extreme heat or cold
 9-10      Accuracy: Gain +1 accuracy on a particular skill or action due to training or
           enhancement, this mission only
 11-12     Bombardment: Call in a artillery or orbital bombardment once during mech combat
           (Full Action, range 30 within line of sight, blast 2, 3d6 explosive damage).
 13-14     Extended Harness: Carry an extra pilot weapon and two pieces of pilot gear into this
           mission
 15-16     Ambush: Choose exactly where your next battle will take place, including the terrain
           and cover set up, etc
 17-18     Orbital Drop: Start the mission dropping from orbit into a heavily fortified or hard to
           reach location
 19-20     NHP assistant: Gain an NHP, controlled by the GM, that can give you advice on the
           current situation

Here’s some examples of reserves in play:


Xi Xiaoyun

During downtime, Xi negotiates with the powerful boss of the Red Dog Triad local crime
syndicate. He knows their next mission might take them through their territory. He makes some
checks (and some promises) and gets in the boss’s good graces. The GM tells his player he can
write down the following for reserves:

	        Weaponry: I’ve borrowed a Coldcore sniper rifle from the triad (from the Smith Shimano
Corpro gear catalogue) just for this mission

	        Backing: I have the backing of the triad. If someone messes with me, I can invoke that.

	        Information: I can ask the GM what kind of enemies we might run into on the mission
due to the Triad’s spy networks


Pard Landover

Pard decides to do some scouting during downtime, scavenging a strange, abandoned derelict
in his mech for supplies. After some harrowing exploration and a few skill checks, the GM asks
him to write the following down:

	        Ammo: I’ve recovered extra ammo cases, giving me +1 use to all limited weapons

	        Nanomaterials: I’ve recovered some strange nano materials, giving me +2 repairs to my
mech this next mission



\section{Downtime Actions}

                                           Downtime Actions


You can use these story actions below to represent what your pilot gets up to during a stretch of
downtime. You can also use them to fill gaps in the story where you want to speed things up,
‘montage’, or cover a greater stretch of time.


Unlike a regular skill check, these downtime actions have specific outcomes depending on
whether you hit, miss, or excel, and usually ask you to choose from a list. The actual skill used is
dependent on context. For example, the Get a Damn Drink action could be rolled using Survive
(for alcohol tolerance, most likely), or Getting Word on the Streets (charming the locals).


Each outcome is dependent on the roll, but is up to the GM and the player making the action to
establish the details, using the prompts given. You can go into as much detail about the ins and
outs of the action as you want - the outcome is the only thing that’s important.

As a good rule, a pilot can do only one of these during downtime, and one or two if the downtime
is especially long. You can always keep the outcomes of these actions as reserves for the next
mission.

You can easily write your own downtime actions that are more dependent on your particular
narrative or story, using them modeled ones below.

To start, let’s look at the Power at a Cost action.

\section{Power at a Cost}
POWER AT A COST

This is a really simple downtime action for gaining rewards, opportunities, or additional
resources (such as reserves). You might want reserves, or something more abstract, like time,
safety, information, allies, support, a base of operations, materials, shelter, food, a damn
pack of cigarettes.


Name what you want. You can always get it, but:

BUT (the GM chooses one or two of the following and fills in details, depending on how
outlandish the request is):

    -    It’s going to take a lot more time than they anticipated
    -    It’s going to be really damn risky
    -    You have to have to give something up or leave something behind (wealth, resources,
         allies)
    -    You’re going to piss someone or something important and powerful off to get it
    -    It’s going to involve going WILDLY off the plan
    -    You’ll need more information to proceed safely
    -    What you put together is going to fall apart damn soon




    -    You’ll need to gather more resources first (you know where to find them, however)
    -    You can’t EXACTLY get what they want, just approximately what you want, a lesser
         version, or less of what you want


You can also use the power at a cost action mid-mission for similar effects (not necessarily
during downtime). The rest of the downtime actions are typically for use only in downtime, but
you could adapt them mid mission if they fit.


BUY SOME TIME
You try and stave off some reckoning, extend your window of opportunity, or merely buy more
time and breathing room for you and your group to act. You might be trying to dodge some heat,
survive stranded in the wilderness, or cause a distraction so another plan can go off. You can use
that distraction or bought time as reserves for the next mission.

On a total result of 9 or lower, you’re out of time, and whatever you’re trying to stall catches up
with you unless drastic measures are taken right now.
On a total result of 10-19, you can buy enough time, but the situation becomes precarious or
desperate. Next time you get this result with the same situation, treat it as a 9 or lower.
On a total result of 20+, you buy just enough time as you need for now, until the next mission. If
you’ve already gotten this result, it becomes a 10-19 for the same situation next time.

GET A DAMN DRINK
You can only make this action where there’s a drink to actually get (in town, a station, a city, or
some other populated area). You blow off some steam, carouse, and generally get into trouble.
You could be doing this to make connections, collect gossip, forge a reputation, or maybe just to
forget what happened on the last mission. Drinks don’t have to be involved, but vice usually is.
There’s usually trouble.

On a total result of 9 or lower, you can decide whether you had good time or not. However, you
wake up in a gutter somewhere with only one of the following (choose!):
    -    Your dignity
    -    Most of your possessions
    -    Your memory
On a total result of 10-19, you get one of the following as reserves and lose one of the following:
    -    A good reputation
    -    A friend or connection
    -    A useful item or piece of information
    -    A convenient opportunity
On a total result of 20+, you get two of the above and don’t lose anything

GET CREATIVE
You tweak something or attempt to make something new, either a physical project, or a piece of
software. It doesn’t have to be something on the a gear list, but it generally can’t be something as
impactful as a piece of mech gear. Once finished, you can use it as reserves.




On a total result of 9 or lower, you don’t make any progress on your project for now. If you already
got this result on the same project, treat it as a 10-19 next time.
On a total result of 10-19, you can make progress on your project, but can’t finish it. You can finish
it next time you have downtime without a roll if you get some things before then (pick 2):
    -    Quality materials
    -    Specific knowledge or techniques
    -    Specialized tools
    -    A good workspace
On a total result of 20+, you can finish your project during this downtime. If it’s complicated, treat
this result as a 10-19, but only choose 1.

GET FOCUSED
You focus on increasing your own skills, training, and self-improvement. You might practice, learn,
meditate, or call on a teacher. Before you make this action, name one thing you’d like to learn or
improve on (a skill, technique, an academic subject or language). This could be something like
starship piloting, cooking, chess, boxing, history, or etiquette. It should generally be a non-combat
action.

The GM will give you a new trigger for a skill, giving you +1 Accuracy when you undertake your
new action (you need to work out which skill to use). For example, the trigger could be +1
Accuracy to roll sharp when playing chess, or +1 Accuracy to roll cool when dancing.

GET ORGANIZED
You start, run, or improve an organization, business, or other venture. Describe it when you start it
up: it’s purpose or goal, mode of operation, and primary business. Track your organization’s
efficiency and influence, from 1-6 (they each start at 1). When you want to make a narrative role
where your organization could help, you can roll your organization’s efficiency or influence like a
skill check. Efficiency is how effective your organization is at what it does (a military organization
with high efficiency would be good at combat, for example). Influence is your organization’s size,
reach, and reputation. An organization with high influence can easily acquire assets or create
opportunities, transport people, or sway public opinion. You can use these advantages as
reserves.

On a total result of 9 or lower, your organization folds immediately unless:
    -    It lowers in efficiency and influence by 1, to a minimum of 1. If it’s already at 1 for both of
         these, you can’t choose this option.
    -    It takes one of the following actions: pay debts, prove worthiness, get bailed out, make an
         aggressive move
On a total result of 10-19, your organization is stable. It gains +2 influence or efficiency, to a
maximum of +6.
On a total result of 20+ your organization gains +2 influence and efficiency, to a maximum of +6

GATHER INFORMATION




You poke your nose around, perhaps where it doesn’t belong. You’re investigating something,
doing research, following up on a mystery, tracking a target, or keeping an eye on something. You
might be doing research in a library, or go undercover in an organization to learn what you can.
Whatever you’re doing, you’re generally trying to gather information on a subject of your choosing.
You can use this information as reserves.

On a total result of 9 or lower, you can choose to get out now, or treat this result as a 10-15. If you
choose the latter, you get your information but you are immediately found out, noticed, captured,
or discovered by an organization that either controls or seeks that information.
On a total result of 10-19, you find the information you’ve looking for. However, (choose one):
    -    You leave clear evidence of your rummaging
    -    You have to dispatch, harm, or implicate someone innocent
On a total result of 20+, you get your information cleanly, no complications

GET CONNECTED
You try and make connections, call upon favors, ask for help, or drum up support for a particular
course of action. You need access to communications or just good old fashioned face to face
conversation to take this action. You can use your connection’s resources or aid as reserves for
the next mission.

On a total result of 9 or lower, you’ve got to do a favor or make good on a promise for your
connection right now. Otherwise, they won’t help you at all. If you take action right away, however,
they’ll go along with what you want.
On a total result of 10-19, your connection will help you, but you’ve got to do a favor or make good
on a promise after they help you. If you don’t, treat any result as a 9 or lower next time with the
same organization.
On a total result of 20+, your connection will help you out, no strings attached.

SCROUNGE AND BARTER
You try and get your hands on some gear or asset for your group by dredging the scrapyard,
chasing down rumors, bartering in the local market, hunting around, or through good old
fashioned force of will. You could try and get some better pilot gear that could help you, a vehicle,
narcotics, goods, or other sundries. It’s got to be something physical that you can acquire, but
doesn’t necessarily have to be on the gear list. If you get it, you can take it on the next mission as
reserves.

On a total result of 9 or lower, you can get what you’re looking for, but (choose one):
    -    It was stolen, probably from someone who’s looking for it
    -    It’s degraded, old, filthy, or malfunctioning
    -    Someone else has it right now and won’t give it up without force or convincing
On a total result of 10-19, you can get what you’re looking for, as long as you trade in a little
(choose 1):
    -    Time
    -    Dignity
    -    Reputation




     -   Health, comfort and wellness
On a total result of 20+, you get what you’re looking for, no problems at all.

\section{Putting it all Together}
                                    PUTTING IT ALL TOGETHER

After downtime naturally wraps up, the game should loop right back into the next mission,
following the same structure (Brief, Preparation, Mission, Debrief, Downtime).

Of course, narrative play doesn’t cover everything in a game of LANCER. There’s inevitably
always conflict, and that conflict is probably going to involve mechs, which brings us to the next
section.

