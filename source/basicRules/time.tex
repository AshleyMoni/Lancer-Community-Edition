\section{Time in Lancer: Mission, Downtime and Scene}

Before we dig into pilot and mech play in detail, it might be important to note that a game of
LANCER is usually split up into missions. A \textbf{mission} might encompass one or several play
sessions. A \textbf{mission} is a goal or objective that can be completed in a discrete amount of time,
such as destroying a target, evacuating civilians, uncovering a conspiracy, or holding the line
against enemy attack. Each time you embark on a mission, you pick the pilot gear you’re
embarking with and the mech you’re bringing with you.

If you’re not actively on a mission, you’re in \textbf{downtime}. This is the narrative time between
missions where the moment-to-moment action doesn’t matter as much, and roleplaying is much
more important. During downtime you can progress plots, projects, or personal stories.

Within missions and downtime, play is split up into \textbf{scenes}. A scene is a continuous section of
play or activity. The word ‘scene’ is used here because it’s helpful to think about it in cinematic
terms. As long as the ‘camera’ or focus is on their players and their action, a scene is happening.
When the ‘camera’ cuts away from the current scene, it’s over (this is a lot easier to judge
naturally than it sounds). A good example of a scene is a single battle or combat. A scene could
even span many locations or be a montage of action where the moment-to-moment action
doesn’t matter too much. When the current activity or course of action naturally ends, that’s
when the scene should end too.

In keeping somewhat with these terms, the game commonly uses the term \textbf{‘actor’} to refer to any
individual character, player or non-player.

\textbf{Completing a mission is the primary way to level up in LANCER.} Once a pilot completes a
mission, they gain a license level.

For more information on missions and downtime, see the section further on in this book.