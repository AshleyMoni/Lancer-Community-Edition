\providecommand{\Accuracy}{\textbf{Accuracy }}
\providecommand{\Difficulty}{\textbf{Difficulty }}
\section{Bonuses}
Every skill check or attack roll in LANCER has two types of bonuses applied to it: \Accuracy or
Difficulty, or a flat number that comes from a skill rating or Grit. Often you can apply both bonuses!

\subsection{Accuracy and Difficulty}
\Accuracy and \Difficulty represent the momentary advantages and disadvantages gained and
lost during rapid, chaotic moments of action:

\fluff{Opposed offensive and defensive electronic warfare systems bombard each other with viruses
and counter viruses, spoofing targeting systems and layering desperate firewalls.

Pilots, matched in skill, duel each other amidst the shifting debris of a shattered frigate, avoiding
incoming fire and slagged, floating bulkheads as they attempt to land their shots.

Their mech about to overload, a pilot struggles against an unshackled AI to regain control of their
machine, pitting their skill against the best efforts of their newly freed system.}

These situations (and more!) cause pilots to accrue \Accuracy and \Difficulty on rolls.

1 \Accuracy adds \textbf{1d6} to the roll it is applied to. \\
1 \Difficulty subtracts \textbf{1d6} from the roll it is applied to \\
\Accuracy and \Difficulty cancel each other out, on a 1 to 1 basis. \\
\Accuracy and \Difficulty do not stack: instead, the greatest result is chosen and applied to the
final roll. \\
\begin{itemize}
\item An attack roll made with \textbf{+2 Accuracy} would not add the results of those two rolls. Instead, you would pick the greatest result between the two and apply it to your final roll.
\item An attack roll made with \textbf{+1 Accuracy} and \textbf{+1 Difficulty} would have no bonus or subtraction applied to it: the single \Accuracy die and single \Difficulty die would cancel each other out before there is a need to roll.
\item An attack roll made with \textbf{+2 Accuracy} and \textbf{+1 Difficulty} would be made as a roll with \textbf{+1 Accuracy}
\end{itemize}

\subsection{Grit}

Pilots are skilled and unique individuals, multi-talented and resilient. Even so, a brand new pilot
cannot measure up to a tempered, battle-hardened veteran when push comes to shove. This is
represented in game by \textbf{Grit}.

Grit is a flat number equal to \textbf{half your character’s license level}, rounded up. License levels go
from 0-12 and track the general resources, skill, and experience of your character. Your grit
improves as you level up, and can’t typically be increased any other way. It represents your direct
combat experience, expertise, and will to survive.