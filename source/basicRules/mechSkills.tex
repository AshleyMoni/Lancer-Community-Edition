\section{Mech Skills}

The part of your pilot directly related to building, piloting, and fighting with mechs are your mech
skills. There are four, and they go from +0 to +6:

Your \textbf{HULL} skill describes your ability to build and pilot durable, heavy mechs that can take
punches and keep going \\
Your \textbf{AGILITY} skill describes your ability to build and pilot fast,
evasive mechs \\
Your \textbf{SYSTEMS} skill describes your ability to build and pilot technical mechs with powerful
electronic warfare \\
Your \textbf{ENGINEERING} skill describes your ability to build and pilot mechs with powerful reactors,
supplies and support systems

Mech skills help you directly in turn based combat and when piloting your mech. They also give
you additional bonuses when building a mech.

\subsection{WHICH TO USE?}

When you make a \textbf{skill check}, and use your \textit{pilot’s natural skill, experience, or abilities}, use your
pilot skills (1d20 + pilot skill) \\
When you make a \textbf{skill check} and rely on your mech’s \textit{systems, survivability, or raw power}, use
your mech skills (1d20 + mech skill) \\
When you make an \textbf{attack roll}, add Grit. You might add other skills, like Hull or Systems, instead,
but only when specified. \\

You can get extra bonuses on all checks or attack rolls from talents, gear, or pilot backgrounds.
Most, if not all bonuses take the form of bonus Accuracy or Difficulty.