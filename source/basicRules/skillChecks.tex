\section{Skill Checks and Attacks}

There are two types of rolls in LANCER:

A \textbf{skill check} is required in a challenging or tense situation that requires some effort to
overcome. To make a skill check, first name your goal (break down the door, hack the computer),
then \textbf{roll 1d20}, and add any relevant bonuses. On a total result of \textbf{10 or higher}, you \textbf{accomplish
your goal}. On a \textbf{20+ you excel at your goal}, giving you better results than you expected. a total
result of \textbf{9 or lower}, you \textbf{don’t accomplish your goal}. A failed roll doesn’t necessarily mean you
fail completely, but further complicates the situation. The GM cannot change the target number
(10) of a skill check, but can add additional Difficulty (see just below) to the check if it’s harder
than normal.

An \textbf{Attack} is an offensive roll against another actor in \textbf{mech combat}, such as firing a weapon,
attempting to hack a target, or wrestle them to the ground. An attack roll is made the same way
as a skill check (1d20 + relevant bonuses), but the target number can differ from 10, and usually
depends on the numerical defensive statistic of your target, such as evasion, or electronic
defense. An attack is successful if it equals or exceeds the target defense. \textbf{Some attacks can
deal a Critical Hit on a 20+}, allowing you to deal more damage or trigger extra effects.

If a rule refers to an ‘attack’, it applies to an individual roll. The rules might also refer specifically
to \textit{ranged} or \textit{melee} attacks (attacks typically made with a weapon or part of your mech) or tech
attacks (attacks made with electronic warfare).

\subsection*{Contested Check}

As a mech or pilot, you may be called on to perform a \textbf{contested check}. Both the attacker and
defender make skill checks, adding bonuses and penalties. The winner of the contest is whoever
has the highest total result - in the case of any ties, the attacker wins.

\subsection*{Failing Checks}

You can always choose to voluntarily fail any skill check (mech or otherwise). You can take this
option if a friendly actor is trying to help you out, or simply if you think it would lead to a more
interesting situation or make more sense narratively.
