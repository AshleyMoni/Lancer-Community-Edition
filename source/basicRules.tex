\part{Basic Rules}
                                     BASIC RULES

The bulk of the rules in this book focus on actions, movement, and interactions between mechs
in a wide variety of hostile and habitable environments. That being said, pilots spend time
outside of their machines as well (though sometimes not voluntarily). Pilots and mechs are two
components of the same character that each play with slightly variations on the same rules.
The first section of these rules will tell you how to make a pilot and how playing as a pilot works,
the second how to make a mech and how mechs work. The third part talks about the basic
structure of the game (missions and downtime), the fourth is the compendium, where character
options can be found, and the last is the GM section for tweaking rules, creating NPCs, and
running missions.

\chapter{Setup}
                                                  SETUP

This game makes use of two types of dice, the 20 sided dice (referred to from hereon as a d20)
and the 6 sided dice (referred to from hereon as the d6). Multiple dice will be referred to in the
following format - 1 six sided die = 1d6, 2 six sided dice = 2d6, etc.


Sometimes the rules will call for you to roll a 1d3. That is simply a 1d6 with the results halved and
rounded up (1-2 =1, 3-4=2, 5-6=3).


Each player should have at least one 1d20 and a number of d6s. Players will also need a
character sheet or a piece of paper to write down information, and it might be helpful to have
paper with a square or hexagonal grid on it (such as graph paper or a pre-prepared battle map)
since this game makes use of tactical combat. Miniatures are not required to play this game but
can make combat easier to visualize.


One player must play the Game Master (referred to from hereon as the GM). The Game Master
acts as a referee, storyteller, and arbitrator of rules. They help create the story and narrative for
the game and play all of the non-player characters (NPCs). For more information on the Game
Master as well as a list of rules, tips, and tools to use as a Game Master, you can refer to the GM
guide in the section at the end of this book. The rest of the players will play the role of pilots, or
characters in that story.

                                          THE GOLDEN RULE
When referring to the rules in this book, specific rules override general statements or rules.


For example, making a ranged attack typically takes into account cover. However, a weapon with
the Seeking keyword ignores cover. In this case, the weapon’s keyword supersedes the more
general rule.

                                                 ROUND UP




Always round up in LANCER (to the nearest whole number).
\chapter{Space and Measurements}
                                 SPACE AND MEASUREMENTS

This game makes use of measurements in ‘spaces’ for ranges such as movement, weapon
ranges, etc. Most things in the game are measured in sizes, with size 1 being a square or hex 1
space wide on each side. By default 1 space = about 10 feet in game. Size indicates the physical
presence of a mech or other actor on the battlefield. It is measured as a number where size 1 = a
square or hex measuring 1 on each face. Size can be smaller than 1, such as 1/2, or larger, such
as 2 or 3. An actor or object is usually as tall vertically as its size, but that’s not always the case.
Size often does not represent the physical size of an actor, but the space they control around
them. An actor takes up a square area on the battle map equal to its size, rounded up, with a
minimum size of 1. For example, a size 1 mech takes up a 1x1 square area and a size 1/2 human
would also take up a 1x1 square area.

It’s recommended to use a map and tokens, icons, or miniatures to track actors while playing this
game for ease of play (though you can certainly play without them). You can easily play this game
on a tactical grid or hex battle map (as the designers of this game do) or simple measure ranges
using a standard ruler or measuring tape.

The scale of space can be changed if the situation needs it - for example, you might decide each
space is 50’ on each side, or a mile, or something similar. The space that an actor occupies does
not necessarily indicate its size, but the space it controls around it. Most actors take anywhere
from a 1x1 to a 3x3 space on the map, with some exceptions.


Measure all ranges indicated for weapons, effects, etc as originating from any exterior side of a
an actor. That means larger actors will have slightly longer range for their weapons or actions. To
be in range, an actor must be physically inside the range of an effect.

\chapter{Skill Checks and Attacks}
   Skill Checks and Attacks

There are two types of rolls in LANCER:


A skill check is required in a challenging or tense situation that requires some effort to
overcome. To make a skill check, first name your goal (break down the door, hack the computer),
then roll 1d20, and add any relevant bonuses. On a total result of 10 or higher, you accomplish
your goal. On a 20+ you excel at your goal, giving you better results than you expected. a total
result of 9 or lower, you don’t accomplish your goal. A failed roll doesn’t necessarily mean you
fail completely, but further complicates the situation. The GM cannot change the target number
(10) of a skill check, but can add additional Difficulty (see just below) to the check if it’s harder
than normal.





An Attack is an offensive roll against another actor in mech combat, such as firing a weapon,
attempting to hack a target, or wrestle them to the ground. An attack roll is made the same way
as a skill check (1d20 + relevant bonuses), but the target number can differ from 10, and usually
depends on the numerical defensive statistic of your target, such as evasion, or electronic
defense. An attack is successful if it equals or exceeds the target defense. Some attacks can
deal a Critical Hit on a 20+, allowing you to deal more damage or trigger extra effects.


If a rule refers to an ‘attack’, it applies to an individual roll. The rules might also refer specifically
to ranged or melee attacks (attacks typically made with a weapon or part of your mech) or tech
attacks (attacks made with electronic warfare).


                                           CONTESTED CHECK

As a mech or pilot, you may be called on to perform a contested check. Both the attacker and
defender make skill checks, adding bonuses and penalties. The winner of the contest is whoever
has the highest total result - in the case of any ties, the attacker wins.


                                             FAILING CHECKS

You can always choose to voluntarily fail any skill check (mech or otherwise). You can take this
option if a friendly actor is trying to help you out, or simply if you think it would lead to a more
interesting situation or make more sense narratively.

\chapter{Bonuses}
   BONUSES

Every skill check or attack roll in LANCER has two types of bonuses applied to it: Accuracy or
Difficulty, or a flat number that comes from a skill rating or Grit. Often you can apply both bonuses!

\section{Accuracy and Difficulty}
                                ACCURACY AND DIFFICULTY

Accuracy and Difficulty represent the momentary advantages and disadvantages gained and
lost during rapid, chaotic moments of action:


Opposed offensive and defensive electronic warfare systems bombard each other with viruses
and counter viruses, spoofing targeting systems and layering desperate firewalls.

Pilots, matched in skill, duel each other amidst the shifting debris of a shattered frigate, avoiding
incoming fire and slagged, floating bulkheads as they attempt to land their shots.

Their mech about to overload, a pilot struggles against an unshackled AI to regain control of their
machine, pitting their skill against the best efforts of their newly freed system.




These situations (and more!) cause pilots to accrue Accuracy and Difficulty on rolls.


1 Accuracy adds 1d6 to the roll it is applied to.

1 Difficulty subtracts 1d6 from the roll it is applied to

Accuracy and Difficulty cancel each other out, on a 1 to 1 basis.

Accuracy and Difficulty do not stack: instead, the greatest result is chosen and applied to the
final roll.

       •  An attack roll made with +2 Accuracy would not add the results of those two rolls.
          Instead, you would pick the greatest result between the two and apply it to your final roll.
       •  An attack roll made with +1 Accuracy and +1 Difficulty would have no bonus or
         subtraction applied to it: the single Accuracy die and single Difficulty die would cancel
          each other out before there is a need to roll.
       •  An attack roll made with +2 Accuracy and +1 Difficulty would be made as a roll with +1
         Accuracy

\section{Grit}
                                                      GRIT

Pilots are skilled and unique individuals, multi-talented and resilient. Even so, a brand new pilot
cannot measure up to a tempered, battle-hardened veteran when push comes to shove. This is
represented in game by Grit.

Grit is a flat number equal to half your character’s license level, rounded up. License levels go
from 0-12 and track the general resources, skill, and experience of your character. Your grit
improves as you level up, and can’t typically be increased any other way. It represents your direct
combat experience, expertise, and will to survive.

\chapter{Pilot Skills}
                                              PILOT SKILLS

The part of your pilot related to their personal abilities and experience are your pilot skills. These
skills are mostly used during narrative play. At level 0, your pilot has 4 skills based on narrative
triggers, representing different aspects of their character’s training or background. Skills apply a
flat bonus from +2 to +6 to pilot skill checks, and you also might get an accuracy or difficulty
bonus depending on your character.

\chapter{Mech Skills}
                                              MECH SKILLS

The part of your pilot directly related to building, piloting, and fighting with mechs are your mech
skills. There are four, and they go from +0 to +6:

Your HULL skill describes your ability to build and pilot durable, heavy mechs that can take
punches and keep going
Your AGILITY skill describes your ability to build and pilot fast, evasive mechs




Your SYSTEMS skill describes your ability to build and pilot technical mechs with powerful
electronic warfare
Your ENGINEERING skill describes your ability to build and pilot mechs with powerful reactors,
supplies and support systems

Mech skills help you directly in turn based combat and when piloting your mech. They also give
you additional bonuses when building a mech.

                                              WHICH TO USE?

When you make a skill check, and use your pilot’s natural skill, experience, or abilities, use your
pilot skills (1d20 + pilot skill)
When you make a skill check and rely on your mech’s systems, survivability, or raw power, use
your mech skills (1d20 + mech skill)
When you make an attack roll, add Grit. You might add other skills, like Hull or Systems, instead,
but only when specified.

You can get extra bonuses on all checks or attack rolls from talents, gear, or pilot backgrounds.
Most, if not all bonuses take the form of bonus Accuracy or Difficulty.

\chapter{Time in Lancer: Mission, Downtime, and Scene}
               TIME IN LANCER: MISSION, DOWNTIME, and SCENE

Before we dig into pilot and mech play in detail, it might be important to note that a game of
LANCER is usually split up into missions. A mission might encompass one or several play
sessions. A mission is a goal or objective that can be completed in a discrete amount of time,
such as destroying a target, evacuating civilians, uncovering a conspiracy, or holding the line
against enemy attack. Each time you embark on a mission, you pick the pilot gear you’re
embarking with and the mech you’re bringing with you.

If you’re not actively on a mission, you’re in downtime. This is the narrative time between
missions where the moment-to-moment action doesn’t matter as much, and roleplaying is much
more important. During downtime you can progress plots, projects, or personal stories.

Within missions and downtime, play is split up into scenes. A scene is a continuous section of
play or activity. The word ‘scene’ is used here because it’s helpful to think about it in cinematic
terms. As long as the ‘camera’ or focus is on their players and their action, a scene is happening.
When the ‘camera’ cuts away from the current scene, it’s over (this is a lot easier to judge
naturally than it sounds). A good example of a scene is a single battle or combat. A scene could
even span many locations or be a montage of action where the moment-to-moment action
doesn’t matter too much. When the current activity or course of action naturally ends, that’s
when the scene should end too.

In keeping somewhat with these terms, the game commonly uses the term ‘actor’ to refer to any
individual character, player or non-player.




Completing a mission is the primary way to level up in LANCER. Once a pilot completes a
mission, they gain a license level.


For more information on missions and downtime, see the section further on in this book.


\chapter{Narrative Play and Mech Combat}
                          NARRATIVE PLAY AND MECH COMBAT

LANCER makes a distinction between narrative, freeform play, and mech combat, where tracking
individual turns and actions is important. Most of the game and story will typically be in narrative
play. During narrative play, players and the GM can take actions naturally and spontaneously as
they come up. Time and scenes might be faster and individual rolls might count for more.

	        During mech combat, players must take turns to act, and are restricted in the number
and type of actions they can take, making each action much more impactful and tactical.

	        Certain types of actions or effects will work differently in narrative play vs mech combat.
For example, attacking someone as a pilot is typically a skill check in narrative play that could
use skill triggers like Applying Fists to Faces, Assault, or Taking Someone Out, and narratively
stretching much further. In the world of mech combat, however, attacking something is typically
an attack roll, adding grit, and each individual roll accomplishes much less.


