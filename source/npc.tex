\chapter{Non Player Characters}
      NON PLAYER CHARACTERS

This section contains the rules for running non-player characters in combat in LANCER. As a
large part of your role as a GM is running those characters, we've included a lot of rules and
resources here to help you create those characters as fully as you can.


The rules for NPCs can look intimidating, but it's more useful to think of this section as a toolbox
for you to put together the NPCs that you want to include in your game, whether that's a horde of
enemy mechs, a powerful and devious adversary, biological monstrosities, or rebel raiders. It's
more like a catalogue for you to pick, choose, and create. If you don't feel like getting too deep
into this, you can take all the basic NPCs presented here and put them in your combats without
any additional work. However, we have also tried to provide a comprehensive and flexible outline
for you to fill in your own details or make your own creations.


                                              Running NPCs

Non-player characters (NPCs) act by slightly differently rules in LANCER than player characters.
Player characters are assumed to be exceptional individuals, whether through ability, training, or
just sheer luck. The rest of everyone else has to follow in their wake, so to speak.

In narrative play, NPC actions typically depend on the rolls of player characters. In other words,
the player's rolls do double duty for both PC and NPC actions. For example, a player failing a
combat roll is cue for the NPCs to tackle them, punch them in the face, or open fire. A player
failing a roll to infiltrate causes NPCs to notice them, or sound the alarm, or call reinforcements.
NPCs, and the GM generally don't make rolls for themselves. Some very basic statistics are
given for pilot-scale NPCs if you need to include them in mech combat, or want to run turn-
based pilot combat.


In mech combat, NPCs act much like player characters, with some notable exceptions. NPCs
can be heroic individuals, but they typically don't have access to the full range of options that
players do.


                                         NPCs in mech combat

NPCs follow the same rules for players in Mech combat. By default, they take one turn per
round, and can make a single move and two quick actions or full action on their turns, like
player characters. However, NPCs act with the following exceptions:


NPCs, unless specified, never act first on the very first turn, and only take a turn when they
are activated. A player will always act first in the round, then the GM gets to activate a hostile
NPC. A friendly NPC can be activated by player characters and acts in lieu of a player turn.
Player/friendly NPC turns and hostile NPC turns will always alternate until one side has
completely activated, at which point the remaining actors can take their turns in any order.




NPCs are limited in the actions they can take on their turn. They can also take actions that
are slightly different to player actions.

NPCs can take the following quick actions:

             -   Boost - Move again, similar to the player action
             -   Hide or Search - As the player action
             -   Quick Tech - Invade, Lock on, Bolster, or another action in their profile

             -   Grapple - As per the player action
             -   Ram - As per the player action

             -   Reload - Reload one weapon with the loading tag

             -   Skirmish - The NPC attacks with one weapon in its profile of size heavy or smaller

And the following full actions:
             -   Recover - The NPC ends 2 conditions on themselves - Impaired, Jammed,
                 Stunned, Slowed, Immobilized, Lock On.

             -   Boot up - The NPC ends the shut down condition on themselves (they can make
                 this action even if shut down)
             -   Cool - Reduce heat to 0
             -   Barrage - Enemies attack once with each weapon in their profile. They may
                 choose the same or different targets.

You will notice that NPCs cannot overcharge or take the Stabilize action (they cannot repair by
default).

                                                   Recharge

Many NPC modules and weapons have the recharge tag. Once an NPC uses a system or
weapon with this tag, they can't use it again until it recharges. At the start of each of their turns,
roll 1d6 and check to see if they gain the use of their system or weapon back. This is listed in the
profile of the weapon or system. For example, a recharge (5+) system can be used again once a
5 or 6 is rolled.


Check only once for all recharge modules per NPC, but roll separately for each NPC.


                                             NPCs and damage.


NPCs, are destroyed when they reach 0 HP by default. By default an NPC has 1 point of
structure, and cannot enter the CRITICAL state (they are just destroyed when reaching 0 HP). An
NPC with more than 1 point of structure follows the same rules for taking structure damage as
players (but still can't go CRITICAL). When they're reduced to 0 hp and check their last point of
structure, they are destroyed.


NPCs mostly deal flat damage instead of rolling for their attacks.





NPCs cannot Critical Hit unless specified. They rely less on luck than players, who are more
traditionally `heroic' characters. Some NPCs, such as veterans, or Ultras can Critical Hit, which
makes them quite dangerous.


                                                 NPCs and heat

Some NPCs (such as mechs) have a heat capacity like players. By default, NPCs have 1 reactor
stress and are shut down when they reach full heat capacity and reactor stress (and cool all heat
when they shut down). If an NPC has more than 1 reactor stress, it takes heat like a player and
can also enter the CORE BREACH state like a player instead of shutting down when it reaches
full stress.

If an NPC has no heat capacity, it instead takes heat as energy damage.


                                                        Tier

NPCs are split into tiers for ease of estimating difficulty. Higher tier NPCs have increased
statistics and deal more damage. Tier 1 is levels 1-4, tier 2 levels 5-8, and tier 3 levels 9-12.


The damage and bonuses of NPCs often scales per tier. This is written as X/tier. For example, an
attack with +2/tier vs evasion +1 accuracy/tier would roll the following per tier:


Tier 1: +2 with 1 Accuracy

Tier 2: +4 with 2 Accuracy

Tier 3: +6 with 3 Accuracy


Damage is written as Tier 1/Tier 2/Tier 3 damage


For example, an NPC might have an attack that looks like this:


Assault Rifle

Main Rifle

+2 vs evasion/tier with +1 Accuracy

Range 10

4/6/8 kinetic damage


At tier 1, the weapon will attack at +2 targeting, +1 Accuracy for 4 kinetic damage

At tier 2, the weapon will attack at +4 targeting, +1 Accuracy for 6 kinetic damage

At tier 3, the weapon will attack at +6 targeting, +1 Accuracy for 8 kinetic damage


Player characters of a lower level than the tier of NPC they are fighting will generally have a much
harder time. For example, you shouldn't generally match up players of license level 3 against tier
II NPCs.





You can mix and match tier to give players a harder or easier challenge if you want (especially if
you want to ease players into a higher tier so the jump isn't as severe).


                                                 NPC Tags

NPCs use the same tags as players, but have a few additional tags they can use that change the
way that they work in combat. Some of these tags indicate that NPCs have one or more
templates applied to them, which are explained in the following section (and you can find at the
end of this section).


Grunt - The grunt tag indicates an NPC with the grunt template (a weak and numerous enemy)

Elite - The Elite tag indicates an NPC with the Elite template. A elite NPC is generally tougher
and more dangerous than a regular enemy.

Ultra - The Ultra tag indicates an NPC with the Ultra template. An Ultra is meant to be fought by
an entire group of players and has vastly increased toughness and destructive power.

Veteran - A Veteran NPC has greater abilities than a normal NPC and is a more unique or
standout character


Mech - An NPC with the mech tag is an ambulatory, mechanized cavalry unit (like the players)

Vehicle - An NPC with the vehicle tag is a vehicle of some kind

Biological - An NPC with the biological tag has no heat capacity, cannot take or benefit from
tech actions unless specified, and is immune to all tech actions except Scan and Lock On.

Squad - An NPC with the Squad tag indicates a large squad of biological or mechanical enemies
or a squadron of mechs or vehicles. Rules for squads are found in the squad class.

Swarm - An NPC with the swarm tag indicates a large swarm of drones or smaller NPCs. Rules
for swarms are found in the swarm class.


                                           Traits and systems

Traits are components of an NPC that can't be described by a system, such as general qualities,
pilot experience, or training. They cannot be disabled by system damage.


                                         Classes and Template

LANCER doesn't have a set `catalogue' or manual of NPCs, but instead presents a list of basic
NPC classes and templates you can use to customize an NPC the way you want them.


An NPC class describes the basic statistics and abilities of an NPC, and usually describes their
function. For example, if you want an NPC mech that functions like a mobile artillery piece, you
should use the Bombard class. If you want an NPC mech that flies and strafes its targets, you
should use the Ace class.





An NPC template can be applied on top of the base class to further customize an NPC by
adding more unique flavor (such as the pirate template), more unique modules, or changing the
NPC into a tougher enemy meant to be fought by many players.


There are a few important templates that change the function of an NPC fairly drastically:


The Grunt template makes an NPC into a weak, easily dispatched enemy. Grunts have 1hp and
deal reduced damage, but otherwise function like a regular NPC of their type. Grunts can be
used when you want to throw numerous enemies at your players to make the experience more
cinematic or increase the size of an encounter without totally overwhelming your players.


The Elite template makes NPCs tougher and more durable, and can be used when you want to
make a standout or especially powerful NPC. Elites gain more HP and structure which gives
them the ability to take critical damage like players.


The Ultra template makes an NPC into a very powerful foe that should be fought by an entire
group of players. It drastically increases the toughness and durability of an NPC, as well as
giving them access to powerful Ultra traits and systems. Ultras can gain additional activations
(turns) that they can take per round, making them deadly unless they are tackled by multiple
players at once.


The Veteran template makes an NPC into a more characterful, durable NPC. You can use it
when you want an NPC to stand out or have a notable or memorable ability. You can apply it on
top of other templates such as Elite to make a very tough or dangerous foe.


                                      Base and Optional modules

All NPC classes come with base modules (system, weapons, and traits) common to all of that
particular class. Under the entry for each NPC class is a list of optional modules for that class.
Adding multiple optional modules can make a more tactically interesting but more complex and
dangerous enemy and is up to your discretion.


                                             Building an NPC

Building an NPC is a pretty simple process:


    1.  Choose NPC class from the section below
    2.  Choose 0-2 optional modules
    3.  Pick a tier and set the stats for your NPC.
    4.  Choose and apply a template to your NPC, if applicable
    5.  Re-flavor, re-name, and customize

You should always feel free to re-name or re-flavor, modules, or classes as you see fit. For
example, your Ace NPCs in a particular encounter might not be called `Aces' but `Royal Guard'.





                                            Adding complexity

Think about how your NPC functions. Most NPCs should have base systems and 1 optional
system unless otherwise noted. However:


If you want a very basic NPC, don't choose any optional modules or templates. You can very
easily run NPCs without adding any extra complexity to a fight by just choosing the basic NPCs
with their stats and base modules.


If you want a weak NPC you can throw at players en-masse, choose the Grunt template. If you
want to give the impression of a true horde of enemies, use the Squad template.


If you want a slightly more complex, advanced, or dangerous NPC, choose additional optional
modules. Generally adding 1-2 more will be sufficient, but you can add more or less as you see
fit. The more optional modules you add, the more complicated the NPC will be to run during a
game.


If you want a unique, strong, standout, or memorable NPC, apply the Veteran, or Elite templates.


If you want a `boss' type NPC, something that is able to fight the entire group at once, choose
the Ultra template


If you're feeling confident, you can even swap systems around from enemy class to enemy class.
For example, you could give the Spectre's cloaking field to an Assault-type enemy. To push it
even further, if you want to change the Assassin NPC into a nasty example of alien wildlife, you
could give it the biological tag, re-name its variable knife to `slashing claws', then give it the `Acid
Spittle' option from the Monstrosity class.


                                            Balancing Combat

A `normal' difficulty combat should be (per player) any mix of:

    -   1 Ultra/4 players
    -   4 grunts/1 player
    -   1-2 normal enemies/1 player
    -   1 elite/1 player

These enemies should be of the same tier as players. You can mix and match this, for example, if
you have four players, you could mix in 4 grunts and three regular enemies.


You can decrease combat difficulty by lowering the number of enemies or lowering their tier
relative to players, and you can increase combat difficulty by adding enemies with more optional
systems, elite enemies, higher tier enemies, enemies with more templates (such as veteran), or
adding more enemies.





This is something up to you to figure out with your particular group of players. Don't take the
above advice on balancing encounters as a rigid set of rules, but rather a starting point or
guideline. Every group is going to want different levels of challenge.


                                          Number of Combats

In an optimal situation, players should have 1-2 combats between rests, and should fight in 3-4
combats before getting a full repair. As GM this is up to your discretion, especially if you're
throwing harder combats at players. Remember the GM agenda - you are not there to punish
players, but to help tell a good story.

\subimport{./npc/}{npcClasses}

\subimport{./npc/}{specialClasses}

\subimport{./npc/}{templates}