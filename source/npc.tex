\part{Non Player Characters}  
      NON PLAYER CHARACTERS  

This section contains the rules for running non-player characters in combat in LANCER. As a  
large part of your role as a GM is running those characters, we’ve included a lot of rules and  
resources here to help you create those characters as fully as you can. 
 

The rules for NPCs can look intimidating, but it’s more useful to think of this section as a toolbox  
for you to put together the NPCs that you want to include in your game, whether that’s a horde of  
enemy mechs, a powerful and devious adversary, biological monstrosities, or rebel raiders. It’s  
more like a catalogue for you to pick, choose, and create. If you don’t feel like getting too deep  
into this, you can take all the basic NPCs presented here and put them in your combats without  
any additional work. However, we have also tried to provide a comprehensive and flexible outline  
for you to fill in your own details or make your own creations.
 

                                              Running NPCs  

Non-player characters (NPCs) act by slightly differently rules in LANCER than player characters.  
Player characters are assumed to be exceptional individuals, whether through ability, training, or  
just sheer luck. The rest of everyone else has to follow in their wake, so to speak.
 
In narrative play, NPC actions typically depend on the rolls of player characters. In other words,  
the player’s rolls do double duty for both PC and NPC actions. For example, a player failing a  
combat roll is cue for the NPCs to tackle them, punch them in the face, or open fire. A player  
failing a roll to infiltrate causes NPCs to notice them, or sound the alarm, or call reinforcements.  
NPCs, and the GM generally don’t make rolls for themselves. Some very basic statistics are  
given for pilot-scale NPCs if you need to include them in mech combat, or want to run turn- 
based pilot combat.
 

In mech combat, NPCs act much like player characters, with some notable exceptions. NPCs  
can be heroic individuals, but they typically don’t have access to the full range of options that  
players do.
 

                                         NPCs in mech combat  

NPCs follow the same rules for players in Mech combat. By default, they take one turn per  
round, and can make a single move and two quick actions or full action on their turns, like  
player characters. However, NPCs act with the following exceptions:
 

NPCs, unless specified, never act first on the very first turn, and only take a turn when they  
are activated. A player will always act first in the round, then the GM gets to activate a hostile  
NPC. A friendly NPC can be activated by player characters and acts in lieu of a player turn.  
Player/friendly NPC turns and hostile NPC turns will always alternate until one side has  
completely activated, at which point the remaining actors can take their turns in any order.  

                                                                                                         


NPCs are limited in the actions they can take on their turn. They can also take actions that  
are slightly different to player actions.
 
NPCs can take the following quick actions:
 
             -   Boost - Move again, similar to the player action  
             -   Hide or Search - As the player action  
             -   Quick Tech - Invade, Lock on, Bolster, or another action in their profile
 
             -   Grapple - As per the player action  
             -   Ram - As per the player action
 
             -   Reload - Reload one weapon with the loading tag
 
             -   Skirmish - The NPC attacks with one weapon in its profile of size heavy or smaller
 
And the following full actions:  
             -   Recover - The NPC ends 2 conditions on themselves - Impaired, Jammed,  
                 Stunned, Slowed, Immobilized, Lock On.
 
             -   Boot up - The NPC ends the shut down condition on themselves (they can make  
                 this action even if shut down)  
             -   Cool - Reduce heat to 0  
             -   Barrage - Enemies attack once with each weapon in their profile. They may  
                 choose the same or different targets.  

You will notice that NPCs cannot overcharge or take the Stabilize action (they cannot repair by  
default).
 
                                                   Recharge  

Many NPC modules and weapons have the recharge tag. Once an NPC uses a system or  
weapon with this tag, they can’t use it again until it recharges. At the start of each of their turns,  
roll 1d6 and check to see if they gain the use of their system or weapon back. This is listed in the  
profile of the weapon or system. For example, a recharge (5+) system can be used again once a  
5 or 6 is rolled.
 

Check only once for all recharge modules per NPC, but roll separately for each NPC.
 

                                             NPCs and damage.
 

NPCs, are destroyed when they reach 0 HP by default. By default an NPC has 1 point of  
structure, and cannot enter the CRITICAL state (they are just destroyed when reaching 0 HP). An  
NPC with more than 1 point of structure follows the same rules for taking structure damage as  
players (but still can’t go CRITICAL). When they’re reduced to 0 hp and check their last point of  
structure, they are destroyed.
 

NPCs mostly deal flat damage instead of rolling for their attacks.
 

                                                                                                              


NPCs cannot Critical Hit unless specified. They rely less on luck than players, who are more  
traditionally ‘heroic’ characters. Some NPCs, such as veterans, or Ultras can Critical Hit, which  
makes them quite dangerous.
 

                                                 NPCs and heat  

Some NPCs (such as mechs) have a heat capacity like players. By default, NPCs have 1 reactor  
stress and are shut down when they reach full heat capacity and reactor stress (and cool all heat  
when they shut down). If an NPC has more than 1 reactor stress, it takes heat like a player and  
can also enter the CORE BREACH state like a player instead of shutting down when it reaches  
full stress.
 
If an NPC has no heat capacity, it instead takes heat as energy damage.
 

                                                        Tier  

NPCs are split into tiers for ease of estimating difficulty. Higher tier NPCs have increased  
statistics and deal more damage. Tier 1 is levels 1-4, tier 2 levels 5-8, and tier 3 levels 9-12.
 

The damage and bonuses of NPCs often scales per tier. This is written as X/tier. For example, an  
attack with +2/tier vs evasion +1 accuracy/tier would roll the following per tier:
 

Tier 1: +2 with 1 Accuracy
 
Tier 2: +4 with 2 Accuracy
 
Tier 3: +6 with 3 Accuracy
 

Damage is written as Tier 1/Tier 2/Tier 3 damage
 

For example, an NPC might have an attack that looks like this:
 

Assault Rifle
 
Main Rifle
 
+2 vs evasion/tier with +1 Accuracy
 
Range 10
 
4/6/8 kinetic damage
 

At tier 1, the weapon will attack at +2 targeting, +1 Accuracy for 4 kinetic damage
 
At tier 2, the weapon will attack at +4 targeting, +1 Accuracy for 6 kinetic damage
 
At tier 3, the weapon will attack at +6 targeting, +1 Accuracy for 8 kinetic damage
 

Player characters of a lower level than the tier of NPC they are fighting will generally have a much  
harder time. For example, you shouldn’t generally match up players of license level 3 against tier  
II NPCs.
 

                                                                                                                 


You can mix and match tier to give players a harder or easier challenge if you want (especially if  
you want to ease players into a higher tier so the jump isn’t as severe).
 

                                                 NPC Tags  

NPCs use the same tags as players, but have a few additional tags they can use that change the  
way that they work in combat. Some of these tags indicate that NPCs have one or more  
templates applied to them, which are explained in the following section (and you can find at the  
end of this section).
 

Grunt - The grunt tag indicates an NPC with the grunt template (a weak and numerous enemy)
 
Elite - The Elite tag indicates an NPC with the Elite template. A elite NPC is generally tougher  
and more dangerous than a regular enemy.
 
Ultra - The Ultra tag indicates an NPC with the Ultra template. An Ultra is meant to be fought by  
an entire group of players and has vastly increased toughness and destructive power.
 
Veteran - A Veteran NPC has greater abilities than a normal NPC and is a more unique or  
standout character
 

Mech - An NPC with the mech tag is an ambulatory, mechanized cavalry unit (like the players)
 
Vehicle - An NPC with the vehicle tag is a vehicle of some kind
 
Biological - An NPC with the biological tag has no heat capacity, cannot take or benefit from  
tech actions unless specified, and is immune to all tech actions except Scan and Lock On.
 
Squad - An NPC with the Squad tag indicates a large squad of biological or mechanical enemies  
or a squadron of mechs or vehicles. Rules for squads are found in the squad class.
 
Swarm - An NPC with the swarm tag indicates a large swarm of drones or smaller NPCs. Rules  
for swarms are found in the swarm class.
 

                                           Traits and systems  

Traits are components of an NPC that can’t be described by a system, such as general qualities,  
pilot experience, or training. They cannot be disabled by system damage.
 

                                         Classes and Template  

LANCER doesn’t have a set ‘catalogue’ or manual of NPCs, but instead presents a list of basic  
NPC classes and templates you can use to customize an NPC the way you want them. 
 

An NPC class describes the basic statistics and abilities of an NPC, and usually describes their  
function. For example, if you want an NPC mech that functions like a mobile artillery piece, you  
should use the Bombard class. If you want an NPC mech that flies and strafes its targets, you  
should use the Ace class.
 

                                                                                                          


An NPC template can be applied on top of the base class to further customize an NPC by  
adding more unique flavor (such as the pirate template), more unique modules, or changing the  
NPC into a tougher enemy meant to be fought by many players.
 

There are a few important templates that change the function of an NPC fairly drastically:
 

The Grunt template makes an NPC into a weak, easily dispatched enemy. Grunts have 1hp and  
deal reduced damage, but otherwise function like a regular NPC of their type. Grunts can be  
used when you want to throw numerous enemies at your players to make the experience more  
cinematic or increase the size of an encounter without totally overwhelming your players.
 

The Elite template makes NPCs tougher and more durable, and can be used when you want to  
make a standout or especially powerful NPC. Elites gain more HP and structure which gives  
them the ability to take critical damage like players.
 

The Ultra template makes an NPC into a very powerful foe that should be fought by an entire  
group of players. It drastically increases the toughness and durability of an NPC, as well as  
giving them access to powerful Ultra traits and systems. Ultras can gain additional activations  
(turns) that they can take per round, making them deadly unless they are tackled by multiple  
players at once.
 

The Veteran template makes an NPC into a more characterful, durable NPC. You can use it  
when you want an NPC to stand out or have a notable or memorable ability. You can apply it on  
top of other templates such as Elite to make a very tough or dangerous foe.
 

                                      Base and Optional modules  

All NPC classes come with base modules (system, weapons, and traits) common to all of that  
particular class. Under the entry for each NPC class is a list of optional modules for that class.  
Adding multiple optional modules can make a more tactically interesting but more complex and  
dangerous enemy and is up to your discretion.
 

                                             Building an NPC  

Building an NPC is a pretty simple process:
 

    1.  Choose NPC class from the section below  
    2.  Choose 0-2 optional modules  
    3.  Pick a tier and set the stats for your NPC.  
    4.  Choose and apply a template to your NPC, if applicable  
    5.  Re-flavor, re-name, and customize  

You should always feel free to re-name or re-flavor, modules, or classes as you see fit. For  
example, your Ace NPCs in a particular encounter might not be called ‘Aces’ but ‘Royal Guard’.
 

                                                                                                          


                                            Adding complexity  

Think about how your NPC functions. Most NPCs should have base systems and 1 optional  
system unless otherwise noted. However:
 

If you want a very basic NPC, don’t choose any optional modules or templates. You can very  
easily run NPCs without adding any extra complexity to a fight by just choosing the basic NPCs  
with their stats and base modules.
 

If you want a weak NPC you can throw at players en-masse, choose the Grunt template. If you  
want to give the impression of a true horde of enemies, use the Squad template.
 

If you want a slightly more complex, advanced, or dangerous NPC, choose additional optional  
modules. Generally adding 1-2 more will be sufficient, but you can add more or less as you see  
fit. The more optional modules you add, the more complicated the NPC will be to run during a  
game.
 

If you want a unique, strong, standout, or memorable NPC, apply the Veteran, or Elite templates.
 

If you want a ‘boss’ type NPC, something that is able to fight the entire group at once, choose  
the Ultra template
 

If you’re feeling confident, you can even swap systems around from enemy class to enemy class.  
For example, you could give the Spectre’s cloaking field to an Assault-type enemy. To push it  
even further, if you want to change the Assassin NPC into a nasty example of alien wildlife, you  
could give it the biological tag, re-name its variable knife to ‘slashing claws’, then give it the ‘Acid  
Spittle’ option from the Monstrosity class.
 

                                            Balancing Combat
 

A ‘normal’ difficulty combat should be (per player) any mix of:
 
    -   1 Ultra/4 players  
    -   4 grunts/1 player  
    -   1-2 normal enemies/1 player
 
    -   1 elite/1 player  

These enemies should be of the same tier as players. You can mix and match this, for example, if  
you have four players, you could mix in 4 grunts and three regular enemies.
 

You can decrease combat difficulty by lowering the number of enemies or lowering their tier  
relative to players, and you can increase combat difficulty by adding enemies with more optional  
systems, elite enemies, higher tier enemies, enemies with more templates (such as veteran), or  
adding more enemies.
 

                                                                                                          


This is something up to you to figure out with your particular group of players. Don’t take the  
above advice on balancing encounters as a rigid set of rules, but rather a starting point or  
guideline. Every group is going to want different levels of challenge.
 

                                          Number of Combats  

In an optimal situation, players should have 1-2 combats between rests, and should fight in 3-4  
combats before getting a full repair. As GM this is up to your discretion, especially if you’re  
throwing harder combats at players. Remember the GM agenda - you are not there to punish  
players, but to help tell a good story.
 
\chapter{NPC Classes}

        NPC Classes 

\subsection{Ace}
\begin{npc}{Ace}

\fluff{A pilot more comfortable the closer they get to their roots, the Ace enemy-type employs high-speed

strafing runs, agile maneuvers, and a reckless approach to piloting their mech. Cocky and self-assured,
Ace-type enemies relish a good duel.}



\npcBox
[hull = -1,
agility = +3,
systems = +1,
engineering = +0,
hp = 16,
evasion = 11,
e-defense = 9,
heat cap = 8,
armor = 0,
speed = 6,
sensors = 10,
size = 1/2 or 1]

Base systems:

SS Corpro flight system
System

Whenever this mech moves or boost, it can fly.


Missile Launcher
Main Launcher

+1 vs evasion/tier

Seeking

Range 10, Blast 1

4/6/8 explosive damage


Optional systems:
Bombing Bay
System, Limited (6)

1/turn

When the mech flies while it moves or boosts, it can drop a bomb on one target below as a free
action it that it passes over or adjacent to. Bombs create a blast 1 area with at least 1 square on
their target. All affected targets must pass an agility check or take 6/9/12 explosive damage and
be knocked prone.





Strafing Module
System, Quick Action

1/round

The mech flies its speed in any direction in a strafing run, dealing 3/5/7 kinetic damage to any
enemy it passes over or adjacent to (no roll required).


Burst/Dash module
System

Recharge (5+)

The mech may activate this module instead of moving normally, engaging powerful thrusters. It
flies its speed in a straight line in any direction. This movement doesn’t provoke reactions,
ignores, engagement, and any mechs engaged with the target when it activates this module
must pass an agility check or be knocked prone.


Chaff Launchers
System, Reaction

Recharge (6+)

In response to any hostile tech action or attack, the mech launches chaff and engages digital
scramblers, causing the attack to miss, and granting it immunity to tech attacks and ranged or
melee attacks with the smart keyword until the end of its next turn.


Attitude Thrusters
System, Reaction

1/round, Recharge (5+)

In response to being hit by any ranged attack, the mech does a barrel roll, flying 3 in any
direction and gaining resistance to all the damage of that attack.


Tier II:

HP: + 2


          Evade    E-D    Heat    H     A     S    E        Armor       Spd       Sense

          14       10     8       -2    +4    +2   +1       0            7        10

All agility checks made as a response to the Ace’s attacks are made at +1 Difficulty.


Tier III:
HP + 2


          Evade    E-D    Heat    H     A     S    E        Armor       Spd       Sense

          18       11     8       -2    +6    +3   +1       0           8         10


\end{npc}

\subsection{Aegis}

                                                 AEGIS

Aegis-style mechs are squat, defend-and-suppress chassis built to hold ground, support their allies, and

suppress enemies. Higher-tier Aegis mech pilots have access to hardlight and blackwall defensive
measures - technology that makes bulwarks out of even the smallest barricade.

 AEGIS

 Mech

 Hull       Agility      Systems       Engineering

 +0         -2           +2            +2

 HP         Evasion      E-defense     Heat Cap.

 14         7            10            10

 Armor      Speed        Sensors       Size

 2          3            8             2

Base systems:
Light Laser
Main cannon

+2 vs evasion/tier

Range 8

2 energy damage + 3 burn/tier


Defense Net
System, Shield, Quick Action

This system spreads a powerful shimmering repulsion shield over a large area. While this system
is active, this mech is immobilized, but all attacks against any target within a burst 3 zone around
the mech originating outside of that zone take +1 Difficulty/tier. Attacks out of the zone or
between targets that are both inside of the zone are unaffected. It can be deactivated as an
action.


Optional systems:
Adaptive Shielding

System, Reaction

1/round

The first time an allied target inside the defense net is damaged, all targets inside gain resistance
to the damage type of the triggering attack until the end of the Aegis's next turn.


Ring of Fire





System

All targets hostile to the Aegis that start their turn inside the Defense Net or enter it for the first
time on their turns take 3 heat and become Shredded while inside the net (they can end this
condition by leaving the net).


Hardlight cover system

System, shield, deployable, quick action

Recharge (5+)

The Aegis creates a line 3 section of hardlight, which grants light cover (1 difficulty) against
incoming fire. Mechs adjacent to this cover have resistance to damage from line, blast, and cone
attacks. The cover is immune to damage.

At tier II onwards, this cover grants heavy cover.


EM Shielding

System

The Aegis' defense net also grants its difficulty to tech attacks made from outside the shield.


HA Blackwall system

System, Full Action

Limited (1)

The Aegis creates a pitch black wall of blinkspace that takes up a free line 10 zone 5 spaces high
that must have at least one point within 5 spaces of the Aegis. The wall breaks line of sight
between its two sides and no effect or attack can be drawn across it. Any mech that starts its
turn on the wall or crosses it for the first time on its turn has a 50% chance to be lost in
blinkspace momentarily. If this occurs, remove the mech from play. It returns at the end of its
next turn, in any free space within range 10 of either side of the wall. If there is no space
available, it returns when there is. The wall disappears when the Aegis is destroyed or it uses an
action to deactivate it, also returning any stranded mechs.


Tier II:

HP: +4


          Evade     E-D    Heat    H    A     S     E       Armor        Spd       Sense

          8         12     10      +0   -1    +3    +3      2            3         10

Tier III:


          Evade     E-D    Heat    H    A     S     E       Armor        Spd       Sense

          8         14     10      +0   -1    +4    +5      3            3         10


\section{Assassin}

                                                 ASSASSIN  

Assassin style mechs trend towards agility, damage, and speed. Their pilots sacrifice comfort for a chassis  

efficiency — unnecessary life-support systems, pilot care systems, communications systems, and others  
are stripped out and replaced with systems that increase processing power, run-silent ability, and increase  
chassis range.   

Assassin pilots work alone or in small groups, piloting their small, sleek mechs into territory thought  
impenetrable by their targets. One of the few doctrines to employ bladed weapons, assassin-style pilots  

train both in the cockpit and out of it to be able to fight with any weapon, in any theater, as efficiently as  
possible.   

 Assassin 

 Mech 

 Hull        Agility      Systems        Engineering 

 +1          +2           +0             -1 

 HP          Evasion      E-defense      Heat Cap. 

  15         12           8              7 

 Armor       Speed        Sensors       Size 

 0           8            12             1 

Base systems:  
Kai Bioplating  
Trait
 
This mech gains +1 Accuracy on all agility checks. It can climb with no penalty, takes no penalty  
for difficult terrain, and doesn’t make dangerous terrain tests.
 

Variable Knife  
Auxiliary Melee
 
AP
 
+1 vs evasion/tier with 1 Accuracy/tier
 
Threat 2
 
4/6/8 kinetic damage
 

Pulse Rifle  
Heavy rifle
 
+1 vs evasion/tier with 3 Accuracy
 
Range 12
 
7/9/11 energy damage
 

                                                                                                               


Optional systems:
 
Boltok Rifle  
Heavy CQB
 
Loading, Knockback 1
 
+1 vs evasion/tier
 
Range 3, Threat 3
 
10/15/20 kinetic damage
 
This heavy, archaic weapon replaces the Pulse Rifle.
 

Spinning Kick  
Trait, Quick Action
 
A target adjacent to the Assassin must pass a hull or agility check with 1 difficulty/tier or be  
knocked back 4 spaces away from the assassin and knocked prone.
 

Shroud module  
System, Quick Action
 
The Assassin creates a burst 3 zone of light cover within range 8. The zone disperses at the end  
of its next turn, or when the Assassin creates a new one. Mechs other than Assassins that enter  
the zone for the first time on their turns or start their turn there must pass a systems check with 1  
difficulty/tier or become Jammed until the end of their next turn.
 

Explosive Knives  
Quick Action
 
Limited (3)
 
A knife can be thrown at any surface or mech as a quick action. Make a ranged attack roll for +2  
vs evasion/tier, attaching on hit. At the start of the assassin’s next turn, the knife explodes,  
dealing 6/8/10 explosive damage in a burst 2 area centered on the knife or mech. A mech can  
remove and disarm the knife by passing a successful systems check as a quick action if they are  
adjacent to it.
 

Tracker Missiles
 
Quick Tech
 
+2 vs e-defense/tier
 
The Assassin makes a tech attack vs e-defense against a target in sensor range. On hit, the  
target has the missiles latch on to them. While attached, the target cannot hide or benefit from  
invisibility, and the Assassin always knows its location up to a 10 mile distance. The missiles can  
be removed during a rest or by taking a quick action and successfully repeating this check.
 

Tier II:
 

          Evade     E-D    Heat    H    A     S     E       Armor        Spd       Sense 

          15       8       7       +2   +3    +1    -1      1            8         12 

                                                                                                                


Tier III:
 

         Evade     E-D   Heat    H    A     S     E      Armor        Spd      Sense 

          17       8     7       +3   +4    +2   -1      2            8        12 


\subsection{Assault}
\begin{npc}{Assault}

\fluff{Assault doctrine mechs and their pilots are the most common main battle chassis found throughout the
galaxy. Fitted with a localized version of a main battle rifle, sidearm, and a suite of systems to enhance

movement, targeting, and defensive systems, an Assault Doctrine chassis is a straightforward, reliable,
hardy combatant.

Assault doctrine pilots are the cheapest to train and outfit: this does not make them any less of a threat
when paired with a kit of their choice.}

\npcBox
[hull = +1,
agility = +1,
systems = +1,
engineering = +1,
hp = 20,
evasion = 8,
e-defense = 8,
heat cap = 8,
armor = 1,
speed = 4,
sensors = 8,
size = 1]

\textbf{Base systems:}

\npcGearBox[name = {Heavy Assault Rifle},
template = {\Main \Rifle\\
Range 10\\
+2 vs evasion/tier with +1 Accuracy\\
7/9/11 \kinetic damage}]

\npcGearBox[name = {Combat Knife},
template = {\Auxiliary \Melee\\
Threat 1\\
+1 vs evasion/tier\\
5/7/8 \kinetic damage}]

\textbf{Optional systems:}

\npcGearBox[name = {Underslung Grenade Launcher},
template = {\Auxiliary \Launcher\\
Loading, Arcing\\
Range 8, Blast 2\\
4/6/8 \explosive damage}]

\npcGearBox[name = {Micro-missile Barrage},
template = {Quick action, Recharge (6+)},
rules = {This mech makes a line 8 ranged attack with +1 vs evasion/tier for 6/9/12 \explosive damage. The
closest target hit by this missile barrage must pass a hull check or be knocked prone.

At tier II onwards, this hull check is made with 1 difficulty.}]

\npcGearBox[name = {High Impact Rounds},
template = {System},
rules = {The Heavy Assault rifle can be fired with high caliber rounds, adding +4 \kinetic damage and the
\AP tag, but requiring a reload after firing it this way (as if it had the loading tag)}]

\npcGearBox[name = {Auto-targeting},
template = {System},
rules = {The Heavy Assault rifle ignores the effects of light and heavy cover}]

\npcGearBox[name = {Rank Discipline},
template = {System},
rules = {The Assault gains +1 Accuracy on all its attacks and checks as long as it's adjacent to at least
one friendly mech.}]

\textbf{Tier II:}
\tierBox
[hp = 23,
evasion = 10,
e-defense = 9,
heat cap = 8,
hull = +2,
agility = +2,
systems = +2,
engineering = +2,
armor = 1,
speed = 4,
sensors = 8]

\textbf{Tier III:}
\tierBox
[hp = 26,
evasion = 12,
e-defense = 10,
heat cap = 8,
hull = +3,
agility = +3,
systems = +3,
engineering = +3,
armor = 1,
speed = 4,
sensors = 8]

\end{npc}

\subsection{Barricade}
                                            BARRICADE

 BARRICADE

 Mech

 Hull       Agility     Systems       Engineering

 +0         -2          +1            +3

 HP         Evasion     E-defense     Heat Cap.

 16         6           9             10

 Armor      Speed       Sensors       Size

 1          4           10            1

Base Systems:
Cycle Lance
Main Cannon

Range 10

+0 vs evasion with 1 Accuracy/tier

6 energy damage

This weapon can attack twice at tier III with the attack action


Mobile Printer
Recharge (6+)
As a free action at the start of its turn, the barricade may print a line 3 section of size 3 cover (3
spaces high) in any adjacent and free space. The cover provides heavy cover and has evasion 5
and 30 HP.


Shock Plating

System

The Barricade is resistant to kinetic damage


Optional Modules:

Drag Drone
Quick Tech

+2 vs e-defense/tier

A target struck by this drone takes 2 AP energy damage for each 1 space that it voluntarily or
involuntarily moves until the start of the Barricade’s next turn. The target is aware of this effect.


Seismic Repulsor
Quick Action, recharge (5+)





All non-flying targets within a burst 3 area centered on the Barricade must pass a hull skill check
with 1 difficulty/tier or be knocked back 3 space directly away from the barricade and knocked
prone. The area in the blast then becomes difficult terrain permanently.


Rapid Extruder
The Barricade can print twice as fast (can print two sections of cover at once)


Mag Mines
Quick Action, Recharge 6+
The Barricade rapidly prints and deploys a field of tiny mines that seek out targets and clamp on
to them in droves, slowing them down and inhibiting movement. The mines take up a 5x5x5
cube. The area is both difficult and dangerous terrain, and any target other than the Barricade
that enters the area or starts its turn there must pass a systems check with 1 difficulty/tier or
become Slowed the end of its next turn.


Snare Drone
Quick Action, Drone

Limited (1)

The Barricade rapidly prints and deploys a snare drone in an adjacent space. This drone is size
1, has 10 evasion and 10 HP. When any hostile mech moves within range 3 of the drone or starts
its turn in range of the drone, it emits a pulse, immediately immobilizing that mech until the drone
is destroyed (no check allowed).


Tier II:

HP: +3


          Evade     E-D    Heat    H    A     S     E       Armor        Spd       Sense

          7         11     11      +1   -2    +1    +5      1            4         10

Tier III:
HP: +3


          Evade     E-D    Heat    H    A     S     E       Armor        Spd       Sense

          8         13     12      +2   -2    +2    +6      1            4         10


\subsection{Bastion}

                                               BASTION  

Bastion doctrine chassis trade the pure defensive edge of an Aegis doctrine chassis for one that  
allows more mobility. Blending area-denial offensive capability with hardened defense systems  
and advanced communication suites, Bastion doctrine chassis make for strong squadron  
commanders.   

 BASTION 

 Mech 

 Hull       Agility     Systems       Engineering 

 +2         -3          +0            +2 

 HP         Evasion      E-defense    Heat Cap. 

 16         6           8             8 

 Armor      Speed       Sensors       Size 

 3          4           8             1 

Base systems:
 
Rotary Grenade Launcher  
Main Launcher
 
Arcing, Loading
 
+1 vs evasion/tier
 
Range 8, Blast 1
 
8/11/14 explosive damage
 

Heavy Assault Shield  
Heavy Melee
 
Threat 1
 
+1 vs evasion/tier
 
3 kinetic damage/tier
 
As a quick action, the Bastion can deploy or retract its shield as a line 2, size 2 piece of heavy  
cover, disarming it of this weapon. The Bastion can leave the shield deployed and move away  
from it if it so wishes (it can pick it up again as a quick action). Any mech that gains the benefit of  
this cover against an attack also has resistance to all damage from that attack. The cover has  
evasion 5 and 20 HP and can be attacked and destroyed as normal.
 

Optional systems:  
Shard Launcher
 
System, Reaction
 

                                                                                                         


The first time a Bastion takes damage in a round, all mechs in a cone (3) area in front of it must  
pass an agility check or take 5/7/9 explosive damage, and half on a successful save.
 

Shieldwall  
Trait
 
As long as it’s holding its shield, the Bastion and one adjacent mech of its choice has resistance  
to all damage from a target of its choosing that it can see. It can only change this target at the  
start of its turn.
 

Immortality  
System, Full Action
 
Recharge (5+)
 
The Bastion and one adjacent mech of its choice are immune to all damage and effects until the  
start of the Bastion’s next turn, guarded by a flickering pane of Blackshield tech. The mech  
adjacent to the Bastion loses this benefit if it breaks adjacency. On the start of the Bastion’s next  
turn, it is stunned until the start of its following turn and cannot gain or benefit from resistance or  
immunity of any kind for the same duration.
 

Defender  
Trait, Reaction
 
Once a round, when an allied mech is damaged, if the mech is in movement range of the Bastion  
and the bastion is not immobilized, it can immediately move adjacent to that mech and take that  
damage instead of the targeted mech. This movement doesn't provoke reactions and ignores  
engagement.
 

Hypo-reinforcement servos
 
System, Shield, Protocol
 
Recharge (4+)
 
The bastion activates this system at the start of its turn as a free action if it has it available. While  
active, it reduces all damage from the very next melee or ranged attack it takes to 0. This effect  
can only stack once.
 

Tier II:
 
HP: +2
 

          Evade     E-D    Heat    H     A     S     E       Armor        Spd      Sense 

          7         8      9       +4    -3    +0   +3       3            4        8 

Tier III:  

          Evade     E-D    Heat    H     A     S     E       Armor        Spd      Sense 

          7         8      10      +6    -2    +0   +4       4            4        8 


\subsection{Berserker}
                                            BERSERKER

Berserker doctrine mechs build to take advantage of advanced heat cycling systems to shunt
system heat tax into an offensive force, increasing their weapon output by orders of magnitude.

       BERSERKER

       Mech

       Hull      Agility      Systems       Engineering

       +3         +2          -2            -1

       HP         Evasion     E-defense     Heat Cap.

       15         10          6             6

       Armor     Speed        Sensors       Size

       1          4           5             1

Base systems:
Chain Axe
Heavy melee

+2 vs evasion/tier with

Threat 1

7/10/14 kinetic damage

This weapon deals an extra +1d6 damage/tier on Critical Hits


Volcanic Re-route

System, Protocol

The Berserker can choose to activate or de-activate this system at the start of its turns. While
this system is active, the Berserker can make two attacks with its chain axe when it takes the
attack action, but it takes 4 heat at the end of its turn.


Optional systems:
Molten Shield
System, Shield, Reaction

Once per round, the Berserker can activate this dispersal shield when it takes damage from any
melee attack. The damage is reduced by half, and the attacker must pass an engineering check
with 1 Difficulty/tier or take Burn 2/tier


Harpoon

Main CQB

+1 vs evasion/tier





Range 5, Threat 3

2/4/6 kinetic damage

Targets struck by this weapon the same size or smaller than the Berserker are pulled in a straight
line adjacent to the Berserker or as far as possible. If the target is pulled adjacent to the
Berserker, the Berserker grapples them automatically.


Calamity module
System, Full Action

The Berserker can only activate this system against a target it is grappling. Auxiliary strength
servos engage and the Berserker hurls its target with incredible violence. The target must make a
hull check with 1 Difficulty/tier. On a failed check, it takes 8/12/16 kinetic damage, is knocked
back 4 spaces away from the berserker, and is stunned until the end of its next turn. On a
successful check it takes the damage and knockback, but is not stunned. The Berserker ends its
grab after making this move, success or failure.


Nail Gun
Main CQB

+1 vs evasion/tier

Range 5, Threat 3

5/7/9 kinetic damage

Targets struck by this weapon must pass an engineering check with 1 difficulty/tier or become
immobilized until the end of their next turn.


Hunter jets
System, Reaction

Once per round, as a reaction to any enemy movement, the Berserker can make the boost
action.


Tier II:

HP: +5


          Evade     E-D    Heat    H    A     S     E       Armor        Spd       Sense

          11       6      6        +4   +4    -2    -1       1           4         5

Tier III:
HP: +5


          Evade     E-D    Heat    H    A     S     E       Armor        Spd       Sense

          12       6      6        +6   +4    -2    +0       1           5         5


\subsection{Bombard}

                                             BOMBARD

The Bombard doctrine calls for unending salvos of punishing artillery bombardments. Chassis
tuned to this style of combat are found miles behind the line in fortified positions, or otherwise
removed from combat: one common strategy for prolonged area-denial bombardment is to place
a battery of bombard chassis on a world’s local moon(s). From that movable satellite, the battery
can rain FRAMEs down upon the world below, well removed from the dangers of combat.


       BOMBARD

       Mech

       Hull      Agility     Systems       Engineering

       +0        -1          +1            +2

       HP        Evasion     E-defense     Heat Cap.

       20        7            12           8

      Armor      Speed       Sensors       Size

       0         2            11           2

Base systems:
Bombard cannon
Superheavy cannon
Ordnance, Arcing, Loading

+2 vs evasion/tier with 1 difficulty

Range 20, Blast 2

10/14/18 kinetic damage


Stabilizers
System, Full Action

As a full action, this mech can engage or disengage its stabilizers. While they are active, the
mech cannot move or be knocked prone, but it gains +1 accuracy and +10 range on its attacks
and cannot fire any weapons at a target within range 5 of itself. If this system is destroyed, the
mech is immobilized.


Molded Armor
System

The Bombard has resistance to explosive damage


Optional systems:
H.A. Siege Shield




System, Shield, Passive

The mech has resistance to all damage that it takes from attacks further away than range 5.


High-Impact shells
Trait

The Bombard cannon gains knock back 3


LMG

Main Cannon

One or two targets

Range 10

+1 vs evasion/tier

4/6/8 kinetic damage


Cluster-seeker Bombs
System

After the artillery fires its bombard cannon, 3 additional targets within range 10 of the primary
target take 2 explosive damage/tier (no check or roll required).


Devastator Protocol
Trait, Full Action, recharge (6+)

Choose 1d3 points within range of the mech, and mark them (so they are clearly visible to all
players). The bombard must be activated last next round. At the start of its next turn, the mech
fires at each of those points with its bombard cannon, as a free action, attacking all targets and
structures within. It cannot choose these points so that the blast area from these attacks overlap
in any way.


Tier II:

HP: +5


          Evade    E-D    Heat     H    A     S     E       Armor        Spd      Sense

          8        14     8        +1   -1    +2    +3      0            3        11

Tier III:
HP: +5


          Evade    E-D    Heat     H    A     S     E       Armor        Spd      Sense

          8        16     8        +1   -1    +3    +5      0            3        11


\subsection{Breacher}

                                              BREACHER  

 BREACHER 

 Mech 

 Hull       Agility      Systems       Engineering 

 +2         +0           -1            +1 

 HP         Evasion      E-defense     Heat Cap. 

 18         9            7             8 

 Armor      Speed        Sensors       Size 

 1          4            5             1 

Base Systems  
Dual Shotguns  
Main CQB
 
+1 vs evasion/tier with 2 difficulty
 
Range 3, Threat 3
 
6/8/12 kinetic damage
 
The Breacher can attack twice with this weapon with the attack action, choosing the same or  
different targets.
 

BREACH Ram  
Quick action, Recharge (5+)
 
The breacher moves forward in a straight line as far as it can, up to its speed. It ignores  
engagement and obstruction (even for obstacles or mechs passed through). Obstacles or  
objects such as cover are punched or smashed through and take 10 AP kinetic damage/tier. If  
the obstacle is an especially hardy composition (such as a starship hull) the breacher can make a  
successful hull check to smash through instead. All other targets passed through must pass a  
hull check with 1 difficulty or be knocked out of the breacher’s path and knocked prone.
 

Optional Systems:
 
Unload  
Full Action
 
The breacher chooses a target in range of its dual shotguns. At the start of the breacher’s next  
turn, if the target is still in range, the breacher can attack that target with its dual shotguns with  
+3 Accuracy (for a total of +2 Accuracy by default) as a free action.
 

Loaded for Bear  
Trait
 

                                                                                                          


A target struck by both the Breacher’s Dual shotguns becomes shredded until the end of the  
Breacher’s next turn
 

Hunter Lock  
Quick Tech
 
+1 vs e-defense/tier with +1 Accuracy
 
The Breacher chooses a target in sensor range and makes a tech attack. On hit, all its attacks  
gain +1 accuracy/tier against that target, and 1/turn it can take the boost action as a Free Action,  
as long as it is made directly towards that target. The Breacher cannot change its target until it or  
the target is destroyed.
 

Thermal Charge  
Quick Action
 
Limited (1), Thrown 5
 
Blast 2
 
2/4/6 explosive damage + 5 Burn/tier
 

Hi Caliber Slugs  
The breacher takes 1 heat when its fires its dual shotguns, but they deal +3 damage
 

Tier II:
 
HP: +2
 

          Evade     E-D    Heat    H    A     S     E       Armor        Spd       Sense 

          11        7      8       +3   +1    -1    +2       1           4         5 

Tier III:  
HP: +2
 

          Evade     E-D    Heat    H    A     S     E       Armor        Spd       Sense 

          13        7      8       +4   +2    -1    +3       1           4         5 

                                                                                                                

\section{Cataphract}
                                            CATAPHRACT

The Cataphract doctrine is common among rapid-strike kits that emphasize mobility, shock, and
tenacity. Cataphract squadrons are feared across the galaxy for their ability to overwhelm
defenses in moments: from an otherwise static line, a squadron of low and angular chassis burst
overhead, heavy cannons and PDF guns carving seemingly impossible paths through a suddenly
futile defense.


 Cataphract

 Mech

 Hull       Agility      Systems       Engineering

 +1         +1           +0            +0

 HP         Evasion      E-defense     Heat Cap.

 16         10           8             9

 Armor      Speed        Sensors       Size

 0          6            8             1-2

Base systems:

Ram cannon
Heavy Melee/Heavy Cannon

+0 vs evasion with 1 Accuracy/tier

Range 8, Threat 2

5/7/9 kinetic damage

This lance-like weapon can be fired with either profile (used as a melee or ranged weapon), but
not both in the same turn.


Impact thrusters
System, Full Action

Recharge (5+)

The Cataphract charges, moving 6 in a straight line in a direction of its choosing. It can ignore
obstructions caused by enemies. Hostile targets that it passes through must pass a hull or agility
check or take 2 kinetic damage/tier and be knocked prone.


Optional systems:
Rotary Barrels
System

1/round when the Cataphract hits with a Ram Cannon attack, it can repeat the attack roll against
a different target in range with +2 difficulty.





Lance shot
System, Full Action, Recharge (5+)

The Cataphract can fire a lance shell from its ram cannon. This changes the attack type of the
weapon to line 10 and causes all affected mechs to pass an agility skill check to dodge the shot
or become immobilized until the start of the Cataphract’s next turn.


High-speed targeting
System

The Cataphract can make one Ram Cannon attack (melee or ranged) after taking the dash
action, but takes 1 difficulty on the attack roll.


Point-defense shield
System, Shield, Quick Tech

+2 vs e-defense/tier

The Cataphract makes a tech attack against a target in range. On hit, the Cataphract has
resistance to all damage from that target. It can only have one actor effected at a time.


Capacitor discharge
System

After the cataphract takes the boost action, all mechs it is adjacent to after the action completes
take 2/3/4 heat.


Tier II:

HP: +4


          Evade    E-D    Heat     H    A     S     E       Armor        Spd      Sense

          11       8      9        +3   +2    +0    +0      0            7        8

Tier III:
HP: + 4


          Evade    E-D    Heat     H    A     S     E       Armor        Spd      Sense

          12       8      9        +4   +4    +0    +0      0            7        8



\subsection{Demolisher}

                                            DEMOLISHER

The prolonged siege of Jadigmora City saw the development of the DEMOLISHER patten. Based
off the long-operational berserker doctrine, DEMOLISHER tuned the chaotic heat flow to a more
sustainable, if limited system, increasing a chassis’ heavy-lift capacity to allow for supermassive
kinetic weapons to be used effectively in combat. Combined with the increased pilot shielding
and system hardening made necessary by the heat tuning, pilots soon discovered more
aggressive applications for concussion-wave ordinance.


 Demolisher

 Mech

 Hull       Agility      Systems       Engineering

 +2         -2           +0            +2

 HP         Evasion      E-defense     Heat Cap.

 16         6            7             8

 Armor      Speed        Sensors       Size

 2          4            8             2

Base systems:

Demolisher hammer
Superheavy Melee

Threat 2

AP, Knockback 2

-2 vs evasion with 1 difficulty (no difficulty at tier II, +1 Accuracy at tier III)

13/18/25 explosive damage

This weapon deals double damage against structures, objects, the environment, and cover and
gains +2 Accuracy against them.


Kinetic compensation
System

If the Demolisher misses with its hammer attack, it gains +1 Accuracy on subsequent attacks
until it hits. This effect stacks.


Optional systems:
Broad-sweep haft
Full Action

The Demolisher’s hammer can be used to make a sweep attack instead of a regular attack,
attacking all targets in threat range (allied or enemy) for +1 vs evasion/tier, 4/6/8 explosive
damage. Targets struck by this attack are impaired until the start of the Demolisher’s next turn.





Concussion missiles
Quick Action

A target in range 8 must pass an engineering check with 1 difficulty/tier or be knocked back 3
spaces and impaired until the start of the Demolisher’s next turn.


Seismic Destroyer

System, Full Action, Recharge (6+)

The Demolisher’s hammer can be used to make a special attack that hits cone 3 or line 5 from
the Demolisher instead of its regular attack. This attack cannot hit flying targets. Targets in the
area must pass a hull or agility skill check or be stunned until the end of the Demolisher’s next
turn.


Shatter module
Trait

The Demolisher’s hammer strikes against prone, immobilized, or stunned targets gain the AP tag
and can Critical Hit, dealing +1d6 bonus damage/tier on Critical Hits


Knockout Blow
Trait, Quick Action, Recharge 6+
A target adjacent to the Demolisher must pass a hull or agility check with 1 difficulty/tier or
become stunned until the end of the Demolisher’s next turn.


Tier II:

HP: +4


          Evade     E-D    Heat    H     A     S     E       Armor        Spd       Sense

          6         7      8       +4    -2    +0    +3      3            4         8

Tier III:
HP: +4


          Evade     E-D    Heat    H     A     S     E       Armor        Spd       Sense

          6         7      8       +6    -2    +0    +4      3            5         8


\subsection{Engineer}

                                                   Engineer

Pilots are creative and driven individuals, but you’re an exceptional case - some would say to the point of

excess. In your spare time you’ve managed to scrape together just enough scrap, requisitioned materials,
and workshop space to apply a little old fashioned ingenuity to your mech.

Spark (Rank I)
On a full repair, you can, with some trial and error, install a prototype weapon system on your
mech. It has the following profile -- the specifics you may choose upon installation. Roll the
Limited property each full repair. You can change the profile of this weapon each time you full
repair.

         Prototype Weapon

         Main (Choose 1; Melee, Rifle, Cannon, Launcher, CQB)

         Limited (1d6+2), Overcharged

         Threat 1 (melee) or range 8 (ranged)

         1d6 kinetic, explosive, or energy damage

This weapon does not take a mount to add to your mech, and counts as its own mount.

Updated Plans  (Rank II)
On a full repair, you can remove and tweak essential components of your system in order to
increase the effectiveness of your prototype weapon.

Choose 1 of the following:

  - Tweak optics:
         You always fire your prototype weapon with +1 Accuracy.

  - Tweak Computer





         Your prototype weapon gains the Smart property.

  - Removing reactor shielding

         Your prototype weapon can fire as a cone 3 weapon, a line 5 weapon, or a blast 1
         weapon (choose when you fire), but costs 2 heat to fire

Final Draft (Rank III): Your prototype weapon is now limited (2d6+1) and does 1d6+2 damage.


\section{Hive}

                                                   HIVE

A recent development following the Deimos Contact Event, Hive doctrine chassis are, in theory,
not too different than their pre-Contact cousins: they establish a local, secure omninetwork,
fabricate and deploy a century or half-century of drones, and coordinate them in achieving their
mission. The difference between pre-Contact and post-Contact drone controllers is ease: pilots
who pursue this doctrine establish complex handler-trainer relationships with their hives, allowing
them to segue between direct control and autonomous operation unimpeded by tactile interface.
Hive pilots exhibit signs of psychological trauma at one month continuous drone-strain operation,
and it is recommended that they practice a strict two-week format cycling with their paired hive.

 HIVE

 Mech

 Hull       Agility      Systems       Engineering

 +1         -1           +2            +0

 HP         Evasion      E-defense     Heat Cap.

 20         8            10            8

 Armor      Speed        Sensors       Size

 0          5            15            1

Base systems:
Hunter Killer Drone Nexus

Main Nexus

Smart, Seeking

Range 15

+2 vs e-defense/tier

7 energy damage

This weapon can attack twice at tier III with the attack action


Drone Barrage

Quick Tech

+2 vs e-defense/tier with +1 accuracy/tier

The Hive directs its drones to run interference on a target of its choice. On hit, the target is
Slowed and Impaired until the start of the hive’s next turn.


Optional systems:

Electro-Nanite Cloud
System





All hostile targets that start their turn within range 3 of the Hive take 2 Burn/tier. In addition, any
system checks they make or tech actions they make in that area suffer from +1 Difficulty/tier.


Grinder Drones
Drone, System

Quick Tech

+2 vs e-defense/tier with +1 Accuracy/tier

The Hive releases a swarm of tiny, hard to hit drones that attach to a target within range 15,
making a tech attack. On hit, the target takes Burn 4/6/8. The Hive can only have one target
attacked by these drones at once.


Razor Swarm
Drone, System, Quick Action, Recharge 5+

The Hive creates a blast 2 area within sensor range. The area remains until the end of combat or
the Hive is destroyed. Targets allied to the hive can use the area for light cover. Otherwise, any
target that starts its turn in the area or moves into it for the first time on its turn takes 3 Burn/tier.
The Hive can deploy any number of Razor Swarms.


Swarm Shield
Quick Action, Recharge 6+

The Hive releases a cloud of miniature drones that cluster around it, providing heavy protection.
The Hive has resistance to damage from the next 1d6 attacks.


Seeker Cloud
Main Nexus

Smart

+0 vs e-defense with 1 Accuracy/tier

Cone 5

3/4/5 Kinetic Damage

Targets damaged by this weapon gain the Lock On condition.


Tier II:

HP: +4


          Evade    E-D    Heat    H    A     S    E       Armor        Spd      Sense

         9         12     8       +2   -1    +4   +0       0           5        20

Tier III:
HP: +4


          Evade    E-D    Heat    H    A     S    E       Armor        Spd      Sense

          10       14     8       +3   -1    +6   +0       0           6        25


\subsection{Hornet}

                                                HORNET

 HORNET

 Mech

 Hull       Agility      Systems       Engineering

 -2         +2           +2            +0

 HP         Evasion      E-defense     Heat Cap.

 10         15           8             5

 Armor      Speed        Sensors       Size

 0          8            10            1/2

Base Systems:
Stinger Pistol
Auxiliary CQB

Range 8, Threat 3

+1 vs evasion/tier with +1 accuracy/tier

2/3/4 energy damage

Targets critically hit by this pistol are impaired until the end of the hornet’s next turn.


SSC Total Suite
The Hornet can fly when it moves or boosts with hover flight.


Impale Systems
Quick Tech, Recharge 5+

+2 vs e-defense/tier

On hit, the target takes 3 heat/tier and becomes jammed until the end of the hornet’s next turn.


Optional Systems:
Evade Suite
Trait

The Hornet ignores engagement and is immune to grapples


Minor Basilisk
Trait

The first attack a target misses against the Hornet per turn deals 2 heat/tier to the attacker.


Interdictor Suite

Reaction





1/round the Hornet can use a reaction to attempt to electronically jam a target in its sensor range
that is attempting to make an attack roll. The target must pass a systems skill check or lose its
attack roll (it can still make other attacks as normal). The difficulty of this test increases by 1
difficulty at tier II onwards.


HEX missiles

Auxiliary Launcher

Smart

+0 vs e-defense with +1 Accuracy/tier

Range 10, Blast 1

3/4/5 heat


Drag Javelin
Full Action

The Hornet fires a javelin at a target it can see within range 10. The target must pass a hull check
with 1 difficulty/tier or become impaled by the javelin. Upon impact, the javelin fires a secondary
grapple at the environment, tethering the target to the ground. While impaled this way, the target
is immobilized and shredded. It can end this condition by repeating the check as a quick action
on their turn to pull the javelin out.


Tier II:


          Evade     E-D    Heat    H     A     S     E       Armor         Spd      Sense

          17        8      6       -2    +4    +3    +0      0             10       10

Tier III:


          Evade     E-D    Heat    H     A     S     E       Armor         Spd      Sense

          20        9      7       -2    +6    +4    +0       1            12       10



\subsection{Marker}
                                                MARKER

Marker doctrine calls for the active, close, and aggressive application of orbital, atmospheric, and
terrestrial fire support on identified and potential enemy positions. Equipped with more-than-
cursory targeting systems, Marker doctrine pilots adopt an tactical/artillery commander role,
operating on the ground with infantry, chassis, and other vehicles to build a map of the
battlescape in order to more accurately place small arms fire, bombs, shells, missiles, beams, and
kinetic kill clouds. They are not usually a threat in face-to-face combat, but the threat they
represent to a combat cannot be underestimated.

       Marker

       Mech

       Hull       Agility     Systems       Engineering

       -2         +2          +3            -1

       HP         Evasion     E-defense     Heat Cap.

       14         12          10            8

       Armor      Speed       Sensors       Size

       0          6           20            1/2 - 1

Base systems:

High Caliber Pistol
Auxiliary CQB

+2 vs evasion/tier

Range 8, Threat 3

3/4/5 kinetic damage


Target Marker
Main rifle

Smart

+1 vs e-defense/tier with 1 Accuracy

Range 20

Targets hit by this attack immediately gain Lock On and are Shredded while they suffer from
Lock On. They then cannot turn invisible or benefit from invisibility, and cannot hide until the end
of the Marker’s next turn. If the Marker is destroyed or this weapon is disabled, immediately end
this condition.


Optional systems:




Orbital Strike
System, recharge 5+

Instead of firing the target marker at an enemy, the Marker can fire it at a point within range 30.
At the start of the Marker’s next turn, that area is hit by an orbital strike. Any mechs still in a burst
3 area centered on that point must pass an agility check or take 12/17/22 energy damage and be
knocked prone, or half and no prone on a successful check.


HOUND missile

Quick Action, Recharge (6+)

Choose a target within range 20. The marker fires a hound missile at the target, which is size 1/2,
has evasion 10, and 10/15/20 HP. The missile primes on the turn that it is fired, deploying in an
adjacent space, then moves 3 at the start of each of the marker’s turns. If its target suffers from
the Lock On condition, its movement increases to 6. The missile can benefit from cover, counts
as flying with hover flight, and can be targeted and shot by systems and weapons. It must move
towards its target, but can maneuver skillfully around cover, fit through holes, etc. If the missile’s
movement causes it to collide with a hostile mech or its target, it detonates for a blast 1
explosion. Mechs caught inside must pass an agility skill check or take 15/20/25 energy damage
and half on a successful check.


Blind
Quick Tech, Recharge (4+)

+2 vs e-defense with +1 Accuracy/tier

On hit, the target only has line of sight to adjacent squares until the end of its next turn.


Rebound Scan
System, Full Action

All targets in range 10 of the marker must pass a systems check or immediately lose the benefit
of all cover, hiding, and invisibility and be unable to take cover, hide, or turn invisible until the
start of the Marker’s next turn.


Smart Missile Cloud
System

At the start of the marker’s turn, one target suffering from Lock On in the Marker’s sensor range
takes 4/5/6 explosive damage from the Marker, no roll required.

Tier II:


          Evade     E-D    Heat    H    A     S     E       Armor        Spd       Sense

          15        13    8        -2   +4    +4    -1      0            7         20

Tier III:

          Evade     E-D    Heat    H    A     S     E       Armor        Spd       Sense

          18        16    8        -2   +5    +6    -1      0            8         20


\subsection{Operator}

                                              OPERATOR

Operator doctrine chassis are a known-unknown. Operators typically field smaller chassis not
commonly encountered on the front lines; do not mistake this for fragility. Pilots who are recruited
and trained into this doctrine, regardless of culture, are dangerous, deadly, the best of their
state’s armed forces or martial tradition. They operate alone or in small teams under the auspices
of black-site state agencies, engaging in the most sensitive and dangerous AMNESIAC-tier
missions. Their chassis — and their bodies — are loaded with some of the most advanced tech
available; if they die in combat, it is not uncommon for their bodies and their chassis to self-
immolate, rendering what technology and data they had into waste and ash.


 OPERATOR

 Mech

 Hull       Agility      Systems       Engineering

 +2         +2           +2            +2

 HP         Evasion      E-defense     Heat Cap.

 14         10           10            9

 Armor      Speed        Sensors       Size

 0          4            10            1

Base systems:

‘Raptor’ plasma rifle
Heavy Rifle

+2 vs evasion/tier

Range 15

8 energy damage

This weapon can attack twice at tier III with the attack action


‘Nightcloak’ type omnishield
System, Shield

While this system is active, the operative has resistance to one of the following types of damage:
kinetic, explosive, energy, heat. The operative must decide which when it starts combat (setting it
is a Free Action) and cannot change it during combat.


Self-erasing
Trait

When this mech is destroyed, it immediately self-immolates into superheated plasma. All targets
adjacent to it when it is destroyed must pass an agility check or take 4/6/8 energy damage or




half on a successful check. This mech is then removed from the battlefield - it is utterly
annihilated.


Optional systems:
Fade generator
System

While this system is intact, at the end of its turn, the Operative becomes invisible until the start of
its next turn. This effect is immediately disabled if the Operative takes damage until the start of
the Operator’s next turn.


Refractive shield
System, Shield

This system is permanently disabled if the operative overheats. While it is active, the operative
gains the Hardened Target trait (scans, lock on attempts, and invasions against it are made at +1
difficulty).


Skirmisher implant
System

Once at any point on its turn, the operative can make a boost action as a free action.


Nova missile
Auxiliary Launcher

Smart, Seeking

+1 vs e-defense with 1 Accuracy/tier

Range 30, blast 1

4/6/8 energy damage


Trace Drive
System

Once on its turn, when the Operative moves or boosts, it instead teleports up to 8 spaces away,
as long as its target destination is a free space. This movement does not provoke reactions and
ignores engagement, such as overwatch.


Tier II:


          Evade    E-D    Heat     H    A     S     E       Armor        Spd      Sense

          12       12     9       +3    +3    +3    +3      1            4        10

Tier III:


          Evade    E-D    Heat     H    A     S     E       Armor        Spd      Sense

          14       14     9       +4    +4    +4    +4      2            5        10


\section{Pyro}

                                                   PYRO  

A combat doctrine adopted following the Hercynian Crisis, the Pyro chassis-pattern is reviled  
across the galaxy as a terror instrument, though some states and organizations still choose to  
integrate them into their armies. Pyro doctrine chassis are heavily armored and insulated, built to  
manage incredible heat tax while projecting volatile mixes of irrepressible flame towards their  
enemies. Pyro chassis are sent in to root out entrenched defenders, defoliate areas rich in flora,  
and cause terror. 
 

 PYRO 

 Mech 

 Hull       Agility      Systems       Engineering 

 +1         -1           -1            +3 

 HP         Evasion      E-defense     Heat Cap. 

 12         8            8             15 

 Armor      Speed        Sensors       Size 

 3          3            8             2 

Base systems:
 
Flamethrower  
Heavy CQB
 
Cone 5
 
+1 vs evasion/tier
 
4 heat (self)
 
Burn 5/7/9 + 2/4/6 heat
 

Explosive Vent
 
System, Full Action
 
The pyro cools to 0 heat. Targets caught in a burst 1 area around it must pass an engineering  
check or take half of the heat the pyro cools as heat to themselves and be knocked prone.
 

Optional systems:  
Insulated Plating  
The pyro has resistance to heat and energy damage (including its self inflicted heat). It is immune  
to Burn
 

Unshielded Reactor  

                                                                                                           


Targets that start their turn adjacent to the pyro or become adjacent to the pyro for the first time  
on their turns take 1 heat/tier.
 

Napalm Bomb  
Main Launcher
 
Arcing, Loading
 
-1 vs evasion with +1 Accuracy/tier
 
Range 8, Blast 1
 
Burn 3/4/5 + 2/4/6 heat
 

Rigged fuel rod launcher  
Heavy Launcher
 
-2 vs evasion with +1 Accuracy/tier
 
Knockback 1
 
Range 10
 
4/6/8 heat
 

Explosive Jet  
System, Quick action, recharge (5+)
 
Enemies in a burst 2 area around the Pyro must pass a hull or agility check or be knocked back  
outside of that area and knocked prone. The Pyro then flies 5 in any direction, but it must land  
after completing that move.
 

Tier II:
 
HP: +3
 

          Evade    E-D    Heat    H     A    S     E       Armor        Spd      Sense 

          8        8      15      +2    +0   -1    +4       3           3         8 

Tier III:  
HP: +3
 

          Evade    E-D    Heat    H     A    S     E       Armor        Spd      Sense 

          8        8      15      +3    +0   +0    +5       4           3         8 

                                                                                                               

\subsection{Rainmaker}

                                             RAINMAKER

The tongue-in-cheek designation describes a combat doctrine that patterns chassis around the
optimal use of rockets and missiles in combat. Favored by all-theater combat units, Rainmakers
are mobile platforms loaded with ordinance, able to handle engagements at any range, against
any target.


       RAINMAKER

       Mech

       Hull       Agility     Systems       Engineering

       +0         +1          +2            -1

       HP         Evasion     E-defense     Heat Cap.

       15         8           10            8

       Armor      Speed       Sensors       Size

       1          2           15            2

Base systems:

Thundershock missile pods

Main launcher

Arcing, Knockback 1

+2 vs evasion/tier

One, two, or three targets in range 15

3/4/5 explosive damage


`Drang' missile rack

Heavy launcher

Ordnance, Loading

Line 15

+1 vs evasion/tier

8/11/14 explosive damage


Optional systems:
HADES missile

Quick Action, Recharge (5+)

The rainmaker targets a cone 3 area drawn in any direction from a point within range 20 of its
location. This system tracks line of sight as if it were an arcing weapon. All targets in that area
must pass an agility check or take 3/5/7 energy damage and 2/4/6 heat, and half on a successful
check.





Rigged payload
System

When the rainmaker is destroyed, it explodes in a burst 2 explosion centered on itself. Targets
caught within must pass an agility check or take 8/12/16 explosive damage, and half on a
successful check.


Volley
System, Full Action, Recharge (5+)

All hostile targets within range 30 of the rainmaker must pass an agility check or take 4/6/8
explosive damage, or half on a successful check. This system tracks line of sight as if it was a
seeking weapon.


Huntsman
Trait

While firing at a target suffering from the Lock On condition, the rainmaker's weapons gain the
Smart and Seeking tags.


Atlas missile
Superheavy launcher

Ordnance, Arcing

Range 30

This weapon system replaces the Drang Missile rack. Instead of targeting a mech, the rainmaker
instead targets a space on the ground within range. The targeted area is visible to all in line of
sight. At the start of its next turn, the missile lands, targeting a burst 2 area centered on that
space. All targets caught within must pass an agility check or take 15/20/25 explosive damage,
and half on a successful check.


Tier II:

HP: +2


          Evade     E-D    Heat    H    A     S     E       Armor        Spd      Sense

          11        11    8        +0   +2    +3    +0      1            2        20

Tier III:
HP: +2


          Evade     E-D    Heat    H    A     S     E       Armor        Spd      Sense

          12        12    8        +0   +3    +4    +1      2            3        20


\section{Ronin}

                                                  RONIN  

Ronin doctrine chassis differ from Berserker and DEMOLISHER pattern chassis in that they have  
been purpose-built by boutique fabricators to excel in melee combat. Ronin patterns are  
common among martial cultures and event-combat firms both. Tactical applications for Ronin  
chassis are difficult in cultures where ranged weapons are preferred, but there is a certain  
prestige earned by pilots who adopt ancient weapons in the modern day and survive. That being  
said the availability of stasis and mag based defensive technology has made the Ronin doctrine  
marginally more viable in modern combat. 
 

 RONIN 

 Mech 

 Hull       Agility      Systems       Engineering 

 +1         +2           -1            -1 

 HP         Evasion      E-defense     Heat Cap. 

 18         10           7             8 

 Armor      Speed        Sensors       Size 

 0          5            12            1 

Base systems:  
Carbon Fiber Sword  
Main Melee
 
+2 vs evasion with 1 Accuracy/tier
 
Threat 1
 
6 kinetic damage
 
This weapon can attack once per tier (up to 3 times at tier 3) with the attack action
 
This weapon can Critical Hit, and deals an extra +1d6 kinetic damage on Critical Hit
 

Mag Parry  
System, Shield, Reaction
 
Once per round, when damaged by a ranged weapon, the Ronin can roll a d6. On a roll of a 5+,  
the Ronin gains resistance to all the damage from that attack, and the target must repeat the  
attack roll against itself, dealing damage on a success.
 

Optional systems:  
Charged Slash  
Full Action, recharge (5+)
 

                                                                                                          


All targets adjacent to the Ronin must make a successful agility check or take 8/12/16 kinetic  
damage, and half on a successful check.
 

Reflex Implant  
System
 
When the Ronin boosts, attackers gain +2 Difficulty to attack it until the start of the Ronin’s next  
turn.
 

Echo Edge  
System
 
When the Ronin damages a target, it marks that target (keep track of it). At the start of its turn,  
the Ronin can consume all these marks as a free action to deal 1d6 AP kinetic damage to all  
targets, no attack roll required. They last until consumed, until the Ronin is destroyed, or until the  
end of the current challenge.
 

Hyper-reflex mode  
System, Quick Action, Recharge (5+)
 
This system remains active until the end of the Ronin’s next turn. While active, the first time each  
turn (the Ronin’s turn or any other actor’s turn) the Ronin is targeted by an attack, it can  
immediately make a Carbon Fiber Sword attack against a target in range before the attack is  
made.
 

Extended Blade  
The Ronin’s Carbon Fiber Sword becomes threat 2. The first time it critically hits with a melee  
attack on a turn, all targets within its threat take 3/4/5 kinetic damage including the target of its  
attack.
 

Tier II:
 
HP: +2
 

          Evade    E-D    Heat     H    A     S     E       Armor        Spd      Sense 

          13       7      8        +2   +4    -1    -1      0            5        12 

Tier III:  
HP: +2
 

          Evade    E-D    Heat     H    A     S     E       Armor        Spd      Sense 

          16       7      8        +4   +5    -1    -1      0            6        12 

                                                                                                                

\section{Scourer}

                                              SCOURER

SCOURER doctrine mechs mount massive-output recursive power plants in order to field deadly
energy weapons. A common specialist-role doctrine, SCOURER chassis are found on fronts
across the galaxy, supporting kinetic-focus ground troops with powerful lensing attacks that
target not only corporeal enemies, but systemic threats.


 SCOURER

 Mech

 Hull       Agility     Systems       Engineering

 +1         +0           0            +1

 HP         Evasion      E-defense    Heat Cap.

 16         9            8            10

 Armor      Speed       Sensors       Size

 1          4            10           1

Base systems:

Thermal Lance
Heavy Cannon

2 heat (self)

+0 vs evasion with +1 Accuracy at tier II and +2 Accuracy at tier III

Range 8

6/8/12 energy damage


Cooling Module
System
If the Scourer did not move or boost at all this turn, it reduces its heat to 0


Optional systems:
Crystal Lense
Once the Scourer hits a target with the Thermal Lance, on subsequent turns it can choose to
focus down that target instead of firing normally. As a full action from the Scourer, that target
takes 6/9/12 Burn, no roll required. This effect breaks if the target gains cover from the Scourer,
the scourer is stunned, this weapon is disabled or destroyed, or the target moves out of range.
The target is aware of this effect.


Supercharged

Trait





Targets struck by the thermal lance are shredded until the end of the scourer’s next turn.


Pulse Laser

Main Cannon

+2 vs evasion/tier

Line 10

4 energy damage

This weapon can attack twice at tier III


Flash Lense
System, Quick Action, Recharge (6+)

The Scourer targets a cone 5 area in a direction of its choosing. All targets in the area must pass
a systems check or be jammed until the start of the Scourer’s next turn.


Melt
System, Quick Action, Recharge (5+)

The scourer targets a piece of cover, terrain, or deployable, then makes an attack roll, dealing
10/20/30 AP energy damage on hit.


Tier II:

HP: +2


          Evade    E-D    Heat     H    A     S     E       Armor        Spd      Sense

          10       8      12       +2   +1    +0    +3      1            4        10

Tier III:


          Evade    E-D    Heat     H    A     S     E       Armor        Spd      Sense

          10       8      12       +3   +1    +1    +4      2            4        10


\subsection{Seeder}

                                                SEEDER  

Seeders are sapper chassis, adept at building defensive nets of mines and traps for point defense  
and aggressive area denial. Laden with ordinance, pilots who adopt Seeder doctrines operate in  
small teams to blanket the battlefield in clouds of fire and shrapnel, placing their launched  
grenades and explosive-tipped FRAMEs with pinpoint accuracy. 
 

 SEEDER 

 Mech 

 Hull       Agility      Systems       Engineering 

 +0         +0           +2            +0 

 HP         Evasion      E-defense     Heat Cap. 

 15         7            10            10 

 Armor      Speed        Sensors       Size 

 2          4            15            1 

Base systems:
 
Grav Grenade Launcher  
Main Launcher
 
Arcing
 
+2 vs evasion/tier
 
Range 10, blast 1
 
2/3/4 explosive damage
 
This weapon deals 5/7/9 damage if it catches more than 1 target in its blast
 

Mine Deployer  
System, free action, recharge (4+)
 
The seeder lays a mine in a space within range 3 of it. It cannot place a mine adjacent to another  
mine. The mine arms at the end of the round. Once armed, the next target other than the seeder  
to become adjacent to it must pass a systems check or trigger the mine. That target, and any  
caught in a burst 1 area centered on the mine must pass an agility check or take 10/15/20  
explosive damage, and half on a successful check.
 

Optional systems:  
Smart Tagging
 
System
 
The seeder’s mines only detonate automatically if there is a hostile target in range (not allied).  
Furthermore, the seeder can choose to stop their detonation as a reaction.
 

                                                                                                          


Grav Mine  
System, Quick Action
 
A target in range 5 and line of sight to the Seeder must pass a systems check or have a mine  
attached to them. They can disarm the mine by taking a quick action and successfully repeating  
this check. The seeder can detonate any grav mines at the start of its turn as a free action,  
causing any affected mech to take 3/5/7 explosive damage, no attack roll required, and be  
knocked 3 spaces in a direction of the Seeder’s choosing.
 

Anti-Infantry mine  
Trait
 
If the seeder chooses, it can lay an anti-infantry mine with its mine deployer. This mine affects a  
burst 3 area instead of burst 1 and deals 1/2/3 explosive damage, but 4/6/8 to targets with the  
biological tag.
 

Grav Lash
 
Quick Tech
 
+2 vs e-defense/tier 
 
On hit, the target is knocked back 5 spaces in a direction of the Seeder’s choosing.
 

Seeker Mines  
System, Full Action, recharge (5+)
 
The seeder fires small seeker mines at one to three targets of the seeder’s choice in range 5 and  
line of sight. Those targets must pass a systems check or have a mine latch on to them. At the  
start of the seeder’s next turn, the mines explode for a burst 1 explosion on the target for 3/5/7  
explosive damage.
 

Tier II:
 
HP: +2
 

          Evade     E-D    Heat    H    A     S     E        Armor        Spd      Sense 

          8         10     10      +1    +0   +3    +1       2            4        15 

Tier III:  
HP: +2
 

          Evade     E-D    Heat    H    A     S     E        Armor        Spd      Sense 

          9         11     10      +2    +0   +4    +2       2            4        15 


\section{Sentinel}

                                              SENTINEL

Sentinel doctrine chassis fill guard roles. Typically found in the retinues of commanders or posted
in defense of batteries, Sentinel chassis employ a suite of technology that ensures their charges
stay alive and operational, even if it means the Sentinel’s death.


 SENTINEL

 Mech

 Hull       Agility     Systems       Engineering

 +2         +1           +0           -1

 HP         Evasion      E-defense    Heat Cap.

 20         9            9            8

 Armor      Speed       Sensors       Size

 0          5            8            1


Base systems:

Combat Shotgun
Main CQB

Range 3, Threat 3

+2 vs evasion/tier with 1 Accuracy

7/9/12 kinetic damage


Retractable Sword
Main Melee

Threat 1

+2 vs evasion/tier

4/6/9 kinetic damage


Eye of Midnight

System, Quick Action

This system can be activated or deactivated as a quick action, and remains indefinitely. While
this system is active, the Sentinel is Slowed, but is not limited in the number of overwatch
reactions it can make per round.


Optional systems:

Punisher Ammunition




System

The first enemy damaged by the Sentinel’s combat shotgun each round must pass a engineering
heck with 1 difficulty or be Slowed until the end of the sentinel’s next turn.


Wrath-lock
System, Full Action

The sentinel may arm this module as a full action. While armed, the next time it fires its combat
shotgun, it makes 3 attacks instead of one.


Impaler
System

The first enemy damaged by an overwatch shot from the Sentinel must pass an agility check
with 1 difficulty or immediately stop moving and become immobilized until the end of the
sentinel’s next turn.


Watchful
Trait

Once per round, in response to any enemy movement, the Sentinel can make the boost action as
a reaction.


Guardian protocols

Trait

At the start of its turn, the Sentinel can nominate an allied mech within range 5. Once per turn
(but any number of times per round) when that mech is targeted by an attack, the Sentinel can
immediately make a single attack for free against the attacker if it is in range as a reaction.


Tier II:


          Evade     E-D    Heat    H    A     S     E        Armor        Spd      Sense

          11        11     8       +3    +2   +1    -1       1            5        8

Tier III:
HP: +4


          Evade     E-D    Heat    H    A     S     E        Armor        Spd      Sense

          13        13     8       +4    +2   +4    -1       1            6        8


\subsection{Sniper}

                                                 SNIPER  

Sniper pattern chassis are common throughout the galaxy’s armies. Favoring stability and  
targeting over mobility, Snipers operate extremely long-ranged kinetic weapons in small, self- 
sufficient teams well removed from any direct combat. Sniper pilots are a proud breed, who  
emphasize economy and elegance over destructive power. Their weapons are often as tuned and  
modified as much as their chassis are; the pilots themselves often exhibit unparalleled control  
over their targeting systems. Partnered NHPs are said to send boast-data between themselves,  
noting the difficulty of landed shots to allied pilots. 
 

 SNIPER 

 Mech 

 Hull       Agility      Systems       Engineering 

 +0         +1           +2            -1 

 HP         Evasion      E-defense     Heat Cap. 

 15         10           10            6 

 Armor      Speed        Sensors       Size 

 0          4            15            1 

Base systems:
 
Anti-Material Rifle  
Heavy Rifle
 
Loading, Ordnance, AP
 
+1 vs evasion/tier with +2 Accuracy/tier
 
Range 30
 
10/15/20 kinetic damage
 

Auto-pistol  
Auxiliary CQB
 
+0 vs evasion with +1 Accuracy at tier II, +2 Accuracy at tier III
 
Range 8, Threat 3
 
3/4/5 kinetic damage
 

Deathmark protocol  
System, Full Action
 
The Sniper marks a target within range 20. While marked, the Sniper’s anti material rifle attacks  
against that target causes it to take 1 structure damage on hit instead of dealing damage  
normally (causing the target to make a structure check and reset its HP). The Sniper can only  
have one mark active at a time, but can transfer it if need be. The target can avoid this effect if it  

                                                                                                          


is in any kind of cover, or if it’s prone (it is aware of this effect), and instead takes damage  
normally.
 

Optional systems:
 
Defensive Grapple  
System, Quick Action, Reaction, Recharge (4+)
 
The Sniper chooses a point within range 5 (vertical or horizontal) and pulls itself to that point with  
a grappling hook as if it had flown. It can use this as an action on its turn or a reaction to an  
enemy’s movement that it can see.
 

Climber  
Trait
 
The sniper can climb any surface with no penalty and walk or stand on such surfaces as if they  
were flat ground, even overhanging or vertical surfaces. 
 

Selective Loader  
System
 
The Sniper can fire one of several types of ammunition, choosing one before it attacks:
 
	        Impact: Targets hit by the AM rifle must pass a hull check or be knocked prone
 
         EMP: Targets hit by the AM rifle must pass a system check or be Jammed until the end of  
         their next turn
 
         Flare: Targets hit by the AM rifle cannot hide or turn invisible until the end of their next  
         turn
 
         Molten: Targets hit by the AM rifle must pass an engineering check or become Shredded  
         until the end of their next turn.
 

Over-penetrating Round  
System, protocol, recharge (6+)
 
The Sniper loads an over-penetrating round into its anti-material rifle. Its next attack with the rifle  
becomes Line 30.
 

Flash Bomb  
System, Quick action, recharge (6+)
 
The sniper fires a bomb at a blast 2 area within range 5 of it. Targets other than the Sniper caught  
inside must pass a systems check or become Slowed and jammed until the end of their next  
turn. The area counts as light cover until the start of the sniper’s next turn
 

Tier II:
 
HP: +2
 

          Evade     E-D    Heat    H    A     S     E        Armor        Spd      Sense 

          13        10     6       +0    +2   +3    +0       0            4        15 

                                                                                                                 


Tier III:  
HP: +2
 

           Evade       E-D     Heat     H      A     S      E         Armor          Spd        Sense 

             16         12     6          +0    +3     +4    +1        0              4          15 

.  

\subsection{Spectre}

                                               SPECTRE

Spectre doctrine chassis emphasize the ability not to be hit over pure defensive shielding.
Employing cutting-edge optical and systemic camouflage, Spectres vanish from the battlefield
and all active/passive scans, flickering in and out of vision and shattering their image and radar
signatures, confusing the eye and the sweep.Their weapons do not mark their doctrine: their
power comes from the ability to operate unseen.


 SPECTRE

 Mech

 Hull       Agility      Systems       Engineering

 -2         +2           +1            +1

 HP         Evasion      E-defense     Heat Cap.

 10         10           10            7

 Armor      Speed        Sensors       Size

 0          6            10            1/2-1

Base systems:

Machine pistol
Auxiliary CQB

+2 vs evasion/tier

Range 5, Threat 3

6 kinetic damage

This weapon can attack twice at tier III with the attack action


ATHENA-class scan
System, quick action, recharge (4+)

One target in range 10 of the specter's choice must make a systems skill check (the specter can
guess they are there and doesn't need line of sight). If they fail, they are revealed from hiding and
lose the benefits of hiding or invisibility and cannot hide or turn invisible until the start of the
spectre's next turn.


Tactical Cloak

System

The spectre is permanently invisible while this system is intact.


Optional systems:




Combat subroutine
Trait

A targets that fails their check against the spectre's Athena-Class scan system gain the lock on
condition.


Fortress
Trait

The spectre gains the Fortress trait (lock on, invasion, and scans are made against it at +3
difficulty). To electronic systems, it doesn't appear to even be there.


System flayer
Trait

Targets that fail their check against the spectre's Athena-Class scan system immediately suffer
from Lock On.


Weakness analyser
System

The specter's attacks from hiding gain an additional +1 Accuracy/tier and gain the ability to
Critical Hit, dealing +1d6 kinetic damage on Critical Hit


Cloaking Field
System, full action, recharge (6+)

The spectre deploys a cloaking field, affecting a burst 3 area around it. All allied mechs in that
field become invisible while inside the area. The field deactivates at the end of the spectre's next
turn.


Tier II:


          Evade     E-D     Heat    H     A     S     E        Armor        Spd       Sense

          11        11      7       -2    +3    +3    +1       0            6         10

Tier III:


          Evade     E-D     Heat    H     A     S     E        Armor        Spd       Sense

          12        12      7       -2    +4    +5    +1       0            6         10


\subsection{Support}

                                               SUPPORT

Support doctrine chassis focus their systems towards keeping their allies combat operational.
Alloy cement, nanite paste, patch plates, vacuum seals — the tools vary, but the result is the
same: you may be messed up, but you’re up, and you can keep fighting because the Support
kept you alive.


 SUPPORT

 Mech

 Hull       Agility      Systems       Engineering

 +1         -1           +0            +2

 HP         Evasion      E-defense     Heat Cap.

 20         7            10            10

 Armor      Speed        Sensors       Size

 1          4            12            2-3

Base systems:

Suppressive cannon
Main cannon

Ordnance

+2 vs evasion/tier

Range 10

6 kinetic damage

This weapon can be fired an additional time at tier III with the attack action

Targets damaged by this cannon suffer from the impaired condition until the end of their next
turn


Latch drone
System, Recharge (5+)

As a free action, the support fires a drone to a point within range 5 of its position, where it
hovers. The drone can be attacked and destroyed. It is a size 1 drone with evasion 10 and 10 HP.
The drone clamps on to the next allied mech to move through or adjacent to that drone’s space
and discharges, healing that mech 5/7/10 HP.


Optional systems:

Remote reboot
Trait, Full action





The support immediately ends 2 of the following conditions on an allied mech within range:
impaired, jammed, shut down, Slowed.


Nano-repair cloud
System, Full Action, recharge (5+)

The support creates a blast 2 area within range 5 of its location. Allied targets that start their
turns in the area or move through it for the first time can gain 4/5/6 HP. The cloud disperses at
the start of the support’s next turn.


Defensive pulse
System, Full action

The support makes a systems skill check with +1 Accuracy/tier. If it is successful, all mechs
within range 5 of the support can immediately end the Jammed and Lock on conditions on
themselves.


Manual Repair

System, Full Action, recharge (5+)

The support targets an adjacent mech or vehicle and makes an engineering skill check. If the
check is successful, that mech can repair up to 4/6/8 HP and also to repair up to 1 destroyed
system, returning it to functionality.


VULCAN drone
System, Drone, Full Action, Recharge (6+)

The support chooses another mech within range 5 of its position, then fires a self-deploying
drone at that mech. The drone is a size 1 object with evasion 10, 10 HP, and 1 armor. The drone
clamps on to the targeted mech. At the start of each of its turns, while the drone is deployed on
that target, that target can heal 3/4/5 HP. In addition, it gains +1 Accuracy on all checks and
attacks. The drone can be shot off and destroyed.


Tier II:

HP: +5


          Evade    E-D    Heat     H    A     S     E       Armor        Spd      Sense

          7        11     10       +1   -1    +1    +4      1            4        10

Tier III:
HP: +5


          Evade    E-D    Heat     H    A     S     E       Armor        Spd      Sense

          8        12     10       +1   -1    +2    +6      1            4        10



\subsection{Technical}
                                              TECHNICAL

Technicals lean into the incredible tumult of systemic warfare, operating both in realtime and
among the omnicloud tempest that descends upon a combat theater. Technicals often pair with
personality-clone NHPs — NHPs that structure profiles based on profiles of their pilots — to
handle the chaotic swirl that results from realtime/omninet combat splitting.

 Technical

 Mech

 Hull       Agility      Systems       Engineering

 -1         +1           +2            +0

 HP         Evasion      E-defense     Heat Cap.

 12          12          12            8

 Armor      Speed        Sensors       Size

 0          5            10            1/2 -1

Base systems:

Autogun
Auxiliary rifle

Smart

Range 10

+2 vs e-defense/tier

6 kinetic damage

This weapon can attack twice at tier III with the attack action


HORUS hacker
Trait

This mech makes tech actions with +1 Accuracy and its invasions deal +1/2/3 heat on hit.


Snap Fire
Trait, Quick Action

This mech makes one autogun attack


Optional systems:

Predatory logic

Quick Tech, Recharge (5+)

+2 vs e-defense/tier with +1 Accuracy





On hit, the target immediately makes an attack with a single weapon of the technical’s choice
against any other target within range (even an allied target).


Puppet system
Quick Tech

+2 vs e-defense/tier

On hit, the target is impaired until the end of its next turn. In addition, the technical can
immediately cause that target to move up to its speed in a direction of the technical’s choice.


SCORPION protocols
System

The technical gains the hardened target trait (invasion, lock on, and scan are made at +1
difficulty). In addition, any failed invasion, lock on, or scan attempt on it inflicts 2/3/4 heat to the
attacker.


Illusory subroutines
Quick Tech

+2 vs e-defense/tier

On hit, all actors allied to the Technical count as invisible to the target until the start of its next
turn.


System link
System, Quick Action

The Technical links systems with an allied mech within its sensor range. It can only link with one
mech at a time. While linked, the allied mech gains +1 Accuracy/tier, and the allied mech can use
the Technical’s systems score for all checks it makes. However, if either linked mech becomes
jammed or impaired, the other also suffers the same condition for the same duration as long as
the link persists. The link is disabled if either mech is destroyed or this system is disabled or
destroyed.


Tier II:

HP: +2


          Evade     E-D    Heat    H     A    S     E        Armor        Spd      Sense

          14        15     8       -1    +2    +4   +0       0            5         10

Tier III:
HP: +2


          Evade     E-D    Heat    H     A    S     E        Armor        Spd      Sense

          16        18     8       -1    +3    +6   +0       0            5         10

   

                                     GENERIC OPTIONAL MODULES:  

Any of the above NPCs can choose one of these modules when choosing an optional module.
 

Armored  
Trait  
This mech gains +1 armor, up to a maximum of +4
 

Boosted Reactor  
Trait
 
This mech gains +2 heat capacity
 

Jump Jets  
System
 
This mech can Fly when it boosts
 

Reinforced  
Trait
 
This mech gains +5 HP
 

SSC Core Flight System  
System
 
This mech can Fly when it moves or boosts and has perfect flight (it doesn’t need to land). It  
generates 2 heat/turn while this module is active at the end of each of its turns.
 

Tactical Cloak  
System, Quick Action
 
This mech becomes invisible. If it takes an action, reaction, or overheats it loses this invisibility  
until the start of its next turn.
 
\chapter{Special Classes}
      SPECIAL CLASSES  

The following NPC classes describe special cases, NPCs that are a little more unique or can’t be  
described as a mech.

\subsection{Human}
                                                HUMAN
This entry describes humans or human-scale enemies, such as pilots in the world of lancer. The
entry is intentionally pretty simple.


To make a human enemy:

            -   Pick a statistic block from the table below that basically describes the enemy.

            -   Pick a weapon for your enemy if they're armed, and describe any other gear they
                have

            -   All human enemies are size 1/2 and roll attacks and checks at +2/per tier

            -   You can use the Heavily Armored, Elite, and Legend modifiers below to make
                enemies tougher


                                           Starting templates

 Name                        HP         Evasion/E-defense            Armor                   Speed

 Civilian                    1          8/8                          0                       3

 Untrained martial           3          9/9                          0                       3
 human (gangster, thug,
 warrior, etc)

 Martial human (pilot,       5          10/10                        0                       4
 soldier, guard, etc)

 Subaltern frame             8          10/8                         1                       4

 Armored human (pilot        5          8/8                          1                       4
 in a hard suit, etc)

 Heavily armored             -          -2 to both                   +1                      -1

 Elite (apply to any)        +2/tier    +1/tier to both              -                       +1

 Legend (apply to any)       +3/tier    +2/tier to both              +1                      +2

                                                Weapons

 Name                        Tags                          Range                    Damage

 Assault Weapon               -                            5                        2

 Melee weapon                 -                            Threat                   2




 Sidearm                      -                             2                       1

 Heavy Weapon                 Ordnance, Loading             8, Blast 1              4

Damage: Weapons can be kinetic, energy, or explosive depending on their function

Type: All weapons have the pilot type

Tags: AP (weapon gets -1 damage, to minimum of 1), Smart, Seeking, Arcing, Ordnance,
Loading (weapon gets +2 damage), Limited, Accurate (weapon gets -1 damage, to a minimum of
1), Inaccurate (weapon gets +1 damage)


A weapon shouldn't go above 5 damage after tags.


                                            Example enemies:
These are just a few examples! Feel free to make your own


 Name                     Template                              Weapons

 Assassin                  Elite, Martial Human                 Variable Sword (Melee
                                                                Weapon, AP), Sniper rifle
                                                                (Heavy Weapon, Loading,
                                                                AP)

 Pirate King               Legend, Armored Human                Greatknife (Melee weapon,
                                                                Accurate), R35 pistol
                                                                (sidearm, loading, limited)

 Guard                     Martial Human                        Assault Rifle (Assault
                                                                weapon)

 Hacker                    Civilian                             Scavenged pistol (sidearm)

 Low-tech (medieval        Martial Human                        Primitive melee (Melee
 or tribal) warrior                                             weapon, primitive, 1
                                                                damage)

 Thug or Bandit            Untrained martial Human              Custom pistol (sidearm)


\section{Squad}

                                                 SQUAD

Squads come in two varieties: Squadrons and Infantry. A squadron represents a squad of mech-
type enemies operating as a group, whereas an infantry squad represents a squad of
approximately human-sized enemies operating together. They are treated like one entity.


All squads (squadrons and infantry) get the following features:

    -   Weak: Squads cannot have more than 1 structure or heat capacity. They take energy
        damage when they take heat.
    -   Exclusive templates: Squads cannot take the Grunt, Veteran, or Ultra templates (they
        can still take Elite, but don’t gain structure)
    -   Split HP: Squads have a large number of members operating together. Once a certain
        amount of damage is done to the squad, one of the members is destroyed (indicated in
        the profile). Once a squad is under half HP, it loses some of its attacks to represent this.
    -   Strength in numbers: Squads have resistance to all damage that is not from line, blast,
        or cone attacks. They are immune to grapple or ram and cannot grapple or ram.
    -   Spread out: Squads occupy a square area equal to their size for purposes of targeting,
        but each individual member is not represented. For the purposes of determining cover
        and obstruction, use the size of each individual member, not the size of a squad as a
        whole.

       SQUADRON

       Squad, Mech

       Hull       Agility     Systems       Engineering

       +1         +1          +1            +1

       HP         Evasion     E-defense     Heat Cap.

       16         6           8             -

       Armor      Speed       Sensors       Size

       1          5           10            6 (individual:
                                            1)

HP per member: 2

Number of members: 8


Base modules:

Primary squad weapon

Main Rifle

+1 vs evasion/tier





Range 10

3/4/5 kinetic, energy, or explosive damage (choose on creation)

This weapon can be fired three times when the squad attacks. If the squad is under 1/2 hp, it can
only be fired once.


Heavy squad weapon

Heavy cannon

+0 vs evasion with 1 Accuracy/tier

Range 12 and blast 2, line 10, or cone 5 (choose on creation)

4/5/6 kinetic, energy, or explosive damage (choose on creation)


Tier II:

HP: +2 (+1 members)


          Evade    E-D    Heat    H    A     S    E       Armor        Spd      Sense

          7        8      -       +2   +2    +2   +2       1           5        10

Tier III:
HP: +2(+1 members)


          Evade    E-D    Heat    H    A     S    E       Armor        Spd      Sense

         8         8      -       +3   +3    +3   +3       1           5        10



\subsection{Infantry}
                                               INFANTRY

Infantry-type enemies represent a squad-level group of human or subaltern infantry: not a single
chassis like the Grunt-type, but (generally) a group of five to ten armed and armored individual
soldiers. Like the Grunt, an Infantry-type enemy alone might not present a threat to a chassis, but
operating as a squad with the right gear and training, infantry groups are a formidable threat.

       INFANTRY

       Squad, Biological

       GRIT +2/tier

       HP         Evasion      E-defense     Heat Cap.

       10         8            8             -

       Armor      Speed       Sensors        Size

       0          4            10            4 (individual:
                                             1/2 )

HP per member: 1

Number of members: 10


Primary squad weapon

Main Rifle

+2 vs evasion/tier

Range 10

3/4/5 kinetic, energy, or explosive damage (choose on creation)

This weapon can be fired three times when the squad attacks. If the squad is under 1/2 HP, it can
only be fired once.


Anti-mech squad weapon

Heavy cannon

AP

+2 vs evasion/tier

Range 20

4/tier kinetic, energy, or explosive damage (choose on creation)


Tier II:

HP: +5 (+5 members)


       Evade    E-D    Heat    Armor       Spd      Sense




        8         10     -        0            4         10

Tier III:
HP: +5 (+5 members)


        Evade     E-D    Heat    Armor         Spd      Sense

        8         10     -        1            5         10

Optional modules for squads (both types):

Armored
Trait

The squad gains +1 armor


Ambushers
Trait

The squad gains +2 Accuracy on all attacks and rolls on the first round of combat only, and gains
+2 Accuracy to avoid being discovered while hiding


Go to ground
Trait, reaction

Once per round as a reaction, the squad gains resistance to all the damage from an incoming
attack, but cannot move or boost on their following turn. They must decide before the damage is
rolled.


Rapid Insertion
Trait

The squad can fly when it moves or boosts.


Technicals
Trait

The squad can take the Invasion action (as per player rules). Disable this action if the squad is
under 1/2 HP.




\subsection{Swarm}
                                                   SWARM

A swarm represents a large, motile group of very small or weak enemies, biological, human,
nanorobotic, or otherwise.

All swarms get the following features:

    -    Weak: Squads cannot have resilience or heat capacity. They take energy damage when
         they take heat.
    -    Exclusive templates: Swarms cannot take the Grunt, Veteran, or Ultra templates (they
         can still take Elite, but don’t gain resilience)
    -    Swarms can take the Biological tag if they represent biological creatures. Otherwise a
         swarm represents a swarm of mechanical entities.
    -    The Many: The only actions a swarm can take are to move and boost, or those specified
         in its profile.
    -    Strength in numbers: Swarms have resistance to all damage that is not from line, blast,
         or cone attacks. They are immune to the grabbed condition.
    -    Spread out: Swarms occupy a square area equal to their size for purposes of targeting,
         but each individual member is not represented. For the purposes of determining cover
         and obstruction, use the size of each individual member, not the size of a squad as a
         whole.
    -    Swarm: A swarm’s area counts as difficult terrain, even if a mech can normally move
         through it

       SWARM

       Swarm

       Hull       Agility      Systems       Engineering

       +1         +1           +0            +0

       HP         Evasion      E-defense     Heat Cap.

       15         6            8             -

       Armor      Speed        Sensors       Size

       -          3            6             5 (individual:
                                             1/2, 1/4, or
                                             smaller)

Base modules:

Swarm

Trait

Targets starting their turn in the swarm’s area or entering it for the first time take 3/5/7 AP kinetic
damage.





Optional modules:
Crawl over

Trait

When a target starts its turn inside the swarm or enters that area for the first time on its turn,
members of the swarm cover and crawl over it, inflicting the impaired condition on it until the end
of its next turn.


Drag down

Trait, Quick Action

1/round, the swarm chooses one target in its area. That target must pass a hull check with 1
difficulty/tier or be knocked prone


Endless Swarm

Trait

The swarm heals 2 HP/tier at the end of its turn.


Split

Trait

At the end of any turn when the swarm is reduced past 1/2 HP, it splits into two swarms of size 2,
each with half the swarm’s current HP. These swarms cannot split again.


Tear apart

Trait

The swarm’s Endless Swarm trait also causes 3/4/5 Burn


Tier II:

HP: +5


          Evade     E-D     Heat    H     A     S     E        Armor        Spd       Sense

          6         -       -       +2    +1    +0    +0       0            3         6

Tier III:
HP: +5


          Evade     E-D     Heat    H     A     S     E        Armor        Spd       Sense

          6         -       -       +3    +1    +0    +0       0            3         6


\subsection{Monstrosity}
                                           MONSTROSITY

Monstrosity-type enemies are massive or horrifying predatory wildlife. Generally wild, some kinds
of monstrous enemies can be domesticated and trained for combat: these are in high demand in
the distal and proximal reaches of the galaxy.


       MONSTROSITY

       Biological

       Hull       Agility      Systems       Engineering

       +2         +1           +0            +0

       HP         Evasion      E-defense     Heat Cap.

       15         10           10            -

       Armor      Speed        Sensors       Size

       1          6            10            1-3

Base mutations:
Claws

Main melee

+2 vs evasion/tier with 1 accuracy

6/8/12 kinetic damage


Optional mutations (choose 1 or more):


Acid Spittle
AP
+0 vs evasion with +1 accuracy/tier

Range 10

3/4/5 energy damage


Adhesive Extrusion
Trait

The first target damaged by the monstrosity on a turn must pass an engineering check or
become Slowed until it heals any amount of hit points or until the end of the current challenge.


Burrower
Trait, Quick Action

The monstrosity can burrow into the ground as a quick action, as long as the ground beneath it
is malleable enough. While burrowed, it counts as invisible, ignores all obstructions above
ground, can take no other actions other than to move, boost, or emerge as a quick action. When




it emerges, it must have a free space to emerge into. It loses these benefits and all adjacent
mechs must pass a hull check or be knocked prone.


Charger
Trait, Quick Action, Recharge (5+)

As a quick action, the monstrosity in a straight line as far as possible up to its speed, ignoring
obstructions and not provoking reactions. Any targets it passes adjacent to or over must pass an
agility check or take 4/6/8 kinetic damage.


Corrosive Bite
Full Action, Recharge 5+

The monstrosity makes a melee attack at an adjacent target for +2 vs evasion/tier. On hit, the
target takes 4/6/8 Burn. In addition, regardless of whether it succeeds this check, it is Shredded
until it regains any amount of HP.


Grasping Claws
Trait

The monstrosity gets +1 Accuracy to grapple, +1 Accuracy to attack targets it is grappling, and
can boost and take reactions while grappling.


Natural Camouflage
Trait

The monstrosity gains +1 Accuracy to checks made while hiding. If it's in its natural terrain, it
counts all terrain as being large enough to hide it (no matter the size difference).


Regenerator
Trait

The monstrosity heals 2/3/4 HP at the end of its turn. This trait does not function if the
monstrosity took energy damage at any point during the round.


Spined
Trait

When the monstrosity takes damage from a melee weapon, it deals 1 AP kinetic damage to its
attacker after the damage resolves.


Swift
Trait

The monstrosity gains +2 speed


Tempered Hide
Trait

The monstrosity has resistance to one of the following damage types: kinetic, energy, explosive
damage.





Winged
Trait

The monstrosity can fly when it moves or boosts.


Increase at tier II :

HP +2

+1 optional mutation


         Evade    E-D     Heat    H   A   S    E   Armor     Spd       Sense

         11       11      -       +   +   +    +   2         7         10
                                  4   2   0    0

Increase at tier III:
HP +2

+1 optional mutation


         Evade    E-D     Heat    H   A   S    E   Armor     Spd       Sense

         13       13      -       +   +   +    +   2         8         10
                                  6   4   0    0

All monstrosities get the following features:

    -    Unique Critical Chart: If the monstrosity has structure (such as an Elite or Ultra), it uses
         the critical chart below instead of the regular one for mechs
    -    Biological: The monstrosity has the biological tag

                                          Monstrosity Critical Chart

 ROLL       RESULT                   EFFECT

 5-6         GLANCING BLOW           The monstrosity flinches in pain, giving it the impaired condition until
                                     the end of its next turn

 4           HEAVY BLOW              The blow knocks the monstrosity down. The attacker (or the GM)
                                     chooses one of the following:

                                         -   The monstrosity is Slowed until the end of its next turn

                                         -   The monstrosity is knocked prone

 3           HEAD TRAUMA             The monstrosity is stunned until the end of its next

 2           DISMEMBERMENT           A limb or chunk is torn off the monstrosity, dealing +1d6 bonus damage
                                     to it and permanently slowing it until it can potentially heal.

 1           BRUTAL HIT              The monstrosity must pass a hull check or be destroyed. It gets +1
                                     difficulty on this check per level of structure damage it has.




Two or     FATAL HIT              The monstrosity is destroyed and instantly killed
more
1s




                                       TEMPLATES


These templates can be applied to any of the above enemy types to add more flavor or change
the way they function in combat. In particular, the Ultra, Elite, Grunt, and Veteran templates can
be used to make tougher or easier enemies. The Veteran template can be applied on top of the
Ultra or Elite templates to make an especially tough enemy.


\section{Ultra}
                                                   ULTRA  

In Lancer, Ultra-type enemies are typically the most dangerous individual enemies a party can  
face.   

Ultra-type enemies are high-tier enemies that do not usually fill command roles. Ultras are  
champions, favored warriors, major domos: they might command a few units in a retinue or lead  
armies from the front, but they do not typically engage in grand strategy.    

Making an Ultra
 
All ultras get the following features added to their base NPC type:
 
    -    Bonus activations: The Ultra can be activated again (take another turn) each round. If  
        the Ultra is facing more than 4 players, it takes another activation (for a total of 3). It  
         regains spent reactions each time it takes a turn.  
    -    Deadly: The Ultra can Critical Hit, dealing an extra +2 bonus damage/tier on Critical Hits.
 
    -    Exclusive templates: The Ultra can’t take the Elite or Grunt templates.  
    -   Juggernaut (trait): At the start of its turn, the Ultra ends one condition affecting it. If it  
         rolls a ‘system destroyed’ result on a structure check, it is instead jammed until the end of  
         its next turn (it doesn’t lose the system or weapon).
 
    -    Legendary (trait): The Ultra can enter the CRITICAL state and is not destroyed when  
         reaching 0 HP.  
    -   Structure: The Ultra gains 4 Structure, or increases its structure up to 4 structure if it  
         already has structure
 
    -   Stress: The Ultra gains 4 reactor stress, or increases its stress up to 4 if it already has  
        stress.  
    -    Bonus module: The ultra should gain +1-2 more optional modules than normal.  
    -    Bonus HP: The Ultra gains +5 HP  
    -    Reflex (trait): The Ultra can make any number of overwatch attacks per round (instead of  
        just 1). It can only attack the same target 1/turn with overwatch.  

Then choose 1-3 Ultra Traits or modules.
 

ULTRA traits:
 

                                                                                                            


Berserker  
Trait  
The Ultra gains +1 Accuracy on all melee attacks. It can make 1 melee attack roll, grapple, or  
ram attack as a Free Action on its turn.
 

Devastator  
Trait
 
Once per turn, when the ultra hits with an attack, all targets visible to the ultra take 2 kinetic,  
explosive, or energy damage.
 

Evasive
 
Trait
 
The Ultra gains +4 evasion, up to a maximum of 20, but reduce its structure to 3.
 

Extra Deadly  
Trait  
The first Critical Hit the Ultra deals per turn does +1d6 bonus damage/tier
 

Fortress  
Trait
 
The Ultra gains the Fortress trait instead of its Hardened Target trait (lock on, invasion, and scans  
are made against it at +3 difficulty). To electronic systems, it doesn’t appear to even be there.
 

Legion  
Trait
 
The ultra gains +4 e-defense, up to a maximum of 20, can take the full tech action (like a player),  
and gains +2 accuracy to all tech actions
 

Limitless  
Trait
 
The Ultra can use overcharge.
 

Unstoppable  
Trait
 
The Ultra always cannot be knocked back, knocked prone, or moved involuntarily.
 

Sight  
Trait
 
Targets cannot hide while inside the Ultra’s sensor range and it ignores invisibility.
 

Superior Construction  
Trait
 
The ultra has resistance to one of the following damage types: kinetic, energy, explosive. It  
cannot gain resistance to more than 2 (from this trait or any other systems).
 

                                                                                                                        


Superior Frame  
Trait
 
The ultra is immune to the Slowed, Shredded, and immobilized conditions
 

Superior Reactor  
Trait
 
The ultra is immune to the stunned and shut down conditions.
 

Superior Targeting  
Trait
 
The ultra ignores light and heavy cover when making ranged attacks.
 

Supreme Maintenance  
Trait  
The ultra is immune to the Jammed condition and can reload one weapon with the loading  
property as a free action on its turn.
 

Supreme Parting Gift  
Trait  
The Ultra can use the Self Destruct action. When it takes this action, it emits an EMP pulse. All  
vulnerable targets (mechs, vehicles) in a burst 5 area around it must pass a system check or be  
shut down.
 

Supreme Skirmisher  
Trait
 
2/round the Ultra can take the boost action as a reaction to any enemy movement or action that  
it can see.
 

Ultra systems and weapons:  

ARGUS armor  
The ultra’s armor increases to 6. Each time it rolls a critical or overheating check, reduce its  
armor by 2, to a minimum of 0.
 

H.A. Siege Shield  
The ultra has resistance to all damage that originates further away than range 5
 

Hellfire Projector  
Heavy CQB
 
Cone 5
 
3 energy damage + 5 Burn/tier
 
This weapon can be fired twice with the attack action, but its areas cannot overlap.
 

                                                                                                                   


HORUS manticore repulsion field  
Hostile targets that start their turn adjacent to the Ultra or become adjacent to it for the first time  
on their turn take 2 energy damage, 2 heat, and must pass a systems check or become impaired  
until the end of their next turn.
 

Obliterator
 
Action, Ordnance, Recharge (6+)
 
Choose a direction, then draw a line 30 spaces long and 2 spaces wide in that direction. Each  
target inside that area must pass an agility skill check or take 1/2 of its current hit points in energy  
damage, or 1/4 on a successful check, rounded up. Any cover, objects, buildings, drones,  
deployables, or items smaller than size 5 in the area are completely annihilated.
 

Ravager Turret  
Heavy Cannon
 
Ultra
 
+1 vs evasion with 1 Accuracy/tier
 
Range 10
 
4/6/8 kinetic damage
 
The Ravager fires at every hostile mech in range when this weapon is fired with the attack action.
 

SSC Slivershielding  
System
 
The Ultra is permanently invisible. Reduce its structure to 2.
 

SSC Ex Hover propulsion  
System  
The Ultra can fly when it moves or boosts, and can hover (it can stop midair, doesn’t have to  
move in a straight line, and doesn’t need to land after moving).
 

Volley module
 
Full Action, System
 
The ultra goes into a stable stance. At the start of its next turn, it fires one weapon at each target  
that is in range and not in cover or prone as a Free Action. It can fire the same weapon multiple  
times for this special action, ignoring the loading tag during this attack.
 

WOLFHOUND missile  
Heavy Launcher
 
Recharge (6+)
 
Instead of firing this weapon normally, choose a target within range 30. The ultra fires a  
wolfhound missile at the target, which is size 1, moves speed 4, has evasion 10, 0 armor, and  
10/15/20 HP. The missile primes in an adjacent space when deployed, then moves 3 at the start  
of each the ultra’s turns. If its target suffers from the Locked On condition, its movement  
increases to 6. The missile can benefit from cover and can be targeted and shot by systems and  
weapons. It must move towards its target, but can maneuver skillfully around cover, fit through  

                                                                                                            


holes, etc. If the missile’s movement causes it to collide with a hostile mech or its target, it  
detonates for a blast 1 explosion. Mechs caught inside must pass an agility skill check or take  
15/20/30 energy damage, and half on a successful check.
 

The Ultra can have only one missile fired at a time. Only check for recharge if this missile has hit  
its target or been destroyed.

\subsection{Elite}
                                                   ELITE

Making an Elite

All Elites get the following features added to their base NPC type:

    -   Structure: The Elite increases to 2 structure

    -   Stress: The Elite increases to 2 Reactor Stress

    -    Bonus Activation - The Elite can be activated twice a round
    -    Exclusive templates: The Elite can’t take the Ultra or Grunt templates.
    -    Bonus system: The Elite gains +1 more optional system than normal

\subsection{Grunt}
                                                 GRUNT

Grunt-type enemies in Lancer are the most common enemies faced en-mass by players. Grunts
are a step above cannon fodder: cheap and trained to be disciplined fighters before deadly ones,
grunts follow orders under threat of death, imprisonment, or some other kind of censure. Alone, a
grunt-type enemy may not be a threat, but in a group, they can present an overwhelming mass of
bodies and fire that threatens even the most powerful individual.

Making a Grunt

All grunts get the following features added to their base NPC type:

    -   Weak: The Grunt’s maximum HP becomes 1, with 1 structure. It loses any heat capacity it
         has (so it takes energy damage instead of heat). It cannot gain structure or heat capacity
         in any way.
    -   Surprisingly Hardy: Grunts don’t take damage from systems, talents, or weapons that
        automatically deal damage (such as the reliable tag). When they would take half damage
        on a successful save from a source of damage, they instead take 0.

    -   Still a mech: Grunts can only be killed by pilot weapons that have a total attack roll of
        20+
    -    Bonus activations: When a Grunt is activated, up to 2 other grunts can also be activated
        at the same time and take their turns at once

    -    Max systems: The Grunt can’t have any more than 1 optional system
    -    Exclusive templates: The Grunt can’t take the Veteran, Squad, Elite, or Ultra templates
    -    Max attacks: The Grunt can’t attack with more than 1 weapon a turn.
    -    Max damage: None of the grunt’s attacks can deal more than 4 damage, heat, or burn
        total per attack at tier 1, or half if the attack is AOE or has more than one target. This
         increases to 6 at tier II and 8 at tier III. Adjust as needed.


\section{Veteran}
                                                   VETERAN

Veteran-type enemies are hardened, experienced fighters that have survived direct engagement
with their foes. Their encounter with your players is not their first rodeo: their ability to withstand
morale shocks is far higher than an untested greenhorn.


Making a Veteran:

All Veterans get the following features added to their base NPC type:

     -   Bonus structure: The veteran gains +1 structure.

     -   Bonus stress: The Veteran gains +1 reactor stress
     -   Veterancy: The Veteran gains +1 accuracy on all checks of one statistic: Hull, Agility,
         Systems, Engineering (choose on creation)

Then choose up to 1 Veteran trait per tier:
AI co-pilot
Trait
The veteran’s mech or vehicle gains the AI tag. The veteran’s mech or vehicle is autonomous and
can function even if the veteran is not piloting it (per the player rules). The AI can be unshackled,
with the same effects as player AI.


Acrobat
Trait
When the veteran moves or boosts, it can fly 3 after the move or boost completes. This flight
does not provoke reactions and ignores engagement.


Deadly
Trait
This mech can Critical Hit, dealing +2 bonus damage/tier to its attacks on crit.


EM-shielding
Trait
The veteran is resistant to heat


Engineer
Trait, Quick Action
The Veteran can take the repair quick action:

         Repair (limited 1): This mech heals 4 hit points per tier and ends one condition currently
         affecting it (impaired, Slowed, immobilized, shut down, jammed)


Feign Death
Trait




The first time in a challenge this enemy would be destroyed, though it appears destroyed, it is
instead shut down, remaining at 1 hp. A successful scan or pilot notice check will reveal it is still
active.


Hacker
Trait
The veteran’s invasion attacks deal +3 heat


Headshot
Trait
1/round when the veteran scores a Critical Hit, the target of its attack must pass an engineering
check or be jammed until the start of the veteran’s next turn.


Hardened Target
Trait
Invasion, scan, and lock on attempts on this target are made with +1 difficulty


Legendary
Trait
The veteran can enter the CRITICAL state (like the player) instead of being destroyed when
ticking off its last structure point.


Lesser Sight
Trait

Within range 3 of the Veteran, targets cannot hide from it, and cannot benefit from invisibility
against it while inside that area.


Limitless
Trait

The veteran can use Overcharge


Lightning Reflexes
Trait
Reaction

The first time in a round the veteran is targeted by any weapon system of size heavy or larger, roll
a d6. On a 5+, the weapon automatically misses this mech.


Parting Gift
Trait
The veteran can use the Self Destruct action.


Rodeo master
Trait




The veteran can jockey (in a hard suit) and gains +1 Accuracy on jockeying attempts


Shock Armor
Trait

The veteran has resistance to damage from main size weapons


Skirmisher
Trait

1/round the veteran can move or boost as a reaction to any enemy movement or action that it
can see.


Slippery

Trait

The veteran does not provoke reactions and ignores engagement with its move (its boost still
provoke reactions)


Steel Jaw
Trait
The most damage an auxiliary weapon can do to the veteran is 1.


Viper’s Speed
Trait
The veteran always takes its turn first, even over player characters. If there is a question of who
acts first (in the story, or combat), it is always the veteran.


\subsection{Exotic}
                                                      EXOTIC   

Exotic-type enemies are, even for a galaxy of wonders, strange and dangerous enemies. Some  
feature unique technologies not yet available to the wider galaxy, others wield primitive weapon  
styles updated to the modern day, and others carry equipment or adopt tactics that are alien to  
Union doctrine.  
 
 
Making an Exotic NPC:
 
All Exotic NPCs get the following features added to their base NPC type:
 
     -   Xenotech: Scans reveal no information about exotic modules  
     -   Hardened Target: Hostile tech actions take +1 difficulty against exotic targets  
     -   Exotic systems: Any NPC with the exotic tag can choose one or two of the optional  
         exotic modules below  

Exotic modules
 
Bio-integrated  
Trait
 
You may only choose this trait for a mech. The mech gains the biological tag (it loses heat  
capacity and cannot take or be the target of tech actions except Lock On and scans).
 

                                                                                                                     


Blinkspace Carver  
System
 
When the NPC moves, it teleports.
 

Extrusion  
Trait
 
The NPC is only a partial extrusion of a higher-concept entity and only partially present in  
physical space. It gains resistance to all damage but all its weapons also deal half damage and it  
can be passed through as if it wasn’t there (it doesn’t provide obstruction).
 

Living Weaponry  
Trait
 
The NPC becomes immune to the Jammed condition. Its weapons have biotechnical and semi- 
organic components that spontaneously generate ammunition.
 

Paracausal Weapon  
Trait
 
Choose one of the NPC’s weapons. Damage from that weapon cannot be reduced in any way  
(by armor, resistance, or otherwise)
 

Ouroboros Brand  
System
 
1/round the NPC can force a re-roll by themselves or any allied or hostile target (but must choose  
the second result) by causing localized distortions in the flow of time.
 

Regenerator  
Trait
 
The NPC heals 2/3/4 HP at the end of its turn. This trait does not function if the NPC took energy  
damage at any point during the round.

\section{Drone}
                                                    DRONE   

Drone-type enemies are mechs or subaltern frames that are autonomous extensions of  
companion/concierge units or combat NHPs.  

Making an Drone
 
All Drones get the following features added to their base NPC type:
 
    -    Bonus HP: The Drone gets +5 HP to represent its lack of pilot. It can take damage that  
         would normally blow through a cockpit and kill a human.  
    -    No Pilot: The Drone permanently suffers from the Impaired condition. It cannot make  
         complex decisions or moral judgements. However, it is also immune to all systems and  
         actions that affect the pilot (it doesn’t have any).  

                                                                                                                


     -   Vulnerable to Tech: All hostile tech actions against the Drone can be made with +1  
         Accuracy  

\subsection{Mercenary}
                                                MERCENARY  

Mercenary-type enemies are foes that offer their services to the highest bidder. They may have  
attachments to a faith or flag, but when it comes to doing their business, they’re loyal only to  
gold.    
 
 
Making a Mercenary:
 
All Mercenaries get the following features added to their base NPC type:
 
     -   Opportunist: If an allied target is adjacent to the mercenary’s target, it gets +1 Accuracy  
         on all attack rolls against that target  
     -   Mercenary systems: A mercenary can choose from the list of mercenary systems and  
         traits below when choosing optional systems and modules  

Mercenary systems and traits:
 
Bounty Hunter
 
Trait
 
Before combat, choose a player character that there’s a bounty out for. The mercenary gains +1  
Accuracy on all attacks and checks against that character and their mech.
 

Efficient Killer
 
Trait
 
The mercenary gains +1 Accuracy on attacks and all its damage becomes AP against a target  
with 1 structure remaining.
 

Call in Favor
 
Action, Limited (1)
 
The Mercenary calls in a favor in the form of an orbital strike or artillery strike, targeting a blast 2  
area within range 30 of its position. All mechs caught in the area must pass an agility check or  
take 6/10/15 explosive damage, or half on a successful check.
 

Scout Drone
 
Action, Limited (1)
 
The Mercenary fires a scout drone to an empty space within its sensor range. The drone has  
evasion 10, 10 hp, and no armor. Within a burst 3 area centered on the drone, no target can turn  
invisible or hide from the mercenary, and the mercenary can attack targets in the area with +1  
accuracy. The mercenary can reposition the drone as a free action at the start of its turn.
 

Tactical Retreat
 
Reaction, Limited (1)
 
When the mercenary is taken below half health, this system automatically activates. All targets  
within a burst 2 area around the mercenary must pass a systems check or take 2 kinetic damage  

                                                                                                                   


and become jammed until the end of their next turns. The mercenary can then move its speed in  
any direction. This movement doesn't provoke reactions and ignores engagement.
 

\subsection{Commander}
                                              COMMANDER

Commander-type enemies operate on a grand scale, controlling fleets and armies across worlds
and in interstellar space. They might not be the best individual combatant, but they can bring out
the best in the best of their forces.

Making a Commander:

All Commanders get the following features added to their base NPC type:

    -    Bonus structure: The Commander gains +1 structure
    -    Bonus stress: The Commander gains +1 reactor stress
    -    Command: Once per round, the Commander can cause an allied target it can see to re-
         roll any single attack roll or check as a reaction. The commander can’t use this reaction if
         it is Jammed.
    -    Commander Traits: The commander can choose one of the following commander traits.
         These traits don’t function if the commander is Jammed.

Commander Traits

Bolster network
Trait
While the commander is alive, all allies the commander can see gain the hardened target trait (+1
difficulty on all hostile tech actions or attacks against them). The commander does not benefit
from this trait.


Covering Fire
Trait, Reaction

1/round, an ally the commander can see can attack with one weapon as a reaction against a
target hostile to the commander that just made an attack against that ally or a different ally.


Press on!
Trait, Free Action

1/round, the commander can pass a systems check as a free action to end the stunned or
jammed conditions on a target it can see in its sensor range.


Reposition
Trait, Free Action

1/round one ally that a commander can see can make the boost action as a reaction on the
commander’s turn.


Rank and File
Trait

Allied targets adjacent to the commander gain +1 Accuracy on all attacks and checks.


\subsection{Pirate}

                                                   PIRATE

Pirate-type enemies live on the boundaries, in the forgotten bolt-holes of occupied space. They
are far-ranging, operating around blink gates, interstellar shipping lanes, and near flagged but
not-yet colonized worlds.

Pirate-type groups are usually small, with their base of operations hidden a short-to-medium
distance from the area that they are first encountered. Usually they are motivated by profit and
materiel goods, not ideology or religion -- a good thing for a savvy negotiator.

Pirate-type enemies commonly operate in mixed groups of fast, deadly subline ships, fighter/
bombers, and clutches of marines and mechanized chassis. They prefer hit-and-run style
overwhelming ambush attacks, aim to capture rather than kill, and don’t like to engage in pitched
battles


Pirate-type enemies can be found across the galaxy wherever the law is spread thin, resources
travel, and people are desperate.

Making a Pirate

All pirates get the following features added to their base NPC type:

    -    Deadly (trait): The pirate can Critical Hit, dealing +2/3/4 bonus damage on a 20+ ranged
         or melee attack roll
    -    Pirate modules: The pirate can choose from the below modules when choosing optional
         systems

Gain the following options for optional systems/modules:


Boarding Clutch

Heavy Melee

+0 vs evasion with +1 Accuracy/tier

5/7/9 kinetic damage

This weapon can be used as an improvised grappling hook. It can be thrown at a target within
range 5. The affected target must pass a hull or agility check or be pulled directly to the attacker
and grappled.


Splinter Rounds
System

Critical Hits from this mech deal an additional +3 bonus damage


Borer Missiles
Main Launcher





Smart, Limited (1)

+1 vs e-defense with 1 Accuracy

Range 8

On hit, 1d3 miniature drones are attached to the target and begin boring into the mech’s interior.
In 1d6+3 rounds, if there are any missiles still attached to the target, they bore inside the cockpit
savage the pilot, reducing the pilot to 0 HP, causing them to make a Down and Out check. A
single drone can be removed by making a successful systems or engineering skill check as a
quick action. The drones can also be confused by shutting the affected mech down, which
deactivates them. The pilot in the affected mech is aware of how much time they have before the
drones bore through.


Prying Claws
System, Full Action, Limited (1)

An adjacent target makes a hull check. On a failed check, the targeted pilot immediately exits
their mech as if they had taken the dismount action (place them adjacent to their mech).


Slaver Signal
System, Full Action

Recharge 5+

Range 10

This system affects one piloted vehicle or mech within range 10. The affected target must make
a systems check with 1 difficulty/tier to shut out the signal or be affected. On failure, all the pilots
in the target are affected by the signal, falling into an unresponsive stupor. While in this state,
their mech is stunned. If the target takes damage, the pilots will be jolted awake, immediately
ending this effect. Another adjacent actor can use an action on their turn to make a systems
check to try and blot out the signal, waking the affected target up on success. If the source of
the signal is destroyed, any targets affected will also wake up.


\subsection{Spacer}
                                                 SPACER

Spacer-type enemies are born and bred in the hard vacuum of deep space. They are adept at
maneuvering in and around the difficult, kinetic, low-gravity environments found only in space:
blink stations, space stations, among asteroid fields, in low orbit over worlds, and between ships
in military and commercial fleets.

Making a Spacer:

All Veterans get the following features added to their base NPC type:

    -    Maneuverable: The Spacer does not suffer the impaired condition for operating in space,
         underwater, or in zero-g environments and always counts as having an EVA module in
        those environments.
    -   Optional modules: The Spacer can choose from optional Spacer modules when
        deciding additional modules

Spacer modules

Concussion gun





Main Rifle

Cone 3

+1 vs evasion/tier with 1 Accuracy/tier

Mechs hit by this weapon are 3 spaces away from the direction of the zone.


Gravity Rifle

Main Rifle

Range 10

+1 vs evasion/tier with +1 Accuracy/tier

The target of this attack must pass a hull check with 1 difficulty/tier or be pulled in a straight line
towards the wielder of this weapon up to 10 spaces, or as far as it can move. If this causes it to
collide with an obstacle or another mech, it is additionally knocked prone.


Sealant Trap

Mine, Limited (1)

This self deploying trap can be placed in a vacant space. It counts as a mine and can be
scanned to be detected and disabled with a successful systems check. Once any target walks
over or adjacent to the trap, it explodes, creating a burst 2 area centered on it. All targets caught
inside must pass an agility check or be covered in thick, fast-drying hull sealant and immobilized.
The only way to break out of the sealant is to do enough damage to break it (evasion 5, 15 hp).


Thumper grenades

Grenade, Limited (1)

Thrown 5

Once thrown to an impact point as a quick action, these grenades explode outwards with a
concussive pulse in a blast 1 area around the space where they are targeted. Targets caught in
the area must pass an agility check or be knocked back 3 spaces directly away from the impact
point. If this causes them to collide with an obstacle or another mech, they are additionally
knocked prone.


\section{Outworlder}
                                            OUTWORLDER

Outworlder-type enemies are found in the distal/proximal areas of the galaxy. They are, typically,
more rugged and independent than other enemy types, able to operate on a shoestring -- or
without one! -- for extended periods of time.


Making an Outworlder


    -    Resourceful: The Outworlder gains +1 use to all (limited) use weapons, actions, and
        deployables (including the repair action below - not yet included in the profile).
    -   The outworlder gains the repair action:
             -   Repair (limited 1): This mech heals 4/6/8 hit points and ends one condition
                 currently affecting it (impaired, Slowed, immobilized, shut down, jammed)



\section{Vehicle}
                                                VEHICLE   

Vehicle-type enemies are in-atmosphere military or civilian vehicles found throughout the galaxy. 
 

Making an Vehicle
 
You can use any of the mech classes and apply this template to convert that mech into a vehicle.  
They lose the mech tag and gain the vehicle tag. All vehicles get the following features added to  
their base NPC type:
 
    -    Limited Maneuverability: A vehicle must always move in a straight line (though it can  
         move and boost in separate directions). A vehicle cannot climb or swim. It gets +2  
        Accuracy on checks to avoid being knocked prone, but cannot right itself without  
        assistance.  
    -   Crew: A vehicle might be operated by more than one crew member (typically a minimum  
         number of crew equal to 1/2 of the vehicle’s size). As long at least half the crew is alive, the  
        vehicle can still function.  
    -    No manipulators: A vehicle cannot initiate a grapple, pick up, or manipulate items  
    -   Type: You can give a vehicle one or more of the following types:  
             -   Flier: A flying vehicle can fly when it moves or boosts  
             -   Transport: A transport vehicle can hold one squad or a number of entities whose  
                 total size (added together) is less than its size  
             -   Treads or Hover: A Treaded or Hover vehicle ignores difficult terrain  

\subsection{Ship}
                                                    SHIP

Ship-type enemies are any military or civilian vehicles that are flight capable and operate primarily
in space, outside the bounds of atmosphere (though many can operate in-atmos if need be).


Making a Ship

You can use any of the mech classes and apply this template to convert that mech into a ship.
They lose the mech tag and gain the vehicle tag. All ships get the following features added to
their base NPC type:

    -    Flier: A ship always counts as having an EVA module in space. In atmosphere, it can fly
        with either normal flight or hover flight (depending on type). If a ship is grounded without
         landing, it is immobilized. If it is immobilized, knocked prone, or stunned mid-air, it
        crashes.
    -    Massive size: A ship is typically much larger than a mech. If its size is less than 4,
         increase its size to 4. This template describes ships that can be engaged on a ship-to-
         mech level about up to size 6-8. Much larger ships (size 10/20+) are usually too heavily
        armored for mech-mounted weapons to harm
    -   Crew: A ship might be operated by more than one crew member (typically a minimum
         number of crew equal to 1/2 of the vehicle’s size). As long at least half the crew is alive, the
        ship can still function.

    -    Bonus HP: Increase the Ship’s HP by +5
    -    No manipulators: A ship cannot initiate a grapple, pick up, or manipulate items
    -    Limited melee attacks: A ship cannot make melee attacks other than Ram





-   Transport: A transport ship can hold one squad or a number of entities whose total size
    (added together) is less than its size

                                                         

                                                                                                                

