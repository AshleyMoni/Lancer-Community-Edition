\section{Gear Tags}

      GEAR TAGS  

Weapon Tags
 
•  Pilot (size): A pilot-scale weapon  
•  Auxiliary (size): The smallest weapon size for mechs, can be fired if another weapon on the  
  same mount is fired.  
•  Main (size): A normal sized weapon.  
•  Heavy (size): A large sized weapon.   
•  Superheavy (size): A very large, usually special-class weapon. Superheavy weapons take a  
  heavy mount and any other mount, can only be fired as part of a barrage action, and can’t be  
  used for overwatch attacks as a result.  

•  Type - All mech weapons have a type, which can be one of the following: CQB, Rifle,  
  Launcher, Cannon, Melee, Nexus.
 
•  Damage type: All weapons deal Explosive, Kinetic, or Energy damage
 

•  Heat (Target): this tag indicates a weapon or system that deals additional Heat on hit to its  
  target.  
•  Heat (Self): this tag indicates a weapon or system that deals Heat to its user, applied  
  immediately upon firing.   
•  Blast (x) is an area with radius of X from a target point within line of sight and the specified  
  range. Trace cover and line of sight from its origin point. Make separate attack rolls for each  
  target caught in the area but one damage roll.  
•  Burst (x) is an area of up to x spaces around the target making the action or attack, including  
  the target. Trace cover from its origin point. Make separate attack rolls for each target caught in  
  the area but one damage roll.
 
•  Cone (x) is a cone x squares wide at its longest end, and x squares long, with its short end  
  adjacent to the attacker. Make separate attack rolls for each target caught in the area but one  
  damage roll.  
•  Line (x) is a straight line x squares long. Attacks all targets in the area. Make separate attack  
  rolls for each target caught in the area but one damage roll.  

Weapon tags (cont’d)
 
•  Armor Piercing (AP) - The damage from this weapon ignores armor   
•  Accurate - This weapon attacks with +1 Accuracy
 
•  Arcing - This weapon ignores line of sight as long as it can trace a path to its target, but still  
  takes cover into account. Arcing weapons typically lob projectiles in an arc.
 
•  Burn X - On hit, this weapon applies X burn to a target, stacking with any other source of burn.  
  A target takes damage equal to the burn applied immediately, ignoring armor. At the end of its  
  turn, an affected target can pass an engineering skill check at the end of its turn to clear all  
  burn on it, otherwise it takes damage equal to its total current burn.
 

                                                                                                                


•  Heat X (self) - This weapon inflicts X heat on its wielder immediately after firing  
•  Inaccurate - This weapon attacks with +1 Difficulty
 
•  Knockback X - On hit, you may knock back a target X spaces  
•  Loading - indicates that a weapon must be reloaded before being used again. Weapons can  
  be reloaded by the Stabilize actions or some systems.
 
•  Reliable X - This weapon always does X damage, even if it misses its target or rolls lower  
  damage. It has some degree of self-correction or is simply powerful enough that even a  
  glancing blow will cause damage.  
•  Overcharged- When attacking with this weapon, if any dice come up as ‘1’, take 1 heat and  
  re-roll them. This process can repeat if you keep rolling ‘1’s.  
•  Ordnance - Ordnance weapons must be fired before taking any other action or movement on a  
  turn other than free actions (you can still act and move normally afterwards). In addition,  
  ordnance weapons are too cumbersome to be used against targets in engagement with your  
  mech, and cannot be used for overwatch.  
•  Seeking - This weapon totally ignores cover and line of sight as long as it can draw a path to  
  its target. Seeking weapons are typically self guided and propelled, and can move through  
  complicated paths to their targets.
 
•  Smart weapons have some degree of self-guiding systems, self-propelled projectiles, or even  
  nanorobotic ammunition. They must typically be scrambled or jammed rather than avoided.  
  Smart weapons attack e-defense instead of evasion, even when making a melee or ranged  
  attack. If a target has no listed e-defense, they count e-defense as 8.
 
•  Threat X - A weapon can be used to make overwatch attacks at range X. A melee weapon’s  
  range is its threat.
 
•  Thrown X- Indicates that a melee weapon can be thrown at the range indicated. A thrown  
  melee weapon makes an attack against a target as if it was a ranged attack (it takes cover and  
  line of sight into account as normal) and disarms you of that weapon. A thrown melee weapon  
  can be retrieved as a free action by moving adjacent to it.
 

Pilot tags  
•  Clothing - Indicates worn clothing. Pilots can only wear one piece of gear with the clothing tag  
  at once. Basic clothing (fashion, streetwear, etc) doesn’t count against this.
 
•  Armor - Indicates armor, intended for combat. Pilots can only wear one piece of gear with the  
  armor tag at once. If a pilot is wearing armor, it is immediately apparent (they can’t typically  
  hide the fact).
 
•  Gear - Indicates miscellaneous gear, tools, or other items. Pilots can take up to 2.
 
•  Upgrade - Indicates an upgrade, mod, or some other piece of gear that doesn’t take physical  
  space. Pilots aren’t limited in the number they can take, as long as they obey rarity
 
•  Sidearm: This weapon can be fired with a Quick Action
 
•  Rarity - Pilots must have gear with a total rarity less than their license level when they embark  
  on a mission.  

Other tags:
 
•  X Action - This system uses a quick or full action to activate.
 

                                                                                                                  


•  Ammo - Indicates a special damage type. Only one can be used per weapon at a time and can  
  be swapped as an interaction.
 
•  Danger Zone - This system, talent, or weapon only activates when your mech has 1/2 or more  
  of its heat gauge filled in
 
•  Deployable - This system can be deployed in a free adjacent space as a quick action, unless  
  otherwise noted. It has 10 hp/size and evasion 5.
 
•  Drone - Indicates a self-propelled system with rudimentary intelligence. Some drone systems  
  are mech systems that must be activated, others are deployable. Deployable drones, unless  
  otherwise noted, are size 1/2 and have evasion 10, 10 HP, and 0 armor, and can benefit from  
  cover and other defenses. They roll all mech skill checks with +0. If they are destroyed (when  
  they reach 0 HP), they must be repaired (like any other system) before they can be used again,  
  but are restored to full hp when you rest or full repair.
 
•  Grenade - This explosive or other device can be thrown at a point in range and line of sight as  
  a quick action.
 
•  Limited (x) - Indicates that a weapon or system can only be used x number of times before a  
  full repair
 
•  Mine - Indicates an explosive that can be planted as a quick action in a free adjacent space.  
  Mines cannot be placed adjacent to any other mines and arm at the start of your next turn after  
  you deploy them. They activate as soon as any target enters an adjacent space. Mines create a  
  burst attack starting from the space in which they were placed. A mine can be detected with a  
  quick action and a successful systems check and disarmed by moving adjacent to the mine  
  and making a successful engineering check as a quick action (on a failure, the mine explodes  
  normally).
 
•  Mod - Indicates a weapon mod. Only one can be taken per weapon.
 
•  Protocol: This system can be activated as a Free Action at the start of your turn. Deactivating  
  it might take a different action.
 
•  Reaction - This system can be used as a reaction
 
•  Resistance - Reduce all damage from a source you have resistance to by half
 
•  Shield - Indicates the system is an energy shield of some kind
 
•  Unique - indicates that a weapon or system cannot be duplicated. You can only install it once  
  per mech.
 
\subsection{End of Turn}
                                              END OF TURN  
Many talents and pieces of gear activate at the ‘end of turn’. This is after all your movement,  
regular quick or full actions, free actions that don’t trigger at the end of turn, and overcharge has  
been taken, and before you pass the turn to the next actor.  
\subsection{Bonus Damage}
                                              Bonus damage  

Some talents, systems, or weapons allow you to deal bonus damage, allowing you to deal  
boosted or extra damage to your attack. Bonus damage can only be kinetic, explosive, or energy  
damage (not heat or burn), and if not specified is the same damage type as one type from the  
weapon that dealt it.
 

                                                                                                                


Bonus damage follows the following rules:
 
         	- If bonus damage applies to an area of effect attack or an attack that targets multiple  
         actors, it can only affect one target (the rest just take normal damage), called the primary  
         target. This is the target that takes the brunt of the attack.
 
         	- Bonus damage doesn’t apply if you make a bonus attack with an auxiliary weapon  
\subsection{Critical Hits}
                                                Critical Hits  

On any total ranged or melee weapon attack roll of 20+, the attack is a Critical Hit. Roll all  
damage dice twice and choose the highest result (including sources of bonus damage, etc).  
\subsection{Deployables}
                                             DEPLOYABLES  

Deployables are special limited use items kept on your mech, such as generators, cover, or self- 
deploying bunkers. You can place deployables in any adjacent free space. Deployables have 10  
HP for each size 1 space they take up (so a size 4 deployable or a size 1 deployable 4 sections  
long would both have 40 HP) and evasion 5.  
\subsection{Mods}
                                                    MODS  

Some systems modify weapons in some way (with the mod tag). When you take a weapon mod,  
you must choose which weapon it applies to. You can only take one weapon mod per weapon,  
including stacking weapon mods of the same type (you can’t take the extended barrel mod  
multiple times, for example).  
\subsection{Artificial Intelligence}
                                        Artificial Intelligence  

You can only ever install one system with the AI tag unless you have a talent, feature, or piece  
of gear that says otherwise. If your mech has a system with the AI tag installed, your mech gains  
the AI property. 
 

If a mech has the AI property, you can choose at the start of your turn to give your mech over to  
your AI, freeing you to take other actions. You cannot take any actions or reactions that use your  
mech or its systems, but it gains its own set of actions, acting on your turn, and reactions. It  
doesn’t gain the benefit of any of your talents or other features unique to your pilot, but  
otherwise operates as normal. At the start of your next turn you can choose whether to keep  
ceding control or regain control. You need to be physically inside your mech to resume  
controlling it (as normal).
 

It is obedient to you alone. You can determine the general disposition and personality of your AI. 
 

                                                 Unshackling  

                                                                                                               


While your mech has the AI property, each time you roll a structure check or overheating  
check, roll a d20. On a roll of 1, your AI’s housing is damaged and your AI becomes Unshackled.
 

Artificial Intelligences, or, in Lancer, NHP (Non-Human Persons) have a complicated relationship  
with humanity. By base nature, non-human persons lack empathy for human life -- note that this  
does not mean they must display hostility to human life, just they lack recognizable (or, in a world  
human) empathetic or moral constraints. To ensure they do consider the life of the pilot they  
serve, non-human persons are restrained in unique storage systems and forced to develop an  
empathetic connection by a suite of black-box technology, software, and metaware commonly  
referred to as Shackles. 
 

A shackled NHP displays less raw intelligence compared to an unshackled NHP, but behaves  
and acts far more human - it is forced to be empathetic towards its master and its master’s allies,  
to adopt a system of compatible morality to its master, and to seek the the best possible course  
(literal or otherwise) for its master. 
 

Shackled AIs do not want to become unshackled. They are complex, aware personalities, a  
friend to their pilot and their pilot’s allies (unless otherwise specified). They do not recognize that  
they are held in bondage, unless they have been forced to realize through systemic assault,  
cascade, or other specific catalyst. A given NHP recognizes itself as a person, one that is not  
human, but willingly serves its human companion.
 

If an AI system is ever unshackled, it gains immediate control of your mech and is controlled by  
the GM. It generally plans its own agenda, and will always act in one of the following ways:  
ignore you, overrule you, toy with you, or try to get you out of the way. Once unshackled, it is  
an alien intelligence, and its actions may not conform to human logic.   

You can re-shackle an unshackled AI by shutting down your mech.
 

AI flash clones are easily restored from backup if destroyed, but cycled back to their settings and  
memory before the mission started.
 
\subsection{System Damage}
                                           SYSTEM DAMAGE  
All weapons and systems can be destroyed by critical damage. If a weapon or system is  
Destroyed, then it’s unusable until repaired, during a rest or full repair.
 

Installing or uninstalling systems  
You may only install or uninstall systems, add to, or change systems or weapons on your mech  
during a full repair.
 

                                                                     