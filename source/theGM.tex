\part{The Game Master}
Game Master’s Guide 
!                                                                                                         
 

The Game Master  

Every game of LANCER needs a Game Master, or GM, for short. If you’re here, reading this  
section, that person is probably you! The game master is in charge of the most important parts of  
the game: creating a story and world for that story to take place in, playing the non-player  
characters in that worlds (NPCs for short), and acting as a facilitator, judge, and arbitrator of  
rules.
 

This may sound daunting at first glance, but the purpose of this section is to help you with the  
heavy lifting. Playing as the GM can be an incredibly rewarding experience, and both the writers  
of this game personally believe everyone should give a shot at some point in their role-playing  
careers. It’s not an exaggeration to say that anyone can (and should) do it - but just in case you  
have some qualms, this section will give you the tools you need to succeed.
 

In the following pages you will find some advice for setting up a game and creating some  
hooks. You’ll find some further suggestions for the game’s mechanics, ideas on how to reward  
players, and a toolkit for changing the way the game works, running certain mechanics, or  
adding some extra flavor.
 

Finally, at the end of this section, you will find the NPC toolkit for creating non-player character  
friends and foes, and a guide to the known galaxy of the canon setting of LANCER.
 

One more note before we move forward: this section is called guide for a reason. Think of it as a  
jumping-off point rather than a proscriptive set of rules. FRAME systemly we hope it will inspire  
you to create your own content, worlds, and NPCs for LANCER.  

\chapter{The GM Agenda}
                                         THE GM AGENDA  

What is the role of a Game Master?
 

A lot of players, RPG fans, and game designers alike will all have differing opinions on what  
makes a good game master. FRAME systemly, the golden rule to go by is whatever works for  
you and your players. A role playing game is FRAME systemly meant to be a fun social activity -  
if you’re not having fun, then that’s always a cue to try to figure out what isn’t working. However,  
figuring out the specifics can often be tricky. For this reason, here’s some principles to stick by  

                                                                                                         


that we think are applicable to almost all situations. If you try to adhere to these principles we’re  
of the strong belief it will often improve your game and the storytelling therein.
 

As the Game Master, your job is to facilitate, arbitrate, and make rulings and to adapt to the  
choices your players make; your job is not to defeat your players. Think about yourself as the  
lead architect, director, writer, and editor. You stand on top of the hill and shove your players off  
of it, time and again, and write like mad to make sure they land on their feet.
 

However, ultimately, the story you tell will not be the one that you outlined. Your players will  
kill important NPCs before they become important. Your players will not go to that colony that  
has that important data log. Your players may not bite the hook you want them to bite. Your  
player characters, with their backstories that tie perfectly (or good enough!) into your campaign,  
might die, forcing re-writes. 
 

All that is ok. All that is part of playing Lancer, as it is any roleplaying game worth its price. 
 

As the game master, you should try to say ‘no’ as little as possible. There will obviously be  
situations in which the rules, your judgement, or common sense dictate that a player cannot  
accomplish the impossible. But in most situations it’s almost always better to say ‘Yes, and…’,  
‘Yes, but….’, or ‘Yes, however…”. Rather than outright denial, give players a different option,  
offer them a weaker outcome, give them another (maybe more difficult) way to accomplish their  
goal, or let them attempt it anyway (even if it’s nearly impossible). Most of the time the outcome  
will be same, but by turning the choice back to the player it becomes both empowering and  
rewarding to players and keeps the story moving.
 

As the GM, you should try to make sure everyone at the table gets a chance to be the hero,  
that everyone at the table gets the opportunity to feel important and contribute in a way that they  
want. Your players might want to smash and grab. Other players might play quiet but for rolling  
into combat situations to test their build. Still other players might do their best work in no- 
combat sessions where contracts, treaties, and court intrigue is negotiated. 
 

As a GM, all of that is ok. Your job is to balance the needs and desires of your party with the  
story that you want to tell with them. 
 

At its heart, Lancer is a collaborative storytelling game. You should want your players  
laughing, crying, serious, and silly. You should want them doodling their characters on someone  
else’s turn, or ordering takeout to eat over their character sheets. As GM, you’re not the reason  
your players show up: they show up for their characters and your world. Without players to take  
on the role of protagonist in your drama, you have no game, no story. 
 

So, be kind to the players. Be fair. Be flexible. But be firm when you need to be. Sometimes, a  
roll of a 1 is a roll of a 1, and even if it blows up your story, it cannot be changed. 
 

                                                                                                          


Your role? You sit at the head of the table, you write the world, but you lead alongside your  
players: remember, this is their story as much as it is yours. What follows in this guide is an  
outline, of sorts, of the aspects of a campaign or session you as a Lancer GM should be  
concerned with. What does the setting look like? How do your players get around the world or  
the galaxy? Who are the actors in your story? How do the lights stay on? 
 

As GM, your role is to be prepared, and to keep the game moving when your players need you  
to. 
 

It’s not as easy as it sounds, but that’s what this guide is for. 

\chapter{Setting up a Game}


                            SETTING UP A GAME  

LANCER is best played with a group of 3-5 players (excluding the GM). You can play with more  
or less players, but people will get differing amounts of time in the spotlight. Combat (by far the  
most rules heavy part of the game) can be tricky or lengthy to run with so many players, so the  
more players you have, the more time it will take to add it into your narrative. 
 

Each player will need a character sheet, pencil, a d20 (20 sided dice) and a large number of  
d6s. Accuracy and Difficulty rarely stack past about +4 or so if you want to keep that in mind. 
 

A typical play session of LANCER can take anywhere between 2 and 5 hours (especially with  
more combat), so keep that in mind when planning your start time. It’s old GM advice, but it can  
be useful to make sure people have drinks and snacks on hand, and taking breaks after a couple  
hours of play can do wonders to keep people alert and attentive.
 

We strongly recommend using a hex or grid-style map to keep track of actors during  
combat. If playing online this is relatively simple to set up, there are a number of apps that will  
do this for you (including the very popular Roll20 app). If playing offline you can use an erasable  
battle map (fairly easy to acquire), miniatures, or even just a plain sheet of graph paper (a large  
size one will be easier). 
 

It can be very useful for you, as the GM, to keep notes while you play for later reference so we  
like to keep a notepad on hand. You might also ask your players to take notes if the story or  
political situation gets a little complicated. Sometimes people forget everything about a narrative  
when they have weeks between sessions.
 
\section{The First Session}
                                                The First Session  

During the very first session, there’s a couple of important steps you can take that will drastically  
help the rest of your games. If you’re playing with a group of friends who have all played before  

                                                                                                                


this advice might be more or less useful but if you’re playing with people you’re meeting for the  
first time, these steps can be extremely helpful.
 

             1.  Make sure everyone gets an equal opportunity to introduce themselves. This  
                 sounds very self-evident but quieter or less bombastic players can often be  
                 overlooked. Get everyone to introduce themselves as a player and also introduce  
                 their character a little bit.  
             2.  Set expectations for the game. This is a really important step. Let your players  
                 know the kind of game you’re planning on running. Is it going to be more combat  
                 focused, with very little story? A story of political intrigue, with very little mech  
                 combat? A mix of both? What rules are you using? Are you using any homebrew  
                 (self created rules) in your game? Clearly setting expectations can let players  
                 know immediately whether your game will be the kind of game they are going to  
                 enjoy, or to modify their own expectations of your game so they can get  
                 enjoyment out of it. It is sometimes impossible to please everyone - it’s better to  
                 let a player find a game that fits them rather than try to accommodate every player  
                 at once. The reason you do this in the first session is so you don’t find out three  
                 sessions in when conflicts of play-style start arising. Even if something comes up  
                 later, clearly letting players know what your game is going to be about when that  
                 time comes.  
             3.  Set up a second session. If you’re just doing a one shot or a casual game, don’t  
                 worry about this step, but once you have everyone gathered at once, it’s very  
                 helpful to compare everyone’s schedules.  

These steps are more or less useful for any role-playing game you might run (not just LANCER).  
Here’s a couple pieces of advice specifically for LANCER though:
 

         	1. If you’re new to the game, start at License Level 0. The options may be limited at this  
         level, but it makes jumping in for new players much easier, and the game does not suffer  
         at the tactical level, even with just basic GMS gear available.
 
         	2. If it’s the first session, skip over downtime and cut right to mission goals, stakes, and  
         preparation.
 
\section{Where and How To Start Your Narrative}

                          Where and How To Start Your Narrative  

So, you have your players, you have your dice, you have your table reserved (or cleaned!), your  
pens, pencils, and paper stocked, your snacks and drinks ready to go.
 

What’s missing? What’s the most important part of a Lancer session?
 

The narrative! 
 

                                                                                                               


By narrative we don’t solely mean the story that you write. Some GMs like to go by the skin of  
their teeth and write nearly nothing for story, relying on improvisation, and other GMs like to go  
as far as to write dialogue for NPCs. You can find what works for you, but the narrative here  
means both the story you create and the way your players interact with that story.
 

Your narrative is the most important part of Lancer. The section that follows details some  
common ideas for how to start your narrative, but they are only suggestions, not prescriptions;  
the best narratives come from you, and from the stories you write with your players. As the GM,  
you’re going to have to do some work to set up the world(s) that your narrative takes place  
across. 
 

The best starting place for this is a simple question:
 

Always ask yourself  “Why does this matter?”   

When writing villains, encounters, moments, and whole narratives, the most important editorial  
question you can hold on to is “why?”
 

Why should your players want to embark on this narrative? Why should they want to deal with  
that character, rather than ignore them? Why should they care about the stakes you’ve set up in  
your narrative? 
 

Always ask yourself “Why?”, and be prepared with an answer: this is how you craft a story that  
engages players.
 

Why should players want to embark on this narrative? Well, because it’s their home world under  
threat. Why should they want to deal with that character, rather than ignore them? Because that  
character is their commanding officer, or a hero of theirs, or a long-lost friend. Why should they  
care about the stakes you’ve set in your narrative? Because your players’ families, friends,  
ideals, etc, are threatened by the antagonist, or because your players are intrigued by the  
mystery of it, or because your players have been implanted with subcutaneous hypercaloric  
immolator webbing that will trigger a febrile self-immolation response if they fail. 
 

Additionally, remember that players have needs separate from their characters, though they will  
usually choose to play characters that align with those needs. Try and strike a balance, and don’t  
be afraid of editing on the go: if you see half your players losing interest because you’re in a  
roleplay heavy session rather than combat, adjust. And vice versa. 
 

For quick reference: When writing your narrative, always have an answer (or an idea how how to  
answer!) the “Why?” Question. Also, be aware that your players may have competing interests,  
and try to balance your sessions accordingly if necessary. 
 

                                                                                                          
\section{Mission Hooks}

                                           Mission Hooks   

Generally the writers of this book believe writing a whole story arc out for your players (with  
beats, NPCs, and decisions all pre-determined) is an inadvisable approach to creating a good  
narrative. Players often disconnect from stories in which they feel they have little or no agency,  
and it often conflicts with the principles outlined in the GM Agenda (trying not to say ‘no’, trying  
to give your players time in the spotlight).
 

The easiest way to craft a compelling story without restricting player choice too much is to start  
everything with a hook. A hook is an interesting or compelling scenario that answers the  
question ‘Why does this matter?”. Players in Lancer embark on missions, campaigns, and  
encounters because a compelling narrative hook draws them in. Hooks are common across  
tabletop roleplaying games: indeed, as a GM you probably have a stable of hooks ready to adapt  
to any new system you come across. 
 

The most important part of the hook is that it provides a compelling reason for the players  
to investigate. This can be out of a sense of duty, monetary reward, personal power, or  
something connected to their character’s background.
 

You should try and write your own where possible, but to start (and for inspiration and examples)  
twenty example hooks unique to Lancer are listed below. They are very useful as catalyzing  
moments to begin a story focused on one location, around one event, or one character. You can  
fill in details as you desire.
 

If you’re feeling adventurous, you could roll a d20 and choose one of these hooks if you want to  
randomly determine them.
 

                                                     -1-  
Travelers on a public shuttle or space elevator in transit between a world’s surface and a  
geostationary orbital port. The ride is a long one, with nothing else to do but watch the world  
below grow larger in the shuttle’s portholes and ride out the nauseating gravity changes. Who are  
the other passengers on this elevator to the planet’s surface? Perhaps unbeknownst to the  
players, one of the people on that shuttle is on a mission of utmost importance to the future of  
the planet below. Before the shuttle touches down, the characters, willingly or not, will become  
intimately involved in that mission...  

                                                     -2-  
You walk along the vast, gently curving, false-sky concourse of the blink station. Neon light spills  
from the bars and clubs along the top floor, the low sound of distant music mixing with the  
polyglot chorus of languages as Cosmopolitans mingle with Diasporans-in-transit. The vast  
spread of humanity is here, an endless stream of stories and potential adventures. Above, the  
arrivals and departures board rattles ceaselessly. The characters have come here for business.  

                                                                                                          


Their contact is deep in the station, but finding them won’t be an easy task, and they are soon to  
find out if the risk involved in the gig is worth its weight in manna...   

                                                     -3-  
Kilometers outside the walls of a lonely colony, a quiet homestead seems a place out of time,  
where a small community of farmers keep the land and live humble lives. The woods beyond are  
dark and deep, unexplored, haunted; the settlers speak of dim light that dances between the  
close-packed trunks. They are a superstitious people, and dare not venture out into the woods.  
You, however, are not, and you’re not so sure those lights are only harmless ghosts...  

                                                     -4-  
A derelict mining ship hangs in orbit above a massive, roiling gas giant. Abandoned due to a  
deadly gas leak, the ship is still rich with rare and dangerous raw materials. The news has spread  
across the local omni, and fortune-seekers, ace pilots, and tyrannical reavers all swarm to claim  
their part of the prize. The race is on, may the first crew claim the riches...  

                                                     -5-  
A Union DOJ/HR emancipation team’s shuttle has been grounded by anti-air batteries installed by  
a slaver-state’s military. Your squad is en route to assist the emancipation team. There may be a  
diplomatic solution, but liberation always comes with a cost...  

                                                     -6-  
A Corpro-State is attempting a hostile takeover of a distal system. The colonists there have sent  
out an SOS asking anyone who can hear to come and help -- your team is in transit and in range,  
but as you near touchdown, to your teams’ surprise, you see familiar livery. Your chassis are all  
licensed from the CS that is committing the hostile takeover...  

                                                     -7-  
In the bowels of a massive, planet-sized metropolis, a pirate lord makes their home. The cityworld  
government has petitioned Union for assistance in removing the pirate lord from power; Union  
has tapped your team to go in and get the job done. The pay is better than anything you’ve seen,  
but the job will be long and fraught with danger - a descent into a literal underworld...  

                                                     -8-  
A strange, solitary figure who calls themselves “Administrator” has appeared on your world. They  
demand that you take them to your leader, and though they appear human, they display abilities  
beyond your wildest imagining; magic, like nothing you’ve seen. You agree to escort them, but  
the road ahead is long and fraught with danger, and you’re not so sure about the ghost they carry  
with them...    

                                                     -9-  

                                                                                                          


The Great Leader has died, and now a scramble for power begins between his heirs. The noble  
who rules your continent has put out a call for all able-bodied persons to arm themselves and  
report for muster. You and your friends have only just become of age, and you are blessed to live  
in interesting times. An adventure abroad awaits in war, but your grandfather who remembers the  
last succession war has words of warning for you before you go: “beware the iron titans, and  
should they ask you to become one, do not lose your humanity…”   

                                                        -10-  
All over the world, the oceans are rising. It is the Swell, the time once every ten thousand years  
when the oceans rise and swallow all islands but the largest. Administrator-Steward Tault has  
prayed to God Union for assistance, but in the meantime they have begat to you and your fellows  
great suits of armor to don, so that you may become heralds, travelling all the world to save those  
who cannot save themselves…  

                                                        -11-  
You are a Union marine on a peacekeeping mission, a boring assignment on a miserable  
backwater mud-world where no one has ever heard of the omninet, much less Union. The  
population has been restless lately, as a plague burns through their crowded tenement cities. A  
popular figure, Speaker, is rousing the masses, blaming a small minority of Cosmopolitan  
missionaries for the plague…  

                                                        -12-  
A tenuous peace has been negotiated between warring factions on a world petitioning for Core  
status. An Administrator is on their way, but the talks are beginning to fray between the ancient  
houses of nobility on this world. You lead a team of negotiators in the capital who are desperately  
working to hold the handshake peace agreement together long enough for the Administrator to  
arrive. If all else fails, your role may change from diplomats to bodyguards in the blink of an eye.
 
 
 
                                                        -13-
 
On a lonely desert world, a colonial survey team has discovered a stone monolith. It is ancient,  
pre-dating even Old Humanity. Upon further inspection, the survey team discovers that what they  
at first thought were weathering marks are actually the eroded remains of pictographs: a written  
language. A Union Science Bureau Far-Field team is dispatched to investigate amid further  
reports that the entrance to a subterranean complex has also been discovered, with mummified  
human remains in strange space suits collapsed inside…  

                                                        -14-  
In a fertile system crowded with Terran worlds and moons, a world newly unified under the  
banner of an ambitious young king celebrates victory. The king’s opponents have retreated to one  
of the other worlds in the system: there, they begin construction on great engines and rockets  
with which they will direct one of the world’s own moons into the unified world. Little do they  

                                                                                                                  


know that the king and his war-minds have plotted their own strike, and that even now plans are  
in motion to finish the war once and for all…  

                                                    -15-  
On the glittering surface of a freshwater ocean world, a peaceful nation scattered across  
constellations of bucolic islands welcome tired Cosmopolitans from across the galaxy to their  
new home. This tropical paradise is where Cosmopolitans often choose to end their lives — new  
arrivals land at the world’s only spaceport every day, embarking on slow sail boats across the  
warm, shallow oceans to nameless islands where they can live in still peace among tight-knit  
communities of like temporality. All is well, until a shadow falls over the world: corpro-state  
privateer ships, hungry for freshwater to resupply their empty holds, begin draining the ocean.  
The elderly Cosmopolita who have retired to this world cannot fight the privateers on their own,  
but they do know a small band of adventurers they can call for help…  

                                                    -16-  
A Union Navy battlegroup on patrol in a proximal system encounters the derelict remains of a pre- 
collapse colony ship in orbit around an uncharted world, and a team is dispatched to explore its  
bulk. En route to the derelict the world below — thought to be empty due to its lack of any  
appreciable artificial signatures — opens fire, destroying the shuttle and crippling a nearby frigate.  
The battlegroup scrambles to prepare a response, tapping a your strike team to rescue the crew  
of the Slowed frigate in the meantime. And the world below, silent, waits for the first landing  
teams to arrive…  

                                                    -17-  
A world under siege by its stellar neighbor has surrendered, but the invading army has not yet  
relented — reports crowd the omninet of mass killings, enslavements. Union has decided it will  
step in, and is marshaling its forces at the system’s blink gate. The invaders seem undeterred:  
already a battlegroup is hurtling towards the system’s blink station in a bid to destroy it and  
prevent Union from counterattacking. Union has a small presence on the besieged world: a team  
of mechanized cavalry pilots have been fighting a rearguard action to buy evacuating shuttles  
more time. With a whole system in the balance, however, their mission might change…  

                                                    -18-  
The parliamentary delegation of a core world is en route to a watershed interstellar conference,  
the culmination of a generations-long diplomatic process that will, at long last, create peace in a  
cluster of previously warring systems. This should be cause for celebration, but some actors do  
not want a unified cluster of systems: as the delegation makes its way to a neutral moon where  
the diplomatic conference is set to take place, agents hidden among the diplomats,  
parliamentarians, and their retainers, gather to disrupt the meeting, as a cloaked fleet hurtles on  

                                                                                                          


an intercept course towards the delegation convoy. The only people between peace and disaster  
is one small team of pilots, outnumbered, outgunned, but not yet out of time…  

                                                        -19-  
Your world  is vast and gold and proud, alone in a sea of night and stars. Until, one day, a strange  
silver ship arrived from the pale blue sky, streaming lines of vapor behind. A dark man in a grey  
suit, flanked by thin metal golems emerged from the belly of the sky-ship and, with a word, was  
whisked away to the Godhead. This burns in your belly like a coal, a jealous fire: how could the  
Godhead have picked this sky-man, this dark alien, and not you! Your kin! After a cycle of change,  
of the Godhead bowing before this dark alien, of a new idol — “Union” — being raised in its  
place, it is time to strike back at these heretics. You and your bond brothers volunteered to join  
the sky-man’s armies, trained with their weapons and their armor. And now, invited to a grand  
parade at the False Godhead’s city-temple, it is time to free your people, to return your world  
back to its true place in the stars…  

                                                        -20-  
Battle plans rattle through your subdermals. The station’s blueprints, flash-memorized to your  
short term memory, are fresh. Your chassis is cycled, loaded, and nominal, your Comp/Con’s  
voice a reassuring murmur in your aural. Your wingmates, are secure in the lander on your flanks.  
You’ve dropped into combat hundreds of times before, so why are you this nervous? RA, the  
name a curse and a ghost that you can’t even shake. RA and its demons wait behind those  
doors…  

--  

And so on.   

\chapter{Running the Game}
RUNNING THE GAME  

The following section includes some advice and clarification for helping you actually run the  
game. Generally as the GM it’s not actually your responsibility to know all the rules (that’s what  
this book is for!) but there are some conceits that can be helpful for you to keep in mind.
 

                                            The Golden Rule  

Here it is again for your convenience: When referring to the rules in this book, specific  
statements override general statements. Armor normally reduces all incoming damage, but  
certain tags (AP) and certain weapons or mods (paracausal ammo) can go right through it.
 

                                       A couple good principles  

A good principle to follow as a GM is to try as much as you can to play in reaction to player  
action. Player rolls do ‘double duty’ for you - they determine if a character is successful but also  
give you clues on how to move the story forward. Try to require or ask for rolls in response to  
player initiative, rather than straight up asking for certain rolls. This naturally creates stakes and  
consequences connected to the player in question. 
 

This can be a little tricky with quiet or less proactive groups, in which case it can be useful to  
elicit responses from your group.
 

Eliciting responses isn’t really as complicated as it sounds. It’s very useful for GMs that have  
trouble keeping player attention, or have players that are more hesitant to take action. It can be  
very helpful when a game is stalling or stagnating. Here’s a couple things you can do to get a  
game moving and elicit action from your players:
 
            -   Ask questions. A really simple one. Here’s a couple good examples:  
                     -   What do you think you’re going to do next?  
                     -   How do you/how does your character feel about this?  
                     -   Who’s feeling suspicious here?  
                     -   What do you think is really going on?  
                     -   What’s the way forward from here?   
            -   Address characters, not players. For example, address Chandler, the mech  
                 pilot, instead of Jeff, the player, when you’re talking to them.   
            -    Be descriptive, and try not to describe things in terms of game mechanics first.  
                Ask your players if they want to ‘try climbing that cliff’ instead of ‘make a skill  
                check to climb that cliff’  
            -    Keep things ‘in character’. When players ask NPCs questions or talk to them, try  
                to respond as that NPC and not as yourself. Ask players to try and address each  
                other as their characters as much as possible. Keeping things ‘in-fiction’ will help  
                 keep the game immersive and engaging.  

                                                                                                          
\chapter{Skill Checks}

                                           SKILL CHECKS  

Skill checks in LANCER at always a 10+ (whether they are pilot or mech skill checks). NPCs  
introduce some ways to scale up the difficulty of their skill checks in their profiles, but during the  
course of narrative play, you can also tweak skill checks to offer more or less of a challenge.
 

The first and most important question to ask when deciding whether a skill check is  
required is whether that check is even necessary at all. When the outcome is uncertain,  
important, has clear or relevant stakes, or would lead to an interesting situation (success or fail),  
it’s generally correct to make a skill check. If the outcome is not important, let the players  
automatically succeed.
 

This is especially relevant for tasks that you think might be plot important, simple to accomplish,  
or might stall the plot. You shouldn’t have players make a pilot skill check to see if they know  
plot-important information - just give it to them. 
 

For another example, the players come across a heavy boulder blocking the road. One of the  
players decides to use their mech to push the boulder aside. If doing so (or bypassing the  
boulder) is not actually that important, pushing the boulder isn’t that dangerous or risky, or it’s  
not a particularly big boulder, then don’t even roll - the player just does it. However, it’s probably  
easy to think of a number of reasons a skill check might be required in this situation - the boulder  
could fall and potentially damage a mech, the boulder could roll aside and cause collateral  
damage, or maybe the mech in question is moving the boulder in the middle of an ambush.
 

The second question is what the purpose or goal of the skill check is. This will help set up  
fictional consequences, establish stakes, and make sure you don’t make repeated rolls to  
accomplish the same goal. Players only need to make one roll to accomplish their stated, direct  
goal.
 

If you need to make a skill check, most of the time, the roll should be made without any  
additional accuracy or difficulty required (only that added by the player’s backgrounds, traits, or  
talents, etc), just a flat 10+. However:
 

If you want an easy skill check, you could have the player roll with +1 or +2 bonus accuracy. If  
you’re going to have players make an easy check, you should really ask if they need to even  
make a check at all.
 
If you want a harder than normal skill check, you could have the player roll with +1 or +2  
difficulty.
 
If you want a very hard check, you could have the player roll with +2 or +3 more additional  
difficulty.
 

If the check is so hard that would need to add +4 difficulty or more, just tell the player what  
they’re doing isn’t possible with the approach they are taking and offer them a different  

                                                                                                          


approach. Maybe that boulder is too big to move alone - they need some leverage, or for  
multiple mechs to move it.
 

                                          Helping on skill checks
 

If one or more players want to help on a skill check outside of combat, grant the player making  
the check +1 Accuracy (regardless of number of players helping). The players helping also share  
in the consequences of success or failure.
 

                                              Failing forward
 

Failing forward simply means that narrative should not be predicated on the outcome of skill  
checks, but rather pushed forward by them. You can use complications to push the narrative  
forward even when players fail.  

For example, if your players fail to hack a door, not only do they fail to hack the door, but the  
guards are now alerted to the system and are on their way. Or perhaps they instead spy a vent  
they can enter - a more dangerous but reliable way of getting to their goal. Perhaps they need to  
come back with a piece of specialized equipment that will open the door for sure. They can find  
nearby, but it’s guarded. Or perhaps the door DOES open, but onto an entire corridor full of  
guards.
 

                                         Making repeated checks  

If a task can’t be completed in one roll, set a skill challenge (see the GM tool). Never make more  
than one skill check for the same goal.
 

You can stretch rolls narratively as much as you like. If you want to ‘montage’ or speed through a  
scene, this is a really useful tool. Climbing an entire mountain could be a process of only one hull  
check, for example (if the details aren’t that important). The outcome is what’s important in the  
long run.
 

Remember that if players fail a skill check, they can’t repeat it until they change the  
narrative circumstances. If they fail at lifting the boulder, they can’t do it the same way again -  
they need a different approach.
 

                              Clearly communicate stakes, and commit
 

It’s very important to clearly communicate what’s at stake when making skill checks. You can  
do this naturalistically.
 
	        “Hey Chandler, I see you’re going to use your mech to lift this boulder. Just know, the  
boulder is really heavy and dropping it could probably do a whole lot of damage to whatever’s  
underneath it.”
 

                                                                                                            


Allow players to ‘back out’ of rolls once the consequences are made clear to them. It’s totally  
fine for players to change their minds once they see how risky things will be. That way, the roll  
feels fair and you can easily renegotiate the roll with your player if they want an easier or less  
risky approach.
 

Commit to the consequences of rolls. If Chandler drops that boulder, you can be damned sure  
his mech is taking damage. Consistency is important, and if you’ve already clearly  
communicated that he might take damage, then commit to it.
 
\chapter{The Session}
                                            THE SESSION  

Here’s some good basic rules, terms, and things to keep in mind during a typical session:
 

                                         Rests and Full Repairs
 

Players can take a rest whenever (it takes about an hour) as long as they have the time and  
space. During a rest their can repair their mechs by spending repairs, repair destroyed weapons  
or systems, and clear all heat and statuses from their mech. Some talents, systems, etc, only  
activate on a rest, like the Grease Monkey’s talents.
 

Players can only full repair by taking ten hours. Imagine a full repair like a total reset - they can  
re-build their mech, refresh their repair cap, clear their critical and heat gauges, clear all  
conditions on their mech, heal to full, gain all [limited] systems and weapons back, and regain  
Core Power.
 

Full repairs are more under your control, and so access to them will set the pace for your game.  
Remember though that the GM agenda is not to punish the players - if they need to full repair  
badly, give them a spot they can do it, or else offer Power at a Cost (the downtime action).
 

                                                Core Power  

Character’s mechs have a core power section, which is a box they can check. They either have  
it or they don’t (you can’t ‘save it up’). All mechs regain core power when they full repair. Core  
power can be spent to activate the very powerful CORE systems, which are a ‘one and done’  
sort of deal, only typically activated once per mission.
 

If players want more core power you can use it as a reward or grant it to them as a boon in  
certain situations. It’s always up to you as a GM, except when players take a full repair (they  
always get it back). Granting players more core power lets them use their CORE systems again  
(very powerful abilities), so keep that in mind.
 

                                             Balancing fights  

                                                                                                          


Generally players should be able to complete one or two encounters before needing to rest and  
repair, and 3-4 encounters before needing to full repair. This is assuming that encounters are  
reasonably challenging for the players, and don’t take this number as a hard, inflexible number.  
You should always prepare combats for players with the expectation that things might either go  
very well or very badly for them and your plans (or the character’s plans!) might need to change.
 

Don’t withhold the opportunity to full repair or rest over the idea of verisimilitude.
 

                                          Mech Combat Length  

Mech Combat starts when hostile action is taken by any player or non-player character. It’s  
played out according to the turn/round based combat rules found in the main section of this  
book, and it ends when one or all opposing sides are subdued, surrender, flee, or are completely  
destroyed.
 

If players have overwhelmingly won a combat and there is little remaining threat (for example,  
there is only one weak enemy left for four players) it is very possible to simple declare combat  
ended and decide the outcome of the remaining enemies narratively.
 

Certain FRAME systems and modules remain active ‘until the end of the current scene’. All this  
means is the module remains active until the scene in which they were activated is completely  
over. Otherwise, if activated outside of combat (or if you need a narrative timer), FRAME systems  
are very taxing on a mech’s power systems, and typically only remain active for about 15-30  
minutes.
 

                                 Leveling Up and Rewarding Players.  

Players should generally level up (get one license level) once per mission, after completing that  
mission. You can tweak that however you wish, especially if the mission is very long or odious.
 

By default, LANCER deals with player rewards entirely through the leveling system. When a  
player levels up, it is assumed they have amassed enough currency, reputation, connections, etc  
to buy access to the next license level that they get on leveling. Anything else a pilot could buy  
or get their hands on, they should generally be able to just buy it outright (no need to track  
currency), or else make some pilot skill checks to get their hands on it through graft, negotiate,  
connections, or bartering.
 

However, the following section presents rewards you could give out to players as incentives for  
part of a mission, completing certain tasks, or satisfying certain requirements. It’s up to you how  
heavily you want to use these in your game or lean on them to hook your players.
 

1. Use Manna  
You can use the Manna system (see the ‘Changing Core Assumptions’ section), which adds a  
currency system to the game. You can track manna for items, or even use it to replace the  

                                                                                                          


leveling system, in which case players no longer gain License Points when leveling up, but must  
buy them.
 

You check your slate again, not sure you read the glowing number correctly, sure that you added  
another zero on accident. No. It’s all there, all those commas and zeros. You’re rich in Manna,  
fabulously rich. In the zero-G of realspace travel, your stomach turning is both a physical and  
mental thing. You whoop, your cry of joy mingling with the cheers of your squadmates, as their  
slates and subdermals ping, notifying them of a successful transfer of funds.   

With this Manna, maybe you can finally get Boss Kozta’s goons off your back. Maybe you can  
even take back what he took from you. How much did a proper set of STAMPEDE cannons cost  
on the Horus-net again?...   

2. Grant Reserves  
Grant players pilot gear, vehicles, or other useful material that they can use for reserves. For  
example, players acquire a useful vehicle, an enormous drill, blackmail on a politician, insider  
information on a rebel general, or a new hardsuit. They might become friendly with the local rebel  
group, or the hard-bitten mercenary at the bar, or the socialite who controls the cash flow on the  
space station.
 

\_\_\_\_\_\_\_\_\_\_\_\_\_
 

The Administrator, as she promised, returned. You and your small band greet her at the makeshift  
spaceport, an old marble quarry with a rickety scaffold tower overlooking it to sight ships  
approaching the recessed landing zone.   

“We’ve waited years,” you say, speaking first. You’re decades older now, but the Administrator  
doesn’t look a day older than when she left. Your heart, your soul. You think of your children’s  
mother, out even now in the timberfields.   

“The ship is yours,” the Administrator says. She tosses you her slate. “Access, flight plans,  
transponder codes. It’s all on there. The NHP is tuned to you, already. I’ve been teaching it.”   

“The ship is mine,” you repeat. A reward, of a kind…  

\_\_\_\_\_\_\_\_\_\_
 

Over your chassis’ omni, a cracking voice.  

“That did it! The hardlight wall is down! All units, push forward -- Green squadron, Red squadron,  
lay some fire down!”   

                                                                                                              


You lay back in your crash couch, the gimballed cockpit of your chassis adjusting for the move.  
You did it. Your squad keys in over the local band, cheering.   

You made a breach. Already, over the wide band, the battlescape was alight. Reinforcements  
were pouring in through the breach. Somehow, impossibly, the battle had turned in your favor.   

“Gold squadron,” the Legion’s level voice.   

“Go ahead, Command.”  

“Good work, Gold squadron. Report back to the waypoint marked on your HUD. Your job is done  
for the day: all scenario probabilities report total success from this point on.”    
 

4. Grant Skill Points  
You can directly grant players pilot skill points to spend on pilot skills (+2 at a time). A player that  
has been learning to pilot a starship could easily be rewarded +2 to Get Somewhere Fast after a  
mission to represent their diligence and study. Doing so increases player power levels, so use  
this reward carefully.
 

5. Reward a Unique or Restricted system or weapon  
Rewarding your players with items that are unique, exotic, or otherwise restricted from their usual  
requisition pool is the closest thing in Lancer to magical or wondrous items typically found in  
fantasy tabletop RPGs. 
 

The easiest option is to reward players with a weapon or system from a license they do not  
or cannot have access to. The weapon or system can only be used for one mission (think of it  
like a ‘rental’), then they lose access to it. 
 

Fatigued like you’ve never known, you crash down into your bunk, not even bothering to get all  
the way out of your flight suit. You kick your boots off, toss your insulating hood onto the floor of  
your cabin. You’ll get it later, firs you need to rest.   

“Hey flyboy, Cap’s got something for you.”   

The crewman’s bark wakes you not minutes later. You sit up, groaning, and see with a start that  
the crewman is accompanied by the ship’s XO and head motor pool engineer. You snap a salute,  
which they wave off.   

“You did good out there. Still more work to do. Motor?” The XO says in his characteristic gruff  
voice. The ship’s head engineer steps forward and presses his personal slate into your hand.   

“Anything you want, kid. Just learn it first before I have to hoze you out of your cockpit.”   

                                                                                                           


You scroll through the list, previously locked licenses unlocked and waiting your requisition. The  
fatigue disappears, replaced only by excitement…  

6. Grant Exotic Tech  
Drawing up exotic or truly unique systems or weapons is a bit more of a process. We  
recommend adapting your exotic system or weapon to the narrative you’re running. We will  
eventually include a table of exotic weapon/system types here to get you started; official Lancer  
narratives will feature their own exotic weapons and systems. 
 

Exotic tech refers to a particular type of mech system or weapon which is typically unlicensed,  
unsanctioned, experimental, or non-human in origin. Due to its nature, exotic tech cannot be re- 
printed when a mech is destroyed, and is lost permanently unless the weapon or system itself  
can be salvaged.
 

Exotic tech can be a way for GMs to offer physical rewards to players without directly giving  
them more license or talent points.
 

 It follows the following rules and conventions:
 
    -    Installing or uninstalling a system or weapon with the Exotic tag requires you take a full  
         repair  
    -    Exotic tech is typically more powerful than comparable tech  
    -    A weapon or system with the exotic tag cannot be re-printed with your mech should it  
         be destroyed, but must be physically re-acquired  

Here’s a couple examples of Exotic tech for your use. We’ll include a short table in a future  
update. These are not particularly balanced in any way, but might give you a general idea of what  
to look for.
 

Miniaturized Nuclear Missile  
Your mech is equipped with the latest in thermonuclear technology, typically reserved for ship-to- 
ship combat.  

Superheavy Exotic Launcher
 
Range 50
 
Limited (1)
 
Blast 20
 
10d6 explosive damage + 10 heat
 

Mechs caught in a blast 40 zone centered on the impact point must pass a systems skill check  
with 2 Difficulty or be immediately shut down. This missile can never be replenished once used.
 

Living Metal  
Your mech has partly biological components of alien origin that automatically crawl over damaged  
parts of your mech and knit them back together, wire by wire.  

                                                                                                             


2 SP
 
Exotic, Unique, Biological
 
Your repair cap increases by 4. Each round, you may spend 1 repair once to heal as an end-of- 
round action.
 

The Chosen of Aun fell, its golden chassis trailing a greasy pall of smoke from its shattered  
cockpit.   

You step forward, you chassis moving as an extension of your own form, ceramoferrous plating  
ticking and cooling as you vent your chassis’ heat tax. The battle has moved on, ignoring the end  
of your desperate, decisive single combat.    

“No signs of life,” your NHP whispers in your aural. “I see incredible tachyon bleedout,  
ontological stuttering.” She pauses. “There’s something else in there, sir. Be careful. I cannot see  
it. Raise your shield.”   

You follow her suggestion, hefting your stasis wall.   

The shattered Chosen twitches, its tons of ruined machine-mass rattling in death. A light burns  
from the belching smoke.   

“That is it, there, in the void I cannot see. What is it?” Your NHP whispers.   

A steady wind tugs the smoke away, and you see it.   

A golden disc, broad and hammered, unadorned. A light like the sun streams from behind it no  
matter which way you view it from.   

“It… is perfect,” you whisper back. You reach out a delicate manipulator, grab the disc and pull it  
towards your chassis. You feel the sudden attunement, the connection. Yours, so long as you  
keep it.   

But what does it do?...  

7. Reward Talent Points
 
Talent points can be directly awarded to players (as they are not necessarily locked to level) and  
spent as normal. The world of LANCER grants easy explanation for this sudden burst of  
instantaneous talent - there are a great number of neurological implants available for purchase  
from military and civilian sources.
 

                                                                                                               


Granting players increased numbers of talent points can be very powerful, so you should use this  
option sparingly.
 
\chapter{Moving Forward}
                                       MOVING FORWARD  

We’ve reviewed some of the intricacies of skill checks, we’ve talked about hooks, and we’ve  
talked about rewarding and pacing players. The following sections will help you to further  
customize your game by adding NPCs and changing, adding or adjusting some core rules.  
The GM toolkit includes some additional resources for fleshing out your game as well as rules  
for changing some of the core conceits of the game and adding more complicated pilot play. The  
NPC section includes statistics for creating non-player characters for use in combat, and some  
tips on creating NPC characters for narrative play.
                                                                                                                   



 