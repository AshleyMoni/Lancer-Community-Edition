\section{Other Actions} OTHER ACTIONS  
\subsection{Activate}
                                             ACTIVATE  
Some systems or pieces of gear take either a quick or full action to use or activate. Such  
systems are marked with the quick or full action tags.
 
\subsection{Shut Down}
                                           SHUT DOWN  

Shutting Down your mech is a risky move, though one that is sometimes necessary to prevent  
potentially catastrophic systemic overload or AI unshackling.   

You can shut down as a quick action. When you take the Shut Down action, your mech powers  
completely off and enters the Shut Down state. While Shut Down:
 
       •  Your mech is stunned. However, you can still take the Boot action to reboot your mech.  
       •  Your mech is immune to all tech actions or attacks and can’t benefit from friendly tech  
         actions. Any tech effects or conditions caused by a tech action (such as lock on, etc)  
         affecting the mech immediately end.
 
       •  Your evasion becomes 5  
       •  Your mech immediately cools (get rid of all heat)  
       • Any unshackled AI you have installed are re-shackled.  
\subsection{Boot Up}
                                              BOOT UP
 
You can power up a shut down mech as a full action, ending the shut down condition on it.  
Mechs that are powered off must be powered on with a boot up action before becoming active.  
You must be piloting a mech to boot it up.
 

                                                                                                         

\subsection{Mount or Dismount}
                                     MOUNT OR DISMOUNT   

Mounting or Dismounting a mech is a turn of phrase commonly used by pilots. You don‘t “get  
in“ or “climb aboard“, you mount. You‘re the cavalry, after all. It takes a quick action to mount or  
dismount. You must be adjacent to your mech to Mount it, and when you Dismount your mech,  
you are placed adjacent to it. If there’s no free space, you cannot dismount your mech.
 

If you want to Eject when you dismount your mech, you can do so, flying 6 in a direction of your  
choice. However, its a one-way system meant to be used in case of emergency, and leaves your  
mech permanently impaired until you full repair (and the eject system can’t be used again until  
you take a full repair).
 
\subsection{Self Destruct}
                                          SELF DESTRUCT  

Self-destructing by overloading your reactor is a final, catastrophic play a pilot can trigger. You  
can initiate self destruct as quick action, causing your reactor to start melting down. At the end  
of your next turn, or up to two turns after (you choose), your mech will explode as though it  
suffered a reactor meltdown, annihilating it, killing any pilot inside, and causing a burst 2  
explosion for 4d6 explosive damage around it. Characters caught in the explosion can pass an  
agility check to halve this damage.  
\subsection{Prepare}
                                                PREPARE  

When you prepare an action, you’re holding in preparation for a specific time or trigger (a more  
advantageous shot, for example). You can only prepare a quick action, and it costs a quick  
action to prepare. This counts as that action’s duplicate, so you can’t, for example, skirmish and  
then prepare a skirmish action.
 

Until the start of your next turn, you can take the prepared action as a reaction. You must set a  
trigger for this reaction phrased as a ‘When X, then Y’ sentence. X must be an enemy or allied  
reaction, action, or movement. For example: “When my ally moves adjacent to me, I want to  
throw a smoke grenade,” or “When an enemy moves adjacent to me, I want to ram them”.
 

It is apparent to a casual observer when you are preparing an action (you are clearly taking aim,  
cycling up systems, etc). You can’t take reactions while you’re holding a prepared action, but can  
take them normally afterwards. If you want to take a reaction and drop your prepared reaction,  
you can also do so. If the trigger doesn’t activate, you lose your prepared action.
 
\subsection{Overcharge}
                                            OVERCHARGE  
It is possible for skilled pilots to push their mech beyond factory specifications for a short period  
of time in order to gain a tactical advantage. Moments of hyperspec action won‘t tax your  
mech‘s systems too much, but sustained action beyond prescribed limits will take its toll. 
 

                                                                                                              


You may Overcharge your mech only once per turn. Overcharging incurs 1 heat. The next time  
you overcharge before you make a full repair, this cost increases to 1d3 heat. The next time, the  
cost increases to 1d6 heat, and thereafter to 1d6+3 heat. Taking a full repair resets this counter.
 

Overcharging immediately allows you to make any quick action of your choice as a free  
action, even one you already made this turn.