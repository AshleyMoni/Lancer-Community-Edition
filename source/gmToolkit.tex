\part{GM Toolkit}
                                     GM TOOLKIT  

Here are some tools for running your game and creating different and more interesting scenarios  
for your players to go through. These rules will help clarify certain situations, customize or add  
additional detail or flavor to your game.
 
\chapter{Piloting a mech}
                                        PILOTING A MECH  
Piloting a mech is a simple matter of mounting it as an action. If a mech is powered off (shut  
down), it must also be booted as an action.
 

If you pilot a mech you are not licensed for (such as an enemy mech) the lack of correct  
neurological interfacing means that mech is permanently impaired and Slowed while you pilot it.  
Pilots piloting an NPC mech only have access to NPC actions, not those of a player character (ie,  
they cannot take the Stabilize action, overcharge, etc).
 
\chapter{Objects and Damage}
                                    OBJECTS AND DAMAGE  

All objects, cover, deployables, etc, have evasion 5 and HP equal to 10x their total size (so a size  
4 object or an object made up of 4 size 1 objects would have 40 HP). Some deployable objects  
(such as drones) specify different armor or HP, which supersedes this rule. This could cover  
terrain, walls, or any other construction on the battlefield. If the object is tough or hardy (such as  
solid rock), you could give it 1 or 2 armor. If it’s fortified, such as a bulkhead, bunker, or starship  
freighter wall, give it 3 armor.  

You can ignore this rule at your leisure when it applies to entities not created by players or out of  
combat. If players want to bust through a wall to get the drop on their enemies, you can always let  
them make a hull check to do it without firing a shot.  

\chapter{Changing Core Assumptions}
                             CHANGING CORE ASSUMPTIONS  

The mechanics of LANCER assume a couple of things that might not be present in your  
campaign. If you want to tweak these things it’s entirely up to you. The following tools can help  
you change around some core conceits of the game.
 

                           PILOTS DON’T HAVE ACCESS TO A PRINTER
 

It’s assumed that pilots generally have access to a printer to create new mechs. This might not  
be the case in your campaign or even your setting, however! Maybe players are outlaws or  
renegades, with limited access to resources. Maybe the printer in their area is broken or  
damaged. Maybe they are operating on the fringes of civilization, where any kind of technology is  
hard to come by.
 

                                                                                                          


Printing a mech allows a player to get back in the game, so to speak, when their mech is  
destroyed. Remember that players can repair and rebuild their mech completely, as long as it is  
at least partly intact and they have access to it, whenever they take a full repair, regardless of  
whether they have a printer or not. 
 

If pilots don’t have a printer and their mech is destroyed (or they can’t access the mech), use the  
Power at a Cost tool at the beginning of this section (the goal: I want to rebuild my mech) to get  
access to people, materials, a workshop, etc where mechs can be manually built or repaired.  
Building a mech can also be a downtime activity (see the section above).
 

                                          DEATH IS MORE LIKELY
 

Here’s an optional rule you can use if you want to slightly tweak LANCER’s default ‘heroic death’  
rules: 
 
             -    If you take more than your maximum HP in a single hit (after armor) as a pilot,  
                 you’re dead, no matter what. 
 
             -    If you’re dead, that’s it. No cloning or revivification.
 
             -    If you take two points of structure in a single hit, your mech is destroyed, no  
                  matter what. 
 
Make sure you know what kind of game your players and you are playing before adding this sort  
of rule in.
 

                                    I WANT TO SIMULATE CURRENCY  

LANCER does away with currency management like in other RPGs in favor of tying everything to  
the leveling system. If your players want to buy something, they can just buy it (unless it’s  
expensive or rare, then do some role playing or use Power at a Cost). It’s assumed pilots are  
still paid (in manna, currency etc), you just don’t track it.  

If you don’t like that system, want something more granular, or want something to replace the  
License Level system, you can track Manna instead. Maybe your pilots don’t have benefactors or  
access to a market where they can freely buy mech licenses, for example.
 

                                                     MANNA  

Manna is a universal currency in the canon of LANCER promoted by Union to integrate client  
states and regulate business, in common use in certain parts of the galaxy.
 

Manna is represented by a capital M preceding the denomination, like so: M1, M2, M3, M100,  
M500, and so on. Manna is a digital currency, though it has been localized in some areas as a  
physical currency. There are also fractions of M1: M.75, M.50, M.25, M.10, and M.05.
 

Here’s what certain things typically cost at average purchasing power in Manna:
 

                                                                                                                


M.01: A cup of black coffee. Beans were grown in zero-g so it doesn’t taste the best.
 
M.05: A beer. Probably artificial but the spacers like it that way.
 
M.10: A decent, hot meal.
 
M.25: A night’s stay in a station capsule, pretty damn cramped and noisy
 
M.50: Standard bribe to gatesec
 
M1: Ticket into an exclusive offworlder nightclub
 
M2.5: Assault rifle, lightly used, sights are slightly crooked
 
M10: Personal kinetic shielding, generally reliable
 
M100: A military grade hard suit
 
M1000: A one-seater starship
 
M10,000: A full starship, crew of 5
 
M1,000,000: A freighter or warship
 
Anything higher than M1000 is usually difficult to get your hands on.
 

                                       USING MANNA TO LEVEL  

If you want to set a cost on mech parts or licenses, you can set a manna cost instead of using  
license level for certain licenses. Doing so effectively changes the leveling system to be based on  
manna, so keep that in mind.
 

To rent (use) a piece of equipment from a license for one mission costs M100 for rank I, 300  
for rank II, and 900 for rank III.   

To buy a piece of equipment costs M250 for rank I, 500 for rank II, and 1000 for rank III. 
 

If you rent a piece of equipment, it’s gone after one mission. If you buy a piece of equipment, it’s  
not re-printed if your mech is destroyed. Renting or buying a weapon doesn’t level up a player  
(they don’t get the FRAME unlocks). You can’t rent or buy a mech FRAME, you only get them by  
permanently unlocking them (as if you’d leveled up normally).
 

To permanently unlock a rank of a license, it costs 1500 (no matter the rank). If you  
permanently unlock a license, you level up (using the same leveling rules, getting 1 core point, 1  
talent point, and possible targeting bonuses, system points, or core mounts). You get access to  
all the gear from that license permanently. You can re-print anything you’ve permanently  
unlocked. Permanently unlocking a license is the same as ‘buying’ a mech so if players want to  
‘buy’ a mech FRAME, tell them it’s going to cost about 1500 to get the rank I license to access it.
 

Manna rewards could vary per mission, but if you want to keep the same leveling pace, you  
should award players about 1500 mana per mission (with more or less at your discretion).  

\chapter{Engagement}
                                         ENGAGEMENT
 

If you want to mix things up in a mission where the starting situation on the ground is unclear (a  
hot drop, an invasion, a foray into enemy territory), you can use the engagement rule.
 

                                                                                                         


Engagement happens right before the Boots on the Ground step of mission. Make an  
engagement roll where everyone can see. This is a simple d6 roll. Roll it and consult the  
following chart to establish what the situation is like the moment players get there.
 

 D6                                                    Starting Situation 

 6                                                     Situation normal, no complications other than  
                                                       expected 
 4-5                                                   Minor complications or unwelcome surprises 

 2-3                                                   Major complications or unwelcome surprises 

 1                                                     Situation FUBAR 

Engagement cuts out unnecessary planning or stalling and cuts right to when the players arrive  
on the scene. When we make an engagement roll, we immediately establish a situation and put  
the players in that situation, ready to take action and respond. 
 

This doesn’t have to throw the players right into combat (and probably shouldn’t the majority of  
the time). For an example, let’s say the players have embarked on a mission to escort a refugee  
caravan through a heavily guarded checkpoint manned by local partisans. The GM decides the  
moment players get boots on the ground is when they meet up with the caravan outside of the  
checkpoint. Based on the engagement roll, it could go the following ways.
 

6 - No major issues, the caravan is unmolested and ready to move
 
4-5 - The caravan is far larger than the players initially expected. It will move slowly and become  
hard to guard.
 
2-3 - The caravan is delayed and the players will have to track it down or wait under threat of  
bandit attack
 
1 - The caravan is under direct bandit attack the moment players arrive on the scene.
 

                                    Changing the engagement roll
 

The engagement roll can be increased by adding extra d6s (and choose the highest). It can also  
be decreased by subtracting dice. If the total pool is 0 or lower, roll two d6s and choose the  
lowest.
 

Check the chart below for ideas on how to modify the roll. This is mostly qualitative, based on  
the nature of the mission, but if the roll’s going to be adjusted, it should be fairly obvious for both  
the players and the GM (it can’t be arbitrarily changed).
 

Most of the time engagement should just be a straight roll (1d6).  

                                         Engagement modifiers
 

                                                                                                          


 Situation                                                                             Effect 

 The mission is in an exceptionally safe or stable location                            +1d6 

 The mission is in a notably unstable, dangerous, or distant location                  -1d6 

 The characters have good scouting, information, or details about the                  +1d6 
 mission 
 The characters have exceptionally poor information about the mission                  -1d6 

 Powerful forces are contesting or helping the players on their mission                -1d6/+1d6 

 The mission is routine                                                                +1d6 

 The mission is an emergency, impromptu, or rushed                                     -1d6 

\chapter{Faction Tracker}
                                       FACTION TRACKER  
There are many factions in the world of LANCER, many of which are outlined in the official canon  
below. You may, during the course of your game, find it relevant to keep track of factions in your  
game (or may run entire games revolving around factions). If you want to codify things a bit, you  
can use this tool.  

Factions can be tracked simply by Power and Hold. Power is the wealth, force, and influence of  
the faction, simply put. Hold is how resilient that faction is (strong, normal, or weak), how well it  
holds on to that position.  

Power has the following ranks (it’s not linear), as well as some examples.  

            Power      Scale             Examples 

            -3         Sub-local        A gang, small militia, or militant group 

            -1         Local            A small colony, a huge bandit gang, a small military 

            0          State            A ruler, warlord, or king; a pirate lord, a mercenary company,  
                                        a large colony, a reaver pilgrimage 

            1          Planetary        A unified planetary government, a god-king, a small  
                                         planetary-state, a pirate haven 

            3          System           A major shipping company, a trade collective, a minor corpo- 
                                        state 

            5          Multiple         A major corpo-state such as Harrison Armory, a Karrakin  
                       Systems          trade barony, Aun Ecumene, a pre-collapse civilization 

            10         Galaxy            Union 

            15         Metaphysical      RA 

The players are probably a faction with power -3 to -1.  

                                                                                                          


If a faction undertakes a major project that does not bring them into direct conflict with another  
faction, such as exploration, mining, trade, construction, research, expansion, etc roll 2d6+ that  
faction’s power to see how it goes. On double 1s, the action fails no matter what. On a total result  
of 2-6, the action is still in progress, or a failure. On a 7-9 the action is successful, but might take  
more time or resources than expected. On a 10+, the action is flatly successful.  
If two factions perform an action that would bring them into conflict with each other (such  
as a war), each rolls 2d6 and adds their power. On double 1s, the faction loses no matter what,  
otherwise the faction with the higher result scores victory (however that is defined). This doesn’t  
have to be direct conflict but could be a trade war, bidding contest, bid for influence, race for  
resources, etc.  

If a faction has strong hold, they roll 3d6 and pick the highest. If a faction has weak hold, they roll  
3d6 and pick the lowest. If a faction has normal hold, they roll as normal. Hold depends on how  
well-entrenched a faction is. A government such as Union or a planetary government usually has  
strong hold. A collapsing state, chaotic bandit gang, or unorganized military fleet has weak hold.  
Any other faction has normal hold. Any faction that goes to war immediately goes to weak hold. If  
a faction is insurgent (a rebellion, secret operation, etc), they always have strong hold, but lose  
that hold if they rise to power 1 or above.  

You can use this tool to check how well factions undertake certain actions to provide a sense of a  
living world for the players, or even allow the players to influence the outcome of events. If a  
faction has some major advantage or disadvantage that the players grant them, you could give  
them strong or weak hold, depending on the player’s actions. It might be possible for factions at  
weak hold, if they lose a conflict, to go down a rung on the power ranking (from planetary to state,  
for example).  

You might find it useful to track the faction’s attitudes towards the players as the story progresses.  

\chapter{SITREP}


                                              SITREP   

Many battles in LANCER will be simple affairs, with one side facing off against the other until one  
or the other is broken or destroyed. However, there are times when, as a GM, you might want to  
add additional objectives or make combat scenarios more interesting or engaging. This is a tool  
for creating combat scenarios featuring deployment zones, objective zones, and predicted enemy  
approaches (for more tactical mech combat, akin to a wargame).   

You can use these scenarios to represent key parts of a mission or adjust them to fit them to your  
story.  

Key terms:  

Player Deployment Zone - Where the players deploy at the start of the mission, unless otherwise  
noted.   
Extraction Zone - Where the players need to be or enter to end certain missions and  
successfully extract.   
Enemy Deployment Zone - Where the hostile NPC forces can deploy initially  
Reserves - NPC forces that are held in reserve. They don’t start on the map but can appear as  
reinforcements in later rounds in an Ingress zone.  
Ingress Zone - Where the hostile NPC forces can reinforce from  
Objective - Any key location or thing that the players need to interact with. Could be a zone or  
object. If it’s a zone, a character contests it if they are at least 1 space inside.  

Most of these missions assume you have a roughly rectangular map no longer than about 40  
spaces on its longest side. You should also fill the map with terrain or cover - some missions  
explicitly ask you to do this in certain areas, but the rest is up to you.  

                                                SCENARIOS:  
Random Scenario
 

 D6                                                     Scenario 

 1                                                      Escort 

 2                                                      Control 

 3                                                      Extract 

 4                                                      Hold Out 

 5                                                      Gauntlet 

 6                                                      Recon 


\section{Escort}                                          


                                                               ESCORT  
                                                               “Listen up, Administrator: when we open these  
                                                               doors, you need to stay with us, and you need to  
                                                               do exactly what we tell you to do. If I go down,  
                                                               don’t help me -- listen to Monk. If Monk gets hit,  
                                                               listen to Cross. If Cross gets hit, listen to Crown.  
                                                               If Crown dies, keep running and don’t stop, and  
                                                               remember: if you make it out of the city, we win  
                                                               the war. Ready?”   
                                                               -Archived audio, Captain Pyotr “Pat” Malov,  
                                                               Cornucopian Revolutionary Guard (KIA)  

                                                               An ESCORT mission requires bringing an  
                                                               Objective safely to the Extraction Zone and  
                                                               extracting all player characters safely.  

                                                               Objective: The Objective is a size 1/2 - size 2  
                                                               object, person, or NPC. It has 10 HP/size,  
                                                               evasion 10, e-defense 10, and no armor. The  
                                                               enemy forces want the objective and will not  
                                                               willingly damage it. If any actor starts adjacent  
                                                               to the Objective, they can move the objective  
                                                               with them when they make their regular move  
                                                               on their turn, maintaining adjacency. If the  
                                                               objective is adjacent or becomes adjacent to  
                                                               two actors of opposing sides, it immediately  
                                                               stops moving and can’t move until there is only  
                                                               one side in adjacency to it. Otherwise it does  
                                                               not move on its own.  
                                                               Enemy Forces: The GM should hold about 2x  
                                                               the enemy forces for a normal encounter. They  
                                                               can deploy up to 1x initially and hold the rest in  
                                                               reserve.  
                                                               Deployment: The players deploy first, placing  
                                                               both their characters and the Objective in the  
                                                               deployment zone, then the GM deploys in the  
enemy deployment zone.  
Reserves: The GM can bring in 1 NPC (or up to 4 grunts) at the start of any new round in one of  
the Ingress Zones. They cannot choose the same zone twice in a row.  
Extraction: Any player character can extract as a free action at the end of their turn while in the  
extraction zone. This removes them from the battlefield (they have gotten away safely). If they  
extract with the objective adjacent to them and no other actor is contesting it, they take the  
objective with them.  
Victory conditions:   
          - The players win if they extract the objective.  
          - At the end of the 6th round, if the objective hasn’t been extracted, the enemy forces win.   
          If any players are left on the battlefield after the 6th round ends, they are captured or  
         overrun.  

                                                                                                                      


          - Nobody wins if the objective is destroyed. The players must still extract to leave safely.  

CONTROL  

         “Take the hill.”  
                      -    Common final order  

A CONTROL mission requires maintaining control of four objective zones for six rounds. The  
zones could be important mission points like transmission towers or gun batteries, terminals or  
hangars.  

Objectives: There are 4 objective zones (by default you can use roughly 4x4 areas), each placed  
anywhere in one quadrant of the map. If they are not roughly symmetrical, the map will be  
unbalanced.  
Deployment: Roll off (1d6) between the players and enemy forces. The loser deploys first in their  
appropriate zone. The GM can hold enemy forces in reserve if they want but doesn’t have more  
than normal available.  
Zone control: If only actors of one side are inside a zone, they control that zone. If there are  
actors from two or more sides inside a zone, the zone is contested.  
Objective Scoring: At the end of each round, for each zone a side controls, give them 1 point. If  
they control all four zones, give them a bonus +1 point.  
Victory conditions: The side with the highest score at the end of round 6 wins.
 

                                                                                                                  

\section{Extract}
   

                                                                     EXTRACT  

                                                                     “Catapult Actual, this is Catapult-2:  
                                                                     requesting immediate evac from current  
                                                                     GRIDCOR”  
                                                                            “Cat-2 confirm GRIDCOR.”  
                                                                     “Confirming… Confirmed, Catapult Actual.  
                                                                     Air-ground clear and holding. We, uh, have  
                                                                     multiple down. KIA and wounded.”  
                                                                            “Heard, Cat-2. Lifeflight on the way.  
                                                                     Confirm: VIP secure.”  
                                                                     “VIP --”  
                                                                            “Say again Cat-2.”  
                                                                     “VIP secure -- We’ve got more company  
                                                                     down here, Actual -- going dark, respond to  
                                                                     GRIDCOR.”  
                                                                            “Confirmed, 2. We’re on our way,   

                                                                     An extraction mission is similar to an escort  
                                                                     mission, though different in a couple of key  
                                                                     areas.   

                                                                     Objective: The Objective is a size 1/2 - size  
                                                                     2 object, person, or NPC. It has 10 HP/size,  
                                                                     evasion 10, e-defense 10, and no armor.  
                                                                     The enemy forces want the objective and  
                                                                     will not willingly damage it. If any actor starts  
                                                                     adjacent to the Objective, they can move the  
                                                                     objective with them when they make their  
                                                                     regular move on their turn, maintaining  
                                                                     adjacency. If the objective is adjacent or  
                                                                     becomes adjacent to two actors of opposing  
                                                                     sides, it immediately stops moving and can’t  
                                                                     move until there is only one side in  
                                                                     adjacency to it. Otherwise it does not move  
                                                                    on its own.  
Enemy Forces: The GM should hold about 2x the enemy forces for a normal encounter. They  
hold them all in reserve (round 1 there will be no enemy forces).  
Deployment: The players deploy first, placing their characters in the deployment zone. The GM  
places the Objective in the objective zone.  
Reserves: The GM can bring in 2 NPCs (or up to 4 grunts) at the start of any new round in two of  
the Ingress Zones.  
Extraction: Any player character can extract as a free action at the end of their turn while in the  
extraction zone, which is the same as the deployment zone. This removes them from the  

                                                                                                                           


battlefield (they have gotten away safely). If they extract with the objective adjacent to them and  
no other actor is contesting it, they take the objective with them.  
Victory conditions:   
         - The players win if they extract the objective.  
         - At the end of the 8th round, if the objective hasn’t been extracted, the enemy forces win.   
         If any players are left on the battlefield after the 8th round ends, they are captured or  
        overrun.  

                                                                                                          
\section{Holdout}

HOLDOUT  

          MJ sat in his open cockpit, chewed his gum, and ignored the briefing. Was for the local Oxes, kids  
          and eagers who needed a pep talk, not him and his Emancipators. They got the real story.   

          Real story wasn’t a pretty pic.     

          Extract was six hours out and burning hard. The Crown horse guards got a hundred mix ‘n match  
          middle-tier Armory shells, couple thou’ cavalry -- poor damn horses.   

          MJ had a thousand rounds of hardpoint EX, two racks of Kodandams, and his MC-TB for when the  
          real mess came.  

          “Yo MJ,” his comms squawked. “Button up. We got company, four hundred meters out and closing.  
          Horses.”   

          “Uh huh,” MJ said. He slapped the cabin seal without looking and got comfortable. Helm on, haptics  
          on. Mag loaded. Feel bad for the horses, because they didn’t know what they were getting into.   

          Six hours to extract, clock starting now.     

A desperate mission type. When ordered to Hold Out, players will have to defend an area against  
an onslaught of enemies -- in a best-case scenario, this is to buy time for allies to complete an  
objective elsewhere. In a worst-case scenario, it is to the death.   

                                                                                                                               


Enemy Forces: The GM should hold about 2x the enemy forces for a normal encounter. They  
hold half in reserve  
Deployment: The players deploy first, then the GM deploys half their total forces.  
Fortifications: The area around the objective zone should have size 1-2 heavy cover  
Objective: The objective zone is a roughly 10 spaces by 5 spaces area in the middle of the map  
(you can adjust this as needed). Players start with 4 points. For every enemy inside of the  
objective zone, the players get -1 points (the score could go negative).  
Victory conditions:  If at the end of round 6 the player characters are alive and their score is 1 or  
greater, they win, otherwise they are captured or overrun.  

                                                                                                          

\section{Gauntlet}
                                                              

                                                             GAUNTLET  

                                                             >// [WHITEOUT CONTINGENCY:: REAR  
                                                             ECHELON HAS FALLEN. ALL UNITS AT  
                                                             MINIMUM/NEGATIVE EFFICACY]   

                                                             >//[ALL-BAND ORDER::PROCEED AT TOTAL  
                                                             STRIKE CAPACITY TO ID EXTRACT AREA]  

                                                             >//[THIS IS A IMPERATIVE COMPLY ORDER.  
                                                             YOU ARE ON YOUR OWN++OUR POSITION  
                                                             IS OVERRUN++HURRY]   

                                                             Generally a mission done under duress, or  
                                                            when no other options are available to the  
                                                             players. Commonly engaged in unfriendly  
                                                            terrain. A Gauntlet mission demands the  
                                                             player party move from their deployment zone  
                                                             across unfriendly territory to secure an enemy  
                                                             position.  

                                                             Enemy forces: The GM should have a  
                                                             normal amount of enemy forces, but hold half  
                                                             in reserve.  
                                                             Deployment: The GM deploys half their  
                                                            forces first, then the players deploy  
                                                             Fortifications: The area around the enemy  
                                                             deployment zone is fortified with size 1 and 2  
                                                             heavy cover  
                                                             Reserves: At the end of round 1, the GM  
                                                             deploys the rest of their forces in any of the  
                                                             Ingress zones  
                                                            Victory conditions: The players win if there  
                                                             are more player characters inside the  
                                                             objective zone at the end of turn 6 than enemy  
                                                            characters (count ultras for 4 players, elites for  
2, and grunts for 1/4). Otherwise the enemies win.  

                                                                                                                
\section{Recon}

RECON  

         Aurelia whistled low, taking in the fresh coat of matte on their chassis.   

          “Optical camo overlay primary, Inkwell secondary,” Engineer Coates said, finishing her walkaround.  
          “Even if they scramble your anoptics, it’s going to be a nightmare to sight her.”   

         Aurelia reached up to touch their chassis’ brachial mount. Their sylph spilled across their arm,  
         touched the chassis. An inkbloom of the midnight color clouded through it.  

         Aurelia grinned. They would be a nightmare.   

A recon mission is, generally speaking, a dangerous endeavor, where a small team enters hostile  
territory to identify targets or retrieve key information  

Objectives: The GM marks 4 4x4 objective zones on the map. The GM secretly chooses one of  
the objectives to be the real objective (from A-D). Characters can control an objective by being the  
only side inside the objective zone at the end of the round. They can determine whether the  
objective is real or not by being inside the zone and taking a full action (this information is freely  
shareable once discovered). They don’t need to control the zone to investigate it.  
Enemy Forces: The GM has normal enemy forces, and can hold any number in reserve  
Deployment: The players deploy first, then the NPCs  
Reserves: The GM can bring in 1 NPC or up to 4 grunts in the ingress zone at the start of any  
new round.  

                                                                                                                        


Victory conditions: The players win if they are in control of the real objective at the end of round  
6. Otherwise, the enemy forces win.  

\chapter{An Endless Generator, An Unquiet Forge}

An Endless Generator, An Unquiet Forge   

Generator Tools for GMs and Players   

\subsection{Deep Field Survey }

Deep Field Survey   

Iterative worldbuilding, courtesy of the Union Astrocartography Department.   

World Type   
     1.  A barren world, with no atmosphere, few valuable resources, no water, under a ceaseless  
         barrage of terrible radiation.   
     2.  A barren world, with little atmosphere, and a distant sun. It is a cold, dead place, where  
         lonely wind blows perpetual across flat planes of dark ice and stone.   
     3.  A barren world rich in mineral wealth, cooling after a long period of tectonic activity.  
         Massive thunderstorms lash the world, and methane ice storms accumulate into world- 
         sculpting glaciers.  
    4.   A barren world encased in ice, with a deep subglacial sea. The surface trembles with  
         asteroid impacts, and the world’s orbit trails ice and dust behind it.   
     5.  A barren world, old and close to its sun, where the mountains have eroded to sand, and  
         the dunes are endless and white.   
     6.  A temperate world, lush, forever hazy under a thick mist that all but blocks out the sun.    
     7.  A temperate world, with myriad biomes, and skies piled high with columnar clouds.   
     8.  A temperate world, dry across great swathes of plainsland and desert, where vast alluvial  
         deposits speak of rivers that once were.   
     9.  A temperate world of archipelagos scattered across a salton sea  
     10. A temperate world with a mild climate, rich in native flora and fauna, and old, stable  
         biomes.   
     11. An inhospitable world, whose atmosphere is thick and choking, with deep valleys that hold  
         pockets of breathable air.  
     12. An inhospitable world of storm-tossed nitrogen oceans and wind-polished cadmium  
         islands, where tides surge and recede for years and no land is safe from the flood or  
         retreat.    
     13. An inhospitable world of lava and lightning, a new world, where there is no life to be found.  
     14. An inhospitable world, scoured by a distant gamma ray burst, where the life that once was  
         there has long since died off, and the land simmers with fires that never stop.   
     15. An inhospitable world, cracked open by an ancient impact, its core bleeding heat into  
         vacuum as it slowly dies.    
     16. A temperate moon with a stable atmosphere, the curve of its horizon visible from even a  
         modest mountaintop. Its parent world looms massive above it, ever-present in its day and  
         night sky.   
     17. An icy moon, cold and dim, one of many around its parent world. It absorbs the impacts of  
         asteroids that would otherwise hit its counterpart.   
     18. An ocean world, where there is no land.  

                                                                                                                 


    19. A barren world with a stable, if thin atmosphere. Massive geometric features and perfectly  
        flat metal planes create an uncanny landscape; deep canals funnel moaning wind around  
        the world.   
    20. An ancient world, star red and swollen in its sky, where all things have a terrible symmetry,  
        as if nature itself had oriented -- or been oriented -- around something now absent.   

Defining Natural Feature  
    1.  100 year storms with increasing regularity -- typically manifests as massive hurricanes.   
    2.  Very active tectonics -- earthquakes are more common on this world than others. On  
        ocean worlds, this means a higher prevalence of tsunamis.   
    3.  Inert core -- this world’s core has stopped spinning, and as a result it has shed its  
        magnetic field. Compasses do not work on this world, and UV exposure is dangerous  
        without significant protection.   
    4.  Monobiome -- this world is an oddity among worlds, in that has a single, global biome with  
        a single, global climate. It could be a forest that spans the world, or a desert, a swamp,  
        and so on.    
    5.  Worldscar - the world has suffered a recent (within a million years) impact from a massive  
        stellar body. Its climate has leveled, but the scar of the impact is still a massive, visible  
        feature on the world’s surface.   
    6.  Royal Court - the world has hundreds of terrestrial moons, ranging from hundreds of yards  
        in diameter to thousands of kilometers. The night sky is bright, and the day is marked by  
        gentle, dappled light.   
    7.  Under Twin Suns - the world orbits a binary star.  
    8.  Ringed -  the world is banded in a series of planetary rings, visible as a thin white line  
        across the night sky. It is an especially beautiful world to view from afar.    
    9.  Remote - this world was a remote colonial, industrial, or scientific prospect. It is far  
        removed from any other civilization, and takes great effort and time to reach.  
    10. Cosmopolitan - this world is located close to interstellar shipping lanes, and is a frequent  
        landmark, resupply point, or pit stop for interstellar travellers, who ether remain in orbit or  
        have an easy route down to the surface of the world.   
    11. Hecatoncheires - The world is marked by a series of massive mountains, sheer peaks  
        rising kilometers into the sky.   
    12. Epochal Sunset - the world is gripped by the transition between eras -- ice to temperate,  
        temperate to ice, and so on. Expect a varied climate.   
    13. A Monument of Shame - the world is a dumping ground for passing ships, for a local CS,  
        or abandoned following massive climate destabilization.   
    14. Quarantined - the world has been hidden behind Union red tape, for reasons unknown or  
        known.     
    15. Breathable Atmosphere - the world, unless already noted, has a breathable atmosphere. A  
        human can breathe without relying on an EVA, scrubbers, or other assistance or  
        augmentation.   
    16. High Gravity - the world, in addition to any other features, has higher than standard gravity,  
        ranging from +1-2G greater than Cradle standard.    
    17. Low Gravity - the world, in addition to any other features, has lower than standard gravity,  
        ranging from .1-.99G less than Cradle standard.   

                                                                                                         


    18. Hard Sun - in addition to any other features noted in its description, this world’s  
        atmosphere provides little to no UV protection. Venturing outside in the daylight is a risk  
        without proper optical and dermal shielding.   
    19. Dreamland - in addition to any other features noted in its description, this world has  
        something, well, otherworldly about it that makes it stand out. Islands that float suspended  
        in the air, or oceans of liquid mercury, or motile fauna, and so on.   
    20. Dust and Echoes - the world has ancient, pre-collapse ruins on it, with no indication as to  
        their origin or nature.  

Defining Anthropocentric Feature   
    1.  Colonial Settlement - initial. There is a settlement here, but it was only recently seeded.  
        Drones and subalterns work tirelessly, clearing space and assembling the initial colony  
        footprint. A small cluster of buildings house the landfall team -- they are likely welcoming,  
        as they’ve probably not seen anyone else for years. The population numbers in the  
        dozens.   
    2.  Colonial Settlement - first generation. The colony settlement here is young, with its first  
        generation of native-born colonists now of-age and set to the work of building their future  
        home. The population numbers in the hundreds to thousands.   
    3.  Colonial Settlement - stable. The colonial settlement here is in its second or third  
        generation, stable, with a population in the tens of thousands.   
    4.  Outpost - Union Far Field Team Mission. There is a FF Team outpost on this world, but  
        nothing else in the way of human population.   
    5.  Outpost - Omninet Relay Node. The world is empty save for a lone omninet relay node. A  
        small team crews the node, providing security and on-site technical support. They rotate  
        on a regular schedule.   
    6.  Outpost - Union Navy SigInt Station. The UN maintains a small signals and intelligence  
        station on this world. There may or may not be other populations here -- that is up to you --  
        but the SigInt station is in a remote, high-altitude part of the world.   
    7.  Outpost - Union Astrocartography Station. The world is marked with an orbital  
        astrocartography station -- a suite of automated telescopes, sensors, and observational  
        equipment used to map more distant stars and worlds.   
    8.  Outpost - Checkpoint/ Forward Post -- The world hosts a small Union garrison, most likely  
        auxiliaries who report to a remote Union officer. Their duty is to scout, stand watch, and  
        await further orders. They spend most of their time exercising, cleaning, maintaining their  
        gear, and waiting for something to happen.   
    9.  Installation - Research Facility - The world hosts a Union, Corpro-State, or other state or  
        private entity’s research facility. It may be a secure, top secret site, or it may not.   
    10. Installation - Proving Ground. The world, or a significant part of it, is given over to a  
        proving ground, test field, ordnance firing range, or other large-area hazardous testing,  
        training, measuring, or dumping installation. Run by a small long-term skeleton crew, there  
        typically is a larger short-term crew or population present during a given season or  
        exercise.   
    11. Installation - Deep Field Relay Installation. The world features a secure, long-range,  
        planet-based sensor installation, typically located in a remote, high-altitude location.  
        Populated by a small crew, the installation can be privately run or Union.   

                                                                                                          


     12. Installation - Union Embassy. Generally only found on more developed worlds  
         approaching or at Core status, Union Embassies are modest, centrally-located buildings  
         typically sited in capital cities on capital worlds. They function as embassies do.   
     13. Installation - Corpro-State Campus. This world features a publicly-accessible or off-limits  
         CS campus with a large permanent population. This can be an enclave, an exclave, or  
         integrated into a larger urban environment, and typically is an administrative center with  
         some level of public exposure (even just notariety)   
     14. Base - Union Naval System Command. The world is the site of a regional system  
         command center for the Union Navy. Part base, part garrison, part shipyard, part  
         recruitment center, part medical center, a UN SysComm facility is a large, removed,  
         military base with a defensive perimeter, sunken launch pads, and hardened buildings and  
         bunkers. It has a massive permanent population and a garrison numbering in the  
         thousands, usually with a complement of ships in orbit with (at minimum) global strike and  
         delivery capacity  
     15. Base - Capital. This world is the capital world of a system, and hosts the physical center of  
         the system’s government, generally a campus, estate, block, or other large collection of  
         hardened buildings charged with ministering to the state’s population. Generally speaking,  
         there is some level of public access to this building, but only in more liberal states.   
     16. Base - Uplift. This world has a spaceport, a sprawling launch facility open to the public  
         (access subject to local laws, of course, beyond the basic ability to purchase a ticket  
         offworld).    
     17. Civic - Municipalities. The population of this world live in modest towns and cities scattered  
         across biomes. The population numbers in the hundreds of millions. There may be one or  
         two signature cities, but there are vast stretches of wilderness between them, and each  
         municipality has developed their own take on global cultures.    
     18. Civic - Arcology. The world’s population lives in one or more arcologies -- unified, self- 
         contained ecosystem-cities -- in harmony with the world around and inside of them. They  
         may be strictly monitored and walled-off from the larger world, or not. Worlds with  
         arcologies as the primary urban centers tend to be well developed, with a history at least  
         centuries long. The global population numbers in the hundreds of millions to single-digit  
         billions.   
     19. Civic - Metroswathe. Unlike an arcology or municipality, a world marked by metroswathes  
         is heavily developed, with a significant percentage of the landmass given over to a single,  
         amalgamated urban environment (<10-15%) that is home to billions. That it functions --  
         and there are likely large chunks of it that doesn’t -- is a miracle. A metroswathe is usually  
         not planned to be a swathe, but there are certainly large sections of it that have been  
         developed intentionally, rather than just metastasizing. There are usually vast, baroque  
         criminal, bureaucratic, and community organizations working both above and below board  
         (and street level!).   
    20. Union - Administrator’s Residence. This world, if it wasn’t previously, is habitable, and has  
         a stable population of millions. A Union Administrator makes their residence here.    

COMBAT AND ROLEPLAYING ENVIRONMENTS  

Roleplaying and combat can take place anywhere in the galaxy. Most human affairs occur in one  
of three locations: on a terrestrial world (or moon, asteroid, comet, etc), on a satellite station, or  
onboard a spaceship. 
 

                                                                                                                    


Habitable worlds are terrestrial worlds, moons, asteroids, or comets that are tectonically stable  
enough to support structures; these worlds do not need to have an atmosphere in order to be  
considered habitable -- in Lancer, “habitable” means a world on which humans can live, not  
necessarily live comfortably or independently of life-sustaining systems. 
 

Habitable worlds that are stable and have a breathable atmosphere that protects and sustains  
human life are called Terran worlds. These are rare and precious, usually the site of stellar  
nations’ capitals or preservation worlds. 
 

Satellite stations orbit larger stellar bodies: orbital habitats, shipyards, blink stations, omninodes,  
orbital science stations, orbital military installations, and others all fall into this category. 
 

Typically, civilian stations have a large permanent populations and act as galactic transit hubs.  
Travellers pass through these stations on their way to their berths or while their ships are  
refueled, resupplied, and rearmed; the civilian population on the station typically works around  
this transient population at cafes, bars, shops, and entertainment venues, and in maintenance,  
logistical, engineering, and harbor navigation roles. 
 

Military stations typically have small populations posted on a semi-permanent basis, usually for a  
period of months or years before being rotated out.   
 

Example environments:  
 

    $\bullet$    A lush jungle world, thick with kilometer-tall trees and layered canopies thick enough to  
        support buildings. The light fades to a darkness complete the deeper you dare venture.  
    $\bullet$    A dead moon with a thin skein of atmosphere, just enough to form and hold clouds at  
        ground level. The wind is terrible and constant, and carves strange shapes from the  
        moon’s soft grey rock.    
    $\bullet$    In borean fields of polar ice, a crashed Far-Field Shuttle broadcasts its lonely SOS. A band  
        of slavers makes across the permafrost, their flyers beating low over the white landscape.  
        A blizzard approaches.        
    $\bullet$    On the metal and stone flanks of a rocky world, where unique magnetic properties hold  
        iron pillars suspended in the air. Buildings here must be made of wood and plastic, and  
        your mechs struggle to adapt to the interference.    
    $\bullet$    Aboard an abandoned station in decaying orbit, its inertial gravity fluctuating in response  
        to the decay. The station lists to the side as distal components begin to break away. The  
        station’s NHP is firm in its declaration of intent: it wants to see the night sky from the  
        world below.     
    $\bullet$    On a massive blink station thronging with travellers, merchants, and all manner of people  
        in transit. Thousands travel the station’s main concourse, some characters less savory  
        than others, and hundreds of alleyways lead to chambers and venues that showcase the  
        wonders of the galaxy.   

                                                                                                         


$\bullet$    Among the dunes of an arid world, a nomad’s camp crouches on the banks of an oasis,  
    indigo in the shadows. They are hunting something grand in the sky, tracking it by its  
    droppings.   
$\bullet$    In the hive-like streets of a neon-drenched capital world, under a driving artificial rain in  

    the commerce district. NHPs in armature bodies consort with human partners, dealing in  
    business, trade, and secrets. The city is endless.    
$\bullet$    In the deep black of space, aboard a capital ship as its main guns thunder away at the  
    enemy fleet. Outside, vacuum and silence, as the debris and the dead float cold and still.    
$\bullet$    Outside the hull of a titanic generation ship, under the shield of its bow guard as it  
    accelerates through an asteroid field.    
$\bullet$    On an atmospheric moon of a gas giant, in the fields under the planet-rise as day slips to  
    oversky. The people of this moon have never seen night, as the world their moon orbits  
    only brings a dimmer day.     
$\bullet$    In a lonely colony on a nondescript Terran world, as snow falls on the fledgling habitat  
    during the celebration of their first year settled. It is a time of celebration, and a time of  
    worry: what happens if it never stops snowing?    
$\bullet$    In the gilded palace of an interstellar king, as his hosts march in formation below the  

    viewing plaza. Your shuttle waits on a distant landing pad and you think to yourself, are  
    you guests, or are you prisoners?    

                                                                                                     


HAZARDOUS ENVIRONMENTS  

It is possible (and likely) for mechs to operate in hazardous environments such as being  
submersed in hostile atmosphere, vacuum, or water. To operate without killing its pilot, a mech in  
these circumstances needs life support -- if a mech’s life support system is active and running,  
it can run essentially indefinitely (a pilot will likely die of dehydration/starvation before they run  
out of oxygen, thanks to the efficiency of standard 02 scrubbers). A mech in a hazardous  
environment without life support has enough residual support for a number of hours equal to its  
engineering score, or 30 minutes if that score is less than 1.
 

                                                 ZERO -G  

A Mech operating in zero-g, underwater, or space is impaired unless it has a propulsion system  
or a system allowing it to Fly. Mechs in space or zero-g cannot fall.
 

Mechs without a propulsion system or flight are Slowed in space or zero-g, but can fly when  
they move or boost.
 

                                              Environments:
 

 Roll    Name                                Description/Effect 

 1       Dangerous Flora or Fauna            The planet has a high amount of dangerous animal or  
                                             plant life, some of it perhaps titanic, predatory, or  
                                             particularly hostile. You can represent this by using the  
                                             Monstrosity NPC type from the NPC toolkit. You can  
                                             also represent the presence of hostile flora on a  
                                             battlefield as size 1 or 2 static entities with evasion 10,  
                                             5 hp. Any target that moves adjacent to them must  
                                             pass a hull check or take 3 kinetic damage and become  
                                             immobilized by sticky sap, webbing, a pit trap, or the  
                                             like until the flora is destroyed. 

 2       Extreme Cold                        Mechs and pilots will quickly freeze without a source of  
                                             heat nearby, and culture on this world accommodates  
                                             this. Any mech that does not move or boost on its turn  
                                             becomes immobilized at the end of its turn. It can end  
                                             this condition by taking an action and a successful hull  
                                             check to break out of the ice.
 
                                             All mechs have resistance to heat. 

 3       Extreme Heat                        Civilization has retreated mostly underground in this  
                                             blistering atmosphere. All systems and weapons that  
                                             generate or inflict heat generate +1 more heat than  
                                             normal. 

 4       Thin Atmosphere                     All targets gain resistance to explosive damage. 

                                                                                                          


5       Extreme Sun                         Mechs take 1d6 heat at the end of any turn that they  
                                            are not standing in the shade 

6       Corrosive Atmosphere               The thick atmosphere on this world corrodes armor. All  
                                            weapons gain the AP tag. 

7       Particulate Storm                  This planet is swept by brutal, scouring storms of sand,  
                                            rock, or metal. While one of these storms is active,  
                                            mechs treat all terrain as heavy cover. Pilots cannot  
                                            step outside without great personal risk if they are not  
                                            in their mech. 

8       Electric Storm                     This planet is swept by unusually strong electrical  
                                            storms. While one of these storm is active, at the end of  
                                           the round, choose a target at random. That target must  
                                            pass a systems check with 1 difficulty per size or be  
                                            stunned until the end of their next turn by a bolt of  
                                            lightning. 

9       Disruptive Storm                   The storms on this planet are so thick that electronic  
                                            systems cannot function. All tech actions and system  
                                            checks are made at +1 difficulty. 

10      Dangerous Storm                    This planet is swept by storms of fire, meteors, acid  
                                            rain, ice, or other destructive particles. While one of  
                                           these storms is active, all mechs not in cover at the end  
                                            of the round take 3 AP kinetic damage. 

11      Earthquakes                        At the end of each round, while earthquakes are active  
                                            on this world, roll a 1d6. On a 1, all mechs on the  
                                            battlefield must pass a hull check or be knocked prone. 

12      Ocean World                        This entire world is covered in water, with less than 5%  
                                            of the surface being solid ground. The water is a  
                                            hazardous environment. Mechs with Flying or EVA can  
                                            move normally, otherwise mechs without will sink to the  
                                            bottom and count movement as difficult terrain. They  
                                            can walk perfectly normally on the bottom (if slowly)  
                                            and most mechs are pressure rated to extremely high  
                                            specifications. 

13      Molten World                        Parts of this world’s crust pokes through in showers  
                                            and pools of liquid rock. Any mech that enters a zone of  
                                            molten rock or lava for the first time on its turn or starts  
                                            its turn there takes 5 AP energy damage and 1d6 heat. 

14      Primordial World                   This world is mostly a bubbling soup of semi-organic  
                                            mud and gases. The atmosphere is toxic and humans  
                                            must use a breathing apparatus or sealed suits outside  
                                            of their mechs. Boiling mud covers this world, creating  
                                            zones of both difficult and dangerous terrain. 

15      Low Gravity                        All mechs gain the ability to fly when they boost on this  
                                            world, as if they had jump jets. Mechs don’t take  
                                            damage from falling, and only fall 3 spaces a round. 

                                                                                                        


 16      High Gravity                        Mechs cannot boost on this high gravity world and are  
                                             immobilized instead of Slowed 

 17      Tomb World                          This world has extremely high radiation, possibly as a  
                                             result of nuclear war, atmospheric degradation, or  
                                             something more sinister. Humans not wearing a sealed  
                                             hard suit or some other kind of environmental suit lose  
                                             5 HP/hr on this world. 

 18      Spire World                         This world is made up of a large number of islands or  
                                             spires held aloft in a gaseous substrate and suspended  
                                             through magnetic force. Perhaps it was a world  
                                             shattered by a superweapon or natural disaster. Most of  
                                             the land is not connected, but a loose collection of  
                                             floating rocks, some of them large enough to hold  
                                             cities. Navigation systems go haywire on this world. 

 19      Sinking World                       This world is covered in fine sand or thick mud. While  
                                             on the surface, mechs that move 1 space or less during  
                                             their turn are Slowed. Mechs that are Slowed and move  
                                             1 space or less are immobilized and start sinking,  
                                             eventually becoming completely engulfed. A mech can  
                                             take a full action and make a hull check to end this  
                                             effect on itself or another adjacent mech. 

 20      Holy World                          This world is beautiful and has no particular dangerous  
                                             features, but held sacrosanct by the local population.  
                                             Damaging its rocks, trees, and pristine grasslands will  
                                             incur the ire or wrath of the locals 

Generally speaking, tune the environmental hazards to the world(s) that your campaign takes  
place on: jungle worlds will have hazards that are appropriate for jungle worlds, aquatic worlds  
will have hazards appropriate for aquatic worlds, and so on. 
 

                                                                                                          
\section{In The Pipe, 9 To 5 }

In The Pipe, 9 To 5   

What work do mech pilots encounter, and what types of organizations do they work for?   

What kind of constraints might get put on them depending on a) the work that they do and b) the  
people/orgs that they work for? What are common complications that they might encounter on a  
job/mission/assignment?  

These tables below will help develop mission structures in a pinch, further complications can be  
added as necessary. These reflect the more mundane missions, typically encountered by  
mercenary groups, Union DoJ/HR liberation teams, Union Navy regulars or auxiliary troopers,  
private militaries, and other conventional fighters.    

Source  
    1.  A Union Administrator in need of a team of fixers.  
    2.  A mercenary company, known for its upstanding reputation.   
    3.  An upstart mercenary company with a lot to prove, few resources, but this one golden  
        contract.  
    4.  A Union Far Field team manager in need of extended security.  
    5.  A local ruler, whose own soldiers can’t finish the job.  
    6.  A board member of a corpro-state, who needs this job done off-book.   
    7.  The heir to a throne, in need of a team of champions.  
    8.  A local crime boss, who needs extra muscle for this job.  
    9.  The mouthpiece for a mysterious figure, who needs a disposable team for a covert  
        mission.  
    10. A system administrator NHP, who needs to procure an asset from one of its distant  
        colonies.  
    11. Your commanding officer, who orders you to complete a necessary mission.   
    12. A sudden burst of code that overwhelms your chassis’ onboard computers, depositing  
        mission parameters in a window on your HUD that won’t go away  
    13. A familiar dead drop, unused for years but active now, with your orders scrawled on  
        hardcopy inside.   
    14. The dying wish of a comrade or loved one.   
    15. A system administrator NHP, who has issued an SOS and requires immediate, Union- 
        sanctioned aid.   
    16. A HORUS cell leader -- you think -- who speaks to you through a remote-piloted subaltern.    
    17. An NHP, just prior to cycling, who would whisper a single name over and over  
    18. Of your own volition, out of desire  
    19. Of your own volition, out of a sense of duty  
    20. Of your own volition, out of a need for revenge  

Hook  
    1.  Escort a VIP from a compromised location to a new safe one  
    2.  Respond to an SOS from an unknown source, location noted in message.   
    3.  Retrieve a valued or strategic object, item, or information from a secure, hostile location  

                                                                                                          


    4.  Investigate a tip from a valued informant, which could go south.   
    5.  Escort a long-flight weapon or ordinance to its target   
    6.  Run security for a secure location expecting an attack   
    7.  Head into a derelict to retrieve important data    
    8.  Bring down a piece of massive infrastructure (bridge, skyhook, dam, etc)  
    9.  Go loud to provide cover for a covert mission of utmost importance  
    10. Assassinate a VIP in broad daylight, to send a message  
    11. Attack a hostile defensive position in order to destroy a key objective  
    12. Board a hostile ship or station and take it over; or, destroy it   
    13. Be first on the ground on a world hostile to human life; create a beachhead  
    14. Deal with hostile local fauna or megafauna plaguing a colony.   
    15. Hunt down a team of notorious, feared, or respected mech pilots, and kill them.   
    16. Provide cover for an evacuation.  
    17. Rescue and extract a downed pilot from a warzone.   
    18. During a massive attack, strike the enemy’s critical weak point to make a breakthrough.  
    19. Liberate a people held hostage from their cruel ruler, with Union’s backing.  
    20. Intervene in a desperate attempt to stop an incoming missile or attack.   

Location  
    1.  On a habitable, terrestrial, populated world, with a few large cities, a capital with a  
        spaceport, and outlying towns. Vast stretches of unexploited, but explored wilderness.   
    2.  On a terrestrial world with little to no atmosphere. Rocky, uninhabited, with a smattering of  
        automated sig/int, omni, and navigational stations.   
    3.  On a high-traffic, low-orbit civilian station, full of travellers bound for distant stars and  
        Cosmopolitans waiting to clear quarantine.   
    4.  Inside a massive arcology, thick with life, an oasis on an otherwise grim world.     
    5.  In hard vacuum, near a dorsal monitoring terminal of a Blink station’s Dyson panel,  
        thousands of kilometers above the city decks.   
    6.  Aboard a capital ship, all decks vented of air and ready for combat.   
    7.  On an arid terrestrial world close to its sun, in a white sand desert of massive marching  
        dunes and glass spires.   
    8.  On a cold, distal world of black stone and vast, geometric planes of nitrogen ice.   
    9.  Inside the upper atmosphere of a massive gas giant, where long-flight hydrogen refineries  
        scrape the valuable fuel from impossible blue depths.   
    10. Among the teeming streets, bazaars, and boulevards of an ancient, persistent  
        metroswathe, laden with millennia of histories, both personal and public.   
    11. On a ocean world, landmasses covered by glaciers and the mountains they’ve carved as  
        they march to the sea. Storms marble the world’s grey skies, and turbulent tectonic activity  
        sends world-sweeping tidal waves to crash into glacial faces.   
    12. On a rocky world defined by massive craters, whose skies are lit by the brilliant passage of  
        frequent meteor showers.   
    13. On a hot, terrible world of choking gases and thick atmospheres, where light is eaten by  
        deep yellow fog, and the ground is a barren mix of oily green stone and rotting metals.   
    14. Aboard a vibrant Cosmopolitan space station, a hub of intergalactic trade and art, with  
        constituant asteroidal states leashed in its orbit.   
    15. In the arcadian planes of a temperate world, distant from any human habitation, but  
        marked by the promise of future colonization.  

                                                                                                         


    16. On the storied land of an old Core world, once a thriving metropole, now long since scaled  
         back. Empty cities collect sand and creeping vines; the world’s population resides in a  
         single valley, content.     
    17. On a bucolic temporal reserve, a neo-pastoral world with a population that has no  
         knowledge of Union or the politics and current events of the galaxy at large.   
    18. In the hard vacuum of space, among the ruins of the fleet.  
    19. In the hard vacuum of space, between the ink-black hulls of capital ships on long patrol  
    20. On the hardpack surface of a comet, whose horizon is never far away, whose chasms and  
         valleys are picked out in sharp relief by the hard light of a nearby star.   

Complication  
    1.   The local leader is hostile to Union, and will send their agents, military, or assassins to try  
         and stop you.  
    2.   The location of the mission is under local quarantine, due to the presence of a virulent,  
         deadly illness.   
    3.   The location of this mission is secret, and your presence there will be deniable: expect no  
         support or public reward for completing your objectives.   
    4.   The local gravity is much higher than you’re used to, making movement and breathing  
         more difficult.   
    5.   The atmosphere is corrosive and thick, rendering all but the most powerful energy  
        weapons useless.   
    6.   The world is marked by odd, unnatural geometries -- certain features, when the wind  
         blows over them, “sing”, and the world is never silent.   
    7.   The world is a frequent (enough) target of meteorite impacts: one is forecasted within the  
         mission window, and its impact will plunge the world into a decade of dust-night and  
         storms.   
    8.   The world is in turmoil, roiling with political discontent, and you’ve just stepped in the  
         middle of it.   
    9.   The station is a common type among long-term-residency stations: a cylinder, whose  
         population lives on the gentle curve of the cylinder’s interior. You’ll need to take that into  
         account when the mission gets hot.    
    10. The weapon, VIP, ship, etc, that you’re escorting will intersect with an opposing weapon,  
        VIP, ship, etc, that the enemy is escorting: you must ensure that your objective remains  
         secure, while theirs is done away with.    
    11. The station you’re on begins to break apart after the shooting starts.  
    12. The forces employed by your enemy are unwilling combatants, coerced into fighting you.   
    13. Your mission runs counter to the mission of another arm of your state, company, or  
         agency; you may encounter internal resistance.   
    14. Your way in is incapacitated or made unavailable, leaving you without a clear way out.   
    15. Your intel was wrong -- the objective you seek is not where you were told it would be.   
    16. The city’s (or station’s) administrative NHP dislikes you, for reasons unknown, and seeks  
         to impede your progress.    
    17. The world has a much longer day/night cycle, lasting on the order of standard months or  
         more -- you’re approaching the end of one period and the beginning of the other, with  
         significant cultural meaning placed on the transition, and wide variance in how the local  
         biome reacts.   

                                                                                                               


    18. Completing the mission requires some interstellar transit, which would divorce you from  
        the subjective timeline of your friends and family back home.   
    19. This mission is set up to fail, but you don’t know that yet. Someone higher up in your  
        organization wants you dead.   
    20. The world has been visited previously by RA, and it bears markings of the MONIST entity's  
        passing.  

\section{Identifty Friend/Foe}

Identify Friend/Foe  
Force generator for NPC factions.   

Faction Type   
    1.  Union Regulars, Regiment   
    2.  Union Auxiliary, Regiment  
    3.  Union Department of Justice and Human Rights, Liberator Team   
    4.  Union Intelligence Bureau, Field Team  
    5.  Union Naval Intelligence, Field Team  
    6.  Union Far Field Team, Security   
    7.  Harrison Armory, Acquisition Team  
    8.  Harrison Armory, Colonial Legion  
    9.  Harrison Armory, Board Intelligence Field Team  
    10. IPS-N, Marine Security Detachment   
    11. IPS-N, Company Trunk Security  
    12. SSC, Diplomatic Corps Detachment - Security Team    
    13. HORUS, Local Cell   
    14. Ungratefuls, Local Cell    
    15. MSMC, Company Detachment   
    16. Voladores, Sparri Espadas   
    17. Sparri Mercenaries, Free Band   
    18. Local, Planetary Defense Force (Ground, Navy, or both)   
    19. Local, Honor Guard  
    20. Local, Opposition Group   

Signature  
    1.  Survivors. Loosely organized, scattered in groups of twos and threes, with broken, or  
        empty small arms. No anti-armor.   
    2.  Ready partisans, with civilian weapons and the capacity to produce improvised explosive  
        devices and lightly armored vehicles (technicals, motorcycles). Small to medium numbers.   
    3.  Resistance fighters. Scavenged or stolen military-grade small arms and explosives, light  
        anti armor, provisions and scavenging ability. Lightly armored vehicles, few stolen military  
        grade vehicles. Small to medium numbers.   
    4.  Light infantry, with light anti-armor, a few light vehicles (motorcycles, armored cars, etc),  
        and provisions for long field deployment. Medium numbers.   
    5.  Mobile, mechanized infantry, light to medium anti-armor, military grade small arms and  
        explosives. Mounted in armored personnel carriers. Provisions for field repairs and long  
        travel. Some APCs devoted to medium anti-armor. Medium to large numbers.    

                                                                                                         


    6.  Line infantry. Medium armored infantry, with military grade small arms and explosives.  
        Supported by medium armored vehicles, typically with artillery behind their line. Typically  
        supported by medium/line mechs. Provisioned for long-term deployments, garrisons, and  
        battles. Large numbers.   
    7.  Heavy infantry. Medium to heavy armor, military small arms and explosives, medium to  
        heavy anti-armor, uncommon/exotic weapons and systems. Powered armor is common,  
        typically operate in support of line and heavy mechs. Medium numbers.   
    8.  Shock infantry. Light to medium armor, military small arms and explosives, medium anti- 
        armor. Often employ light, mobile power armor, commonly in support of scout and mobile  
        line mechs. Typically mounted in armored personnel carriers or all-theater dropships.  
        Medium to large numbers.   
    9.  Drop infantry. Light to medium armor, military small arms and explosives, often equipped  
        with light powered armor. Commonly in support of scout ot medium mechs. Commonly  
        provisioned for long-term deployments. Typically deployed behind enemy lines or in close  
        proximity to priority targets via orbital drop or atmospheric insertion. Small to medium  
        numbers.   
    10. Scout Mechs. Half-size to size 1. Emphasis on speed, range, and sensors. Light armor.  
        Small numbers. Military grade weapons. Rapid infiltration and exfiltration. Small numbers.   
    11. Medium/Line Mechs. Size 1 to size 2. Emphasis on all-round ability, light to medium armor,  
        small numbers. Military grade weapons, with some exotic or uncommon. Rapid infiltration  
        and exfiltration. Small numbers, unless operating in a line capacity. Versatile roles.   
    12. Heavy/Siege Mechs. Size 2+. Emphasis on powerplant and fire output. Light to heavy  
        weapons, some exotic, medium to heavy armor. Small numbers, operate protected by  
        lighter mechs, armor, or infantry. Slow infiltration and exfiltration.   
    13. Armored Division - Light. Emphasis on speed and offensive ability, usually operate alone  
        or with light aerial support.    
    14. Armored Division - Medium. Emphasis on all-round capability and staying power. Usually  
        operate in support of or supported by infantry and light-to-line mechs.   
    15. Armored Division - Heavy. Emphasis on armor and offensive capability. Usually operate in  
        support of infantry, specifically meant to eliminate enemy armor or mechs.   
    16. GM choice.  
    17. GM choice.  
    18. GM choice.  
    19. GM choice.  
    20. GM choice.   

Strength   
    1.  Patrol. 1 to 5 soldiers, whose goal it is to observe and report, avoid engagements at all  
        cost, and correctly identify the location and strength of the enemy.   
    2.  Patrol. 1 scout mech, light atmospheric flyer, pair of light armored cars, pair of motorbikes,  
        or other patrol vehicle.    
    3.  Squad. At least 10 to 20 soldiers, on foot.   
    4.  Squad. At least 10 to 20 soldiers, supporting a single scout or line mech, or a single  
        medium armored unit.   
    5.  Squad. At least 10 to 20 soldiers, mounted in a single dropship, or two APCs.   

                                                                                                           


6.  Company. At least 100 to 200 soldiers, with support personnel and fire support (artillery,  
    mortars, air support, or orbital). They will occupy ground, provide relief, engage in combat,  
    and operate as an efficient unit.   
7.  Company. At least 100 to 200 soldiers, mounted 10:1 in armored personnel carriers  
    (APCs), or 20:1 in dropships.  
8.  Company. At least 10 to 20 light mechs, operating in teams of 2-4. Scout configuration,  
    some carrying heavy precision weapons.  
9.  Company. At least 50 heavy soldiers, operating 10:1 in support of 5 medium or heavy  
    mechs.   
10. Company. At least 20-40 armored vehicles, operating in teams of 5.   
11. Regiment. 5 to 8 companies, organized under an on-site, (typically) rear echelon   
    regimental commander. Usually supported by significant off-site fire support, well- 
    provisioned, outfitted with broad range of small arms and heavy weaponry, and operational  
    from a fortified static base or orbital carrier.    
12. Regiment. 5 to 8 companies organized under an on-site commander, who travels in an up- 
    armored, command-suite version of the same APC or dropship as their soldiers. Some of  
    the APCs or dropships are up-armored, anti-armor/gunship variants.    
13. Regiment. 5-8 companies of light mechs, organized into teams of 5-10, operating as  
    flankers, hussars, pickets, and/or first recon. Tip of the spear tactics, with more specialized  
    weapons and systems seeded throughout.  
14. Regiment. 5-8 companies of soldiers operating in support of medium or heavy assault  
    mechs at a 10:2 ratio.   
15. Regiment. 5-8 companies of armored vehicles, operating in teams of 5. Supported by a  
    rear echelon or orbital motor pool.   
16. Battalion. 2-4 Regiments organized under an offsite, high-ranking commander, typically  
    operating well behind the lines in a hardened site or orbital. Significant logistic, tactical,  
    and fire support. Continental reach and response time within a day.   
17. Battalion. 2-4 regiments organized under an on-site commander, who pilots a high-license  
    mechanized chassis and is attended by a retinue of similarly grizzled, ranking veterans. At  
    least 1 regiment is composed of medium or heavy assault mechs.   
18. Battalion. 2-4 regiments of armored vehicles, supported by an off-site battalion  
    commander with significant tactical, strategic, logistic, and fire support. Multiple motor  
    pools and/or heavy-lift shuttles allow for for sustained operations and rapid insertion.   
19. Army. 2-5 battalions, with significant material, logistic, strategic, tactical, systemic, and fire  
    support. Usually matched with an orbital/aerial presence, multiple forward operating  
    bases, a main operating base, and well-resourced. Will contain a mix of infantry, armor,  
    and mechanized chassis. Has a global reach.   
20. Army. 2-5 battalions Mobile, organized around a mix of medium and heavy chassis, with  
    support from regiments of mechanized infantry, dropships, and heavy shuttles. Supported  
    by an orbital battlegroup, has a global reach.    

                                                                                                        
                                                                                                          
                                                                                           