\part{Mech Combat}
MECH COMBAT  

It’s entirely possible to play through a session of LANCER without even touching mech combat.  
Some groups may prefer a more role-play, politically heavy game in which most of the fights are  
decided with skill checks.
 

However, Lancers are people with a particular talent, and it’s almost inevitable that talent will be  
put to good use during a mission. Sometimes, you want combat to be more in-depth, and each  
decision to matter more. You want your skills and ingenuity at building and fighting with a mech  
put to good use. That’s the signal that its the perfect time to get into Mech Combat.  

Unlike narrative play, mech combat is tactical, and turn based. As the name implies, these rules  
are only used for combat, and probably when mechs are involved. You can certainly run pilot on  
pilot fights using mech combat, but the options are far less interesting. 
 

Here are some key differences between running combat narratively (with skill checks) and mech  
combat:
 
        	- Instead of using your pilot skills to narratively resolve combat actions, you must take  
        turns, and you have limited options on your turns to attack, move, and activate  
         components of your mech.
 
         - Your pilot and mech make attack rolls, adding grit, to fight opponents. This might mean  
        that even though your pilot has a higher combat related skill (assault, for example), they  
         must use a lower value. This is because mech combat is at much higher stakes, and at a  
         much higher scale! Only your pilot’s direct experience fighting in a mech (and their four  
         mech skills) is going to help them succeed.
 

                                                  Use a map  

It is recommended you use a map of some sorts, and draw out or use items such as miniatures,  
tokens, etc to track the position of players during a fight. A grid (hex or square) will also help  
immensely. You can run LANCER without a grid, using each space 1 to 1 for inches or cm on a  
ruler and measure directly, but it will be far less consistent.  

                                              Starting Combat  

To start combat, the GM merely needs to declare that it has been initiated. Hostile intent, such  
as firing a weapon at a target, attempting to grapple them, or charging a target will typically  
automatically initiate combat. Establish where the various NPCs and players are when combat  
starts before you start rolling or picking turn order, it will help visualize things better.  

\chapter{The Turn}
                                                THE TURN  

                                                                                                              


During combat, players always take the very first turn. One player or friendly NPC (nominated  
by all players) gets to act first. If the players can’t agree, the GM chooses. After that player  
finishes their turn, the GM may activate a hostile, GM-controlled NPC, allowing them to take a  
turn. Each NPC can usually only be activated once, unless they have special traits. The player  
that previously acted then nominates a player or friendly NPC to act next, and so on. Each  
actor gets 1 turn in a round, alternating between players and hostile NPCs, with players each  
choosing the next player or friendly NPC to act.
 

If there are only actors of once side left, the remaining actors take their turns in any order. After  
all actors have completed a turn, this constitutes 1 round. The round then begins again,  
alternating, so if one side ended the last round, the other side starts the new round. This may, for  
example, mean that hostile NPCs take the first turn in the new round if the players outnumber  
them.  

On a turn, players and NPCs can perform a move, and either two quick actions or one full  
action, with no duplicate actions allowed. Players can overcharge their mechs to gain an extra  
quick action at the cost of heat, and all actors can also take any number of Free Actions or  
reactions.
 

MOVE - A player can move their character up to their full movement speed.
 
QUICK ACTION - A quick action represents an action that takes a few moments, such as  
quickly firing a weapon, using a system, or moving a little further
 
FULL ACTION - A full action represents an action that takes your full attention, such as a  
sustained barrage of fire, or field repairing your mech
 
FREE ACTION - A free action can be made at any point during your turn, but only on your turn. It  
doesn’t count as a quick or full action, so you can still make those as normal. Free actions can  
also be used to make a duplicate action (for example, a free action could allow you to boost if  
you have already made that action). You only get free actions if some part of your character  
grants you them.
 
REACTION - Reactions are special moves that can be made out of turn order in response to  
incoming attacks, movement, or other prompts. You can make each reaction only a specified  
number of times per round, but take as many overall as you want. By default, mechs have two  
reactions they can take once a round: brace, and overwatch but they may gain more from  
systems or talents. Reactions resolve before the triggering action completes by default, but  
some may resolve after.
 

                                                      PILOTS
 
On foot, a pilot has the following statistics in mech combat:
 
	        HP: 6 + grit
 
	        Evasion: 10
 
         E-defense: 10
 
	        Armor: 0
 
	        Size: 1/2
 

                                                                                                                  


	        Speed: 4
 

These statistics might change depending on the gear and armor a pilot brings with them.
 

Pilot weapons and armor are at a scale that they can’t be relied on to take down mechs - and  
mech weapons are at a scale that they normally completely pulverize a pilot-scale foe. The  
following rules apply to pilots (some of these refer to mech rules later in this section):  
     -   Pilots have the biological tag. They are immune to Tech actions (even beneficial ones),  
         though they can still be targeted by electronic systems such as drones or smart weapons.  
         If a pilot would take Heat, they instead take an equivalent amount of energy damage.  
     -   When a pilot is called on to make a mech skill check, they use Grit instead of the required  
         statistic  
     -   Pilots can’t aid a mech, give, or receive any bonuses that would apply to mech-sized  
         weapons (such as from Talents)  
     -   Pilots and pilot weapons and gear don’t benefit from Talents  
     -   Pilots can’t cause a mech to become engaged and don’t provide obstructions to mechs no  
         matter the size.  

It is possible for a pilot, with enough experience, to gain enough technology and experience to  
become capable of fighting on nearly even terms with some mechs, but such pilots are usually  
stuff of legend.  

                                                Pilot, Mech, and AI  

As components of the same character, pilots and mechs share the same move and actions. You  
can split them up if you so choose. If you want to use a quick action to skirmish with your mech,  
use another quick action to dismount, then use your move to run to cover as your pilot, you can  
absolutely do so.  

A mech needs to be piloted for you to take actions with it, with the pilot physically present inside  
the cockpit, unless that mech has the AI property. If your mech has the AI property, at the start of  
your turn you can choose to turn your controls over to your AI. If you do so, your pilot can no  
longer take actions or reactions with your mech until the start of your next turn, but your mech  
gets its own set of actions and reactions, freeing you up to take normal action as a pilot. However,  
your AI cannot benefit from any of your talents while it pilots your mech.  

\chapter{Movement}
MOVEMENT  

Characters move a number of spaces equal to their speed value. They can freely move through  
(but not end their turn in) the space of friendly targets, but treat hostile targets as obstructions. 
 
\section{Engagement, size, and movement}
                             Engagement, Size, and Movement   

All characters have to worry about engagement. If you move adjacent to a hostile target, you  
become engaged. Being engaged gives penalties to ranged attacks (+1 difficulty), and if you  
become engaged with a target the same size or larger than you, you must stop moving and lose  
any additional movement you have left. Targets smaller than you cannot stop you from moving,  
so larger mechs can move around more easily.
 

An actor that is larger than another actor does not count the smaller actor as an obstruction and  
ignores engagement from that smaller actor. For example, a size 1 mech can freely pass through  
the space of a size 1/2 human, and a size 2 mech can freely pass through the space of that size 1  
mech. Mechs can always move through pilots or human NPCs on foot.
 

Typical sizes:  
1/2: A human, a hard suit, small mechs, an EVA suit
 
1: Typical light mechs, assault mechs, line mechs
 
2: Battle tanks, many vehicles, heavy mechs
 
3: Siege mechs, heavy vehicles
 
4-5: Titanic mechs, mech-oriented flyers
 
\section{Disengage}
                                               Disengage  
A character can spend a full action to disengage, allowing them to ignore engagement when  
moving and allowing their movement to not trigger reactions (such as overwatch).  
\section{Involuntary movement}
                                      Involuntary movement  
Some actions or attacks push, pull, or shove a character in a certain direction. Involuntary  
movement such as pushing, pulling, or knocking a character forces them to move in a direct line  
in a direction specified by the triggering action or attack. Mechs that are pushed, pulled, or  
knocked around do not provoke reactions and ignore engagement with their movement, though  
they must still obey obstructions.  
                                                        
 
\section{Traversal and the enviorment}
                          TRAVERSAL AND THE ENVIRONMENT  

Mech combat takes place on many types of worlds with many hostile and hazardous  
environments. Here’s what you need to worry about:
 

                                                                                                             


Difficult Terrain reduces a mech’s speed. 1 space of movement through Difficult Terrain costs 2  
spaces worth of movement speed. Difficult terrain could be rough, marshy or swampy ground,  
icy landscape, or treacherous, rocky scree. What constitutes difficult terrain for pilots and mechs  
might be different.
 

Dangerous Terrain prompts an engineering check to navigate the first time on a turn an actor  
enters it on their turn, or if they start their turn there. Should a player fail that check, they take 5  
kinetic, energy, explosive, or burn damage on failure, depending on the hazard. Intense radiation,  
boiling gases, lava, or falling rocks are good examples of dangerous terrain. An actor only needs  
to make one check a turn for dangerous terrain.
 

Obstructions block passage. Obstructions are typically environmental, but can include NPCs  
and other players. Obstacles smaller than the moving object do not block movement, and  
can be passed through freely. Friendly NPCs or allied players never cause obstruction, but  
you can’t end your movement in their space.
 

Lifting and  Dragging  
A mech can drag another character or item up to 2x its size (but is Slowed while doing so), and  
lift a character or item overhead that’s its size or smaller, but remain immobile. While dragging or  
lifting another object or character, a mech cannot take reactions. Pilots follow the same rules but  
cannot drag or lift objects larger than size 1/2.
 

Climbing like difficult terrain, costs 2 spaces of movement for every space moved. Climbing  
especially difficult surfaces might require a successful hull or agility skill check not to fall.
 

A mech can jump half its speed horizontally in a straight line, ignoring ground based obstacles  
that it could jump over (such as pits, gaps, etc), and cannot jump higher than its size (so a size 1  
mech can jump up to 1 space high during that movement).
 
 
 
Falling causes damage if a character falls 3 or more spaces and cannot recover before it hits the  
ground. A character takes 3 AP kinetic damage for each 3 spaces it falls, up to a maximum of 12  
AP kinetic damage. Typically a character falls about 10 spaces a round, but a mech cannot fall in  
a zero-g environment (or even a low-g environment) and speeds may differ depending on where  
you are.
 
\section{Cover}
                                                     COVER  

Cover is obscurement from observation or gunfire. In narrative terms, cover refers to smoke  
screens, hard cover (a building, a wall, a bulkhead, etc) between the attacker and the target, soft  
cover (trees, earthen mounds, etc) between the attacker and the target, obscured vision,  
electronic countermeasures, or any other obstruction physical, mental, electronic, etc, between an  
attacker and their target.  

                                                                                                                    


Smoke, foliage, trees, blinding light, dust clouds, low hills, low walls, etc are all examples of light 
cover. Light cover is typically not solid enough to reliably block fire, but causes enough visual 
interference or reduces profile enough to make aiming difficult. 

Tall walls of buildings, ruined buildings, bulkheads, reinforced emplacements, destroyed mechs 
or vehicles, spacecraft, etc are all examples of heavy cover. Heavy cover is solid enough to 
block shots and hide behind. 

Light Cover adds +1 Difficulty to an attacker’s roll to hit for ranged weapon attacks. 
Heavy Cover adds +2 Difficulty to an attacker’s roll to hit for ranged weapon attacks. 

If a character has a better form of cover, it is not superseded by a weaker form of cover unless 
specifically mentioned. For example, a mech gains light cover from a talent. If a mech fires at that 
mech in heavy cover, they will still treat that target as in heavy cover, as that cover is better than 
the light cover granted to that target by their talent. 

\section{Splitting up movement and action}
                              Splitting up movement and action 

A character may take its actions at any point during its movement, and complete that movement 
after that action completes. However, each action itself cannot be split into several parts. 

For example, a mech with 3 weapons and 6 movement can move 3 spaces, then attack, then 
move 3 more spaces. However, if that mech takes the barrage action, its action must complete 
before it can move further, i.e. it must fire all weapons at once (it can’t move 2 spaces, fire a 
weapon, move two spaces, fire a weapon, etc). 

Actions and reactions themselves cannot be split into parts, and each action must resolve before 
the next takes effect. For example, a reaction typically interrupts and resolves before the action 
that triggered it resolves. 

\section{Teleport}
                                                TELEPORT 

Some experimental mechs have the ability to teleport. When a mech teleports, it instantly moves 
to a point within the specified range (it needs free space that will fit the whole of its body to be 
able to do so). Teleporting does not provoke reactions and ignores engagement. It ignores 
obstructions entirely and ignores line of sight. A mech can attempt to teleport to a space it can’t 
see, but if that space is already occupied, the teleport fails. The mech loses their action and takes 
5 AP kinetic damage. 

\section{Flight}
                                                   FLIGHT 

Some characters have the ability to Fly. When you Fly: 

                                                                                                                


    -    You ignore ground-based terrain, and you can totally ignore obstruction from ground  
         based targets while flying. You only become engaged with targets if you move physically  
         adjacent to them while you’re flying.
 
    -    You ignore obstructions while flying at all points while you’re flying. If you need to pass  
         over a a size 3 obstacle to get to the other side, feel free to do that (it’s assumed you just  
         juke around). However, you can only ignore obstructions if it’s physically possible for you  
         to do so (you can’t go right through a wall).
 
    -    You can end your movement anywhere within a vertical or horizontal range of you equal  
         to your fly speed, in any combination. For example, a mech with a fly speed of 6 could  
         end its movement anywhere within 6 spaces of its location, up to 6 spaces high.
 
    -    You cannot be knocked prone while flying
 

Flying also has some downsides:
 
             -    Flight movement must start and end along a straight line, though the direction  
                 can be changed for each separate movement. For example, a flying mech could  
                  move in one direction, then boost in another. 
 
             -    If a flying character is ever immobilized, stunned, shut down, or otherwise cannot  
                  move, it falls.
 
             -    If a flying mech takes structure damage, it must pass an agility check or fall.
 

                                                  Hover Flight  

Some very advanced mechs have Hover. Hover mechs do not need to move in a straight line,  
and can remain still while airborne (they don’t have to move on their turn and can move any  
distance).
                                                                                                                       
\chapter{Attacks}


                                            ATTACKS  

Attacks in LANCER come in three types: Melee, Ranged, and Tech. Melee attacks are typically  
made against a target in your threat range, grit vs evasion. Ranged attacks are made agains a  
target in range, grit vs. evasion. Tech attacks are typically made against a target in your sensor  
range, tech attack vs. e-defense. 
 
\section{Range and Patterns}
                                      RANGE AND PATTERNS  
Measure weapon range from one of the edge spaces of your mech to the edge of your target by  
default, unless specified.   
Some weapons or systems have special attack patterns: Line, Cone, or Blast, or Burst. These  
attacks hit areas, and affect all targets in that area unless specified, rolling an attack separately  
for each target.  
    -    Line is a straight line X spaces long. All targets the line intersects with are attacked. Make  
         separate attack rolls for each target caught in the area.  

                                                                                                                 


     -   Cone is a cone X squares wide at its longest point and X squares long, drawn from a point  
         1 space wide at its shortest point (it’s origin). Make separate attack rolls for each target  
         caught in the area.  
    -    Blast is an area of radius X spaces, drawn from a point in range and line of sight. Check  
         cover and line of sight for the actual attack from the point of the blast, not the attacker.  
         Make separate attack rolls for each target caught in the area.  
    -    Burst is an area affecting the space over your mech and X spaces around your mech.  
         You’re not affected by your own burst attacks unless specified. Check cover and line of  
         sight from your mech. Make separate attack rolls for each target caught in the area.  

Some attacks with a line, cone, or blast pattern have a range listed. The starting point for the  
attacks can be drawn from a point within the range specified. For example: a blast 2, range 10  
attack, would attack a blast 2 area centered on any point within range 10.  

Some mech licenses or systems include increases to range. This range increase does not affect  
the size of cones, lines, or blast attacks (though it might allow you to place such attacks at further  
range if a range is specified).  
\section{Threat}
                                                     Threat  

Threat indicates the default distance at which a melee weapon can be used, and a melee or  
ranged weapon can be used to make an overwatch reaction. Default threat for all weapons is 1  
unless noted otherwise, and it can be increased from certain talents or pieces of gear. Measure  
threat from a mech’s exterior, so larger mechs will cover slightly more area than smaller mechs.  
\section{Valid Targets}
                                                Valid Targets  

You can attack any target in range (ranged), sensor range (tech), or weapon threat (melee) as  
long as you have line of sight to that target. Valid targets are other characters (player or non- 
player) such as other mechs, monsters, or people; objects that are not being held, worn, or part of  
a mech; and a point in the environment or on the ground.  
\section{Line of Sight}
                                                Line of sight  

If your character can’t trace of line of sight to a target (ie, you cannot see any part of the target),  
then it cannot be attacked (melee, ranged, tech, or otherwise). Weapons with the arcing tag can  
still attack a target or point you don’t have line of sight to as long as they could actually draw a  
path there (they couldn’t fire through 50ft of bulkheads, for example), but still take cover into  
account. They typically attack by lobbing projectiles over obstacles. Weapons with the powerful  
seeking tag totally ignore cover and line of sight, as long as they could draw a path to their  
target. Seeking weapons are typically self guided, self propelled, and can navigate complicated  
spaces.
 

                                                                                                                   
\section{Invisibility}

                                                 Invisibility  

Some characters have the ability to turn invisible. An invisible character is detectable by heat  
patterns and some visual artifacts, but extremely hard to target - all attacks of any kind have a  
flat 50% chance to miss outright (roll a dice or flip a coin) - checked before rolling. Additionally,  
an invisible mech can always hide.  
\section{Attacks}
                                                  ATTACKS  

You can attack as a mech by making the Skirmish, Barrage, or Tech actions while piloting your  
mech. You can attack by taking the Fight action as a pilot.
 
Ranged attack: Choose a target in your weapon range and line of sight. Then roll 1d20, adding  
your grit vs your target’s evasion, plus any Accuracy or Difficulty.
 
                      •  Being adjacent to a hostile target causes a character to be engaged. If your  
                        mech is engaged, it takes +1 difficulty on all ranged attack rolls.  
                      •  Light cover imposes +1 Difficulty a ranged attack roll. Heavy cover imposes  
                        +2 Difficulty to the attack roll.  
Melee attack: Choose a target in the weapon’s threat and line of sight, then roll 1d20, adding  
your grit vs. your target’s evasion, plus any Accuracy or Difficulty.
 
                      •  Melee attacks ignore cover
 
Tech attack: Choose a target in your sensor range and line of sight, then roll 1d20, adding your  
tech attack vs your target’s e-defense, plus any Accuracy or Difficulty. Tech attacks ignore  
cover.
 

To hit, your total roll must equal or exceed your target’s evasion or e-defense.
 
\subsection{Bonus Damage}
                                             Bonus damage  

Some talents, systems, or weapons allow you to deal bonus damage, allowing you to deal  
boosted or extra damage to your attack. Bonus damage can only be kinetic, explosive, or energy  
damage (not heat or burn), and if not specified is the same damage type as one type from the  
weapon that dealt it.
 

Bonus damage follows the following rules:
 
         	- If bonus damage applies to an area of effect attack or an attack that targets multiple  
         actors, it can only affect one target (the rest just take normal damage), called the primary  
         target. This is the target that takes the brunt of the attack.
 
         	- Bonus damage doesn’t apply if you make a bonus attack with an auxiliary weapon  

\subsection{Critical Hits}
                                                Critical Hits  
On any total ranged or melee weapon attack roll of 20+, the attack is a Critical Hit. Roll all  
damage dice twice and choose the highest result (including sources of bonus damage, etc).
 
\chapter{Actions}
 ACTIONS  

•   Players can take two quick actions or one full action on their turns  
•   You cannot make duplicate actions unless you make them as a free action or reaction. For  
    example, you can only boost 1/turn, but you can boost again if you have a free action that  
    allows you to boost, or if you overcharge to do so  
    
\section{Basic Quick Actions}

                       BASIC QUICK ACTIONS  
\subsection{Skirmish}

                                             SKIRMISH  

When you take the skirmish action, you attack with a single weapon from your mech.   
        - You can also make an attack with another auxiliary weapon from the same mount. That  
        weapon can’t deal bonus damage. Auxiliary weapons are light and can be used to make  
        quick, numerous attacks.
 
        	- Superheavy weapons are too cumbersome to be fired with a skirmish action and must  
        be fired as part of a barrage action.
 
\subsection{Boost}

                                                BOOST  

When you take the boost action, you can move your speed. Boosting allows you to move again,  
in addition to taking a move action on the same turn. Certain talents and systems only activate  
on boosts (not regular movement).
 
\subsection{Ram}

                                                 RAM  
Ramming is a melee attack made against an adjacent target with the aim of knocking down or  
back an enemy mech.   
If your attack is successful, your target is knocked Prone and you may also knock your target  
back up to 1 space directly away from you.  
\subsection{Grapple}

                                              GRAPPLE  
When you Grapple, you attempt to grab hold of an enemy mech and overpower it, disarming,  
subduing, or damaging it so that it cannot do the same to you.   

In order to perform a Grapple, choose an adjacent target and make a melee attack. On hit:  
    -   Both parties are engaged  
    -   While grappled or grappling, neither party can boost or take reactions  
    -   The smaller party is immobilized, but moves when the larger party moves, mirroring their  
        movement. If both parties are the same size, they can make a contested hull check when  
        they attempt to move, counting as the large party for their turn if they win.  
    -   The grapple breaks if either target breaks adjacency (is knocked back for example)  
    -   The attacker can end the grapple as a Free Action, and the defender can end the grapple  
        as a quick action by making a successful hull or agility check.  

                                                                                                        


    -    If there are multiple parties involved in a grapple, the same rules apply, but when counting  
         size, count up all opponents of a side in a grapple. For example, if my all and I are both  
         size 1 and grappling a size 2 target together, we would count our total size (2) and could  
         attempt to drag our target around.   
\subsection{Quick Tech}

                                              QUICK TECH  

The Quick Tech actions cover electronic warfare, countermeasures, and other actions that can  
be taken by a pilot, often aided by their mech’s powerful computing and simulation cores. Many  
pilots choose NHP (non-human person) assistants or more conventional comp/con units to help  
them with these tasks. All mechs have access to the basic tech actions. Further tech actions can  
be enhanced by taking systems that upgrade them. 
 

Some tech actions are attacks (often called tech attacks) and benefit from generic bonuses to  
attack rolls. All tech actions must choose a target within Sensor Range to be effective, and roll  
systems vs. e-defense. To use a tech action, choose a target in your sensor range (including  
yourself) and choose one of the following options:
 

Bolster  
You use the formidable core processing power of your mech’s systems to boost one other  
target’s systems. That target can take +2 Accuracy on its next skill check of any kind before the  
end of its next turn. A mech can only benefit from bolster once at a time.
 

Scan
 
You can use your mech’s powerful internal systems to deep scan your enemies.  
To Scan, make a tech attack against a target in your sensor range. On a successful attack, ask  
your GM to reveal one of the two to you:
 
             -   Your target’s full statistics (HP, Speed, Evasion, Armor, HASE, etc), weapons, and  
                 systems
 
             -   Hidden information about the target, such as information caches it is carrying,  
                 current mission, pilot ID, etc. 
 
This information is only current when you receive it (for example, if the target loses HP again,  
your information won’t update).
 

Lock On  
Make a tech attack against a target in range. On hit, the target suffers from the Lock On  
condition, enabling some systems and talents. Any attacker can end Lock On on a target when  
they attack that target to gain +1 Accuracy on their very next attack roll against that target.
 

Invade
 
Make a tech attack against a target in range. On success, your target takes 1d3 heat and you  
may choose one of the following options:
 

                                                                                                                


         Fragment Signal/Feed Misinformation: You feed false information, obscene messages,  
         or phantom signals to your target’s core computer, inflicting the Impaired Condition on  
         your target until the end of their next turn.
 

         Aggressive Code: You attack your target’s servos and engines, inflicting the Slowed  
         condition on your target until the end of their next turn.
 

         Attack systems: You go for the throat, the core computer. Inflict an additional 1d3 heat  
         on your target
 
\subsection{Hide}

                                                      HIDE  

In order to perform the Hide action, you need cover or concealment. The cover needs to be large  
enough to totally conceal your mech (such as a smoke cloud or building) or you won’t be able to  
hide. Lack of line of sight is always sufficient, and if you’re invisible, you can always attempt to  
hide.
 

Hiding is always successful. After you hide, you gain the hidden condition. A hidden target can’t  
be directly targeted by attacks or hostile actions, but can still be incidentally hit by attacks that  
target an area. NPCs cannot perfectly locate a hidden target but only know their approximate  
location. 
 

Performing any attack (melee, ranged, or tech), the boost action, or taking a reaction will break  
hiding. You can take other actions as normal. You must end your turn in cover to keep hidden.  
You automatically lose hidden if you end your turn in a place where you wouldn’t benefit from  
cover (ie, a mech comes around a wall and can now draw unbroken line of sight to you), your  
cover is destroyed, or you move from cover. If you’re hiding and invisible, you also lose hidden if  
you lose invisibility.
 
\subsection{Search}
                                                   SEARCH  

To detect a hidden target takes a quick action and makes a contested check.  
         Mech: The searching party needs you to be in their sensor range and makes a systems  
         check. A hidden mech makes an agility check.
 
         Pilot: The searching party needs you to be in range 5 and makes a pilot skill check, using  
         skills such as notice. A hidden pilot makes a skill check and can use bonuses such as  
         infiltrate.
 
Once a hidden target is detected, it loses the hidden condition.

\section{Basic Full Actions}BASIC FULL ACTIONS  
\subsection{Barrage}
                                               BARRAGE  

When you take the barrage action, you can attack with two weapons, or one superheavy  
weapon. You can choose the same target or different targets when you make these attacks.
 
        	- When you attack with a weapon, you can also attack with another auxiliary weapon in  
        the same mount. That weapon can’t deal bonus damage.
 
        	- The barrage action takes your mech’s full attention and the engagement of all its  
        systems, so it requires Full Action to use. 
 
        	- Superheavy weapons can only be fired as part of a barrage action, as they require the  
        full attention of your mech’s systems.  
\subsection{Full Tech}
                                              FULL TECH  

Choose and perform two options from the Quick Tech list (or choose from other systems that  
would take a quick tech action to use). You can repeat options, but must choose different targets  
for each option.
 

Alternately, use a system or tech option that takes a Full Tech action to activate.
 \subsection{Improvised Attack}

                                      IMPROVISED ATTACK  

If your mech is unarmed or does not have a melee weapon, it can use an action to make an  
improvised attack action with a rifle butt, fist, or other improvised melee weapon against a target  
in melee. You may use the butt of a weapon, a slab of concrete, a length of hull plating, etc, to  
complete this improvised attack. How you flavor the attack is up to you!
 

An improvised attack costs a full action by itself to perform, and is separate to the skirmish or  
barrage actions above. It counts as a melee attack. An improvised attack is a melee attack that  
deals 1d6 kinetic damage on hit.  
\subsection{Stabilize}
                                              STABILIZE  
During a heated battle or prolonged mission, it may become necessary to enact emergency  
protocols in order to purge your mech‘s systems of excess heat, to repair your chassis where you  
can, and/or buy your system time to eliminate hostile code.  

To that end, a pilot may spend a Full Action to Stabilize and do one of the following:  
•  Cool your mech, resetting the heat gauge  
•  Spend 1 Repair to refill HP to maximum.  

And one of the following:  
•  Reload all weapons with the Loading Tag  

                                                                                                           


•  End all Burn currently affecting your mech  
•  Perform an Engineering Check. If successful, end one of the following conditions on yourself  
  or an adjacent ally.  
             •  Jammed  
             •  Impaired  
             •  Lock On  
             •  Immobilized  
             •  Slowed  
\subsection{Disengage}
                                            DISENGAGE  

When you disengage, you attempt to move safely. Until the end of your turn, your movement  
ignores engagement and does not provoke reactions, such as overwatch.
 
\section{Other Actions} OTHER ACTIONS  
\subsection{Activate}
                                             ACTIVATE  
Some systems or pieces of gear take either a quick or full action to use or activate. Such  
systems are marked with the quick or full action tags.
 
\subsection{Shut Down}
                                           SHUT DOWN  

Shutting Down your mech is a risky move, though one that is sometimes necessary to prevent  
potentially catastrophic systemic overload or AI unshackling.   

You can shut down as a quick action. When you take the Shut Down action, your mech powers  
completely off and enters the Shut Down state. While Shut Down:
 
       •  Your mech is stunned. However, you can still take the Boot action to reboot your mech.  
       •  Your mech is immune to all tech actions or attacks and can’t benefit from friendly tech  
         actions. Any tech effects or conditions caused by a tech action (such as lock on, etc)  
         affecting the mech immediately end.
 
       •  Your evasion becomes 5  
       •  Your mech immediately cools (get rid of all heat)  
       • Any unshackled AI you have installed are re-shackled.  
\subsection{Boot Up}
                                              BOOT UP
 
You can power up a shut down mech as a full action, ending the shut down condition on it.  
Mechs that are powered off must be powered on with a boot up action before becoming active.  
You must be piloting a mech to boot it up.
 

                                                                                                         

\subsection{Mount or Dismount}
                                     MOUNT OR DISMOUNT   

Mounting or Dismounting a mech is a turn of phrase commonly used by pilots. You don‘t “get  
in“ or “climb aboard“, you mount. You‘re the cavalry, after all. It takes a quick action to mount or  
dismount. You must be adjacent to your mech to Mount it, and when you Dismount your mech,  
you are placed adjacent to it. If there’s no free space, you cannot dismount your mech.
 

If you want to Eject when you dismount your mech, you can do so, flying 6 in a direction of your  
choice. However, its a one-way system meant to be used in case of emergency, and leaves your  
mech permanently impaired until you full repair (and the eject system can’t be used again until  
you take a full repair).
 
\subsection{Self Destruct}
                                          SELF DESTRUCT  

Self-destructing by overloading your reactor is a final, catastrophic play a pilot can trigger. You  
can initiate self destruct as quick action, causing your reactor to start melting down. At the end  
of your next turn, or up to two turns after (you choose), your mech will explode as though it  
suffered a reactor meltdown, annihilating it, killing any pilot inside, and causing a burst 2  
explosion for 4d6 explosive damage around it. Characters caught in the explosion can pass an  
agility check to halve this damage.  
\subsection{Prepare}
                                                PREPARE  

When you prepare an action, you’re holding in preparation for a specific time or trigger (a more  
advantageous shot, for example). You can only prepare a quick action, and it costs a quick  
action to prepare. This counts as that action’s duplicate, so you can’t, for example, skirmish and  
then prepare a skirmish action.
 

Until the start of your next turn, you can take the prepared action as a reaction. You must set a  
trigger for this reaction phrased as a ‘When X, then Y’ sentence. X must be an enemy or allied  
reaction, action, or movement. For example: “When my ally moves adjacent to me, I want to  
throw a smoke grenade,” or “When an enemy moves adjacent to me, I want to ram them”.
 

It is apparent to a casual observer when you are preparing an action (you are clearly taking aim,  
cycling up systems, etc). You can’t take reactions while you’re holding a prepared action, but can  
take them normally afterwards. If you want to take a reaction and drop your prepared reaction,  
you can also do so. If the trigger doesn’t activate, you lose your prepared action.
 
\subsection{Overcharge}
                                            OVERCHARGE  
It is possible for skilled pilots to push their mech beyond factory specifications for a short period  
of time in order to gain a tactical advantage. Moments of hyperspec action won‘t tax your  
mech‘s systems too much, but sustained action beyond prescribed limits will take its toll. 
 

                                                                                                              


You may Overcharge your mech only once per turn. Overcharging incurs 1 heat. The next time  
you overcharge before you make a full repair, this cost increases to 1d3 heat. The next time, the  
cost increases to 1d6 heat, and thereafter to 1d6+3 heat. Taking a full repair resets this counter.
 

Overcharging immediately allows you to make any quick action of your choice as a free  
action, even one you already made this turn. 

\section{Reactions}
                                        REACTIONS  

Reactions are special moves that can be made out of turn order in response to incoming triggers  
such as attacks or movement. Upon use, reactions are, unless specified otherwise, expended  
until the beginning of your next turn. You can only make one reaction per turn (your turn or  
another actor’s), but any number per round, as long as you have unspent reactions to perform.  
All mechs can use the Brace and Overwatch reactions once per round by default.
 
\subsection{Brace}
                                                  BRACE  

Once per round, you can choose to brace your mech’s systems in response to incoming fire. You  
can choose to brace against an attack after the attack hits you and you learn what the damage  
is.
 

If you choose to Brace, you gain resistance to all damage from the triggering attack (damage is  
halved, rounding up) and all other attacks against you are made at +1 difficulty until the end of  
your next turn. However, the stress of bracing means until the end of your next turn you cannot  
take reactions and on that turn you can only make one quick action (no regular move, no full  
actions, no free actions, and no overcharge).
 \subsection{Overwatch}

                                              OVERWATCH  

All mechs are able to perform Overwatch. Overwatch represents your mech’s ability to control  
and defend the space around it from enemy incursion, whether through pilot skill, reflex, or finely  
tuned sub-systems. By default, a mech can make 1 overwatch reaction per round.
 

If any enemy starts any movement (move, boost, etc) inside the threat of one or more of your  
weapons, you can immediately make a skirmish action as a reaction against that target using  
that weapon and any others on the same mount that count the target inside their threat.
 

Threat is 1 by default for all weapons unless listed otherwise.

\section{Free Actions}
   Free Actions  

Free Actions are actions often granted by systems, talents, gear, or overcharge. Characters may  
perform any number of Free Actions on their turn, but only on their turn, and only those granted  
to them. The most common type of Free Action is a protocol, which can be activated or  
deactivated only at the start of a turn.

\section{Pilot Actions}
   PILOT ACTIONS  

Pilots can take the following actions : BOOST, HIDE, SEARCH, ACTIVATE, DISENGAGE,   
PREPARE, OVERWATCH, BOOT UP, SHUT DOWN, MOUNT/DISMOUNT  
These are the same as the mech actions.
 

They also get the following actions:
 
\subsection{Fight}
                                      FIGHT (FULL ACTION)  
Make a melee or ranged attack with one weapon against a target in line of sight and range.
 
\subsection{Jockey}

                                    JOCKEY (FULL ACTION)  
It is possible (though very foolhardy) to aggressively attack an enemy mech while on foot. To  
jockey a mech as a pilot, you must be adjacent to it. You must make a contested check with the  
mech, using GRIT (your GM could allow another skill, such as maneuver, flash, or brawl if you  
argue for it). The mech contests with hull. If you win the contest, you’re now riding the mech,  
sharing its space and moving when it moves. The mech you’re riding can shake you off by  
repeating the contest successfully as an action, and you can jump off as part of your movement  
any time. 
 

Attempting to jockey takes a full action. The turn you successfully jockey, you can choose one of  
the below for free, and repeat one each turn after that you continue to jockey as a full action.
 

Distract: You inflict the Impaired or Slowed condition on your target until the start of your next  
turn.
 
Shred: Deal 2 heat to your target by ripping at wiring, paneling, etc
 
Damage: Deal 4 kinetic damage to that mech by firing or slashing at joints, hatches, etc
 
\chapter{Statuses}

  STATUSES  

During combat, characters might inflict statuses on opponents, giving temporary conditions. Most  
statuses indicate when they are inflicted and when they end in the weapon, system, or piece of  
gear that inflicts them.  

CORE BREACH  
    -   You cannot cool heat or lose heat
 
    -   Make an overheating check each time you take heat
 
CRITICAL  
    $$\bullet$$   You cannot gain HP or repair  
    $\bullet$    Make a structure check each time you take damage  
Engaged  
    -   While Engaged in melee, all your ranged attacks are made with +1 difficulty
 
    -   If you become engaged with a target the same size or larger, you immediately stop  
        moving and lose any unspent movement from that move.
 
Hidden  
    -   You cannot be targeted by hostile attacks or actions, and enemies only know your  
        approximate location. Performing any attack, taking reactions, or ending your turn out of  
        cover will lose hidden.
 
Immobilized 
 
    $\bullet$    Your maximum speed becomes 0. You cannot move or boost.  
Impaired  
    $\bullet$    +1 Difficulty on all actions, attacks, and skill checks  
Invisible  
    $\bullet$    All attacks against an invisible target have a 50% chance to miss outright (check before  
        the attack roll).
 
Jammed 
 
    $\bullet$    The only attacks a Jammed mech can make are improvised attacks, grapples, or rams
 
    $\bullet$    A Jammed mech cannot use comms to talk to other players (can only talk to GM)   
    $\bullet$    A Jammed mech cannot make or benefit from Tech actions
 
    $\bullet$    A Jammed mech cannot take reactions
 
Lock On  
    -   An attacker can consume the Lock On status on a target to gain +1 Accuracy on its next  
        attack against that target. Lock On may also activate other talents or abilities.
 
Shut Down   
    $\bullet$    Your mech is stunned, but you can still take the Boot Up action  
    $\bullet$    Your mech’s evasion becomes 5  
    $\bullet$    Your mech cools  
    $\bullet$    You cannot be affected by tech actions or system attacks or effects  
    $\bullet$    Shutting down (or being shut down) re-shackles any unshackled AI.
 
Shredded  
    $\bullet$    A shredded mech cannot benefit from armor or resistance
 

                                                                                                          


Slowed  
     $\bullet$    Your maximum speed becomes 2 (after all modifiers)  
     $\bullet$    You cannot boost
 
Stunned 
 
     $\bullet$    You cannot overcharge, take free actions, reactions, move, or take any actions with your  
         mech. A pilot in a stunned mech can still mount or dismount their mech, eject, or take  
         actions normally.  
     $\bullet$    Attackers receive +1 Accuracy to attack you
 
     $\bullet$    You automatically fail all hull or agility checks
 
Prone 
 
     $\bullet$    Attackers receive +1 Accuracy to attack prone targets  
     $\bullet$    A mech knocked prone is Slowed while prone but can still move. A prone mech can stand  
         up as its regular movement. A mech cannot stand up while immobilized.  
\chapter{Stat and Terminology Glossary}
                          STAT AND TERMINOLOGY GLOSSARY  

Armor - The amount of damage you reduce all incoming sources of damage by. For mechs,  
cannot go higher than 4  
Electronic Defense - The number that most electronic warfare attacks must beat to be  
successful  
Evasion - The number that most melee and ranged attacks must beat in order to hit with an  
attack  
Grit - 1/2 your level. Added to melee and ranged attacks, system points, and hp  
Heat Capacity - The amount of heat your mech can take before making an overheating check.  
Hit Points (HP) - The amount of damage you can take as a pilot before going down and out, and  
the amount of damage a mech can take before it takes 1 point of structure damage.  
Modifier - The number added to mech skill checks and attacks, equals your targeting, hull, agility,  
engineering, or systems  
Range - The range of your ranged attack, measured from yourself. Depends on weapon  
Repair Cap - The maximum number of repairs to your mech you can make per mission  
Resistance - Resistance to damage or a type of damage means it is reduced by half, rounded up,  
after armor is applied. You can only have resistance to damage once (it doesn’t stack multiple  
times)  
Structure - When a target with structure goes to 0 HP, it makes a structure check, then takes 1  
structure damage. When a target runs out of structure, it either goes into the CRITICAL state  
(player mechs) or is destroyed outright.  
Tech attack - Electronic Warfare attacks, modified by your systems  
Threat - The range of your melee and overwatch attacks with certain weapons, measured from  
yourself. Base threat for all weapons is 1, but it may be greater depending on the weapon.  
Sensor Range - The range in which you can make electronic warfare attacks, lock on, and use  
some systems  
Size - The area that your mech takes up, rounded up for determining space. For example, a size  
2 mech is an area 2 spaces on each side approximately 2 spaces high  
Speed - How far your mech moves when it moves (in spaces)  

                                                                                                              
\chapter{Ending Combat}

                                           ENDING COMBAT
 

Combat ends when one side has accomplished their main objective. Usually this means they  
have defeated, destroyed, or routed the other side, but not necessarily. Combat doesn’t have to  
end with total destruction of one side or another - it’s perfectly fine for a GM to call combat early  
if the outcome is not in question, and return to narrative play. NPCs often have their own goals  
and are typically concerned with self-preservation.
