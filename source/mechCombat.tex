\chapter{MECH COMBAT}
It’s entirely possible to play through a session of LANCER without even touching mech combat. Some groups may prefer a more role-play, politically heavy game in which most of the fights are decided with skill checks.

However, Lancers are people with a particular talent, and it’s almost inevitable that talent will be put to good use during a mission. Sometimes, you want combat to be more in-depth, and each decision to matter more. You want your skills and ingenuity at building and fighting with a mech put to good use. That’s the signal that its the perfect time to get into \textbf{Mech Combat}.

\textbf{Unlike narrative play,} mech combat is tactical, and turn based. As the name implies, these rules are only used for combat, and probably when mechs are involved. You can certainly run pilot on pilot fights using mech combat, but the options are far less interesting. 

Here are some key differences between running combat narratively (with skill checks) and mech combat:
\renewcommand{\labelitemi}{\(-\)}
\begin{itemize}
    \item Instead of using your pilot skills to narratively resolve combat actions, you must take turns, and you have limited options on your turns to attack, move, and activate components of your mech.
    \item Your pilot and mech make attack rolls, adding \textbf{grit}, to fight opponents. This might mean that even though your pilot has a higher combat related skill (assault, for example), they must use a lower value. This is because mech combat is at much higher stakes, and at a much higher scale! Only your pilot’s direct experience fighting in a mech (and their four mech skills) is going to help them succeed.
\end{itemize}

\subsubsection{Use a map}

It is recommended you use a map of some sorts, and draw out or use items such as miniatures, tokens, etc to track the position of players during a fight. A grid (hex or square) will also help immensely. You can run LANCER without a grid, using each space 1 to 1 for inches or cm on a ruler and measure directly, but it will be far less consistent.

\subsubsection{Starting Combat}

To start combat, the GM merely needs to \textbf{declare} that it has been initiated. Hostile intent, such as firing a weapon at a target, attempting to grapple them, or charging a target will typically automatically initiate combat. Establish where the various NPCs and players are when combat starts before you start rolling or picking turn order, it will help visualize things better.

\subimport{./mechCombat/}{turn}

\subimport{./mechCombat/}{movement}

\subimport{./mechCombat/}{attacks}

\subimport{./mechCombat/}{actions}

\subimport{./mechCombat/}{statuses}

\subimport{./mechCombat/}{glossary}

\subimport{./mechCombat/}{ending}
