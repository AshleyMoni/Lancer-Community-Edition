\part{The Mech}

THE MECH

A mechanized cavalry unit -- a mech -- is the primary agent around which the Union Navy
bases its ground forces.

Mechs depending on the FRAME, stand anywhere from eight to forty feet tall and are bi-, qadra-,
or hexapedal. The majority of them are brachial, featuring one to two pairs of arms able to
manipulate to-scale weaponry and interact with the natural environment. Some pilots and units
prefer to integrate their mech’s weapon systems to the unit’s chassis and, depending on the size
or power of the weapon system, such an integrated system might be required. Most mechs are
piloted by a single pilot, but there are larger, highly advanced platforms that require one or more
additional pilots to control.


Mechanized cavalry units are agile, quick, and responsive systems for their size. They are able to
traverse most all solid and vacuum environments; their mobility is often augmented or entirely
dependent on maneuvering jets (fuel depends on environment). Still, they are heavy, and in order
to run they are powered by a cold fusion generator. Their power plant is heavily shielded and
resistant to damage, reliable, and essentially inexhaustible, but should the reactor be
compromised the results are often catastrophic.


Mechanized cavalry typically support mounted or unmounted infantry as a heavy weapons
platform or force multiplier. They operate on their own in hostile environments in squadrons of
two to five, often dropped either far behind enemy lines or at the front, where the fighting is
thickest and they can be used as line-breaker shock troops.


Mechs are, by and large, military equipment, modular and restricted by license. However, they
are common enough in civilian construction, hazardous materials cleanup, exploration, and other
roles that they are not shocking to the average human. Mechanized Cavalry, though, are
different: their pilots are regarded on the same level as knights or the flying aces of old.

\chapter{Piloting a Mech}
                                          PILOTING A MECH

You must be physically inside a mech to pilot it. Mechs that are powered off or inactive are
typically shut down (they can’t do anything and it’s easy to hit them). You can mount or
dismount a mech and boot up a mech by taking action to do so. While inside your mech, you
don’t have line of sight to anything outside your mech (and nothing outside has line of sight to
you. As long as your mech is not destroyed, you cannot be targeted, damaged, or affected by




any attacks or effects that originate from outside your mech while inside it. If your mech is
destroyed, you lose this benefit (it’s got holes blown in it!) and it merely grants cover.


\chapter{Mech Skills}
                                                  Mech Skills

A mech operates on an entirely different level in LANCER: a powered, armored hulk with
incredibly powerful systems, weaponry, and physical strength. A mech is capable of far more than
a pilot on foot, and can perform feats of incredible strength, speed, and resilience where skill
alone will not suffice.

Your character’s pilot skills relate to their personal aptitudes on foot, whereas your character’s
mech skill refer to their abilities piloting and building mechs. A Mech skill check can be referred to
by the name of it’s statistic, such as a Hull check, a an Agility check, etc.

Each mech skill has a primary benefit and use. Unlike pilot skills, mech skills also give your mech
bonuses while building or creating a mech.

HULL describes your ability to pilot and build mechs with high strength, structure, and frame
reinforcement. Mechs with a high hull can survive a lot of punishment and perform feats of
enormous strength.

Roll Hull when: Smashing through or pulverising an obstacle, vehicle, or building; lifting, dragging,
pushing, or hurling an enormous amount of weight; grabbing, wrestling, or grappling a mech,
starship, or mech-sized creature; resisting a huge amount of force, staying upright through
cataclysmic weather

AGILITY describes your ability to pilot and build mechs with speed, reactivity, quickness, and
dexterity. Mechs with a high agility are lightning fast for their size, and can react quickly and
accurately.

Roll Agility when: Chasing, pursuing, or fleeing with incredible speed; performing acrobatics in
your mech; hiding or moving silently; performing a feat of fine manual dexterity with your mech;
dodging out of the way of danger

SYSTEMS describes your ability to build mechs with powerful computing and electronic systems,
often aided by sentient or sub-sentient programs. Mechs with powerful Systems scores have a
mind-boggling amount of computing power, some of which may not even exist in realspace.

Roll Systems when: Infiltrating hardened and powerful electronic systems and targets, including
other mechs; boosting or suppressing a signal; performing electronic warfare; scanning or analyze
information; boosting or weakening the electronic systems of a vehicle, mech, starship, or base;
interacting safely with Non-Human Persons or electronic life forms; analyzing the nature of
unfamiliar electronic systems.




ENGINEERING describes your ability to build mechs with powerful reactors, high tech weaponry,
ammunition, and utility tools. Mechs with a high engineering score typically incorporate a lot of
advanced technology, are extremely resilient and reliable, can draw upon huge amounts of energy
and heat, and can push their systems far beyond their factory specifications.

Roll Engineering when: pushing your mech past its limits; withstanding extreme conditions such
as heat, cold, void, or radiation; keeping your mech running well past its breaking point or for
extreme amounts of time; traveling or move safely through hazardous or hostile conditions;
boosting the reactor output of another mech, starship, or base; conserving and efficiently using
ammo, power, and other resources

Collectively, these four skills are often referred to as H.A.S.E. You may add the appropriate rating
to a 1d20 roll for the total when making a skill check that calls for one of these ratings. You often
make HASE checks as a part of mech combat, and certain actions will call for checks, such as
hiding or grappling.

\chapter{The Modular Mech}
  THE MODULAR MECH

Mechs in LANCER are powerful machines, but what makes them more powerful, and your
character more unique than the regular mech pilot, is your ability to source mech weapons, parts,
and gear from many different manufacturers and combine them together. LANCER has a
modular mech system, where each mech comes with a list of gear, gated by license, that once
acquired can be freely swapped between any other mechs a pilot owns.

Pilots in LANCER are as much an expert at building their machines as piloting them (or else have
a team supporting them). Currency is not tracked in LANCER, so a pilot’s ability to get better parts
and build a stronger mech mainly increases with License Level. License level represents wealth,
resources, connections, influence, and all the powers a pilot has at their disposal to get better and
more advanced mech parts. A pilot’s mech and their proficiency with mechs are tracked with a
character’s mech skills and their license level.

When you level up from levels 1-12, your pilot gains +1 license point to spend on a license from
any mech manufacturer, from rank I to rank III (with each rank requiring the previous one). Each
level of a license you unlock grants you access to various systems and weaponry, with rank II
giving you a FRAME (a base for your mech), and rank I-III giving you more advanced weapons
and systems. These weapons and systems are interchangeable. You can add as many to your
mech as you have mounts or system points to do so, from any source among licenses you have
unlocked. This is the main way you build ‘your’ mech.


After picking a frame and gear, you apply bonuses from your mech skills, and any core
bonuses you have unlocked (every 3 levels) to get a finished mech.


All pilots, even level 0 pilots, always have access to the GMS ‘Everest’ FRAME and the general
GMS gear list. At level 0, this is the only gear available for pilots until they gain license levels and
can unlock more gear.

\section{Without Limits}
                                          WITHOUT LIMITS


You never need to re-acquire mech gear or FRAMEs that you have access to, as long as you
have the license for that gear. You are assumed to generally always have access to that gear,
through influence, patronage, wealth, or rank. In addition, modern 3-d printing technology has
advanced enough in LANCER that entire mechs can be printed wholesale. Even if you lose your
mech in combat, it can easily be re-printed during downtime. You can think about your licenses
as your character’s ‘class’ if this were a more traditional RPG.

\section{Mech Structure}
                                        MECH STRUCTURE




The basic structure of a mech is 2 arms and 2 legs. You can modify this however you choose,
within reason (ask your GM). The general look, structure and layout of your mech has no
bearing on game play.


Let’s go through each mech component in detail:

\subsection{Frame}
                                                 FRAME

Your mech’s FRAME is its chassis, armor, mounts, and reinforcement. It determines your mech’s
appearance and function, from a heavy siege fighter, to an agile flier, to a cloaking technical-
focused mech. Think about the FRAME as choosing ‘which’ mech you’re going to pilot.


In game terms, a FRAME is the base of your mech. FRAMEs becomes available by unlocking
level II of mech licenses.


Your mech’s FRAME determines its size and armor. Your FRAME also has base statistics,
system points and mounts, which determine what kinds of weapons and systems you can add
to your mech. The kinds of weapons and systems you can add depend on the kinds of mounts
your FRAME has and the free system points it has available.


Finally, your FRAME gives you a powerful CORE system ability to use in combat that they can
typically only activate once per mission. Activating this core system requires Core Power, which
you start every mission with, but can only use once.

\subsubsection{Size and Armor}
                                        SIZE AND ARMOR

Let’s go over the base FRAME features. Every FRAME starts with differing number of mounts,
system points, armor, a different CORE system and can be size 1/2 to the enormous size 3.:


Size: How big your mech is, in spaces, on each size. Humans and the smallest mechs are size
1/2. Typical mechs are size 1. Size does not directly correlate to how big an actor is, but the
space it controls around it. Most mechs are size 1 (about 10’ by 10’ in default size) while being
about 15-20’ tall in the fiction.

Armor: A mech’s armor reduces all incoming sources of damage by that amount. Armor mostly
depends on FRAME, and can’t go higher than 4. Damage with the AP tag or burn ignores armor.

\subsubsection{Mounts and Weapons}
                                   MOUNTS AND WEAPONS
Your mech can add weapons as long as it has space for them on a mount. Weapons come in
several sizes, and each mount will only take a certain number and size of weapons.


All mech weapons in LANCER have size, type, and damage.




•  Pilot (size): A pilot scale weapon. Pilot weapons are small enough that they cannot Critical Hit.

•  Auxiliary (size): The smallest weapon size for mechs, light enough to use alongside larger
  weapons. When you attack with a weapon, you can also attack with an auxiliary weapon in the
  same mount for free.
•  Main (size): A normal sized weapon (for a mech)
•  Heavy (size): A large, heavier weapon typically used to inflict massive damage.
•  Superheavy (size): A very large, usually special-class weapon with high power requirements.
  Can only be fired with the Barrage action.

•  Type - All weapons have a type, which can be one of the following: CQB, Rifle, Launcher,
  Cannon, Melee, Nexus. These describe the general effect range and combat function of the
  weapon.


•  Explosive (Damage/ Weapon type): Explosive weapons deal their damage in a single,
  sudden, and incredibly powerful burst of shrapnel, flame, and/or pressure, blasting in a radius
  around their point of detonation.
•  Kinetic (Damage/ Weapon type): Kinetic weapons fire solid projectiles of various calibers and
  sizes, inert or innervated, that rely on simple collision to deal damage from point-of-impact
  through to point-of-exit. Kinetic weapons utilize chemical and electronic methods of firing or
  launching their projectiles, and are commonly fed by belts, boxes, and/or internal or external
  magazines.
•  Energy (Damage/ Weapon type): Energy weapons are weapons that project beams, lances,
  bolts, waves, or cones of different energy to damage and destroy their targets. Commonly
  powered by external or internal batteries, or hooked directly into a mech’s power core, energy
  weapons demand tremendous amounts of input to provide tremendous amounts of output.


Weapons might also deal Heat or Burn (damage over time), which have additional effects,
detailed in the damage section.
\paragraph{Mounts}
                                                 MOUNTS

Mech FRAMEs are standard-built to mount a limited number of core systems. Too many
weapons or systems will overtax the reactor or add too much stress on the mech’s structure.


Each mech FRAME has differing numbers of the following mount points. You cannot add
weapons or systems to your mech if you don’t have an available mount to do so. You can add a
weapon of any smaller size in a space that could take a larger weapon (for example, you could
add a main or aux weapon to a heavy mount, or add two aux weapons to a main/aux mount).


Aux/Aux mount: This mount takes up to 2 auxiliary weapons

Main/Aux mount: This mount takes 1 main weapon and 1 aux weapon

Main mount: This mount takes 1 main weapon

Heavy mount:  This mount takes 1 heavy weapon





Flexible mount: This mount takes either 1 main weapon or up to 2 auxiliary weapons


Superheavy weapons take a heavy mount and one other mount of any size.


Some mechs have the following mount:


Integrated mount: This mount is part of a mech’s FRAME. It includes the listed weapon by
default, which cannot be removed or replaced. This mount and weapon cannot be removed,
modified, or duplicated in any way.


Weapons mounted on a mech don’t necessarily need to be part of its chassis - they could be
slung in holsters, build into compartments, or held/wielded normally. You can decide which when
you build your mech - it has no effect in the rules. Mounts represent the tax on your mech’s
systems more than an actual physical structure.


\subsection{System Points}
                                          SYSTEM POINTS


Mech FRAMEs also come with a certain number of System Points (SP). System points can be
spent to add additional systems to your mech, and some weapons or heavier systems will take
system points to add to your mech in addition to requiring open mounts. As a pilot levels up, you
can add your grit (1/2 your level) to your total system points. You cannot add systems to your
mech that would cause you to exceed your system points.

\subsection{Core System}
                                             CORE system
Each FRAME comes with a CORE system. This a powerful ability that is unique to the FRAME,
can’t be transferred (you have to use the FRAME to use it), and can only be used typically once a
mission by consuming Core Power.


All mech FRAMEs have a reservoir of high efficiency system power that is designed only to work
for a short period of time. This Core Power is essential to the high-powered systems that many
mechs utilize in emergency situations or points of heavy action.


Your mech either has Core Power or it doesn’t. There’s no way to ‘save’ it up (you either have it
or you don’t). You always get core power when you start a mission or full repair. You only gain
core power when taking a full repair, or unless the GM grants you core power during the course
of a mission.

\subsection{Base statistics and improvements}
                         BASE STATISTICS and IMPROVEMENTS
Every FRAME has different base statistics that it starts with, giving it a unique role.


HP: Like your pilot, your mech has HP (hit points). Your mech, however, has 4 structure, and
isn’t destroyed when reaching 0 HP. Instead, its HP resets when it loses structure and it makes a




critical damage check. When a mech runs out of structure, it goes into the CRITICAL state, and
makes a critical check each time it takes damage.

Repairs: Your mech can heal and repair its systems by spending repairs, like a currency. If it runs
out of repairs, it can no longer heal or repaired destroyed systems.

Speed: How far your mech can move when it moves.

Evasion: How hard it is for most ranged and melee attacks to hit your mech.

Sensor Range: The distance your mech can use tech systems or attacks. If something’s in your
sensor range, you know it’s there unless it’s hiding (even if you can’t directly see it).

Tech attack: Your mech can add its tech attack instead of grit when conducting electronic
warfare.

E-defense: How hard it is for electronic or guided weapons or systems to hit your mech.

Heat Capacity: Your mech can take heat from electronic warfare or its own systems. If it takes
more than its heat capacity, it overheats and suffers adverse effects.

Limited systems: Some weapons only have limited uses between full repairs. Once used up,
they can’t be used again until replenished.


Your pilot’s mech skills can boost some of the statistics of any FRAME you pilot when you build
it, giving you your final mech. This is your pilot’s unique or personal touch that you add to the
stock FRAME to customize it.


The following features of a FRAME get bonuses from your mech skills. This is where your pilot’s
personal abilities really kick in.


GRIT (1/2 your level) can be added to your mech’s attack bonus, HP, and system points


Your HULL skill affects your mech’s HP and durability.
       •  Each point of hull gives you +2 hp.
       •  Each 2 points of hull gives your mech +1 repair capacity
Your AGILITY skill affects your mech’s evasion and speed.
       •  Each point of agility gives +1 evasion
       •  Each 2 points of agility gives your mech +1 speed
Your SYSTEMS skill affects sensor range, tech attack and e-defense.
       •  You can add your systems to your mechs’ sensor range and electronic defense
       •  You can add systems to tech attack
Your ENGINEERING skill affects heat capacity and limited systems
       •  You can add your engineering skill to heat capacity
       •  Each 2 points of engineering gives your mech +1 to maximum uses for limited systems
         and weapons
\subsection{Core Bonus}
                                              CORE BONUS

As your pilot acquires more licenses with a particular manufacturer, they will gain manufacturer-
specific knowledge and skills. In game this is represented by a CORE bonus. Every 3 levels, you




can apply a new CORE bonus to your CORE. Bonuses are a permanent improvements, and
apply no matter what FRAME you are using. They are are unique (you cannot take the same
twice) and can offer interesting ways to customize your mech. You can always choose GMS
CORE bonuses, but to choose a CORE bonus from a manufacturer, you need 3 points in licenses
from that manufacturer (any licenses, you can mix and match), for each CORE bonus you have
from them. For example, to take 1 CORE bonus from IPS-Northstar requires at least 3 points in
IPS-N licenses, to take 2 would require at least 6.

\section{Talents}
                                                 TALENTS

Your pilot’s experience and abilities with piloting a mech is tracked by Talents. This is where skill
and ingenuity can push your mech past its limits. Talents give your character abilities with
particular types of weapons, systems, or styles of play that help define your character further.


Talents, like licenses, go from rank I-III. At level 0, you gain three talent points to spend on
Talents, but can’t take any past rank I. When you level up, you gain +1 talent point to spend on
talents. Talents only apply to your pilot’s capabilities when piloting a mech (with a few
exceptions).


You can find the talent list in the compendium.


                                   PUTTING IT ALL TOGETHER
Constructing a mech may seem daunting at first, but is actually a fairly simple process.

At level 0:
     1.  Pick a FRAME from the licenses you have available to you. At level 0, this should just be
         the GMS Everest, and at level 2 you can already pick a new FRAME. Your FRAME gives
         your mech its armor, SP, mounts, base stats, and a CORE system
    2.   Add bonuses from mech skills
                 HULL: + 2HP/point, +1 repair cap/2 points
                 AGILITY: +1 evasion/point, +1 speed/2points
                 SYSTEMS: +1 tech attack, sensor range, and e-defense per point
                 ENGINEERING: +1 heat cap/point, +1 to all limited gear/2 points
    3.   Pick weapons for your FRAME mounts from those you have access to. You might choose
         different weapons depending on the size of each mount. At level 0, you have access to
         only the GMS weapon list.
    4.   Spend your SP on systems. You cannot spend over the amount your mech has, and any
         excess or unspent SP are lost. At level 0, you can only choose from GMS systems. At
         higher levels, you can add your grit to total SP on all mechs you make.
    5.   A mech can always mount duplicate weapons or systems, unless those weapons or
         systems have the Unique tag.
    6.   Allocate your talent points. At level 0, you can have 3 rank I talents.

    7.   You’re done!





                                             MOVING FORWARD

If you want to skip ahead to creating a mech right away, you can go directly to the compendium.
Though the choices in the full mech compendium are considerable, at level 0, a pilot only has
more limited access to the General Massive Systems gear. If you’re new to the game or learning
the system, it’s recommended you start there. You’re not ‘stuck’ with whatever mech you create,
as you can make a new one every mission if need be.

Before creating a mech, you might find it useful to read through the rules on damage, heat, and
repair in the following section, and the rules for mech combat in the following section.

\section{Damage}
 DAMAGE

You want to avoid or mitigate incoming damage as much as possible, but know this: Sometime,
somewhere, someone is going to punch a few holes in your kit.


Damage in LANCER comes in three types: Explosive, Energy, and Kinetic, each representing a
different type of weaponry or projectile.


Armor, like with pilots, reduces all incoming damage by the amount indicated. Damage with the
AP tag ignores armor.

Resistance reduces all incoming damage by 1/2 of a particular type. Mechs can only have
resistance once (it doesn’t stack) for a particular type of damage.


Damage in LANCER resolves as follows:

             1.  Reductions from armor

             2.  Reductions from systems, talents, and reactions, such as those that grant
                 resistance. Only one reduction of a type or system can be applied at once.

Remaining damage is dealt to your HP, your HIT POINTS.


For example: Your total HP is  15. You take fire from an enemy, who scores a successful hit by
beating your mech’s Evasion. You’re dealt 12 points of kinetic damage. Lucky for you, you have
armor installed on your mech, which subtracts 2 from all incoming damage, reducing the final
amount of incoming damage to  10. Your total HP after all modifications to incoming damage
have been applied is now 5. Take cover!

                                               Heat and Burn

Some weapons also deal Heat or Burn.

    -    Heat is not affected by armor, though it can be affected by resistance. It fills up a mech’s
         heat capacity (see the section below)

    -    Mechs affected by Burn immediately take damage equal to the burn they just took,
         ignoring armor. At the end of their turn, a mech can make an engineering check. On
        success, they clear all Burn on themselves, on failure, they take damage equal to their
        current Burn. Burn can stack with itself (so being hit by Burn 3 twice would increase it to
         Burn 6). Burn is its own damage type.

\subsection{Structure}
                                                Structure

If your mech is ever reduced to 0 HP, unlike a pilot, you are not down and out. Your mech is a
powerful machine that can take multiple hits before it starts to be break down. This is
represented by structure. Player mechs have 4 structure, NPC mechs might have less.





When a mech or other actor with structure is reduced to 0 HP, it takes 1 structure damage,
makes a structure check, then resets its HP to full. It then takes any damage that ‘spills
over’ (this could cause it to lose multiple points of structure and make multiple structure checks
in a turn). Structure damage represents major damage to your mech or its systems.


If your mech takes its last point of structure damage and checks its last box, it goes into the
CRITICAL state (see below). This is a state in which your mech is so heavily damaged that it
begins falling apart with every hit. NPCs that run out of structure typically are destroyed.

\subsubsection{Structure Damage}
                                        STRUCTURE DAMAGE
When your mech is reduced to 0 HP or when it takes any damage in the CRITICAL state, you roll
on the structure damage chart. Structure damage represents the results of unusually powerful or
accurate hits, which can disable a mech rapidly if not dealt with. When you make a structure
check, roll 1d6 per point of structure damage you have marked. When rolling multiple dice,
choose the lowest result, though certain outcomes activate if you also roll multiples 1s.


                                            STRUCTURE DAMAGE

 ROLL            RESULT                   EFFECT

 5-6             GLANCING BLOW            Emergency systems kick in and stabilize your mech. However, your
                                          mech is impaired until the end of your next turn.

 2-4             SYSTEM TRAUMA            Parts of your mech are torn off (potentially limbs). All the weapons
                                          on one mount or a system chosen by you is destroyed. If a system
                                          is used up (it has the limited tag and no charges left) it’s not a valid
                                          target. If there’s nothing left to destroy, this result becomes DIRECT
                                          HIT instead.

 1               DIRECT HIT               This result has different outcomes depending on how much
                                          structure damage your mech has remaining.

                                          3+ - Your mech is stunned until the end of your next turn.

                                          2 - Your mech must pass a hull check or be destroyed.
                                          1 - Your mech is destroyed.

 Two or          CRUSHING HIT             Your mech is damaged beyond repair and is destroyed. You can
 more 1s                                  still exit it as normal.

                                              The CRITICAL state


When your mech ticks off its last point of structure (typically 4), it immediately enters the
CRITICAL state, remaining at 0 hp. While CRITICAL:

             -    Your mech cannot repair or gain Hit Points.

             -    When you take damage, you make a structure check.

This allows a mech to stay fighting at great personal risk to the pilot. A mech can exit the
CRITICAL state only by resting or taking a full repair. When a mech exits the CRITICAL state it
returns to 1 HP.

\subsection{Heat and Overheating}

  HEAT \& OVERHEATING

Heat represents the stress of combat on a mech’s electronic systems and mechanical
components. Generally a mech is equipped with heat sinks, shunts, and coolant systems and to
operate within factory defined standards without generating heat. However, combat and
activated abilities can tax your mech’s heat dispersal systems to the point of causing actual
damage. Electronic warfare attacks, environmental hazards, weaponry, and overcharging can all
cause heat buildup.


Each Mech has a Heat Capacity that determines how much heat they can handle without things
getting dangerous. It can be increased through certain systems and by improving a mech’s
engineering score. A mech with a negative bonus to heat capacity has less than a mech with no
bonus. A mech reactor also can take a certain amount of stress before its reactor core is
breached and it starts to completely melt down. Most mechs have 4.


When a mech takes Heat, mark it off. If you gain heat that puts you up to your heat capacity or
over, check 1 reactor stress, then make an overheating check on the OVERHEATING chart by
rolling 1d6 per point of stress you have. If rolling multiple dice, choose the lowest result. Then
your mech fully cools, erasing all heat from the heat gauge. You take any heat that ‘spills over’ to
your gauge again. This could cause you to overheat more than once.


                                                OVERHEATING


 ROLL       RESULT                  EFFECT

 5-6        EMERGENCY               Cooling systems recover and manage to contain the peaking heat
            SHUNT                   levels. However, your mech is impaired until the end of your next turn.

 2-4        POWER PLANT             Your mech’s power plant becomes unstable, ejecting jets of plasma.
            DESTABILIZE             Your mech is Jammed until the end of your next turn

 1          MELTDOWN                This result has different outcomes depending on how much reactor
                                    stress your mech has remaining.

                                    3+ - Your mech is immediately shut down

                                    2 - Your mech must pass a engineering check or suffer a reactor
                                    meltdown at the end of 1d6 turns after this one (rolled by the GM). You
                                    can reverse it by taking a full action and repeating this check.

                                    1 - Your mech suffers a reactor meltdown at the end of your next turn

 Two 1s     IRREVERSIBLE            Your reactor goes critical. Your mech will suffer a reactor meltdown at
            MELTDOWN                the end of your next turn.

                                               COOLING HEAT




You can reset your heat gauge by taking the Stabilize action in combat or using other systems.
You also automatically cool heat when you rest or full repair. Whenever you cool heat, your gauge
resets, clearing all heat.


                                            The DANGER ZONE

When a mech has 1/2 of its total heat capacity filled, it’s in the danger zone. Certain mech
weapons and talents only activate in this area. While a mech is in this zone, it’s visible - parts of
your mech will be glowing, smoking, or steaming. Reactor vents or other cooling mechanisms
might be visible hot or working overtime.


                                              CORE BREACH

If you check your last (typically 4th) stress box, your mech enters the CORE BREACH state. In
this state your gauge does not reset, you can no longer cool, and whenever your mech takes
heat, it makes an overheating check. You can exit this state by resting or taking a full repair.

\subsubsection{Reactor Meltdown}

                                      REACTOR MELTDOWN

Certain critical and overheating table results can cause a reactor meltdown. This can be
immediate, or involve a countdown (in which case update the countdown at the start of the
round. The meltdown triggers when specified). When a mech suffers a reactor meltdown, any
pilot inside immediately dies, the mech is immediately vaporized in a catastrophic eruption (it
becomes completely unrepairable), and any mechs inside a burst 2 area centered on the mech
must pass an agility skill check or take 4d6 explosive damage, and half on a successful save.


\subsection{Repair}

                                               REPAIR

A Repair is in or out of combat healing to your mech. Repairs represent the resilience of your
mech and its ability to continue functioning while damaged, as well as physical assets such as
parts or tools. You can spend a repair by taking the Stabilize action in combat, repairing your
mech during a rest, or using systems that allow you to repair.


In combat, you can spend 1 repair as part of Stabilize to heal your mech to full HP.


During a rest, your mech cools all heat. You can then spend any number of repairs. 1 repair can:

    -    Refill HP to maximum

    -    Repair a destroyed weapon or system.

4 repairs can be spent to repair a destroyed mech (during a rest only, see below)


A pilot’s Repair Capacity is equal to 4+ HULL. This indicates the number of repairs a pilot can
make before returning to base - so if a mech’s repair capacity is 8, it can only spend 8 repairs
before taking a full repair. If a pilot has no repairs left, they cannot repair their mech! This
capacity refreshes to full when a pilot takes a full repair.


                                                 DESTROYED

When destroyed, a mech counts as permanently stunned and shut down until it is restored to
working condition (these conditions cannot be removed in any way). It then becomes an object
on the battlefield and provides cover accordingly. The wreck can be moved and dragged around.

\subsubsection{Repairing a Destroyed Mech}
                                  Repairing a Destroyed Mech

If a mech is destroyed and the wreck is present (not melted in a reactor explosion, for example), it
can be repaired to working order by spending 4 repairs during a rest. These repairs can be spent
from the mech’s own pool or the pools of any pilots that wish to contribute, in any combination. If a
mech has 0 repairs remaining, it can still be repaired if other mechs spend repairs, for example.
This is unique to repairing a destroyed mech.

Once repaired, the mech is returned to 1 structure at full HP, no matter how much it had before.
Any weapons or systems that are destroyed remain so unless that mech spends its own repairs to
fix them.
\subsubsection{Full Repair}
                                                Full Repair

If you take at least 10 hours of downtime in a secure location, you can Full repair. You can
repair all damage on your mech unless it is completely destroyed, returning it to full HP and




clearing all stress and structure (you can also repair destroyed systems). Your repair cap
refreshes to full, your pilot heals to full (or returns from being down and out), you can reset
your CRITICAL and heat gauges, and end all statuses other than destroyed (including
CRITICAL and CORE BREACH). You also regain core power if you lost it and get back all
(limited) use weapons, that you checked off.

\subsubsection{Printing}
                                                   Printing

If your mech is destroyed, even if you don’t have the wreck with you, you can rebuild it during a
full repair as long as you have access to the proper facilities. Mechs can be printed whole-cloth
from enormous Union printing facilities, which are generally ubiquitous across civilization. A
printer and assembler will perfectly recreate any mech or gear you have licenses for. If you need
to work on a destroyed mech and don’t have a printer, it’s also possible to repair it during a full
repair without a printer, but you need the wreck.


                                            One At a Time, Please

You’re only licensed to print a single active mech at a time. Only the newest mech you printed
will function (any others repaired, etc will cease to function).

\subsubsection{Rests}
                                                     Rests

A rest is defined as at least 1 hour of uninterrupted downtime or light activity (making camp,
routine maintenance, for example). After a rest, as long as you took action to do so:

             -   Cool heat
             -    Heal 1/2 your pilot HP or return from being down and out

             -    Exit the CRITICAL or CORE BREACH state if you’re in it and return to 1 HP and
                 clear 1 heat.
             -   You can spend any number of repairs to repair your mech, as long as you don’t
                 spend over your repair cap. You can also repair a destroyed mech.
             -   You can end any statuses currently affecting your mech automatically.


\subsection{Death}
  DEATH

The destruction of a mech does not always mean the death of a pilot. Pilots can escape and exit
from shutdown, disabled, or even destroyed mechs, presuming they survived. A pilot can always
re-create a mech - the pilot is much harder.

\subsubsection{Cloning}
                                                  CLONING

Pilots are tremendous investments in hardware and training and tend to have powerful and well-
connected patrons. It should not be surprising, then, that the technologies to resuscitate dead
flesh or create imperfect, flash-grown genetic clones of pilots, though often illegal or highly
secretive, do exist, and are often utilized by powerful organizations who don’t wish to give up on
their investments.


Cloning or revivification is a costly and dangerous process. It’s always up to the player whether
they want to bring a character back or simply make a new one. Flash-cloning or revivification
is an experimental process that always creates complications. These caveats are here by
default, and can be tweaked by the GM at their discretion:


    -    A cloned or revived character can only re-join the party after a mission’s completion.

    -    A cloned or revived character knows and learns nothing of the mission that they died on

    -    A cloned or revived character always comes back with a Quirk.

             -   The quirk could be physical or mental in nature, but whatever the quirk is, it
                 should be a story hook or something narrative in design (it shouldn’t have any
                 major gameplay effects).

             -   Quirks are always complicating - though your character might adjust to them in
                 time, they are a shock to the system.
    -    If a cloned or revived character would be cloned or revived a second time, they can no
         longer be played as a player character. The trauma and personality shift from being
         brought back to life is too great. In other words, you’re one and done.


If you want to roll for a random Quirk, you can roll 1d20 or choose from the below chart. You can
use these as examples for your own quirks and are free to figure out between you and your GM
what quirk your pilot comes back with.


                                                 Random Quirk


 Roll
      Quirk
 (1d20)

 1          Part (or all) of your body was too damaged or badly cloned and needed to be amputated and
            replaced with cybernetics. These are high quality prostheses, and are not visibly synthetic to a
            casual observer. The extent of the damage is unknown to you.




2          The process required you be fitted with a visible cybernetic augment, such as an arm, leg,
           eyes, or the like. It is conspicuous and often attracts unwanted attention.

3          By accident or malintent, you have been cloned into someone else’s body. They might be
           someone noteworthy or important.

4          You are cloned or revived with a nasty, disfiguring scar, a mutation, or a hideous appearance
           that clearly marks you as vat-grown.

5          Administrative mishaps lead to complete and drastic change in appearance in your new body

6          An extra, withered limb grows out of your chest shortly after your cloning or resurrection. It
           sometimes moves on its own.

7          A conspicuous barcode is now printed on your body. The barcode has meaning to powerful
           organizations, but you are not initially privy to its meaning.

8          Under certain light conditions, it is possible to read a script or inscription printed just under
           your skin. The script is all over your body and contains a scientific formula, a map, or other
           information contested by powerful organizations or entities.

9          Your new body is too frail to survive the exposure to direct light and air and requires you wear
           an environmental suit outside of sterilized environments or your mech.

10         DNA from a non-human or possible xenobiological source was used in your resuscitation. Your
           revivers will not tell you the exact details or what effects it will have on you long term, and treat
           you more as a science experiment. You now have a useful, visible (though able to be hidden)
           cosmetic variation.

11         You are stricken with persistent dreams, visions, and images of your death in vivid detail
           whenever you try and sleep or rest. You know they are all real, but cannot reconcile the
           existential gulf between what your previous “you” experienced, and your new subjectivity.

12         You are replaced by a digital ‘homunculus’, an electronic imprint and reconstruction of your
           personality that occupies a subaltern, a kind of robotic shell.

13         You are plagued by the constant understanding or belief that the ‘real’ you is actually dead,
           and you are merely a shadow aping a dead person, implanted with the memories of someone
           else. You cannot establish the difference between the “you” that died and the “you” that exists
           now.

14         Due to a clerical mishap, you are implanted with the residual memories of an entirely different
           and powerful or influential person. This reveals very dangerous and potentially unwanted
           information to you that is contested or sought after by powerful entities.

15         The process goes awry and you are revived tabula rasa. In desperation, the techs dump a
           stock personality construction into you. Change your background (adjust your skills
           accordingly)

16         The process of revival is not without complications. Your natural lifespan is dramatically
           shortened, and you know you will have to undergo another flash-cloning in the near future.

17         Something changed you, and you have persistent and intrusive mental contact with another
           entity or entities. It could be human or non-human in nature.

18         You often are struck with searing headaches during which you see brief flashes of what you
           are pretty sure is the future. Sometimes it comes to pass, sometimes it doesn’t.




 19        Knowingly or unknowingly, you are implanted with a mental trigger that when heard or
           activated, causes you to go into a receptive state, either following a pre-programmed course
           of action (kill, lie, etc) or to listen to and follow exactly the commands of the person who
           activated you. These commands must be simple, and the person who gives them (PC or NPC)
           is determined by the GM.

 20        You are brought back with complete amnesia of the time before you were re-born, causing a
           ‘tabula rasa’ situation in which you must be re-trained and cultured, a costly process. Your
           skills completely reset. Re-assign them as if you were level 0 and just leveled up to your
           current level.






