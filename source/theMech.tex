\chapter{The Mech}

\textbf{A mechanized cavalry unit -- a mech -- is the primary agent around which the Union Navy bases its ground forces.}

Mechs depending on the FRAME, stand anywhere from eight to forty feet tall and are bi-, qadra-, or hexapedal. The majority of them are brachial, featuring one to two pairs of arms able to manipulate to-scale weaponry and interact with the natural environment. Some pilots and units prefer to integrate their mech's weapon systems to the unit's chassis and, depending on the size or power of the weapon system, such an integrated system might be required. Most mechs are piloted by a single pilot, but there are larger, highly advanced platforms that require one or more additional pilots to control. 

Mechanized cavalry units are agile, quick, and responsive systems for their size. They are able to traverse most all solid and vacuum environments; their mobility is often augmented or entirely dependent on maneuvering jets (fuel depends on environment). Still, they are heavy, and in order to run they are powered by a cold fusion generator. Their power plant is heavily shielded and resistant to damage, reliable, and essentially inexhaustible, but should the reactor be compromised the results are often catastrophic. 

Mechanized cavalry typically support mounted or unmounted infantry as a heavy weapons platform or force multiplier. They operate on their own in hostile environments in squadrons of two to five, often dropped either far behind enemy lines or at the front, where the fighting is thickest and they can be used as line-breaker shock troops. 

Mechs are, by and large, military equipment, modular and restricted by license. However, they are common enough in civilian construction, hazardous materials cleanup, exploration, and other roles that they are not shocking to the average human. Mechanized Cavalry, though, are different: their pilots are regarded on the same level as knights or the flying aces of old.

\subimport{./theMech/}{piloting}

\subimport{./theMech/}{mechSkills}

\subimport{./theMech/}{modular}

\subimport{./theMech/}{damage}
