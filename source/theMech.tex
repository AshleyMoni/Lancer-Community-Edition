\section{The Mech}
THE MECH  

A mechanized cavalry unit -- a mech -- is the primary agent around which the Union Navy  
bases its ground forces.  

Mechs depending on the FRAME, stand anywhere from eight to forty feet tall and are bi-, qadra-,  
or hexapedal. The majority of them are brachial, featuring one to two pairs of arms able to  
manipulate to-scale weaponry and interact with the natural environment. Some pilots and units  
prefer to integrate their mech’s weapon systems to the unit’s chassis and, depending on the size  
or power of the weapon system, such an integrated system might be required. Most mechs are  
piloted by a single pilot, but there are larger, highly advanced platforms that require one or more  
additional pilots to control. 
 

Mechanized cavalry units are agile, quick, and responsive systems for their size. They are able to  
traverse most all solid and vacuum environments; their mobility is often augmented or entirely  
dependent on maneuvering jets (fuel depends on environment). Still, they are heavy, and in order  
to run they are powered by a cold fusion generator. Their power plant is heavily shielded and  
resistant to damage, reliable, and essentially inexhaustible, but should the reactor be  
compromised the results are often catastrophic. 
 

Mechanized cavalry typically support mounted or unmounted infantry as a heavy weapons  
platform or force multiplier. They operate on their own in hostile environments in squadrons of  
two to five, often dropped either far behind enemy lines or at the front, where the fighting is  
thickest and they can be used as line-breaker shock troops. 
 

Mechs are, by and large, military equipment, modular and restricted by license. However, they  
are common enough in civilian construction, hazardous materials cleanup, exploration, and other  
roles that they are not shocking to the average human. Mechanized Cavalry, though, are  
different: their pilots are regarded on the same level as knights or the flying aces of old. 
 
\subsection{Piloting a Mech}
                                          PILOTING A MECH  

You must be physically inside a mech to pilot it. Mechs that are powered off or inactive are  
typically shut down (they can’t do anything and it’s easy to hit them). You can mount or  
dismount a mech and boot up a mech by taking action to do so. While inside your mech, you  
don’t have line of sight to anything outside your mech (and nothing outside has line of sight to  
you. As long as your mech is not destroyed, you cannot be targeted, damaged, or affected by  

                                                                                                                


any attacks or effects that originate from outside your mech while inside it. If your mech is  
destroyed, you lose this benefit (it’s got holes blown in it!) and it merely grants cover.
 

\subsection{Mech Skills}
                                                  Mech Skills  

A mech operates on an entirely different level in LANCER: a powered, armored hulk with  
incredibly powerful systems, weaponry, and physical strength. A mech is capable of far more than  
a pilot on foot, and can perform feats of incredible strength, speed, and resilience where skill  
alone will not suffice.  

Your character’s pilot skills relate to their personal aptitudes on foot, whereas your character’s  
mech skill refer to their abilities piloting and building mechs. A Mech skill check can be referred to  
by the name of it’s statistic, such as a Hull check, a an Agility check, etc.  

Each mech skill has a primary benefit and use. Unlike pilot skills, mech skills also give your mech  
bonuses while building or creating a mech.  

HULL describes your ability to pilot and build mechs with high strength, structure, and frame  
reinforcement. Mechs with a high hull can survive a lot of punishment and perform feats of  
enormous strength.  

Roll Hull when: Smashing through or pulverising an obstacle, vehicle, or building; lifting, dragging,  
pushing, or hurling an enormous amount of weight; grabbing, wrestling, or grappling a mech,  
starship, or mech-sized creature; resisting a huge amount of force, staying upright through  
cataclysmic weather  

AGILITY describes your ability to pilot and build mechs with speed, reactivity, quickness, and  
dexterity. Mechs with a high agility are lightning fast for their size, and can react quickly and  
accurately.  

Roll Agility when: Chasing, pursuing, or fleeing with incredible speed; performing acrobatics in  
your mech; hiding or moving silently; performing a feat of fine manual dexterity with your mech;  
dodging out of the way of danger  

SYSTEMS describes your ability to build mechs with powerful computing and electronic systems,  
often aided by sentient or sub-sentient programs. Mechs with powerful Systems scores have a  
mind-boggling amount of computing power, some of which may not even exist in realspace.  

Roll Systems when: Infiltrating hardened and powerful electronic systems and targets, including  
other mechs; boosting or suppressing a signal; performing electronic warfare; scanning or analyze  
information; boosting or weakening the electronic systems of a vehicle, mech, starship, or base;  
interacting safely with Non-Human Persons or electronic life forms; analyzing the nature of  
unfamiliar electronic systems.  

                                                                                                                     


ENGINEERING describes your ability to build mechs with powerful reactors, high tech weaponry,  
ammunition, and utility tools. Mechs with a high engineering score typically incorporate a lot of  
advanced technology, are extremely resilient and reliable, can draw upon huge amounts of energy  
and heat, and can push their systems far beyond their factory specifications.  

Roll Engineering when: pushing your mech past its limits; withstanding extreme conditions such  
as heat, cold, void, or radiation; keeping your mech running well past its breaking point or for  
extreme amounts of time; traveling or move safely through hazardous or hostile conditions;  
boosting the reactor output of another mech, starship, or base; conserving and efficiently using  
ammo, power, and other resources  

Collectively, these four skills are often referred to as H.A.S.E. You may add the appropriate rating  
to a 1d20 roll for the total when making a skill check that calls for one of these ratings. You often  
make HASE checks as a part of mech combat, and certain actions will call for checks, such as  
hiding or grappling.
 
