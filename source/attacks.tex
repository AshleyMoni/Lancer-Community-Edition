\section{Attacks}


                                            ATTACKS  

Attacks in LANCER come in three types: Melee, Ranged, and Tech. Melee attacks are typically  
made against a target in your threat range, grit vs evasion. Ranged attacks are made agains a  
target in range, grit vs. evasion. Tech attacks are typically made against a target in your sensor  
range, tech attack vs. e-defense. 
 
\subsection{Range and Patterns}
                                      RANGE AND PATTERNS  
Measure weapon range from one of the edge spaces of your mech to the edge of your target by  
default, unless specified.   
Some weapons or systems have special attack patterns: Line, Cone, or Blast, or Burst. These  
attacks hit areas, and affect all targets in that area unless specified, rolling an attack separately  
for each target.  
    -    Line is a straight line X spaces long. All targets the line intersects with are attacked. Make  
         separate attack rolls for each target caught in the area.  

                                                                                                                 


     -   Cone is a cone X squares wide at its longest point and X squares long, drawn from a point  
         1 space wide at its shortest point (it’s origin). Make separate attack rolls for each target  
         caught in the area.  
    -    Blast is an area of radius X spaces, drawn from a point in range and line of sight. Check  
         cover and line of sight for the actual attack from the point of the blast, not the attacker.  
         Make separate attack rolls for each target caught in the area.  
    -    Burst is an area affecting the space over your mech and X spaces around your mech.  
         You’re not affected by your own burst attacks unless specified. Check cover and line of  
         sight from your mech. Make separate attack rolls for each target caught in the area.  

Some attacks with a line, cone, or blast pattern have a range listed. The starting point for the  
attacks can be drawn from a point within the range specified. For example: a blast 2, range 10  
attack, would attack a blast 2 area centered on any point within range 10.  

Some mech licenses or systems include increases to range. This range increase does not affect  
the size of cones, lines, or blast attacks (though it might allow you to place such attacks at further  
range if a range is specified).  
\subsection{Threat}
                                                     Threat  

Threat indicates the default distance at which a melee weapon can be used, and a melee or  
ranged weapon can be used to make an overwatch reaction. Default threat for all weapons is 1  
unless noted otherwise, and it can be increased from certain talents or pieces of gear. Measure  
threat from a mech’s exterior, so larger mechs will cover slightly more area than smaller mechs.  
\subsection{Valid Targets}
                                                Valid Targets  

You can attack any target in range (ranged), sensor range (tech), or weapon threat (melee) as  
long as you have line of sight to that target. Valid targets are other characters (player or non- 
player) such as other mechs, monsters, or people; objects that are not being held, worn, or part of  
a mech; and a point in the environment or on the ground.  
\subsection{Line of Sight}
                                                Line of sight  

If your character can’t trace of line of sight to a target (ie, you cannot see any part of the target),  
then it cannot be attacked (melee, ranged, tech, or otherwise). Weapons with the arcing tag can  
still attack a target or point you don’t have line of sight to as long as they could actually draw a  
path there (they couldn’t fire through 50ft of bulkheads, for example), but still take cover into  
account. They typically attack by lobbing projectiles over obstacles. Weapons with the powerful  
seeking tag totally ignore cover and line of sight, as long as they could draw a path to their  
target. Seeking weapons are typically self guided, self propelled, and can navigate complicated  
spaces.
 

                                                                                                                   
\subsection{Invisibility}

                                                 Invisibility  

Some characters have the ability to turn invisible. An invisible character is detectable by heat  
patterns and some visual artifacts, but extremely hard to target - all attacks of any kind have a  
flat 50% chance to miss outright (roll a dice or flip a coin) - checked before rolling. Additionally,  
an invisible mech can always hide.  
\subsection{Attacks}
                                                  ATTACKS  

You can attack as a mech by making the Skirmish, Barrage, or Tech actions while piloting your  
mech. You can attack by taking the Fight action as a pilot.
 
Ranged attack: Choose a target in your weapon range and line of sight. Then roll 1d20, adding  
your grit vs your target’s evasion, plus any Accuracy or Difficulty.
 
                      •  Being adjacent to a hostile target causes a character to be engaged. If your  
                        mech is engaged, it takes +1 difficulty on all ranged attack rolls.  
                      •  Light cover imposes +1 Difficulty a ranged attack roll. Heavy cover imposes  
                        +2 Difficulty to the attack roll.  
Melee attack: Choose a target in the weapon’s threat and line of sight, then roll 1d20, adding  
your grit vs. your target’s evasion, plus any Accuracy or Difficulty.
 
                      •  Melee attacks ignore cover
 
Tech attack: Choose a target in your sensor range and line of sight, then roll 1d20, adding your  
tech attack vs your target’s e-defense, plus any Accuracy or Difficulty. Tech attacks ignore  
cover.
 

To hit, your total roll must equal or exceed your target’s evasion or e-defense.
 
\subsection{Bonus Damage}
                                             Bonus damage  

Some talents, systems, or weapons allow you to deal bonus damage, allowing you to deal  
boosted or extra damage to your attack. Bonus damage can only be kinetic, explosive, or energy  
damage (not heat or burn), and if not specified is the same damage type as one type from the  
weapon that dealt it.
 

Bonus damage follows the following rules:
 
         	- If bonus damage applies to an area of effect attack or an attack that targets multiple  
         actors, it can only affect one target (the rest just take normal damage), called the primary  
         target. This is the target that takes the brunt of the attack.
 
         	- Bonus damage doesn’t apply if you make a bonus attack with an auxiliary weapon  

\subsection{Critical Hits}
                                                Critical Hits  
On any total ranged or melee weapon attack roll of 20+, the attack is a Critical Hit. Roll all  
damage dice twice and choose the highest result (including sources of bonus damage, etc).
 
