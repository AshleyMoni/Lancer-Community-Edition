\usepackage{enumerate}
\usepackage{pgfkeys}
\usepackage{xcolor,colortbl,longtable,float,supertabular,stringstrings,enumitem}

\providecommand{\fluff}[1]{{\itshape #1}}

\providecommand\corpName{}
\providecommand\mechName{}

\newcommand{\nosubsection}[1]{%
  \refstepcounter{subsection}%
  \addcontentsline{toc}{subsection}{#1}%{\protect\numberline{\thesubsection}#1}%
  \markright{#1}}


\newenvironment{mech}[2]
{
  \renewcommand{\corpName}{#1}
  \renewcommand{\mechName}{#2}
  % \corpName \space \mechName \\



  % remove spaces and punutation from argument, to translate mech name to filename
  % goddamn this is ugly though. http://mirror.las.iastate.edu/tex-archive/macros/latex/contrib/stringstrings/stringstrings.pdf for details
  \Treatments{1}{1}{0}{0}{0}{0}
  \substring[q]{#2}{1}{20}

  \IfFileExists{images/\thestring.png}{
    % If we have an image, skip the text title because the image already contains it, then load the image
    % The images are all large enough, with the floating H option, to force a newpage
    %\subsection{#1 \space #2}

    \clearpage % somehow we still need this.
    \nosubsection{#1 \space #2}
  
    \begin{figure}[H]
      \begin{center}
        \makeatletter
          \includegraphics{\thestring} 
        \makeatother
        \end{center}
    \end{figure}

  }{
    % If we can't find an image, clear the page and generate a subsection title 
    \clearpage
    \subsection{#1 \space #2}

  }

  
}
{}

\newenvironment{license}
{
License: \vspace*{-5mm}
\begin{enumerate}[label=\Roman*., itemsep=0mm, align=right]
}
{
\end{enumerate}
}

\pgfkeys {
  % we define a family, and enter it
  /frame/.is family, /frame,
  % we define some defaults
  default/.style =
  {
    hp = undefined,
    evasion = undefined,
    speed = undefined,
    heat cap = undefined,
    sensors = undefined,
    armor = undefined,
    e-defense = undefined,
    size = undefined,
    repair cap = undefined,
    tech attack = undefined,
    sp = undefined,
    traits = undefined,
    mount one = -,
    mount two = -,
    mount three = -,
    core system name = \empty,
    core system text = \empty,
    core active name = \empty,
    core active text = \empty,
  },
  % we define some keys
  hp/.store in = \frameHP,
  evasion/.store in = \frameEvasion,
  speed/.store in = \frameSpeed,
  heat cap/.store in = \frameHeatCap,
  sensors/.store in = \frameSensors,
  armor/.store in = \frameArmor,
  e-defense/.store in = \frameEDefense,
  size/.store in = \frameSize,
  repair cap/.store in = \frameRepairCap,
  tech attack/.store in = \frameTechAttack,
  sp/.store in = \frameSP,
  traits/.store in = \frameTraits,
  mount one/.store in = \frameMountOne,
  mount two/.store in = \frameMountTwo,
  mount three/.store in = \frameMountThree,
  core system name/.store in = \frameCoreSystemName,
  core system text/.store in = \frameCoreSystemText,
  core active name/.store in = \frameCoreActiveName,
  core active text/.store in = \frameCoreActiveText,
}

%\newcommand\frameBox[1][] {
%  \pgfkeys{/frame, default, #1}
%  \mechName \\
%  HP: \frameHP \\
%  Evasion: \frameEvasion \\
%  Speed: \frameSpeed \\
%  Heat Cap: \frameHeatCap \\
%  Sensors: \frameSensors \\
%  Armor: \frameArmor \\
%  E-Defense: \frameEDefense \\
%  Size: \frameSize \\
%  Repair Cap: \frameRepairCap \\
%  Tech Attack: \frameTechAttack \\
%  System Points: \frameSP \\
%  Traits: \frameTraits \\
%  Mount 1: \frameMountOne \\
%  Mount 2: \frameMountTwo \\
%  Mount 3: \frameMountThree \\
%  \textbf{\frameCoreSystemName} \\
%  \frameCoreSystemText \\
%  \textbf{\frameCoreActiveName} \\
%  \frameCoreActiveText \\
%}

\definecolor{Gray}{gray}{0.85}
\newcommand\frameBox[1][] {
\pgfkeys{/frame, default, #1}

% DO NOT use \begin{center} or \begin{table}!!!
{
      \centering
      \tablefirsthead{\hline }
      
      \tablehead{\hline}

      \tabletail{\hline}

      \tablelasttail{\hline}

      \begin{supertabular}{| p{25mm} | p{25mm} | p{25mm} | p{25mm} | p{25mm} |}
        \multicolumn{5}{|c|}{\textbf{\mechName}} \rule{0pt}{6mm}\\ 
        \hline
        \rowcolor{Gray}

        \textbf{HP:} \frameHP & \textbf{Evasion:} \frameEvasion & \textbf{Speed:} \frameSpeed & \textbf{Heat Cap:} \frameHeatCap & \textbf{Sensors:} \frameSensors \\ 
        \hline
        \textbf{Armor:} \frameArmor & \textbf{E-Defense:} \frameEDefense & \textbf{Size:} \frameSize & \textbf{Repair Cap:} \frameRepairCap & \textbf{Tech Attack:} \frameTechAttack \\ 
        \hline
        \rowcolor{Gray}
        \multicolumn{5}{|c|}{\textbf{TRAITS:}} \rule{0pt}{6mm}\\
        \hline
        \multicolumn{5}{|p{150mm}|}{\frameTraits} \\ 
        \hline
        \rowcolor{Gray}
        \multicolumn{5}{|c|}{\textbf{SYSTEM POINTS:} \frameSP} \rule{0pt}{6mm}\\
        \hline
        \rowcolor{Gray}
        \multicolumn{5}{|c|}{\textbf{MOUNTS:}} \rule{0pt}{6mm}\\ 
        \hline
        \multicolumn{2}{|p{50mm}|}{\frameMountOne} & \multicolumn{2}{|p{50mm}|}{\frameMountTwo} & \multicolumn{1}{|l|}{\frameMountThree} \\ 
      % \end{supertabular}
      % \vspace*{3mm}
      % \begin{supertabular}{| p{25mm} | p{25mm} | p{25mm} | p{25mm} | p{25mm} |}
        \hline
        \rowcolor{Gray}
        \multicolumn{5}{|c|}{\textbf{CORE system}} \rule{0pt}{6mm}\\ 
        \hline
        \multicolumn{5}{|c|}{\textbf{\frameCoreSystemName} } \\
        \multicolumn{5}{|p{150mm}|}{
          % These empty mathmode blocks ensure latex allocates some vspace when the line is otherwise empty.
          \fluff{\frameCoreSystemText} $ $} \\%\newline
        \multicolumn{5}{|p{150mm}|}{\textbf{\frameCoreActiveName} $ $} \\
        \multicolumn{5}{|p{150mm}|}{\frameCoreActiveText $ $} \\ \end{supertabular}

\vspace*{1cm}
\par
}
}
